\BoSSSopen{BoundaryAndInitialData/BoundaryAndInitialData}
\graphicspath{{BoundaryAndInitialData/BoundaryAndInitialData.texbatch/}}

\BoSSScmd{
restart
 }
\BoSSSexeSilent
\BoSSScmd{
/// This tutorial demostrates the definition, resp. the import of 
/// data for boundary and initial values. 
/// In order to demonstrate the usage, 
/// we employ the exemplaric Poisson solver.
 }
\BoSSSexe
\BoSSScmd{
using BoSSS.Application.SipPoisson;
 }
\BoSSSexe
\BoSSScmd{
/// We use a temporary database for this tutorial:
 }
\BoSSSexe
\BoSSScmd{
var tempDb = CreateTempDatabase();
 }
\BoSSSexe
\BoSSScmd{
/// We use the following helper function to create a template for 
/// the multiple solver runs.
 }
\BoSSSexe
\BoSSScmd{
Func<SipControl> PreDefinedControl = delegate() \{\newline 
\btab SipControl c = new SipControl();\newline 
 \newline 
\btab c.SetDGdegree(2);\newline 
 \newline 
\btab c.GridFunc = delegate() \{\newline 
\btab \btab // define a grid of 10x10 cells\newline 
\btab \btab double[] nodes = GenericBlas.Linspace(-1, 1, 11);\newline 
\btab \btab var grd = Grid2D.Cartesian2DGrid(nodes, nodes);\newline 
 \newline 
\btab \btab // set the entire boundary to Dirichlet b.c.\newline 
\btab \btab grd.DefineEdgeTags(delegate (double[] X) \{\newline 
\btab \btab \btab return BoundaryType.Dirichlet.ToString();\newline 
\btab \btab \});\newline 
 \newline 
\btab \btab return grd;\newline 
\btab \};\newline 
 \newline 
\btab c.SetDatabase(tempDb);\newline 
\btab c.savetodb = true; \newline 
 \newline 
\btab return c;    \newline 
\};
 }
\BoSSSexe
\BoSSScmd{
/// Again, we are using the workflow management
 }
\BoSSSexe
\BoSSScmd{
WorkflowMgm.Init("Demo\_BoundaryAndInitialData");
 }
\BoSSSexe
\BoSSScmd{
/// % ======================================
/// \section{Textual and Embedded formulas}
/// % ======================================
 }
\BoSSSexe
\BoSSScmd{
 % 
 }
\BoSSSexe
\BoSSScmd{
SipControl c1 = PreDefinedControl();
 }
\BoSSSexe
\BoSSScmd{
/// Provide initial data as a text:
 }
\BoSSSexe
\BoSSScmd{
c1.AddInitialValue("RHS","X => Math.Sin(X[0])*Math.Cos(X[1])",\newline 
\btab \btab \btab \btab \btab \btab TimeDependent:false);
 }
\BoSSSexe
\BoSSScmd{
/// Finally, all initial data is stored in the 
/// \code{AppControl.InitialValues} dictionary and 
/// all boundary data is stored in the 
/// \code{AppControl.BoundaryValues} dictionary.
 \newline 
/// The common interface for all varinats to specify boundary
/// and initial data is \code{IBoundaryAndInitialData}.
/// The snippet above is only a shortcut to add a \code{Formula} object,
/// which implements the \code{IBoundaryAndInitialData} interface.
 }
\BoSSSexe
\BoSSScmd{
c1.InitialValues;
 }
\BoSSSexe
\BoSSScmd{
c1.InitialValues["RHS"];
 }
\BoSSSexe
\BoSSScmd{
/// In \BoSSSpad, such objects can also be extracted from 
/// static methods of classes; note that these should not depend on any other
/// object in the worksheet.
 }
\BoSSSexe
\BoSSScmd{
static class BndyValue \{\newline 
\btab public static double BndyFunction(double[] X) \{\newline 
\btab \btab return 1.0;\newline 
\btab \}\newline 
\}
 }
\BoSSSexe
\BoSSScmd{
Formula BndyFormula = GetFormulaObject(BndyValue.BndyFunction);
 }
\BoSSSexe
\BoSSScmd{
c1.AddBoundaryValue(BoundaryType.Dirichlet.ToString(),\newline 
\btab \btab \btab \btab \btab "T",\newline 
\btab \btab \btab \btab \btab BndyFormula);
 }
\BoSSSexe
\BoSSScmd{
 % 
 }
\BoSSSexe
\BoSSScmd{
var J1 = c1.RunBatch();
 }
\BoSSSexe
\BoSSScmd{
LastError;
 }
\BoSSSexe
\BoSSScmd{
WorkflowMgm.BlockUntilAllJobsTerminate(3600*4);
 }
\BoSSSexe
\BoSSScmd{
/// What happened to the job?
J1.Status;
 }
\BoSSSexe
\BoSSScmdSilent{
/// BoSSScmdSilent
NUnit.Framework.Assert.IsTrue(J1.Status == JobStatus.FinishedSuccessful);
 }
\BoSSSexe
\BoSSScmd{
/// % ==================
/// \section{1D Splines}
/// % ==================
 }
\BoSSSexe
\BoSSScmd{
/// Splines can be used to interpolate nodal data onto a DG field;
/// currently, only 1D is supported.
 }
\BoSSSexe
\BoSSScmd{
SipControl c2 = PreDefinedControl();
 }
\BoSSSexe
\BoSSScmd{
// create test data for the spline\newline 
double[] xNodes = GenericBlas.Linspace(-2,2,13);\newline 
double[] yNodes = xNodes.Select(x => x*0.4).ToArray();
 }
\BoSSSexe
\BoSSScmd{
var rhsSpline = new Spline1D(xNodes, yNodes,\newline 
\btab \btab \btab \btab \btab \btab \btab  0,\newline 
\btab \btab \btab \btab \btab \btab \btab  Spline1D.OutOfBoundsBehave.Extrapolate);
 }
\BoSSSexe
\BoSSScmd{
 % 
 }
\BoSSSexe
\BoSSScmdSilent{
/// BoSSScmdSilent
double err = 0;\newline 
// test the spline: a line must be interpolated exactly.\newline 
foreach(double xtst in GenericBlas.Linspace(-3,3,77)) \{ \newline 
   double sVal = rhsSpline.Evaluate(new double[] \{xtst , 0, 0 \}, 0.0);\newline 
   double rVal = xtst*0.4;\newline 
   err += Math.Abs(sVal - rVal);\newline 
\}\newline 
NUnit.Framework.Assert.Less(err, 1.0e-10, "Slpine implementation fail.");
 }
\BoSSSexe
\BoSSScmd{
 % 
 }
\BoSSSexe
\BoSSScmd{
c2.AddInitialValue("RHS", rhsSpline);
 }
\BoSSSexe
\BoSSScmd{
var J2 = c2.RunBatch();
 }
\BoSSSexe
\BoSSScmd{
WorkflowMgm.BlockUntilAllJobsTerminate(3600*4);
 }
\BoSSSexe
\BoSSScmd{
J2.Status;
 }
\BoSSSexe
\BoSSScmdSilent{
/// BoSSScmdSilent
NUnit.Framework.Assert.IsTrue(J2.Status == JobStatus.FinishedSuccessful);
 }
\BoSSSexe
\BoSSScmd{
 % 
 }
\BoSSSexe
\BoSSScmd{
/// % =====================================================
/// \section{Interpolating values from other Calculations}
/// % =====================================================
 }
\BoSSSexe
\BoSSScmd{
/// For demonstrational purposes, we use the result (i.e. the last time-step) 
/// of a previous calculation as a right-hand-side for the next calculation.
 }
\BoSSSexe
\BoSSScmd{
var j2Sess = J2.LatestSession;
 }
\BoSSSexe
\BoSSScmd{
j2Sess;
 }
\BoSSSexe
\BoSSScmd{
j2Sess.Timesteps;
 }
\BoSSSexe
\BoSSScmd{
var lastTimeStep = j2Sess.Timesteps.Last();
 }
\BoSSSexe
\BoSSScmd{
/// We encapsulate the value T in the \code{ForeignGridValue} object,
/// which allows interpolation between different meshes:
 }
\BoSSSexe
\BoSSScmd{
var newForeignMesh = new ForeignGridValue(lastTimeStep,"T");
 }
\BoSSSexe
\BoSSScmd{
/// Use different mesh in the control file:
 }
\BoSSSexe
\BoSSScmd{
SipControl c3 = PreDefinedControl();
 }
\BoSSSexe
\BoSSScmd{
c3.GridFunc = delegate() \{\newline 
   // define a grid of *triangle* cells\newline 
   double[] nodes = GenericBlas.Linspace(-1, 1, 11);\newline 
   var grd = Grid2D.UnstructuredTriangleGrid(nodes, nodes);\newline 
 \newline 
   // set the entire boundary to Dirichlet b.c.\newline 
   grd.DefineEdgeTags(delegate (double[] X) \{\newline 
\btab    return BoundaryType.Dirichlet.ToString();\newline 
   \});\newline 
 \newline 
   return grd;\newline 
\};\newline 
// we also save the RHS in the database\newline 
c3.AddFieldOption("RHS", SaveOpt: FieldOpts.SaveToDBOpt.TRUE);
 }
\BoSSSexe
\BoSSScmd{
/// finally, we define the RHS:
 }
\BoSSSexe
\BoSSScmd{
c3.AddInitialValue("RHS", newForeignMesh);
 }
\BoSSSexe
\BoSSScmdSilent{
/// BoSSScmdSilent
double orgProbe = newForeignMesh.Evaluate(new double[] \{0.5,0.5\}, 0.0);\newline 
double newProbe = lastTimeStep.GetField("T").ProbeAt(new double[] \{0.5,0.5\});\newline 
NUnit.Framework.Assert.Less(Math.Abs(orgProbe - newProbe), 1.0e-10, "Check (1) on ForeignGridValue failed");
 }
\BoSSSexe
\BoSSScmd{
var J3 = c3.RunBatch();
 }
\BoSSSexe
\BoSSScmd{
WorkflowMgm.BlockUntilAllJobsTerminate(3600*4);
 }
\BoSSSexe
\BoSSScmd{
J3.Status;
 }
\BoSSSexe
\BoSSScmdSilent{
/// BoSSScmdSilent
NUnit.Framework.Assert.IsTrue(J3.Status == JobStatus.FinishedSuccessful);
 }
\BoSSSexe
\BoSSScmd{
/// \paragraph{Side Note:} Since the quadrilateral mesh used for the original
/// right-hand-side is geometrically embedded in the triangular mesh 
/// the interpolation error should be zero (up to machine precision).
 }
\BoSSSexe
\BoSSScmd{
var firstTimeStep = J3.LatestSession.Timesteps.First();
 }
\BoSSSexe
\BoSSScmd{
DGField RhsOnTriangles = firstTimeStep.GetField("rhs"); // case-insensitive!\newline 
DGField RhsOriginal    = lastTimeStep.GetField("T");
 }
\BoSSSexe
\BoSSScmd{
// note: we have to cast DGField to ConventionalDGField in order to use\newline 
// the 'L2Distance' function:\newline 
((ConventionalDGField)RhsOnTriangles).L2Distance((ConventionalDGField)RhsOriginal);
 }
\BoSSSexe
\BoSSScmdSilent{
/// BoSSScmdSilent
var H1err = ((ConventionalDGField)RhsOnTriangles).H1Distance((ConventionalDGField)RhsOriginal);\newline 
NUnit.Framework.Assert.Less(H1err, 1.0e-10, "Check (2) on ForeignGridValue failed.");
 }
\BoSSSexe
\BoSSScmd{
/// % ===================================
/// \section{Restart from Dummy-Sessions}
/// % ===================================
 }
\BoSSSexe
\BoSSScmd{
/// Dummy sessions are kind of fake siolver runs, with the only purpose 
/// of using them for a restart.
 }
\BoSSSexe
\BoSSScmd{
DGField RHSforRestart = firstTimeStep.GetField("RHS");
 }
\BoSSSexe
\BoSSScmd{
/// We save the DG field \code{RHSforRestart} in the database;
/// This automatically creates a timestep and a session which host the DG field:
 }
\BoSSSexe
\BoSSScmd{
var RestartTimestep = tempDb.SaveTimestep(RHSforRestart);
 }
\BoSSSexe
\BoSSScmd{
RestartTimestep
 }
\BoSSSexe
\BoSSScmd{
RestartTimestep.Session
 }
\BoSSSexe
\BoSSScmd{
/// This time step can be used as a restart value.:
 }
\BoSSSexe
\BoSSScmd{
var c4 = PreDefinedControl();
 }
\BoSSSexe
\BoSSScmd{
c4.InitialValues.Clear();\newline 
c4.SetRestart(RestartTimestep);
 }
\BoSSSexe
\BoSSScmd{
var J4 = c4.RunBatch();
 }
\BoSSSexe
\BoSSScmd{
WorkflowMgm.BlockUntilAllJobsTerminate(3600*4);
 }
\BoSSSexe
\BoSSScmd{
J4.Status;
 }
\BoSSSexe
\BoSSScmdSilent{
/// BoSSScmdSilent
NUnit.Framework.Assert.IsTrue(J4.Status == JobStatus.FinishedSuccessful);
 }
\BoSSSexe
\BoSSScmd{
/// \paragraph{Note:}
/// Since no mesh interpolation is performed for the restart, it is much faster
/// than \code{ForeignGridValue}, but less flexible 
/// (a restart is always performed on the same mesh).
///
/// To avoid multiple mesh interpolations (e.g. when multiple runs are required)
/// one coudl therefore speed up the process by doing the 
/// mesh interpolation once (use \emph{ProjectFromForeignGrid}) in BoSSSpad and
/// save the interpolation in a dummy session.
 }
\BoSSSexe
