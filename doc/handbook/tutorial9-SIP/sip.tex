\BoSSSopen{tutorial9-SIP/sip}
\graphicspath{{tutorial9-SIP/sip.texbatch/}}

\BoSSScmd{
/// \section*{What's new}
/// \begin{itemize}
/// \item Symmetric Interior Penalty method (SIP)
/// \item investigating matrix properties
/// \end{itemize}
/// \section*{Prerequisites}
/// \begin{itemize}
/// \item basics SIP method
/// \item spatial operator, chapter \ref{SpatialOperator}
/// \item implementing numerical fluxes and convergence study, chapter \ref{NumFlux}
/// \end{itemize}
///
///%_______________________________________________________
/// \section{Problem statement}
/// We consider the 2D Poisson problem:
/// \begin{equation*}
/// \Delta u = f(x,y)
/// \end{equation*}
/// with $f(x,y)\neq 0$ is an arbitrary function of $x$ and/or $y$ or constant.
/// Within this exercise, we are going to investigate the Symmetric Interior Penalty discretization method (SIP) for the Laplace operator:
/// \begin{equation*}
/// a_{\text{sip}}(u,v)
/// = \int_{\domain} \underbrace{\nabla u \cdot \nabla v}_{\text{Volume\ term}}\dV
///   - \oint_{\Gamma \setminus \Gamma_{N }} \underbrace{
///         \mean {\nabla u} \cdot n_{\Gamma}\jump{v}
///      }_{\text{consistency\ term}} + \underbrace{
///         \mean{\nabla v} \cdot \vec{n}_{\Gamma} \jump{u}
///      }_{\text{symmetry\ term}} \dA
///   + \oint_{\Gamma \setminus \Gamma_{N}} \underbrace{
///        \eta \jump{u}\jump{v}
///     }_{\text{penalty\ term}} \dA
/// \end{equation*}
/// The use of these fluxes including a penalty term stabilizes the DG-discretization for the Laplace operator.
///
///%____________________________________________
/// \section{Solution within the \BoSSS{} framework}
/// First, we initialize the new worksheet:
 }
\BoSSSexeSilent
\BoSSScmd{
restart;
 }
\BoSSSexeSilent
\BoSSScmd{
/// We need the following packages:
using ilPSP.LinSolvers;\newline 
using ilPSP.Connectors.Matlab;
 }
\BoSSSexe
\BoSSScmdSilent{
/// BoSSScmdSilent BoSSSexeSilent
using NUnit.Framework;
 }
\BoSSSexeSilent
\BoSSScmd{
/// \subsection{Implementation of the SIP fluxes}
/// We are going to implement the SIP-form
/// \begin{equation*}
/// a_{\text{sip}}(u,v)
/// = \int_{\domain} \underbrace{\nabla u \cdot \nabla v}_{\text{volume\ term}}\dV
///   - \oint_{\Gamma \setminus \Gamma_{N }} \underbrace{
///         \mean {\nabla u} \cdot n_{\Gamma}\jump{v}
///      }_{\text{consistency\ term}} + \underbrace{
///         \mean{\nabla v} \cdot \vec{n}_{\Gamma} \jump{u}
///      }_{\text{symmetry\ term}} \dA
///   + \oint_{\Gamma \setminus \Gamma_{N}} \underbrace{
///        \eta \jump{u}\jump{v}
///     }_{\text{penalty\ term}} \dA
/// \end{equation*}
/// First, we need a class in which the integrands are defined.
/// This also includes some technical aspects like the \code{TermActivationFlags}.
public class SipLaplace :\newline 
\btab \btab BoSSS.Foundation.IEdgeForm,   // edge integrals\newline 
\btab \btab BoSSS.Foundation.IVolumeForm  // volume integrals\newline 
\{\newline 
    /// We do not use parameters (e.g. variable viscosity, ...)
    /// at this point: so this can be null
\btab public IList<string> ParameterOrdering \{ \newline 
\btab \btab get \{ return new string[0]; \} \newline 
\btab \} \newline 
    /// but we have one argument variable, $u$ (our trial function)
\btab public IList<String> ArgumentOrdering \{ \newline 
\btab \btab get \{ return new string[] \{ "u" \}; \} \newline 
\btab \}\newline 
    /// The \code{TermActivationFlags} tell \BoSSS\ which kind of terms, 
    /// i.e. products of $u$, $v$, $\nabla u$, and $\nabla v$
    /// the \code{VolumeForm(...)} actually contains.
    /// This additional information helps to improve the performance.
\btab public TermActivationFlags VolTerms \{\newline 
\btab \btab get \{\newline 
\btab \btab \btab return TermActivationFlags.GradUxGradV;\newline 
\btab \btab \}\newline 
\btab \}\newline 
    /// activation flags for the 'InnerEdgeForm(...)'
\btab public TermActivationFlags InnerEdgeTerms \{\newline 
\btab \btab get \{\newline 
\btab \btab \btab return (TermActivationFlags.AllOn);\newline 
\btab \btab \btab // if we do not care about performance, we can activate all terms.\newline 
\btab \btab \}\newline 
\btab \}\newline 
\btab public TermActivationFlags BoundaryEdgeTerms \{\newline 
\btab    get \{\newline 
\btab \btab    return TermActivationFlags.AllOn;\newline 
\btab \btab \}\newline 
\btab \}\newline 
    /// For the computation of the penalty factor $\eta$,
    /// we require
    /// some length scale for each cell and 
    /// the polynomial degree of the DG approximation.
\btab public MultidimensionalArray cj;\newline 
\btab public int PolynomialDegree;\newline 
    /// The safety factor for the penalty factor should be in the order of 1.
    /// A very large penalty factor increases the condition number of the 
    /// system, but without affecting stability.
    /// A very small penalty factor yields to an unstable discretization.
\btab public double PenaltySafety = 2.2; \newline 
    /// The actual computation of the penalty factor, which should be 
    /// used in the \code{InnerEdgeForm} and \code{BoundaryEdgeForm} functions.
    /// Hint: for the parameters \code{jCellIn}, \code{jCellOut} and \code{g},
    /// take a look at
    /// \code{CommonParams} and \code{CommonParamsBnd}.
\btab double PenaltyFactor(int jCellIn, int jCellOut) \{\newline 
\btab \btab double cj\_in        = cj[jCellIn];\newline 
\btab \btab double penalty\_base = PenaltySafety*PolynomialDegree*PolynomialDegree;\newline 
\btab \btab double eta          = penalty\_base * cj\_in;\newline 
\btab \btab if(jCellOut >= 0) \{\newline 
\btab \btab \btab double cj\_out = cj[jCellOut];\newline 
\btab \btab \btab eta           = Math.Max(eta, penalty\_base * cj\_out);\newline 
\btab \btab \}\newline 
\btab \btab return eta;\newline 
\btab \}\newline 
    /// The following functions cover the actual math.
    /// For any discretization of the Laplace operator, we have to specify:
    /// \begin{itemize}
    /// \item a volume integrand,
    /// \item an edge integrand for inner edges, i.e. on $\Gamma_i$,
    /// \item an edge integrand for boundary edges, i.e. on $\partial \Omega$.
    /// \end{itemize}
    /// The integrand for the volume integral:
\btab public double VolumeForm(ref CommonParamsVol cpv, \newline 
\btab \btab    double[] U, double[,] GradU, // the trial-function u \newline 
\btab \btab    //            (i.e. the function we search for) and its gradient\newline 
\btab \btab    double V, double[] GradV     // the test function; \newline 
\btab \btab    ) \{\newline 
 \newline 
\btab \btab double acc = 0;\newline 
\btab \btab for(int d = 0; d < cpv.D; d++)\newline 
\btab \btab \btab acc += GradU[0, d] * GradV[d];\newline 
\btab \btab return acc;\newline 
\btab \}\newline 
    /// The integrand for the integral on the inner edges,
    /// \[
    ///   -( \mean{ \nabla u } \jump{ v }) \cdot \vec{n}_{\Gamma} 
    ///   -( \mean{ \nabla v } \jump{ u }) \cdot \vec{n}_{\Gamma} 
    ///   + \eta \jump{ u }  \jump{v} :
    /// \]
\btab public double InnerEdgeForm(ref CommonParams inp, \newline 
\btab \btab double[] U\_IN, double[] U\_OT, double[,] GradU\_IN, double[,] GradU\_OT, \newline 
\btab \btab double V\_IN, double V\_OT, double[] GradV\_IN, double[] GradV\_OT) \{\newline 
 \newline 
\btab \btab double eta = PenaltyFactor(inp.jCellIn, inp.jCellOut);\newline 
 \newline 
\btab \btab double Acc = 0.0;\newline 
\btab \btab for(int d = 0; d < inp.D; d++) \{ // loop over vector components \newline 
\btab \btab \btab // consistency term: -(\{\{ \textbackslash /u \}\} [[ v ]])*Normal\newline 
\btab \btab \btab // index d: spatial direction\newline 
\btab \btab \btab Acc -= 0.5 * (GradU\_IN[0, d] + GradU\_OT[0, d])*(V\_IN - V\_OT)\newline 
\btab \btab \btab \btab \btab    * inp.Normal[d];\newline 
 \newline 
\btab \btab \btab // symmetry term: -(\{\{ \textbackslash /v \}\} [[ u ]])*Normal\newline 
\btab \btab \btab Acc -= 0.5 * (GradV\_IN[d] + GradV\_OT[d])*(U\_IN[0] - U\_OT[0])\newline 
\btab \btab \btab \btab \btab    * inp.Normal[d];\newline 
\btab \btab \}\newline 
 \newline 
\btab \btab // penalty term: eta*[[u]]*[[v]]\newline 
\btab \btab Acc += eta*(U\_IN[0] - U\_OT[0])*(V\_IN - V\_OT);\newline 
\btab \btab return Acc;\newline 
\btab \} \newline 
    /// The integrand on boundary edges, i.e. on $\partial \Omega$, is
    /// \[ 
    ///   -( \mean{ \nabla u } \jump{ v }) \cdot \vec{n}_{\Gamma} 
    ///   -( \mean{ \nabla v } \jump{ u }) \cdot \vec{n}_{\Gamma} 
    ///   +  \eta \jump{ u }  \jump{v} .
    /// \]
    /// For the boundary we have to consider the special definition for 
    /// the mean-value operator $\mean{-}$ and the jump operator 
    /// $\jump{-}$ on the boundary.
\btab public double BoundaryEdgeForm(ref CommonParamsBnd inp, \newline 
\btab \btab double[] U\_IN, double[,] GradU\_IN, double V\_IN, double[] GradV\_IN) \{\newline 
 \newline 
\btab \btab double eta = PenaltyFactor(inp.jCellIn, -1);\newline 
\btab \btab double Acc = 0.0;\newline 
\btab \btab for(int d = 0; d < inp.D; d++) \{ // loop over vector components \newline 
\btab \btab \btab // consistency term: -(\{\{ \textbackslash /u \}\} [[ v ]])*Normale\newline 
\btab \btab \btab // index d: spatial direction\newline 
\btab \btab \btab Acc -= (GradU\_IN[0, d])*(V\_IN) * inp.Normal[d];\newline 
 \newline 
\btab \btab \btab // symmetry term: -(\{\{ \textbackslash /v \}\} [[ u ]])*Normale\newline 
\btab \btab \btab Acc -= (GradV\_IN[d])*(U\_IN[0]) * inp.Normal[d];\newline 
\btab \btab \}\newline 
 \newline 
\btab \btab // penalty term: eta*[[u]]*[[v]]\newline 
\btab \btab Acc += eta*(U\_IN[0])*(V\_IN);\newline 
 \newline 
 \newline 
\btab \btab return Acc;\newline 
\btab \}\newline 
\}
 }
\BoSSSexe
\BoSSScmd{
/// \subsection{Evaluation of the poisson operator in 1D}
/// We consider the following problem:
/// \begin{equation*}
/// \Delta u = 2,\quad -1<x<1,\quad u(-1)=u(1)=0.
/// \end{equation*}
/// The solution is $u_{ex}(x) = 1 - x^2$. Since this is quadratic, we can represent it \emph{exactly} in a DG space of order 2.
/// As usual, we have to set up a grid and a basis:
var grd1D                     = Grid1D.LineGrid(GenericBlas.Linspace(-1,1,10));\newline 
var gdata1D                   = new GridData(grd1D);\newline 
var DGBasisOn1D               = new Basis(gdata1D,2);\newline 
/// and a right-hand-side:
var RHS                       = new SinglePhaseField(DGBasisOn1D, "RHS");\newline 
RHS.ProjectField((double x) => 2);\newline 
/// We have to ensure to set the \code{PolynomialDegree} in the \emph{SipLaplace}-object.
var i\_SipLaplace              = new SipLaplace();\newline 
i\_SipLaplace.PolynomialDegree = DGBasisOn1D.Degree;\newline 
i\_SipLaplace.cj               = gdata1D.Cells.cj;\newline 
var Operator\_SipLaplace       = i\_SipLaplace.Operator();\newline 
/// We now want to calculate the residual after inserting the exact solution as well as a wrong solution. 
/// The implementation of the exact solution:
var u\_ex         = new SinglePhaseField(DGBasisOn1D, "$u\_\{ex\}$");\newline 
u\_ex.ProjectField((double x) => 1.0 - x*x);\newline 
/// The implementation of a spurious, i.e. a wrong solution; we take the exact solution and add random values in each cell:
var u\_wrong      = new SinglePhaseField(DGBasisOn1D, "$u\_\{wrong\}$");\newline 
u\_wrong.ProjectField((double x) => 1.0 - x*x);\newline 
Random R         = new Random();\newline 
for(int j = 0; j < gdata1D.Cells.NoOfLocalUpdatedCells; j++)\{\newline 
\btab double ujMean = u\_wrong.GetMeanValue(j);\newline 
\btab ujMean += R.NextDouble();\newline 
\btab u\_wrong.SetMeanValue(j, ujMean);\newline 
\btab \}\newline 
/// Evaluating the Laplace operator using the different solutions:
var Residual     = new SinglePhaseField(DGBasisOn1D,"Resi1");\newline 
var ResidualNorm = new List<double>();\newline 
foreach(var u in new DGField[] \{u\_ex, u\_wrong\}) \{\newline 
\btab Residual.Clear();\newline 
\btab Operator\_SipLaplace.Evaluate(u, Residual);  // evaluate\newline 
\btab Residual.Acc(-1.0, RHS);    \newline 
\btab double ResiNorm = Residual.L2Norm();\newline 
\btab ResidualNorm.Add(ResiNorm);\newline 
\btab Console.WriteLine("Residual for " + u.Identification + " = " + ResiNorm);  \newline 
\}
 }
\BoSSSexe
\BoSSScmdSilent{
/// tests BoSSScmdSilent
Assert.LessOrEqual(ResidualNorm[0], 1e-10);\newline 
Assert.GreaterOrEqual(ResidualNorm[1], 1e-1);
 }
\BoSSSexe
\BoSSScmd{
/// \subsection{The matrix of the Poisson Operator}
/// If we do not know the exact solution, we have to solve a linear system.
/// Therefore, we not only need to evaluate the operator,
/// but we need its matrix.
/// The \emph{Mapping} controls which degree-of-freedom of the DG approximation
/// is mapped to which row, resp. column of the matrix.
var Mapping           = new UnsetteledCoordinateMapping(DGBasisOn1D);\newline 
var Matrix\_SipLaplace = Operator\_SipLaplace.ComputeMatrix(Mapping,null,Mapping);
 }
\BoSSSexe
\BoSSScmd{
Matrix\_SipLaplace.NoOfCols;
 }
\BoSSSexe
\BoSSScmd{
Matrix\_SipLaplace.NoOfRows;
 }
\BoSSSexe
\BoSSScmd{
/// We see that the matrix has 27 rows and columns.
 }
\BoSSSexe
\BoSSScmd{
/// \paragraph{Matrix rank and determinant of the matrix 
///  \code{Matrix\_SipLaplace}:}
/// Use the functions \emph{rank} and \emph{det} to analyze the matrix (warning: this can get costly 
/// for larger matrices!).\\
///Interpret the results:
/// \begin{itemize}
/// \item What does it mean, when a matrix has full rank?
/// \item How many solutions can a linear system have?
/// \end{itemize}
double rank = Matrix\_SipLaplace.rank(); \newline 
Console.WriteLine("Matrix rank = " + rank);\newline 
 \newline 
double det = Matrix\_SipLaplace.det();   \newline 
Console.WriteLine("Determinante = " + det);\newline 
/// So the matrix of the SIP discretization has a unique solution.
 }
\BoSSSexe
\BoSSScmdSilent{
/// tests BoSSScmdSilent
Assert.AreEqual(rank, Matrix\_SipLaplace.NoOfCols);\newline 
Assert.Greater(det, 1.0);
 }
\BoSSSexe
\BoSSScmd{
 % 
 }
\BoSSSexe
\BoSSScmd{
/// \section{Advanced topics}
 }
\BoSSSexe
\BoSSScmd{
/// %========================================================
/// %========================================================
/// \subsection{The penalty parameter of the SIP and stability in 2D}
/// %========================================================
/// %========================================================
 }
\BoSSSexe
\BoSSScmd{
/// We define a two-dimensional grid:
var grd2D       = Grid2D.Cartesian2DGrid(GenericBlas.Linspace(-1,1,21), \newline 
\btab \btab \btab \btab \btab \btab \btab \btab \btab \btab  GenericBlas.Linspace(-1,1,16));\newline 
var gdata2D     = new GridData(grd2D);\newline 
var DGBasisOn2D = new Basis(gdata2D, 5);\newline 
var Mapping2D   = new UnsetteledCoordinateMapping(DGBasisOn2D);
 }
\BoSSSexe
\BoSSScmd{
/// We are going to choose the \code{PenaltySafety} for the \code{SipLaplace}
/// from the following list
double[] SFs = new double[] \newline 
\btab   \{0.001, 0.002, 0.01, 0.02, 0.1, 0.2, 1, 2, 10, 20, 100\};\newline 
/// and compute the condition number as well as the determinate.
 }
\BoSSSexe
\BoSSScmd{
/// We consider the example 
/// \[
///     -\Delta u = \pi^2 0.5 \cos(x/2) \cos(y/2) 
///       \text{ with } 
///       (x,y) \in (-1,1)^2
/// \]
/// and $u = 0$ on the boundary.
/// The exact solution is $u_{Ex}(x,y) = \cos(x/2) \cos(y/2)$.
Func<double[], double> exSol = \newline 
\btab \btab (X => Math.Cos(X[0]*Math.PI*0.5)*Math.Cos(X[1]*Math.PI*0.5));\newline 
Func<double[], double> exRhs = \newline 
\btab \btab (X => (Math.PI*Math.PI*0.5*Math.Cos(X[0]*Math.PI*0.5)\newline 
\btab \btab \btab \btab \btab   *Math.Cos(X[1]*Math.PI*0.5))); // == - /\textbackslash  exSol\newline 
SinglePhaseField RHS = new SinglePhaseField(DGBasisOn2D, "RHS");\newline 
RHS.ProjectField(exRhs);\newline 
double[] RHSvec = RHS.CoordinateVector.ToArray();
 }
\BoSSSexe
\BoSSScmd{
/// We check our discretization once more in 2D; the residual should be low,
/// but not exactly (resp. up to $10^{-12}$) since the solution is not 
/// polynomial and cannot be fulfilled exactly.
SinglePhaseField u = new SinglePhaseField(DGBasisOn2D,"u");\newline 
u.ProjectField(exSol);\newline 
i\_SipLaplace.PolynomialDegree = DGBasisOn2D.Degree;\newline 
i\_SipLaplace.cj               = gdata2D.Cells.cj;\newline 
var Matrix\_SIP\_sf     = Operator\_SipLaplace.ComputeMatrix(Mapping2D,\newline 
\btab \btab \btab \btab \btab \btab \btab \btab \btab \btab \btab \btab \btab \btab   null,\newline 
\btab \btab \btab \btab \btab \btab \btab \btab \btab \btab \btab \btab \btab \btab   Mapping2D);\newline 
SinglePhaseField Residual = new SinglePhaseField(DGBasisOn2D,"Residual");\newline 
Residual.Acc(1.0, RHS);\newline 
Matrix\_SIP\_sf.SpMV(-1.0, u.CoordinateVector, 1.0, Residual.CoordinateVector);\newline 
Console.WriteLine("Residual L2 norm: " + Residual.L2Norm());
 }
\BoSSSexe
\BoSSScmd{
/// We also check that the matrix is symmetric:
var checkMatrix = Matrix\_SIP\_sf - Matrix\_SIP\_sf.Transpose();\newline 
checkMatrix.InfNorm();
 }
\BoSSSexe
\BoSSScmdSilent{
/// tests BoSSScmdSilent
Assert.LessOrEqual(checkMatrix.InfNorm(), 1e-8);
 }
\BoSSSexe
\BoSSScmd{
/// \paragraph{Matrix properties for different penalty factors.}
/// Now, we assemble the matrix of the SIP for different 
/// \code{PenaltySafety}-factors. We also try to solve the linear system
/// using an iterative method. As Matlab is called multiple times during this 
/// command, it can take some minutes until it is done.
int cnt     = 0;\newline 
var Results = new List<Tuple<double,double,int,double,bool>>();\newline 
foreach(double sf in SFs) \{\newline 
\btab cnt++;\newline 
\btab i\_SipLaplace.PenaltySafety    = sf;\newline 
\btab i\_SipLaplace.PolynomialDegree = DGBasisOn2D.Degree;\newline 
\btab var Matrix\_SIP\_sf             = Operator\_SipLaplace.ComputeMatrix(\newline 
\btab \btab \btab \btab \btab \btab \btab \btab \btab Mapping2D,null,Mapping2D);\newline 
\btab double condNo1                = Matrix\_SIP\_sf.condest();  \newline 
\btab bool definite                 = Matrix\_SIP\_sf.IsDefinite();\newline 
 \newline 
    /// We solve the system 
    /// \[
    ///    \text{\tt Matrix\_SIP\_sf} \cdot u = \text{\tt RHS}
    /// \]
    /// using a an iterative solver, the so-called 
    /// conjugate gradient (CG) method.
    /// CG requires a positive definite matrix. 
    /// The function \code{Solve\_CG} returns the number of iterations.
\btab SinglePhaseField u = new SinglePhaseField(DGBasisOn2D,"u");\newline 
\btab u.InitRandom();\newline 
\btab int NoOfIter = Matrix\_SIP\_sf.Solve\_CG(u.CoordinateVector, RHSvec);\newline 
 \newline 
\btab SinglePhaseField Error = new SinglePhaseField(DGBasisOn2D,"Error");\newline 
\btab Error.ProjectField(exSol);\newline 
\btab Error.Acc(-1.0, u);\newline 
 \newline 
\btab double L2err = u.L2Error(exSol);\newline 
 \newline 
\btab Console.WriteLine(sf + "\textbackslash t" + condNo1.ToString("0.#E-00") \newline 
\btab \btab \btab \btab \btab \btab  + "\textbackslash t" + NoOfIter \newline 
\btab \btab \btab \btab \btab \btab  + "\textbackslash t" + L2err.ToString("0.#E-00") \newline 
\btab \btab \btab \btab \btab \btab  + "\textbackslash t" + definite);\newline 
\btab Results.Add(new Tuple<double,double,int,double,bool>(sf, condNo1, NoOfIter,\newline 
\btab \btab \btab \btab \btab \btab \btab \btab  L2err, definite));\newline 
\}
 }
\BoSSSexe
\BoSSScmdSilent{
/// tests BoSSScmdSilent
foreach(var r in Results) \{\newline 
\btab if(r.Item1 >= 1 && r.Item1 <= 20) \{\newline 
\btab \btab Assert.LessOrEqual(r.Item2, 1e7); // cond No.\newline 
\btab \btab Assert.LessOrEqual(r.Item3, 6000); // iter\newline 
\btab \btab Assert.LessOrEqual(r.Item4, 1e-4); // L2 err\newline 
\btab \btab Assert.IsTrue(r.Item5); // definite   \newline 
\btab \}\newline 
\btab if(r.Item1 <= 0.1) \{\newline 
\btab \btab Assert.IsFalse(r.Item5); // indefinite   \newline 
\btab \}\newline 
\}
 }
\BoSSSexe
\BoSSScmd{
/// \paragraph{Plotting:}
/// Plot the number of conjugate gradient iterations versus the 
/// \code{PenaltySafety}.
/// A logarithmic scale is used for the horizontal axis.
var format = new PlotFormat(lineColor: LineColors.Blue, \newline 
\btab \btab \btab \btab \btab \btab \btab pointSize: 2, \newline 
\btab \btab \btab \btab \btab \btab \btab dashType: DashTypes.DotDashed, \newline 
\btab \btab \btab \btab \btab \btab \btab Style: Styles.LinesPoints, \newline 
\btab \btab \btab \btab \btab \btab \btab pointType:PointTypes.OpenCircle);\newline 
Gnuplot gp = new Gnuplot(baseLineFormat:format);\newline 
gp.PlotLogXY(Results.Select(r => r.Item1), \newline 
\btab \btab \btab  Results.Select(r => ((double)(r.Item3))));\newline 
gp.PlotNow();
 }
\BoSSSexe
\BoSSScmd{
/// \paragraph{Convergence study, indefinite vs. definite.}
/// We are going to solve the SIP-system for different grid resolutions,
/// comparing an insufficient penalty to a penalty which is large enough.
double[] Resolution = new double[] \{ 8, 16, 32, 64, 128, 256 \};\newline 
List<double> L2Error\_indef  = new List<double>();\newline 
List<double> L2Error\_posdef = new List<double>();\newline 
foreach(int Res in Resolution) \{\newline 
\btab var grd2D = Grid2D.Cartesian2DGrid(GenericBlas.Linspace(-1,1,(int)Res + 1), \newline 
\btab \btab \btab \btab \btab \btab \btab \btab \btab    GenericBlas.Linspace(-1,1,(int)Res + 1));\newline 
\btab var gdata2D = new GridData(grd2D);\newline 
\btab var DGBasisOn2D = new Basis(gdata2D, 2);\newline 
\btab var Mapping2D  = new UnsetteledCoordinateMapping(DGBasisOn2D);\newline 
 \newline 
\btab SinglePhaseField RHS = new SinglePhaseField(DGBasisOn2D, "RHS");\newline 
\btab RHS.ProjectField(exRhs);\newline 
\btab SinglePhaseField uEx = new SinglePhaseField(\newline 
\btab \btab    new Basis(gdata2D, DGBasisOn2D.Degree*2),\newline 
\btab \btab    "Error");\newline 
\btab uEx.ProjectField(exSol);\newline 
 \newline 
\btab i\_SipLaplace.PolynomialDegree = DGBasisOn2D.Degree;\newline 
\btab i\_SipLaplace.cj               = gdata2D.Cells.cj;\newline 
 \newline 
\btab i\_SipLaplace.PenaltySafety    = 0.01;\newline 
\btab var Matrix\_SIP\_indef          = Operator\_SipLaplace.ComputeMatrix(\newline 
\btab \btab \btab \btab \btab \btab \btab \btab \btab Mapping2D,null,Mapping2D);\newline 
 \newline 
\btab SinglePhaseField u\_indef = new SinglePhaseField(DGBasisOn2D,"u");\newline 
\btab Matrix\_SIP\_indef.Solve\_Direct(u\_indef.CoordinateVector, \newline 
\btab \btab \btab \btab \btab \btab \btab \btab   RHS.CoordinateVector);\newline 
\btab var Error\_indef = uEx.CloneAs();\newline 
\btab Error\_indef.AccLaidBack(-1.0, u\_indef);\newline 
\btab L2Error\_indef.Add(Error\_indef.L2Norm());\newline 
 \newline 
    /// In order to have a positive definite system, we are
    /// using $\text{\tt PenaltySafety} = 2$!
\btab i\_SipLaplace.PenaltySafety = 2.0;\newline 
\btab var Matrix\_SIP\_posdef         = Operator\_SipLaplace.ComputeMatrix(\newline 
\btab \btab \btab \btab \btab \btab \btab \btab \btab Mapping2D,null,Mapping2D);\newline 
 \newline 
\btab SinglePhaseField u\_posdef = new SinglePhaseField(DGBasisOn2D,"u");\newline 
\btab Matrix\_SIP\_posdef.Solve\_Direct(u\_posdef.CoordinateVector, \newline 
\btab \btab \btab \btab \btab \btab \btab \btab    RHS.CoordinateVector);\newline 
\btab var Error\_posdef = uEx.CloneAs();\newline 
\btab Error\_posdef.AccLaidBack(-1.0, u\_posdef);\newline 
\btab L2Error\_posdef.Add(Error\_posdef.L2Norm());\newline 
 \newline 
\btab Console.WriteLine(L2Error\_indef.Last().ToString("0.#E-00") \newline 
\btab \btab \btab \btab \btab   + "\textbackslash t" + L2Error\_posdef.Last().ToString("0.#E-00"));\newline 
\}
 }
\BoSSSexe
\BoSSScmd{
/// \paragraph{Convergence plot:}
/// The convergence plot unveils that there is something wrong if the
/// penalty factor is set too low.
/// While the solution of the indefinite system may look right at the first
/// glance, we see that we do not obtain grid convergence for 
/// \code{Error\_indef}.
/// The error of the positive definite system, \code{Error\_posdef}, where the 
/// penalty is chosen sufficiently large converges with the expected 
/// rate.
var format = new PlotFormat(lineColor: LineColors.Blue, \newline 
\btab \btab \btab \btab \btab \btab \btab pointSize: 2, \newline 
\btab \btab \btab \btab \btab \btab \btab dashType: DashTypes.DotDashed, \newline 
\btab \btab \btab \btab \btab \btab \btab Style: Styles.LinesPoints, \newline 
\btab \btab \btab \btab \btab \btab \btab pointType:PointTypes.OpenCircle);\newline 
Gnuplot gp = new Gnuplot(baseLineFormat:format);\newline 
gp.PlotLogXLogY(Resolution, L2Error\_indef);\newline 
gp.PlotLogXLogY(Resolution, L2Error\_posdef);\newline 
gp.PlotNow();
 }
\BoSSSexe
\BoSSScmd{
/// Finally, we are going to compute the convergence rate of the SIP
/// discretization: we take the logarithm of the resolution and the error:
 }
\BoSSSexe
\BoSSScmd{
var log\_h   = Resolution.Select(h => Math.Log10(h)).ToArray();\newline 
var log\_Err = L2Error\_posdef.Select(h => Math.Log10(h)).ToArray();
 }
\BoSSSexe
\BoSSScmd{
var LeastSquaresSys = MultidimensionalArray.Create(log\_h.Length,2);\newline 
LeastSquaresSys.SetColumn(0, log\_h.Length.ForLoop(i => 1.0));\newline 
LeastSquaresSys.SetColumn(1, log\_h);\newline 
double[] dk = new double[2]; // intercept and slope of best-fit line\newline 
LeastSquaresSys.LeastSquareSolve(dk, log\_Err);
 }
\BoSSSexe
\BoSSScmd{
dk;
 }
\BoSSSexe
\BoSSScmdSilent{
/// tests BoSSScmdSilent
Assert.LessOrEqual(dk[1], -2.9);
 }
\BoSSSexe
\BoSSScmd{
/// \section{Further reading}
/// \begin{itemize}
/// \item
/// \bibentry{DiPietroErn2011}
/// \item
/// \bibentry{Arnold_1982} 
/// \end{itemize}
 }
\BoSSSexe
