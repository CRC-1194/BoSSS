% !TeX spellcheck = en_GB

\BoSSSopen{MetaJobManager/MetaJobManager}
\graphicspath{{MetaJobManager/MetaJobManager.texbatch/}}

\paragraph{What's new:} 
\begin{itemize}
	\item running multiple simulations in batch mode
	\item deploying and running simulations on a remote machine, e.g. a Microsoft HPC Cluster
	\item access statistics for multiple simulations
\end{itemize}

\paragraph{Prerequisites:} 
\begin{itemize}
	\item basic knowledge of \BoSSSpad{}
	\item executing runs on your local machine, e.g. the quickstart guide of the \ac{cns}, see chapter 5
\end{itemize}

\BoSSS ~includes several tools which aid the advanced user in running simulations and organizing simulation results among multiple computers and/or compute clusters.

%In this tutorial, the following tasks will be illustrated:
%\begin{itemize}
%	\item The use of the meta job manager to run simulations, 
%	on the local machine as well as on a Microsoft HPC cluster
%	\item The evaluation of a simulation using the session table.
%\end{itemize}

The purpose of the meta job manager (a.k.a. the meta job scheduler)
is to provide a universal interface to multiple \emph{job managers} (aka. \emph{batch systems}).
By this interface, the user can run computations on remote and local systems directly from the \BoSSSpad{}.

Furthermore, \BoSSSpad{}  provides a \emph{session table},
which presents various statistics from all \emph{sessions} in the project
(each time a simulation is run, a session is stored in the \BoSSS ~database).

Before you dive into advanced features of \BoSSSpad{}, you should familiarise yourself with the basics of \BoSSS and \BoSSSpad{} and run some first simulations on your local machine, e.g. the tutorials on scalar advection \ref{ScalarAdvection} or Poisson equation \ref{sec:SIP}.

The examples presented in this chapter use the compressible Navier-Stokes solver (CNS), which has been introduced in the previous chapter \ref{CNS}.



 \section{Initialization}
\BoSSScmd{
restart;
 }
\BoSSSexeSilent
\BoSSScmd{
/// The very first thing we have to do is to initialize the workflow
/// management tools and to define a project name. The meta job manager, like  
/// all other workflow management tools strictly enforces the use of a 
/// project name:
WorkflowMgm.Init("MetaJobManager\_Tutorial");
 }
\BoSSSexe
\BoSSScmd{
/// We verify that we have no jobs defined so far
WorkflowMgm.AllJobs;
 }
\BoSSSexe
\BoSSScmdSilent{
/// BoSSScmdSilent BoSSSexeSilent
// first, do some cleanup\newline 
if(System.IO.Directory.Exists(@"C:\textbackslash tmp\textbackslash TutorialLocal\_db")) \{\newline 
\btab System.IO.Directory.Delete(@"C:\textbackslash tmp\textbackslash TutorialLocal\_db",true);\newline 
\} else \{\newline 
\btab //\newline 
\}
 }
\BoSSSexeSilent
\BoSSScmd{
/// and create, resp. open a \BoSSS database on a local drive.
var myLocalDb = OpenOrCreateDatabase(@"C:\textbackslash tmp\textbackslash TutorialLocal\_db");
 }
\BoSSSexe
\BoSSScmdSilent{
/// BoSSScmdSilent BoSSSexeSilent
databases;
 }
\BoSSSexeSilent
\BoSSScmd{
/// % ==================================================================
/// \section{Loading a BoSSS-Solver and Setting up a Simulation}
/// % ==================================================================
 }
\BoSSSexe
\BoSSScmd{
/// We have to initialize all batch systems that we want to use.
/// Normally, one would put these statements into the 
/// {\tt $\sim$/.BoSSS/etc/DBErc.cs}-file
if(System.IO.Directory.Exists(@"C:\textbackslash tmp\textbackslash deploy"))\{\newline 
\btab System.IO.Directory.Delete(@"C:\textbackslash tmp\textbackslash deploy",true);\newline 
\}\newline 
System.IO.Directory.CreateDirectory(@"C:\textbackslash tmp\textbackslash deploy");\newline 
var myBatch = new MiniBatchProcessorClient(@"C:\textbackslash tmp\textbackslash deploy");
 }
\BoSSSexe
\BoSSScmd{
/// The batch processor for local jobs can be started separately (by launching
/// {\tt MiniBatchProcessor.exe}), or from the worksheet as follows.
/// In the latter case, it depends on the operating system, whether the 
/// \newline {\tt MiniBatchProcessor.exe} is terminated with the worksheet, or not.
MiniBatchProcessor.Server.StartIfNotRunning(false);
 }
\BoSSSexe
\BoSSScmd{
/// In this tutorial, we use the workflow management tools to simulate compressible flows, thus we first have to load the namespaces of the CNS solver
using CNS;
 }
\BoSSSexe
\BoSSScmd{
/// and obtain its type and location
Type solver1 = typeof(CNS.Program);\newline 
solver1.Assembly.Location;\newline 
/// For other solvers, we might have to load the assemblies too, e.g. by 
// LoadAssembly("CNS.exe");
 }
\BoSSSexe
\BoSSScmd{
/// In this example, we start the computation from a hard-coded
/// control object that is already compiled into the code. If we started the 
/// {\tt CNS.exe} directly from the command line, the command line 
/// call would be
/// {\tt ./CNS.exe -control cs:CNS.ControlExamples.IsentropicVortex(...) }.
/// On the command line, the arguments could be problematic, e.g. quotation 
/// marks would need some kind of escape string, spaces are problematic, etc.
/// The meta job manager passes the arguments via environment variables, so
/// the before-mentioned issues do not occur. 
string StartupString\_Local = \newline 
  string.Format("cs:CNS.ControlExamples.IsentropicVortex(@\textbackslash "\{0\}\textbackslash ", 20, 2, 1.0)", \newline 
\btab \btab \btab \btab myLocalDb.Path);
 }
\BoSSSexe
\BoSSScmd{
/// % ==================================================================
/// \section{Deploying the jobs}
/// % ==================================================================
 }
\BoSSSexe
\BoSSScmd{
/// Finally, we define two jobs: one for the local machine, one for the 
/// HPC cluster.
/// In a usual work flow scenario, we \emph{do not} want to (re-) submit the 
/// job every time we run the worksheet -- usually, one wants to run a job once.
/// 
/// The concept to overcome this problem is job activation. If a job is 
/// activated, the meta job manager first checks the databases and the batch 
/// system, if a job with the respective name and project name is already 
/// submitted. Only if there is no information that the job was ever submitted
/// or started anywhere, the job is submitted to the respective batch system.
 }
\BoSSSexe
\BoSSScmd{
var oneJob\_Local = new Job("byCmdLine\_Local", solver1);
 }
\BoSSSexe
\BoSSScmd{
oneJob\_Local.SetCommandLineArguments(StartupString\_Local);
 }
\BoSSSexe
\BoSSScmd{
oneJob\_Local.Activate(myBatch);
 }
\BoSSSexe
\BoSSScmd{
/// All jobs can be listed using the workflow management:
 }
\BoSSSexe
\BoSSScmd{
WorkflowMgm.AllJobs
 }
\BoSSSexe
\BoSSScmd{
/// Here, we block until both of our jobs have finished:
WorkflowMgm.BlockUntilAllJobsTerminate(360);
 }
\BoSSSexe
\BoSSScmd{
/// We examine the output and error stream of the job on the HPC cluster:
/// This directly accesses the {\tt stdout}-redirection of the respective job
/// manager, which may contain a bit more information than the 
/// {\tt stdout}-copy in the session directory.
oneJob\_Local.Stdout;
 }
\BoSSSexe
\BoSSScmd{
/// Additionally we display the error stream and hope that it is empty:
oneJob\_Local.Stderr;
 }
\BoSSSexe
\BoSSScmd{
/// We can also obtain the session which was stored during the execution of the 
/// job:
 }
\BoSSSexe
\BoSSScmd{
var Sloc = oneJob\_Local.LatestSession;\newline 
Sloc;
 }
\BoSSSexe
\BoSSScmd{
/// Finally, we check the status of our jobs:
 }
\BoSSSexe
\BoSSScmd{
oneJob\_Local.Status;
 }
\BoSSSexe
\BoSSScmd{
/// % ==================================================================
/// \section{Accessing the session table}
/// % ==================================================================
 }
\BoSSSexe
\BoSSScmd{
/// We can access all sessions related to the project via the session table:
 }
\BoSSSexe
\BoSSScmd{
var Table = WorkflowMgm.SessionTable;
 }
\BoSSSexe
\BoSSScmd{
/// The size of the table, resp. the number of columns is usually 
/// quite extensive, ...
 }
\BoSSSexe
\BoSSScmd{
Table.GetColumnNames().Length;
 }
\BoSSSexe
\BoSSScmd{
Table.GetColumnNames().Take(12);
 }
\BoSSSexe
\BoSSScmd{
/// ... therefore we use some utility function to extract a sub set of columns:
 }
\BoSSSexe
\BoSSScmd{
Table.ExtractColumns("SessionName", "DGdegree:rho", "L2ErrorDensity").Print();
 }
\BoSSSexe
\BoSSScmd{
 % 
 }
\BoSSSexe
\BoSSScmd{
 % 
 }
\BoSSSexe
\BoSSScmd{
/// % ==================================================================
/// \section{Further Reading: Setting up a simulation on a remote machine via Control Objects}
/// % ==================================================================
 }
\BoSSSexe
\BoSSScmd{
/// Here, we use the workflow management tools to simulate compressible flows \emph{with} immersed boundaries, thus we first have to load the namespaces of the CNS and IBM solver:
using BoSSS.Application.IBM\_Solver;
 }
\BoSSSexe
\BoSSScmd{
Type solver2 = typeof(IBM\_SolverMain);\newline 
solver2.Assembly.Location;
 }
\BoSSSexe
\BoSSScmdSilent{
/// BoSSScmdSilent BoSSSexeSilent
// first, do some cleanup\newline 
if(System.IO.Directory.Exists(@"C:\textbackslash tmp\textbackslash TutorialNetworkDb\_db"))\newline 
\btab System.IO.Directory.Delete(@"C:\textbackslash tmp\textbackslash TutorialNetworkDb\_db",true);
 }
\BoSSSexeSilent
\BoSSScmd{
/// For this tutorial, we use a local drive to simulate the network database:\newline 
var myNetworkDb = OpenOrCreateDatabase(@"C:\textbackslash tmp\textbackslash TutorialNetworkDb\_db");\newline 
/// When using a real network drive, the path to the drive would be given as: 
//var myNetworkDb = OpenOrCreateDatabase(@"\textbackslash \textbackslash \textbackslash $server\textbackslash $ \textbackslash \textbackslash $path\_to\_database\textbackslash $");
 }
\BoSSSexe
\BoSSScmd{
/// Again, we fake the network capabilities:
var myHPC = myBatch;  \newline 
/// For a Microsoft HPC-Server this would read:
//var myHPC   = new MsHPC2012Client(@"\textbackslash \textbackslash dc1\textbackslash scratch\textbackslash hpcdeploy", \newline 
//                          "hpccluster",\newline 
//                          ComputeNodes: new[] \{"hpccluster", "pcmit30"\});  \newline 
///   Remark: the Ms HPC client is deactivated  
///   because it does not work on the build server which creates this  
///   documentation.
 }
\BoSSSexe
\BoSSScmd{
/// For the second job, we use a control object. The advantage of the control  
/// object is that it can be completely constructed (or modified) within
/// \BoSSSpad, without the need of an external file.
/// In order to shorten things, we start with a predefined control object
/// and modify it.
var StartupObject\_Network = HardcodedControl.ChannelFlow();
 }
\BoSSSexe
\BoSSScmd{
/// E.g., we can mofify individual members:
StartupObject\_Network.DbPath              = myNetworkDb.Path;\newline 
StartupObject\_Network.NoOfTimesteps       = 2;\newline 
StartupObject\_Network.MaxSolverIterations = 35;\newline 
StartupObject\_Network.savetodb            = true;
 }
\BoSSSexe
\BoSSScmd{
/// The only technical problems are formulas and expressions used e.g. to set
/// initial values or queries; those objects cannot be serialized, i.e.
/// stored in a file since they are live code. (If we work with
/// control objects, \BoSSSpad ~tries to save them as a file, so they can later
/// be loaded by the solver.)
 }
\BoSSSexe
\BoSSScmd{
/// E.g. the following would not work:
StartupObject\_Network.InitialValues\_Evaluators["VelocityX"] = X => 0.5;\newline 
/// The reason is that the expression {\tt X => 0.5} is only valid within
/// this \BoSSSpad ~document and cannot be saved in a file and loaded again
/// due to technical restrictions of .NET resp. C\#.
///
/// In order to make this work, we have to use a \emph{text} representation of
/// our initial values.
 }
\BoSSSexe
\BoSSScmd{
/// But first, we clear the entries from the predefined object:
StartupObject\_Network.InitialValues\_Evaluators.Clear();\newline 
StartupObject\_Network.Queries.Clear();
 }
\BoSSSexe
\BoSSScmd{
/// Especially for more complex flows, the mathematical form of the solution,
/// resp. the initial value is rather complicated. 
/// Therefore, one could use a couple of auxilliary definitions:
 }
\BoSSSexe
\BoSSScmd{
string CommonCode = @"\newline 
\btab Random r = new Random();\newline 
\btab Func<double[],double> Noise = delegate(double[] X) \{\newline 
\btab \btab double y          = X[1];\newline 
\btab \btab double NoiseScale = (1 - y*y);\newline 
\btab \btab return r.NextDouble()*NoiseScale;\newline 
\btab \};\newline 
";
 }
\BoSSSexe
\BoSSScmd{
/// Using the additional code defined above, we are able to set 
/// the initial values...
StartupObject\_Network.InitialValues.Clear();\newline 
StartupObject\_Network.InitialValues.Add("VelocityX", \newline 
\btab new Formula("X => Noise(X)*(1 - X[1]*X[1])", false, CommonCode));\newline 
StartupObject\_Network.InitialValues.Add("VelocityY", \newline 
\btab new Formula("X => 0.0", false));
 }
\BoSSSexe
\BoSSScmd{
/// ... as well as the boundary conditions for the control object:
StartupObject\_Network.BoundaryValues.Clear();\newline 
StartupObject\_Network.AddBoundaryCondition("Velocity\_inlet");\newline 
StartupObject\_Network.AddBoundaryCondition("Pressure\_Outlet");\newline 
StartupObject\_Network.AddBoundaryCondition("Wall\_bottom");\newline 
StartupObject\_Network.AddBoundaryCondition("Wall\_top");\newline 
StartupObject\_Network.BoundaryValues["Velocity\_inlet"].Values.Add(\newline 
\btab \btab "VelocityX",\newline 
\btab \btab new Formula("X => (1 - X[1]*X[1])", false));
 }
\BoSSSexe
\BoSSScmd{
/// The same holds for the grid: we cannot use a delegate to construct the 
/// grid. Instead, we have to store the grid (including boundary
/// conditions) in the database.
 }
\BoSSSexe
\BoSSScmd{
StartupObject\_Network.GridFunc = null;
 }
\BoSSSexe
\BoSSScmd{
var xNodes = GenericBlas.Linspace(0, 10, 101);\newline 
var yNodes = GenericBlas.Linspace(-1, 1, 21);\newline 
var Grid   = Grid2D.Cartesian2DGrid(xNodes, yNodes);
 }
\BoSSSexe
\BoSSScmd{
Grid.EdgeTagNames.Add(1, "Velocity\_inlet");\newline 
Grid.EdgeTagNames.Add(4, "Pressure\_Outlet");\newline 
Grid.EdgeTagNames.Add(2, "Wall\_bottom");\newline 
Grid.EdgeTagNames.Add(3, "Wall\_top");
 }
\BoSSSexe
\BoSSScmd{
Grid.DefineEdgeTags(delegate (double[] \_X) \{\newline 
\btab var X    = \_X;\newline 
\btab double x = X[0];\newline 
\btab double y = X[1];\newline 
 \newline 
\btab if (Math.Abs(y - (-1)) < 1.0e-6)\newline 
\btab    // bottom\newline 
\btab    return 2;\newline 
 \newline 
\btab if (Math.Abs(y - (+1)) < 1.0e-6)\newline 
\btab    // top\newline 
\btab    return 3;\newline 
 \newline 
\btab if (Math.Abs(x - (0)) < 1.0e-6)\newline 
\btab    // left\newline 
\btab    return 1;\newline 
 \newline 
\btab if (Math.Abs(x - (10)) < 1.0e-6)\newline 
\btab    // right\newline 
\btab    return 4;\newline 
\btab throw new Exception();\newline 
\});
 }
\BoSSSexe
\BoSSScmd{
Grid;
 }
\BoSSSexe
\BoSSScmd{
bool found;\newline 
Guid GridGuid = myNetworkDb.SaveGrid(Grid);
 }
\BoSSSexe
\BoSSScmd{
StartupObject\_Network.GridGuid = GridGuid;
 }
\BoSSSexe
\BoSSScmd{
var oneJob\_Network = new Job("byCmdLine\_HPC", solver2);
 }
\BoSSSexe
\BoSSScmd{
oneJob\_Network.SetControlObject(StartupObject\_Network);\newline 
oneJob\_Network.Activate(myHPC);
 }
\BoSSSexe
\BoSSScmd{
/// We can e.g. obtain the status of the job:
oneJob\_Network.Status;
 }
\BoSSSexe
\BoSSScmd{
/// Again, we block until both of our jobs have finished:
WorkflowMgm.BlockUntilAllJobsTerminate(360);
 }
\BoSSSexe
\BoSSScmd{
/// We can also obtain the session which was stored during the execution of the 
/// job:
var Snet = oneJob\_Network.LatestSession;\newline 
Snet;
 }
\BoSSSexe
\BoSSScmdSilent{
/// BoSSScmdSilent BoSSSexeSilent
if(System.IO.Directory.Exists(myNetworkDb.Path))\{\newline 
\btab DatabaseUtils.DeleteDatabase(myNetworkDb.Path);\newline 
\}\newline 
if(System.IO.Directory.Exists(myLocalDb.Path))\newline 
\btab DatabaseUtils.DeleteDatabase(myLocalDb.Path);\newline 
MiniBatchProcessor.Server.SendTerminationSignal();
 }
\BoSSSexeSilent
