% !TeX spellcheck = en_GB

\BoSSSopen{MetaJobManager/MetaJobManager}
\graphicspath{{MetaJobManager/MetaJobManager.texbatch/}}

\paragraph{What's new:} 
\begin{itemize}
	\item running multiple simulations in batch mode
	\item deploying and running simulations on a remote machine, e.g. a Microsoft HPC Cluster
	\item access statistics for multiple simulations
\end{itemize}

\paragraph{Prerequisites:} 
\begin{itemize}
	\item basic knowledge of \BoSSSpad{}
	\item executing runs on your local machine, e.g. the quickstart guide of the \ac{cns}, see chapter \ref{sec:CNS}
\end{itemize}

\BoSSS ~includes several tools which aid the advanced user in running simulations and organizing simulation results among multiple computers and/or compute clusters.

%In this tutorial, the following tasks will be illustrated:
%\begin{itemize}
%	\item The use of the meta job manager to run simulations, 
%	on the local machine as well as on a Microsoft HPC cluster
%	\item The evaluation of a simulation using the session table.
%\end{itemize}

The purpose of the meta job manager (a.k.a. the meta job scheduler)
is to provide a universal interface to multiple \emph{job managers} (aka. \emph{batch systems}).
By this interface, the user can run computations on remote and local systems directly from the \BoSSSpad{}.

Furthermore, \BoSSSpad{}  provides a \emph{session table},
which presents various statistics from all \emph{sessions} in the project
(each time a simulation is run, a session is stored in the \BoSSS ~database).

Before you dive into advanced features of \BoSSSpad{}, you should familiarise yourself with the basics of \BoSSS and \BoSSSpad{} and run some first simulations on your local machine, e.g. the tutorials on scalar advection \ref{ScalarAdvection} or Poisson equation \ref{sec:SIP}.

The examples presented in this chapter use the compressible Navier-Stokes solver (CNS), which has been introduced in the previous chapter \ref{CNS}.



 \section{Initialization}
\BoSSScmd{
restart;
 }
\BoSSSexeSilent
\BoSSScmd{
/// This tutorial illustrates how computations can be executed on large 
/// (or small) compute servers, aka.
/// high-performance-computers (HPC), aka. compute clusters, aka. supercomputers, etc.
///
/// This means, the computation is not executed on the local workstation 
/// (or laptop) but on some other computer.
/// This approach is particulary handy for large computations, which run
/// for multiple hours or days, since a user can 
/// e.g. shutdown or restart his personal computer without killing the compute job. 
///
/// \BoSSS{} features a set of classes and routines 
/// (an API, application programing interface) for communication with 
/// compute clusters. This is especially handy for \emph{scripting},
/// e.g. for parameter studies, where dozens of computations 
/// have to be started and monitored.
 }
\BoSSSexe
\BoSSScmd{
/// % ==================================================================
/// \section{Batch Processing}
/// % ==================================================================
 }
\BoSSSexe
\BoSSScmd{
/// First, we have to select a \emph{batch system} that we want to use.
/// Batch systems are a common approach to organize workloads (aka. compute jobs)
/// on compute clusters.
/// On such systems, a user typically does \emph{not} starts a simulation manually.
/// Instead, he specifies a so-called \emph{compute job}. The \emph{scheduler} 
/// (aka. \emph{batch system})collects 
/// compute jobs from all users on the compute cluster, sorts them according to 
/// some priority and puts the jobs into some queue, also called \emph{batch}.
/// The jobs in the batch are then executed in order, depending on the 
/// available hardware and the scheduling policies of the system.
///
/// The \BoSSS{} API provides front-ends (clients) for the following 
/// batch system software:
/// \begin{itemize}
/// \item 
/// \code{BoSSS.Application.BoSSSpad.SlurmClient} for the 
/// Slurm Workload Manager (very prominent on Linux HPC systems)
/// 
/// \item
/// \code{BoSSS.Application.BoSSSpad.MsHPC2012Client}
/// for the Microsoft HPC Pack 2012 and higher
///
/// \item
/// \code{BoSSS.Application.BoSSSpad.MiniBatchProcessorClient} for the 
/// mini batch processor, 
/// a minimalistic, \BoSSS{}-internal batch system which mimiks 
/// a supercomputer batch system on the local machine.
/// \end{itemize}
 }
\BoSSSexe
\BoSSScmd{
/// A list of clients for various batch systems, which are loaded at the 
/// \code{restart} command can be configured through the  
/// {\tt $\sim$/.BoSSS/etc/BatchProcessorConfig.json}-file.
/// If this file is missing, a default setting, containing a 
/// mini batch processor, is initialized. 
///
/// The list of all execution queues can be accessed through:
ExecutionQueues;
 }
\BoSSSexe
\BoSSScmd{
/// For the sake of simplicuity, we pick the first one:
var myBatch = ExecutionQueues[0];
 }
\BoSSSexe
\BoSSScmd{
/// Note on the Mini Batch Processor:
/// The batch processor for local jobs can be started separately (by launching
/// {\tt MiniBatchProcessor.exe}), or from the worksheet;
/// In the latter case, it depends on the operating system, whether the 
/// \newline {\tt MiniBatchProcessor.exe} is terminated with the worksheet, 
/// or not.
/// If no mini-batch-processor is running, it is started 
/// at the initialization of the workflow management.
//MiniBatchProcessor.Server.StartIfNotRunning(false);
 }
\BoSSSexe
\BoSSScmd{
/// % ==================================================================
/// \section{Initializing the workflow management}
/// % ==================================================================
 }
\BoSSSexe
\BoSSScmd{
/// In order to use the workflow management, 
/// the very first thing we have to do is to initialize it by defineing 
/// a project name. 
/// This is used to generate names for the compute jobs and to 
/// identify sessions in the database:
WorkflowMgm.Init("MetaJobManager\_Tutorial");
 }
\BoSSSexe
\BoSSScmd{
/// We verify that we have no jobs defined so far ...
WorkflowMgm.AllJobs;
 }
\BoSSSexe
\BoSSScmdSilent{
/// BoSSScmdSilent BoSSSexeSilent
NUnit.Framework.Assert.IsTrue(WorkflowMgm.AllJobs.Count == 0, "MetaJobManager tutorial: expecting 0 jobs on entry.");
 }
\BoSSSexeSilent
\BoSSScmd{
/// ... and create, resp. open a \BoSSS database on a local drive:
var myLocalDb = CreateTempDatabase();\newline 
/// To specify the exact directory of the database, you might use
/// \code{OpenOrCreateDatabase}.
 }
\BoSSSexe
\BoSSScmdSilent{
/// BoSSScmdSilent BoSSSexeSilent
databases;
 }
\BoSSSexeSilent
\BoSSScmd{
/// % ==================================================================
/// \section{Loading a BoSSS-Solver and Setting up a Simulation}
/// % ==================================================================
 }
\BoSSSexe
\BoSSScmd{
/// As an example, we use the workflow management tools to simulate 
/// incompressible channel flow, therefore we have to import the namespace,
/// and repeat the steps from chapter \ref{IBM} in order to setup the
/// control object:
using BoSSS.Application.XNSE\_Solver;
 }
\BoSSSexe
\BoSSScmd{
/// We create a grid with boundary conditions:
 }
\BoSSSexe
\BoSSScmd{
var xNodes       = GenericBlas.Linspace(0, 10 , 41); \newline 
var yNodes       = GenericBlas.Linspace(-1, 1, 9); \newline 
GridCommons grid = Grid2D.Cartesian2DGrid(xNodes, yNodes);\newline 
grid.DefineEdgeTags(delegate (double[] X) \{ \newline 
\btab double x = X[0];\newline 
\btab double y = X[1]; \newline 
\btab if (Math.Abs(y - (-1)) <= 1.0e-8) \newline 
\btab \btab return "wall"; // lower wall\newline 
\btab if (Math.Abs(y - (+1)) <= 1.0e-8) \newline 
\btab \btab return "wall"; // upper wall\newline 
\btab if (Math.Abs(x - (0.0)) <= 1.0e-8) \newline 
\btab \btab return "Velocity\_Inlet"; // inlet\newline 
\btab if (Math.Abs(x - (+10.0)) <= 1.0e-8) \newline 
\btab \btab return "Pressure\_Outlet"; // outlet\newline 
\btab throw new ArgumentOutOfRangeException("unknown domain"); \newline 
\});
 }
\BoSSSexe
\BoSSScmd{
/// And save it to the database:
 }
\BoSSSexe
\BoSSScmd{
myLocalDb.SaveGrid(ref grid);
 }
\BoSSSexe
\BoSSScmd{
/// Next, we create the control object for the incompressible simulation,
/// as in chapter \ref{IBM}:
 }
\BoSSSexe
\BoSSScmd{
// control object instantiation:\newline 
var c = new XNSE\_Control();\newline 
// general description:\newline 
int k                = 1;\newline 
string desc          = "Steady state, channel, k" + k; \newline 
c.SessionName        = "SteadyStateChannel"; \newline 
c.ProjectDescription = desc;\newline 
c.Tags.Add("k"+k);\newline 
// setting the grid:\newline 
c.SetGrid(grid);\newline 
// DG polynomial degree\newline 
c.SetDGdegree(k);\newline 
// Physical parameters:\newline 
double reynolds            = 20; \newline 
c.PhysicalParameters.rho\_A = 1; \newline 
c.PhysicalParameters.mu\_A  = 1.0/reynolds;\newline 
// Timestepping properties:\newline 
c.TimesteppingMode = AppControl.\_TimesteppingMode.Steady;
 }
\BoSSSexe
\BoSSScmd{
/// The specification of boundary conditions and initial values
/// is a bit more complicated if the job manager is used:
/// Since the solver is executed in an external program, the control object 
/// has to be saved in a file. For lots of complicated objects,
/// especially for delegates, C\# does not support serialization 
/// (converting the object into a form that can be saved on disk, or 
/// transmitted over a network), so a workaround is needed.
/// This is achieved e.g. by the \code{Formula} object, where a C\#-formula
/// is saved as a string.
 }
\BoSSSexe
\BoSSScmd{
var WallVelocity = new Formula("X => 0.0", false); // a time-indep. formula
 }
\BoSSSexe
\BoSSScmd{
/// Testing the formula:
WallVelocity.Evaluate(new[]\{0.0, 0.0\}, 0.0); // evaluationg at (0,0), at time 0
 }
\BoSSSexe
\BoSSScmd{
/// A disadvantage of string-formulas is that they look a bit ``alien''
/// within the worksheet; therefore, there is also a little hack which allows 
/// the conversion of a static memeber function of a static class into a 
/// \code{Formula} object:
 }
\BoSSSexe
\BoSSScmd{
static class StaticFormulas \{\newline 
\btab public static double VelX\_Inlet(double[] X) \{\newline 
\btab \btab //double x  = X[0];\newline 
\btab \btab double y  = X[0];\newline 
\btab \btab double UX = 1.0 - y*y;\newline 
\btab \btab return UX;\newline 
\btab \}  \newline 
 \newline 
\btab public static double VelY\_Inlet(double[] X) \{\newline 
\btab \btab return 0.0;\newline 
\btab \}      \newline 
\}
 }
\BoSSSexe
\BoSSScmd{
var InletVelocityX = GetFormulaObject(StaticFormulas.VelX\_Inlet);\newline 
var InletVelocityY = GetFormulaObject(StaticFormulas.VelY\_Inlet);
 }
\BoSSSexe
\BoSSScmd{
/// Finally, we set boundary values for our simulation. The initial values
/// are set to zero per default; for the steady-state simulation initial
/// values are irrelevan anyway:
/// Initial Values are set to 0
c.BoundaryValues.Clear(); \newline 
c.AddBoundaryValue("wall", "VelocityX", WallVelocity); \newline 
c.AddBoundaryValue("Velocity\_Inlet", "VelocityX", InletVelocityX);  \newline 
c.AddBoundaryValue("Velocity\_Inlet", "VelocityY", InletVelocityY); \newline 
c.AddBoundaryValue("Pressure\_Outlet");
 }
\BoSSSexe
\BoSSScmd{
/// % ==================================================================
/// \section{Activation and Monitoring of the the Job}
/// % ==================================================================
 }
\BoSSSexe
\BoSSScmd{
/// Finally, we are ready to deploy the job at the batch processor;
/// In a usual work flow scenario, we \emph{do not} want to (re-) submit the 
/// job every time we run the worksheet -- usually, one wants to run a job once.
/// 
/// The concept to overcome this problem is job activation. If a job is 
/// activated, the meta job manager first checks the databases and the batch 
/// system, if a job with the respective name and project name is already 
/// submitted. Only if there is no information that the job was ever submitted
/// or started anywhere, the job is submitted to the respective batch system.
 }
\BoSSSexe
\BoSSScmd{
/// First, a \code{Job}-object is created from the control object:
var JobLocal = c.CreateJob();
 }
\BoSSSexe
\BoSSScmd{
/// This job is not activated yet, it can still be configured:
 }
\BoSSSexe
\BoSSScmd{
JobLocal.Status;
 }
\BoSSSexe
\BoSSScmd{
///BoSSScmdsilent
NUnit.Framework.Assert.IsTrue(JobLocal.Status == JobStatus.PreActivation);
 }
\BoSSSexe
\BoSSScmd{
/// One can change e.g. the number of MPI processes:
 }
\BoSSSexe
\BoSSScmd{
JobLocal.NumberOfMPIProcs = 1;
 }
\BoSSSexe
\BoSSScmd{
/// Note that these jobs are desigend to be \emph{persistent}:
/// This means the computation is only started 
/// \emph{once for a given control object}, no matter how often the worksheet
/// is executed. 
///
/// Such a behaviour is useful for expensive simulations, which run on HPC
/// servers over days or even weeks. The user (you) can close the worksheet
/// and maybe open and execute it a few days later, and he can access
/// the original job which he submitted a few days ago (maybe it is finished
/// now).
 }
\BoSSSexe
\BoSSScmd{
/// Then, the job is activated, resp. submitted, resp. deployed 
/// to one batch system.
/// If job persistency is not wanted, traces of the job can be removed 
/// on request dureing activation, causing a fresh job deployment at the
/// batch system:
 }
\BoSSSexe
\BoSSScmd{
bool DeleteOldDeploymentsAndSessions = true; // use with care! Normally, you \newline 
\btab \btab \btab \btab \btab \btab \btab \btab \btab \btab \btab  // dont want this!!!\newline 
JobLocal.Activate(myBatch,  // execute the job in 'myBatch'\newline 
\btab \btab \btab \btab   DeleteOldDeploymentsAndSessions); // causes fresh deployment
 }
\BoSSSexe
\BoSSScmd{
/// All jobs can be listed using the workflow management:
 }
\BoSSSexe
\BoSSScmd{
WorkflowMgm.AllJobs
 }
\BoSSSexe
\BoSSScmd{
/// Check the present job status:
JobLocal.Status;
 }
\BoSSSexe
\BoSSScmdSilent{
/// BoSSScmdSilent BoSSSexeSilent
NUnit.Framework.Assert.IsTrue(\newline 
   JobLocal.Status == JobStatus.PendingInExecutionQueue\newline 
   || JobLocal.Status == JobStatus.InProgress);
 }
\BoSSSexeSilent
\BoSSScmd{
/// Here, we block until both of our jobs have finished:
WorkflowMgm.BlockUntilAllJobsTerminate(3600*4);
 }
\BoSSSexe
\BoSSScmd{
/// We examine the output and error stream of the job:
/// This directly accesses the {\tt stdout}-redirection of the respective job
/// manager, which may contain a bit more information than the 
/// {\tt stdout}-copy in the session directory.
JobLocal.Stdout;
 }
\BoSSSexe
\BoSSScmd{
/// Additionally we display the error stream and hope that it is empty:
JobLocal.Stderr;
 }
\BoSSSexe
\BoSSScmd{
/// We can also obtain the session 
/// which was stored during the execution of the job:
 }
\BoSSSexe
\BoSSScmd{
var Sloc = JobLocal.LatestSession;\newline 
Sloc;
 }
\BoSSSexe
\BoSSScmd{
/// We can also list all attempts to run the job at the assigend processor:
JobLocal.AllDeployments;
 }
\BoSSSexe
\BoSSScmdSilent{
/// BoSSScmdSilent BoSSSexeSilent
NUnit.Framework.Assert.IsTrue(JobLocal.AllDeployments.Count == 1, "MetaJobManager tutorial: Found more than one deployment.");
 }
\BoSSSexeSilent
\BoSSScmd{
/// Finally, we check the status of our jobs:
 }
\BoSSSexe
\BoSSScmd{
JobLocal.Status;
 }
\BoSSSexe
\BoSSScmd{
/// If anything failed, hints on the reason why are provides by the 
/// \code{GetStatus} method:
 }
\BoSSSexe
\BoSSScmd{
JobLocal.GetStatus(WriteHints:true);
 }
\BoSSSexe
\BoSSScmdSilent{
/// BoSSScmdSilent BoSSSexeSilent
NUnit.Framework.Assert.IsTrue(JobLocal.Status == JobStatus.FinishedSuccessful, "MetaJobManager tutorial: Job was not successful.");
 }
\BoSSSexeSilent
