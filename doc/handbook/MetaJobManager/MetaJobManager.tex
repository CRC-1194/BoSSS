% !TeX spellcheck = en_GB

\BoSSSopen{MetaJobManager/MetaJobManager}
\graphicspath{{MetaJobManager/MetaJobManager.texbatch/}}

\paragraph{What's new:} 
\begin{itemize}
	\item running multiple simulations in batch mode
	\item deploying and running simulations on a remote machine, e.g. a Microsoft HPC Cluster
	\item access statistics for multiple simulations
\end{itemize}

\paragraph{Prerequisites:} 
\begin{itemize}
	\item basic knowledge of \BoSSSpad{}
	\item executing runs on your local machine, e.g. the quickstart guide of the \ac{cns}, see chapter 5
\end{itemize}

\BoSSS ~includes several tools which aid the advanced user in running simulations and organizing simulation results among multiple computers and/or compute clusters.

%In this tutorial, the following tasks will be illustrated:
%\begin{itemize}
%	\item The use of the meta job manager to run simulations, 
%	on the local machine as well as on a Microsoft HPC cluster
%	\item The evaluation of a simulation using the session table.
%\end{itemize}

The purpose of the meta job manager (a.k.a. the meta job scheduler)
is to provide a universal interface to multiple \emph{job managers} (aka. \emph{batch systems}).
By this interface, the user can run computations on remote and local systems directly from the \BoSSSpad{}.

Furthermore, \BoSSSpad{}  provides a \emph{session table},
which presents various statistics from all \emph{sessions} in the project
(each time a simulation is run, a session is stored in the \BoSSS ~database).

Before you dive into advanced features of \BoSSSpad{}, you should familiarise yourself with the basics of \BoSSS and \BoSSSpad{} and run some first simulations on your local machine, e.g. the tutorials on scalar advection \ref{ScalarAdvection} or Poisson equation \ref{sec:SIP}.

The examples presented in this chapter use the compressible Navier-Stokes solver (CNS), which has been introduced in the previous chapter \ref{CNS}.



 \section{Initialization}
\BoSSScmd{
restart;
 }
\BoSSSexeSilent
\BoSSScmd{
/// The very first thing we have to do is to initialize the workflow
/// management tools and to define a project name. The meta job manager, like  
/// all other workflow management tools strictly enforces the use of a 
/// project name:
WorkflowMgm.Init("MetaJobManager\_Tutorial");
 }
\BoSSSexe
\BoSSScmd{
/// We verify that we have no jobs defined so far
WorkflowMgm.AllJobs;
 }
\BoSSSexe
\BoSSScmdSilent{
/// BoSSScmdSilent BoSSSexeSilent
// first, do some cleanup\newline 
if(System.IO.Directory.Exists(@"C:\textbackslash tmp\textbackslash TutorialLocal\_db")) \{\newline 
\btab System.IO.Directory.Delete(@"C:\textbackslash tmp\textbackslash TutorialLocal\_db",true);\newline 
\} else \{\newline 
\btab //\newline 
\}
 }
\BoSSSexeSilent
\BoSSScmd{
/// and create, resp. open a \BoSSS database on a local drive.
var myLocalDb = OpenOrCreateDatabase(@"C:\textbackslash tmp\textbackslash TutorialLocal\_db");
 }
\BoSSSexe
\BoSSScmdSilent{
/// BoSSScmdSilent BoSSSexeSilent
databases;
 }
\BoSSSexeSilent
\BoSSScmd{
/// % ==================================================================
/// \section{Loading a BoSSS-Solver and Setting up a Simulation}
/// % ==================================================================
 }
\BoSSSexe
\BoSSScmd{
/// We have to initialize all batch systems that we want to use.
/// Normally, one would put these statements into the 
/// {\tt $\sim$/.BoSSS/etc/DBErc.cs}-file
if(System.IO.Directory.Exists(@"C:\textbackslash tmp\textbackslash deploy"))\{\newline 
\btab System.IO.Directory.Delete(@"C:\textbackslash tmp\textbackslash deploy",true);\newline 
\}\newline 
System.IO.Directory.CreateDirectory(@"C:\textbackslash tmp\textbackslash deploy");\newline 
var myBatch = new MiniBatchProcessorClient(@"C:\textbackslash tmp\textbackslash deploy");
 }
\BoSSSexe
\BoSSScmd{
/// The batch processor for local jobs can be started separately (by launching
/// {\tt MiniBatchProcessor.exe}), or from the worksheet as follows.
/// In the latter case, it depends on the operating system, whether the 
/// \newline {\tt MiniBatchProcessor.exe} is terminated with the worksheet, or not.
MiniBatchProcessor.Server.StartIfNotRunning(false);
 }
\BoSSSexe
\BoSSScmd{
/// In this tutorial, we use the workflow management tools to simulate 
/// compressible flows, thus we first have to load the namespaces of 
/// the CNS solver.
using CNS;
 }
\BoSSSexe
\BoSSScmd{
/// and obtain its type and location
Type solver1 = typeof(CNS.Program);\newline 
solver1.Assembly.Location;\newline 
/// For other solvers, we might have to load the assemblies too, e.g. by 
// LoadAssembly("CNS.exe");
 }
\BoSSSexe
\BoSSScmd{
/// In this example, we start the computation from a hard-coded
/// control object that is already compiled into the code. If we started the 
/// {\tt CNS.exe} directly from the command line, the command line 
/// call would be
/// {\tt ./CNS.exe -control cs:CNS.ControlExamples.IsentropicVortex(...) }.
/// On the command line, the arguments could be problematic, e.g. quotation 
/// marks would need some kind of escape string, spaces are problematic, etc.
/// The meta job manager passes the arguments via environment variables, so
/// the before-mentioned issues do not occur. 
string StartupString\_Local = \newline 
  string.Format("cs:CNS.ControlExamples.IsentropicVortex(@\textbackslash "\{0\}\textbackslash ", 20, 2, 1.0)", \newline 
\btab \btab \btab \btab myLocalDb.Path);
 }
\BoSSSexe
\BoSSScmd{
/// % ==================================================================
/// \section{Deploying the jobs}
/// % ==================================================================
 }
\BoSSSexe
\BoSSScmd{
/// Finally, we define two jobs: one for the local machine, one for the 
/// HPC cluster.
/// In a usual work flow scenario, we \emph{do not} want to (re-) submit the 
/// job every time we run the worksheet -- usually, one wants to run a job once.
/// 
/// The concept to overcome this problem is job activation. If a job is 
/// activated, the meta job manager first checks the databases and the batch 
/// system, if a job with the respective name and project name is already 
/// submitted. Only if there is no information that the job was ever submitted
/// or started anywhere, the job is submitted to the respective batch system.
 }
\BoSSSexe
\BoSSScmd{
var oneJob\_Local = new Job("byCmdLine\_Local", solver1);
 }
\BoSSSexe
\BoSSScmd{
oneJob\_Local.SetCommandLineArguments(StartupString\_Local);
 }
\BoSSSexe
\BoSSScmd{
oneJob\_Local.Activate(myBatch);
 }
\BoSSSexe
\BoSSScmd{
oneJob\_Local.BatchProcessorIdentifierToken;
 }
\BoSSSexe
\BoSSScmd{
/// All jobs can be listed using the workflow management:
 }
\BoSSSexe
\BoSSScmd{
WorkflowMgm.AllJobs
 }
\BoSSSexe
\BoSSScmd{
/// Check the present job status:
oneJob\_Local.Status;
 }
\BoSSSexe
\BoSSScmd{
/// Here, we block until both of our jobs have finished:
WorkflowMgm.BlockUntilAllJobsTerminate(360);
 }
\BoSSSexe
\BoSSScmd{
/// We examine the output and error stream of the job on the HPC cluster:
/// This directly accesses the {\tt stdout}-redirection of the respective job
/// manager, which may contain a bit more information than the 
/// {\tt stdout}-copy in the session directory.
oneJob\_Local.Stdout;
 }
\BoSSSexe
\BoSSScmd{
/// Additionally we display the error stream and hope that it is empty:
oneJob\_Local.Stderr;
 }
\BoSSSexe
\BoSSScmd{
/// We can also obtain the session which was stored during the execution of the 
/// job:
 }
\BoSSSexe
\BoSSScmd{
var Sloc = oneJob\_Local.LatestSession;\newline 
Sloc;
 }
\BoSSSexe
\BoSSScmd{
/// Finally, we check the status of our jobs:
 }
\BoSSSexe
\BoSSScmd{
oneJob\_Local.Status;
 }
\BoSSSexe

