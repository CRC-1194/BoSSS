% !TeX spellcheck = en_US

\BoSSSopen{convergenceStudyTutorial/convStudy}
\graphicspath{{convergenceStudyTutorial/convStudy.texbatch/}}

\paragraph{What's new:} 
\begin{itemize}
	\item performing a convergence, resp. parameter study from a larger number of solver runs
	\item organisation of those runs with the\emph{ work flow management tools}
    \item especially, how to use the \emph{session table} of your project
	\item \emph{advanced} plotting facilities and \LaTeX export
\end{itemize}

\paragraph{Prerequisites:} 
\begin{itemize}
	\item basic knowledge of \BoSSSpad{}
	\item executing runs on your local machine, e.g. the quick-start guide of the \ac{cns}, see chapter \ref{sec:CNS}
    \item knowlede of the meta job scheduler, see chapter \ref{sec:WorkflowMgm}
\end{itemize}

The fully programmable workflow management tools in \BoSSS{} are powerful helpers in
the \emph{organization of multiple solver runs} (from dozens to thousands and beyond)
as well as when it comes to extracting data from those runs (aka. \emph{sessions}).

In order to demonstrate this, 
we are going to perform a spatial convergence study on the following Poisson problem:
we solve
\begin{equation*}
   \Delta T = -2 \cos(x) \cos(y)
\end{equation*}
in the square domain $\Omega = (-\pi/2,\pi/2)^2$, 
with homogeneous Dirichlet boundary conditions, 
for which we know the exact
solution is given by
\[
   T_\text{ex} = \cos(x) \cos(y) .
\]
Note that this simple 2D example would be small enough to solve it in \BoSSSpad{} directly.
However, the purpose of this tutorial is to analyze data spread over multiple sessions.
Furthermore, although we only perform a simple convergence study,
this tutorial can also be extended to the analysis of other metrics (e.g. run-times) collected during 
solver runs.

\BoSSScmd{
/// % ========================================
/// \section{Initialization}
/// % ========================================
 }
\BoSSSexeSilent
\BoSSScmd{
/// In order to execute the individual solver runs,
/// we are going to employ the mini batch processor,
/// for running the calculations on the local machine.
/// We also have to initialize the workflow management system and create 
/// a database.
 }
\BoSSSexeSilent
\BoSSScmd{
restart
 }
\BoSSSexeSilent
\BoSSScmdSilent{
/// BoSSSexeSilent BoSSScmdSilent
using NUnit.Framework;
 }
\BoSSSexeSilent
\BoSSScmd{
WorkflowMgm.Init("ConvStudytuorial");
 }
\BoSSSexe
\BoSSScmd{
var db = OpenOrCreateDefaultDatabase();
 }
\BoSSSexe
\BoSSScmd{
/// The following (deactivated) line would delete all Sessions (i.e. solver 
/// runs) which correspond to this project from the database.
/// Hence, on every execution of the worksheet, all simulations would be 
/// re-done. 
/// Normally, without the following line, existing simulations from 
/// the database will be used; therefore, it is save to close and open
/// the worksheet.
/// This is handy e.g. when simulations are running on a cluster for a long 
/// time, and we usually don't want to re-submit the calculation 
/// every time we execute the worksheet.
//WorkflowMgm.Sessions.ForEach(si => si.Delete(true));
 }
\BoSSSexe
\BoSSScmd{
/// For sake of simplicity, we employ the Poisson solver
/// \code{ipPoisson} which is just a benchmarking application, but sufficient 
/// for the purpose of this tutorial.
 }
\BoSSSexe
\BoSSScmd{
using BoSSS.Application.SipPoisson;
 }
\BoSSSexe
\BoSSScmd{
/// We also instantiate a client for the \code{MiniBatchProcessor}:
 }
\BoSSSexe
\BoSSScmd{
var myBatch = new MiniBatchProcessorClient();
 }
\BoSSSexe
\BoSSScmd{
MiniBatchProcessor.Server.StartIfNotRunning(false);
 }
\BoSSSexe
\BoSSScmd{
/// % ========================================
/// \section{Mesh Creation}
/// \label{sec:MeschCreation}
/// % ========================================
 }
\BoSSSexe
\BoSSScmd{
/// We chose resolutions of $2 \times 2$, $4 \times 4$ to $32 \times 32$
/// cells:
 }
\BoSSSexe
\BoSSScmd{
int[] resolutions = new int[] \{ 2, 4, 8, 16, 32\};
 }
\BoSSSexe
\BoSSScmd{
var grids = new GridCommons[resolutions.Length];\newline 
for(int iRes = 0; iRes < resolutions.Length; iRes++) \{\newline 
\btab // create nodes:\newline 
\btab var Nodes = GenericBlas.Linspace(-Math.PI*0.5, Math.PI*0.5, \newline 
\btab \btab resolutions[iRes] + 1); // note: number of nodes = number of cells + 1!\newline 
 \newline 
\btab // create grid:\newline 
\btab GridCommons grd\_i = Grid2D.Cartesian2DGrid(Nodes, Nodes);\newline 
 \newline 
\btab // define all boundaries as Dirichlet:\newline 
\btab grd\_i.EdgeTagNames.Add(1, BoundaryType.Dirichlet.ToString());\newline 
\btab grd\_i.DefineEdgeTags(delegate (double[] X) \{\newline 
\btab \btab byte ret = 1;\newline 
\btab \btab return ret;\newline 
\btab \});\newline 
 \newline 
\btab // save grid in database\newline 
\btab db.SaveGrid(ref grd\_i);\newline 
 \newline 
\btab // remenber reference to grid:\newline 
\btab grids[iRes] = grd\_i;\newline 
\}
 }
\BoSSSexe
\BoSSScmd{
/// % ===========================================
/// \section{Setup and execution of solver runs}
/// \label{sec:SetupAndExec}
/// % ===========================================
 }
\BoSSSexe
\BoSSScmd{
/// First, we implement the exact expressions for the right-hand-side 
/// and the exact solution.
/// The exact solution will be used to compute the error of the simulation.
/// Normally, the exact solution is not known; in those cases, we need to 
/// compute the experimental convergence against the solution on the finest
/// grid.
 }
\BoSSSexe
\BoSSScmd{
static class Expressions \{\newline 
\btab public static double RHS(double[] X) \{\newline 
\btab \btab double x = X[0];\newline 
\btab \btab double y = X[1];\newline 
\btab \btab return -2.0*Math.Cos(x)*Math.Cos(y);\newline 
\btab \}    \newline 
\btab public static double Tex(double[] X) \{\newline 
\btab \btab double x = X[0];\newline 
\btab \btab double y = X[1];\newline 
\btab \btab return Math.Cos(x)*Math.Cos(y);\newline 
\btab \}        \newline 
\}
 }
\BoSSSexe
\BoSSScmd{
var RHSfunc = GetFormulaObject(Expressions.RHS);
 }
\BoSSSexe
\BoSSScmd{
var TexFunc = GetFormulaObject(Expressions.Tex);
 }
\BoSSSexe
\BoSSScmd{
/// We compute 4 different polynomial orders:
int[] Degrees = new int[] \{1, 2, 3, 4\};
 }
\BoSSSexe
\BoSSScmd{
/// Setup of all runs...
 }
\BoSSSexe
\BoSSScmd{
var Runs = new List<SipControl>();\newline 
Runs.Clear(); // start with an empty run list\newline 
foreach(int pDeg in Degrees) \{ // loop over polynomial degrees\newline 
foreach(var grd in grids) \{ //    loop over all grids\newline 
 \newline 
\btab // create object and remember in list:\newline 
\btab SipControl C = new SipControl();   \newline 
\btab Runs.Add(C);\newline 
 \newline 
\btab // set polynomial degree and grid:\newline 
\btab C.SetDGdegree(pDeg);\newline 
\btab C.SetGrid(grd);\newline 
 \newline 
\btab // specify RHS and exact solution (these are realized as initial values\newline 
\btab // in the \textbackslash code\{ipPoisson\} solver:\newline 
\btab C.AddInitialValue("RHS", RHSfunc);\newline 
\btab C.AddInitialValue("Tex", TexFunc);\newline 
 \newline 
\btab // specify boundary condition:\newline 
\btab C.AddBoundaryValue(BoundaryType.Dirichlet.ToString()); // for homogeneous \newline 
\btab //   boundary conditions, we don not need a value, since the default value\newline 
\btab //   zero is sufficient.\newline 
 \newline 
\btab // the exact solution is a speciality of the SIP Poisson benchmark solver;\newline 
\btab // in order to evaluate the exact solution, we have to set the following \newline 
\btab // boolean:\newline 
\btab C.ExactSolution\_provided = true;\newline 
 \newline 
\}\newline 
\}
 }
\BoSSSexe
\BoSSScmd{
/// ...and activate them:
 }
\BoSSSexe
\BoSSScmd{
foreach(var C in Runs)\newline 
   C.RunBatch(myBatch);
 }
\BoSSSexe
\BoSSScmd{
/// The following line ensures that all jobs are complete before 
/// post-processing analysis is started, although, there is a 720-seconds
/// time-out.
WorkflowMgm.BlockUntilAllJobsTerminate(720);
 }
\BoSSSexe
\BoSSScmd{
/// Note that, in a larer production run study, where jobs may run days or
/// weeks, blocking the worksheet is not really usefull.
/// Instead, one might split process into two workseets 
/// (eactly at this line here), one for set-up and
/// job sumbission and another one for the analysis.
WorkflowMgm.AllJobs;
 }
\BoSSSexe
\BoSSScmd{
/// % ===========================================
/// \section{Convergence against exact solution}
/// \label{sec:ExactConvergence}
/// % ===========================================
 }
\BoSSSexe
\BoSSScmd{
/// As already noted, the computation of the $L^2$ error against the 
/// exact solution is handled specially in the \code{ipPoisson} solver.
/// However, the following tutorial can serve as a general template of how to
/// extract data from the session table and visualize it.
 }
\BoSSSexe
\BoSSScmd{
///
/// We aquire a copy of the session table, and from all the columns in there...
 }
\BoSSSexe
\BoSSScmd{
var Tab = WorkflowMgm.SessionTable;
 }
\BoSSSexe
\BoSSScmd{
Tab.GetColumnNames().Take(7); // Take(7) is just to shorten the output.
 }
\BoSSSexe
\BoSSScmd{
/// ...we extract those which sound interesting:
 }
\BoSSSexe
\BoSSScmd{
Tab = Tab.ExtractColumns(\newline 
\btab //"SessionName", \newline 
\btab "DGdegree:T", "Grid:NoOfCells", "Grid:hMin", "DOFs", \newline 
\btab //"ExactSolution\_provided", \newline 
\btab "SolL2err");
 }
\BoSSSexe
\BoSSScmd{
Tab.Print();
 }
\BoSSSexe
\BoSSScmd{
///
/// Note: the session table can also be exported, e.g. to Excel or 
/// Libre/Open Office Calc, by using the \code{ToCSVFile} function.
///
 }
\BoSSSexe
\BoSSScmd{
/// The columns of the session table 
/// can be easily converted to a plot: the $x$-axis is determined
/// by the cell width, the $y$-axis is determined by the $L^2$ error.
/// Furthermore, we want to \emph{group} our plots according
/// to the DG degree, i.e. have one line for each polynomial degree;
 }
\BoSSSexe
\BoSSScmd{
var ErrorPlot = Tab.ToPlot("Grid:hMin", "SolL2err", // column for x- and y\newline 
\btab \btab \btab \btab \btab \btab    "DGdegree:T"); // column for group
 }
\BoSSSexe
\BoSSScmd{
/// We set logarithmic axes:
ErrorPlot.LogX = true;\newline 
ErrorPlot.LogY = true;
 }
\BoSSSexe
\BoSSScmd{
ErrorPlot.PlotNow();
 }
\BoSSSexe
\BoSSScmd{
/// Of course, we can adjust the plot styles:
 }
\BoSSSexe
\BoSSScmd{
ErrorPlot.dataGroups[0].Format.PointType = PointTypes.Diamond;\newline 
ErrorPlot.dataGroups[1].Format.PointType = PointTypes.Box;\newline 
ErrorPlot.dataGroups[2].Format.PointType = PointTypes.LowerTriangle;\newline 
ErrorPlot.dataGroups[3].Format.PointType = PointTypes.Asterisk;
 }
\BoSSSexe
\BoSSScmd{
foreach(var grp in ErrorPlot.dataGroups) \{\newline 
\btab grp.Format.PointSize = 0.8;\newline 
\btab grp.Format.DashType  = DashTypes.Dotted;\newline 
\btab grp.Format.LineWidth = 2;\newline 
\}
 }
\BoSSSexe
\BoSSScmd{
ErrorPlot.PlotNow();
 }
\BoSSSexe
\BoSSScmd{
/// And we can compute the convergence order:
 }
\BoSSSexe
\BoSSScmd{
ErrorPlot.Regression();
 }
\BoSSSexe
\BoSSScmdSilent{
/// BoSSScmdSilent
var reg1     = ErrorPlot.Regression();\newline 
double conv1 = reg1.Single(kv => kv.Key.Contains("T1")).Value;\newline 
double conv2 = reg1.Single(kv => kv.Key.Contains("T2")).Value;\newline 
double conv3 = reg1.Single(kv => kv.Key.Contains("T3")).Value;\newline 
double conv4 = reg1.Single(kv => kv.Key.Contains("T4")).Value;\newline 
Assert.IsTrue(Math.Abs(conv1 - (+2)) < 0.7, "experimental convergence failed on k = 1");\newline 
Assert.IsTrue(Math.Abs(conv2 - (+3)) < 0.7, "experimental convergence failed on k = 2");\newline 
Assert.IsTrue(Math.Abs(conv3 - (+4)) < 0.7, "experimental convergence failed on k = 3");\newline 
Assert.IsTrue(Math.Abs(conv4 - (+5)) < 0.7, "experimental convergence failed on k = 4");
 }
\BoSSSexe
\BoSSScmd{
///
/// Note: these plots can also be exported to LaTeX, in a quality 
/// that is suitable for print publication:
//ErrorPlot.ToGnuplot().PlotCairolatex().SaveTo("C:\textbackslash \textbackslash tmp\textbackslash \textbackslash errplt.tex");
 }
\BoSSSexe
\BoSSScmd{
 % 
 }
\BoSSSexe
\BoSSScmd{
/// % ===========================================
/// \section{experimental convergence plot}
/// % ===========================================
 }
\BoSSSexe
\BoSSScmd{
/// If the exact solution is not known, one can only estimate the convergence
/// behavior experimentally. 
/// \BoSSS{} provides some utility for this, the \code{DGFieldComparison}
/// class, which has a versatile, yet complex interface.
/// However, there is a simple interface in the workflow management toolbox.
 }
\BoSSSexe
\BoSSScmd{
///
/// We can augment the current session table with experimental errors:
WorkflowMgm.hConvergence.Update();
 }
\BoSSSexe
\BoSSScmd{
var Tab = WorkflowMgm.SessionTable;
 }
\BoSSSexe
\BoSSScmd{
/// We observe, that columns have been added to the session table, 
/// starting with a prefix \code{L2Error\_}; 
Tab.GetColumnNames().Skip(46);
 }
\BoSSSexe
\BoSSScmd{
Tab = Tab.ExtractColumns(\newline 
\btab "DGdegree:T", "Grid:NoOfCells",\newline 
\btab "SolL2err", "L2Error\_T");
 }
\BoSSSexe
\BoSSScmd{
/// We observe that the \emph{experimental} $L^2$ error is approximately
/// equal to the $L^2$ error against the exact solution, 
/// except for the highest resolutions. There, the error of the numerical 
/// solution is computed against itself, and thus the error is zero up 
/// to round-off errors.
/// 
/// If we would like to extract convergence plots from this table, we need to
/// exclude the rows with the finest solution using e.g. the 
/// \code{TableExtensions.ExtractRows} method.
Tab.Print();
 }
\BoSSSexe
\BoSSScmd{
/// Rows could be extracted form a table using a selector function:
/// this is an expression, which is true for all rows that we want to extract;
Tab = Tab.ExtractRows(\newline 
  (iRow, RowEntries) => Convert.ToInt32(RowEntries["Grid:NoOfCells"]) != 1024);
 }
\BoSSSexe
\BoSSScmd{
Tab.Print();
 }
\BoSSSexe
\BoSSScmd{
/// % ============================================
/// \subsection{Working without the session table}
/// % ============================================
 }
\BoSSSexe
\BoSSScmd{
/// As an alternative to working with the session table, which is sometimes
/// not versatile enough, we demonstrate a way to extract data 
/// from the sessions in the current project directly.
 }
\BoSSSexe
\BoSSScmd{
// Create a list in which we store a separate plot for each polynomial degree:\newline 
var ExpPlotS = new List<Plot2Ddata>();\newline 
 \newline 
foreach(int pDeg in Degrees) \{ // loop over polynomial degrees\newline 
\btab // extract sessions with DG degree pDeg\newline 
\btab var pDegSessions = WorkflowMgm.Sessions.Where(\newline 
\btab \btab \btab   // function which is true on all sessions we are interested in:\newline 
\btab \btab \btab   Si => Convert.ToInt32(Si.KeysAndQueries["DGdegree:T"]) == pDeg\newline 
\btab \btab ).ToArray();\newline 
 \newline 
\btab // now, create a plot from the selected sessions:\newline 
\btab // (we could also do other things)\newline 
\btab Plot2Ddata pDegPlot = \newline 
\btab \btab pDegSessions.ToEstimatedGridConvergenceData("T", false);\newline 
\btab \btab // false secelsts DOFs instead of resolution h for x-axis\newline 
 \newline 
\btab // remember the freshly created plot object in a list:\newline 
\btab ExpPlotS.Add(pDegPlot);\newline 
\}
 }
\BoSSSexe
\BoSSScmd{
/// We adjust some plot style settings:
ExpPlotS[0].dataGroups[0].Format.PointType = PointTypes.Diamond;\newline 
ExpPlotS[1].dataGroups[0].Format.PointType = PointTypes.Box;\newline 
ExpPlotS[2].dataGroups[0].Format.PointType = PointTypes.LowerTriangle;\newline 
ExpPlotS[3].dataGroups[0].Format.PointType = PointTypes.Asterisk;\newline 
ExpPlotS[0].dataGroups[0].Name = "$k = 1$";\newline 
ExpPlotS[1].dataGroups[0].Name = "$k = 2$";\newline 
ExpPlotS[2].dataGroups[0].Name = "$k = 3$";\newline 
ExpPlotS[3].dataGroups[0].Name = "$k = 4$";\newline 
foreach(var p in ExpPlotS) \{\newline 
\btab p.dataGroups[0].Format.PointSize = 0.8;\newline 
\btab p.dataGroups[0].Format.DashType = DashTypes.Dotted;\newline 
\btab p.dataGroups[0].Format.LineWidth = 2;\newline 
\}
 }
\BoSSSexe
\BoSSScmd{
/// and we can merge all four plot objects into a singe one:
var ExpPlot = ExpPlotS[0]; //           select 0-th object\newline 
foreach(var p in ExpPlotS.Skip(1)) \{ // loop over other (skip 0-th entry)\newline 
\btab ExpPlot = ExpPlot.Merge(p); //      merge \newline 
\}
 }
\BoSSSexe
\BoSSScmd{
ExpPlot.PlotNow();
 }
\BoSSSexe
\BoSSScmd{
/// and we can also verify the slope of the error curves:
ExpPlot.Regression();
 }
\BoSSSexe
\BoSSScmdSilent{
/// BoSSScmdSilent
var regExp   = ExpPlot.Regression();\newline 
double conv1 = regExp.Single(kv => kv.Key.Contains("1")).Value;\newline 
double conv2 = regExp.Single(kv => kv.Key.Contains("2")).Value;\newline 
double conv3 = regExp.Single(kv => kv.Key.Contains("3")).Value;\newline 
double conv4 = regExp.Single(kv => kv.Key.Contains("4")).Value;\newline 
Assert.IsTrue(Math.Abs(conv1 - (-2)) < 0.7, "experimental convergence failed on k = 1");\newline 
Assert.IsTrue(Math.Abs(conv2 - (-3)) < 0.7, "experimental convergence failed on k = 2");\newline 
Assert.IsTrue(Math.Abs(conv3 - (-4)) < 0.7, "experimental convergence failed on k = 3");\newline 
Assert.IsTrue(Math.Abs(conv4 - (-5)) < 0.7, "experimental convergence failed on k = 4");
 }
\BoSSSexe
\BoSSScmd{
/// % ============================================
/// \subsection{Multiplot demonstration}
/// % ============================================
 }
\BoSSSexe
\BoSSScmd{
/// If we have more than one plot object, we can arrange them in an array 
/// to realize multi-plots:
 }
\BoSSSexe
\BoSSScmd{
var multiplot = new Plot2Ddata[2,2];
 }
\BoSSSexe
\BoSSScmd{
multiplot[0,0] = ExpPlotS[0];\newline 
multiplot[0,1] = ExpPlotS[1];\newline 
multiplot[1,0] = ExpPlotS[2];\newline 
multiplot[1,1] = ExpPlotS[3];
 }
\BoSSSexe
\BoSSScmd{
/// Now, we can draw an array of plots:
 }
\BoSSSexe
\BoSSScmd{
multiplot.PlotNow();
 }
\BoSSSexe
\BoSSScmd{
/// this already looks neat, but a few 
/// formatting tweaks to make the multi-plot look nicer:
multiplot[0,1].dataGroups[0].UseY2 = true; // label right on right column\newline 
multiplot[0,1].ShowYtics = false;\newline 
multiplot[0,1].ShowY2tics = true;\newline 
 \newline 
multiplot[1,1].dataGroups[0].UseY2 = true; // label right on right column\newline 
multiplot[1,1].ShowYtics = false;\newline 
multiplot[1,1].ShowY2tics = true;\newline 
 \newline 
multiplot[0,0].dataGroups[0].UseX2 = true; // label on top on top row\newline 
multiplot[0,0].ShowXtics = false;\newline 
multiplot[0,0].ShowX2tics = true;\newline 
 \newline 
multiplot[0,1].dataGroups[0].UseX2 = true; // label on top on top row\newline 
multiplot[0,1].ShowXtics = false;\newline 
multiplot[0,1].ShowX2tics = true;\newline 
 \newline 
// turn logarithm on for the secondary axis; \newline 
multiplot[0,0].LogX2 = true;\newline 
multiplot[0,1].LogX2 = true;\newline 
multiplot[1,0].LogX2 = true;\newline 
multiplot[1,1].LogX2 = true;\newline 
multiplot[0,0].LogY2 = true;\newline 
multiplot[0,1].LogY2 = true;\newline 
multiplot[1,0].LogY2 = true;\newline 
multiplot[1,1].LogY2 = true;\newline 
 \newline 
// set x ranges\newline 
multiplot[0,0].X2rangeMin = 1e0;\newline 
multiplot[0,0].X2rangeMax = 1e2;\newline 
multiplot[0,1].X2rangeMin = 1e0;\newline 
multiplot[0,1].X2rangeMax = 1e2;\newline 
 \newline 
multiplot[1,0].XrangeMin = 1e0;\newline 
multiplot[1,0].XrangeMax = 1e2;\newline 
multiplot[1,1].XrangeMin = 1e0;\newline 
multiplot[1,1].XrangeMax = 1e2;\newline 
 \newline 
// set y ranges\newline 
multiplot[0,0].YrangeMin = 1e-7;\newline 
multiplot[0,0].YrangeMax = 1e0;\newline 
multiplot[1,0].YrangeMin = 1e-7;\newline 
multiplot[1,0].YrangeMax = 1e0;\newline 
 \newline 
multiplot[0,1].Y2rangeMin = 1e-7;\newline 
multiplot[0,1].Y2rangeMax = 1e0;\newline 
multiplot[1,1].Y2rangeMin = 1e-7;\newline 
multiplot[1,1].Y2rangeMax = 1e0;\newline 
 \newline 
// reduce the whitespace in between the plots:\newline 
multiplot[0,0].rmargin = 2;\newline 
multiplot[0,1].lmargin = 2;\newline 
multiplot[1,0].rmargin = 2;\newline 
multiplot[1,1].lmargin = 2;\newline 
multiplot[0,0].bmargin = 0.5;\newline 
multiplot[1,0].tmargin = 0.5;\newline 
multiplot[0,1].bmargin = 0.5;\newline 
multiplot[1,1].tmargin = 0.5;
 }
\BoSSSexe
\BoSSScmd{
multiplot.PlotNow();
 }
\BoSSSexe
\BoSSScmd{
/// % ============================================
/// \section{Summary}
/// % ============================================
 }
\BoSSSexe
\BoSSScmd{
/// This tutorial showed how to set-up a parameter study,
/// by looping over a set of parameters (in this case, different grids
/// and polynomial degrees), see sections \ref{sec:MeschCreation}
/// and \ref{sec:SetupAndExec}.
/// Finally, it only requires a simple loop to send all jobs to a 
/// compute resource.
///
/// Afterwards, c.f. section \ref{sec:ExactConvergence},
/// the \emph{session table} was used to combine measurements
/// taken in each session (here, the $L^2$ error against the exact solution)
/// into a single table.
/// This table can either be exported to spreadsheet analysis software
/// or visualized internally.
 }
\BoSSSexe
