\BoSSSopen{CsharpAndBoSSSpad/CsharpAndBoSSSpad}
\graphicspath{{CsharpAndBoSSSpad/CsharpAndBoSSSpad.texbatch/}}

\BoSSScmd{
/// % ============================================
/// \section*{What's new}
/// % ============================================
/// \BoSSSpad is a scripting environment, based on the C\# language.
/// While it is not necessary to be a master-class programmer for the tutorials 
/// which come up, it is required to have some basic knowledge about the C\#
/// syntax.
///
/// Here, the following topics are covered:
/// \begin{itemize}
/// \item Declaration of primitive variables
/// \item Classes
/// \item Control Flow
/// \item Functions, Methods and Delegates
/// \item Inheritance
/// \item The execution model of \BoSSSpad
/// \end{itemize}
///
 }
\BoSSSexeSilent
\BoSSScmd{
/// % ============================================
/// \section{Declaration of primitive variables}
/// % ============================================
 }
\BoSSSexeSilent
\BoSSScmd{
/// Normally, a program/application written in C\# consists of one or more 
/// text files (source code, ending with \texttt{.cs}) which are read by a 
/// compiler and converted into an executable (ending with \text{.exe}).
///
/// In \BoSSSpad{}, instead, one can enter single statements or snippets of 
/// C\# code and execute them on the fly. (These snippets are not complete 
/// programs on their own, and they would not compile in the traditional way.)
///
/// Before we can execute any statement, we must restart (aka. reset) 
/// the C\# interpreter:
 }
\BoSSSexeSilent
\BoSSScmd{
restart
 }
\BoSSSexeSilent
\BoSSScmd{
/// Any data like single numbers, vectors, strings or composite data objects
/// (like the numerical grid) is stored resp. referenced via a \emph{variable}.
/// A variable declaration consists of a type and a name, and reads e.g. as:
 }
\BoSSSexe
\BoSSScmd{
int i;
 }
\BoSSSexe
\BoSSScmd{
/// The above statement created a variable named `\code{i}' of type 
/// `\code{int}' (for integer numbers). The most commonly used number types in 
/// \BoSSS{} are \code{int} for integers and \code{double} for real numbers.
/// A full list of C\#-build-in variable types for numbers can be found 
/// online (see 
/// \url{https://docs.microsoft.com/en-us/dotnet/csharp/language-reference/keywords/integral-types-table}
/// and
/// \url{https://docs.microsoft.com/en-us/dotnet/csharp/language-reference/keywords/floating-point-types-table}.)
 }
\BoSSSexe
\BoSSScmd{
/// Next, we assign a value to the new variable:
 }
\BoSSSexe
\BoSSScmd{
i = 9;
 }
\BoSSSexe
\BoSSScmd{
/// If we just enter the variable name, \BoSSSpad{} prompts the variable in 
/// the result window (if the variable can anyhow be presented as text):
 }
\BoSSSexe
\BoSSScmd{
i;
 }
\BoSSSexe
\BoSSScmd{
/// Declaration and assignment can also be put into a single line:
 }
\BoSSSexe
\BoSSScmd{
int j    = 9;\newline 
double x = 4.6;
 }
\BoSSSexe
\BoSSScmd{
/// Next, we test assign a new value to the already-defined variable \code{i}:
 }
\BoSSSexe
\BoSSScmd{
i = j + (int)x; // cast x to int; add to j; store result in i\newline 
i;
 }
\BoSSSexe
\BoSSScmd{
/// Note on the snippet above:
/// \begin{itemize}
/// \item \code{x} and \code{i}, \code{j} are of different type; therefore, 
///       \code{x} is \emph{cast} to int, which truncates the decimal digits,
///       before it can be added and stored as integer.
/// \item The second line would be an incomplete statement, according to the 
///       original, non-interpreted C\# language -- this would produce a 
///       compile error. The C\# interpreter, instead, produces a text from 
///       the last 
///       incomplete statement in a snippet.
/// \item a double-slash `\texttt{//}' starts a comment, i.e. the rest of 
///       the line is not interpreted and can contain any text.
/// \end{itemize}
 }
\BoSSSexe
\BoSSScmd{
/// While assigning \code{double} to \code{int} requires a cast operation, 
/// the other way works automatically (this is called an \emph{implicit cast}),
/// since very \code{int} can be represented as \code{double} (but not 
/// vice-versa):
 }
\BoSSSexe
\BoSSScmd{
double y = j + x;\newline 
y;
 }
\BoSSSexe
\BoSSScmd{
/// C\# defines the standard operations on numbers found in most programming
/// languages:
 }
\BoSSSexe
\BoSSScmd{
// Increasing a variables value by one:\newline 
y = y + 1;\newline 
y;
 }
\BoSSSexe
\BoSSScmd{
// Or the += notation:\newline 
y += 1;\newline 
y;
 }
\BoSSSexe
\BoSSScmd{
// Even shorter by the ++ operator:\newline 
y++;\newline 
y;
 }
\BoSSSexe
\BoSSScmd{
/// Another build-in type is \code{string}, for representing text:
 }
\BoSSSexe
\BoSSScmd{
string s = "some Text";
 }
\BoSSSexe
\BoSSScmd{
s;
 }
\BoSSSexe
\BoSSScmd{
/// Which also supports some (but not all) operators:
 }
\BoSSSexe
\BoSSScmd{
string t = " and more string";
 }
\BoSSSexe
\BoSSScmd{
s + t;
 }
\BoSSSexe
\BoSSScmd{
s + t + ": " + j;
 }
\BoSSSexe
\BoSSScmd{
/// % ============================================
/// \subsection{Arrays}
/// % ============================================
 }
\BoSSSexe
\BoSSScmd{
/// For every type, one can define arrays; e.g. an array of integers is 
/// defined and initialized as
 }
\BoSSSexe
\BoSSScmd{
int[] A = new int[5];
 }
\BoSSSexe
\BoSSScmd{
/// The line above creates an array with 5 entries. Initially, all
/// entries are set to the default value 0:
 }
\BoSSSexe
\BoSSScmd{
A;
 }
\BoSSSexe
\BoSSScmd{
/// One can access and manipulate individual entires by index-brackets; 
/// Note that C\# indices (usually) start at zero:
 }
\BoSSSexe
\BoSSScmd{
A[2] = 7;
 }
\BoSSSexe
\BoSSScmd{
A;
 }
\BoSSSexe
\BoSSScmd{
/// It is also possible to initialize an array with non-default values:
 }
\BoSSSexe
\BoSSScmd{
string[] B = new string[] \{ "Zero", "One", "Two" \};
 }
\BoSSSexe
\BoSSScmd{
B;
 }
\BoSSSexe
\BoSSScmd{
/// % ============================================
/// \subsection{Multi-dimensional arrays}
/// % ============================================
 }
\BoSSSexe
\BoSSScmd{
/// C\# also defines arrays with more than one index, e.g.:
 }
\BoSSSexe
\BoSSScmd{
int[,] M = new int[3,4];\newline 
M[2,1] = 3;
 }
\BoSSSexe
\BoSSScmd{
/// In \BoSSS, there is a separate type for multidimensional arrays of 
/// real numbers. In contrast to the multidimensional arrays build into C\#,
/// their \BoSSS{} counterparts offer more operations, e.g.
/// reshaping the array or tensor multiplications.
/// An array with three dimensions can be initialized as:
 }
\BoSSSexe
\BoSSScmd{
MultidimensionalArray Q = MultidimensionalArray.Create(5,3,4);
 }
\BoSSSexe
\BoSSScmd{
/// However, a comprehensive introduction of the 
/// \code{MultidimensionalArray} type is beyond the scope of this section.
/// We may refer the reader to the API reference.
 }
\BoSSSexe
\BoSSScmd{
/// % ============================================
/// \section{Classes}
/// % ============================================
 }
\BoSSSexe
\BoSSScmd{
/// So far, we used only primitive (build-in) datatypes. The core idea of 
/// object orientation is to define composite data structures (called 
/// classes) upon the primitive types and other classes.
/// 
/// Furthermore, classes provide the concept of encapsulation: any member 
/// variable that should be manipulated from code outside of the class 
/// must be decorated with the \code{public} keyword. Otherwise, the member 
/// is \code{private} and can only be manipulated form inside the class.
/// (This is demonstrated in below)
 }
\BoSSSexe
\BoSSScmd{
class MyVector \{\newline 
\btab // public variables. \newline 
\btab public double x;\newline 
\btab public double y;\newline 
\btab public string Name;\newline 
\}
 }
\BoSSSexe
\BoSSScmd{
/// One may think of classes as blueprints, which are used to create 
/// objects. Whenever the \code{new} operator is applied to a class, a new 
/// object is created in computer memory. This is called \emph{instantiation}.
/// Objects are accessed via variables, just in the same way as arrays.
///
/// We declare a variable of type \code{MyVector} and intialize it with an 
/// instance of this class:
 }
\BoSSSexe
\BoSSScmd{
MyVector vec = new MyVector();
 }
\BoSSSexe
\BoSSScmd{
/// Then, we can manipulate its variables (also called \emph{members}):
 }
\BoSSSexe
\BoSSScmd{
vec.x    = 1.2;\newline 
vec.y    = 4.1;\newline 
vec.Name = "BoSSS";
 }
\BoSSSexe
\BoSSScmd{
/// Each object owns its own space in memory; so when we instantiate a second 
/// object of this type, the members of \code{vec} stay the same:
 }
\BoSSSexe
\BoSSScmd{
MyVector vec2 = new MyVector();
 }
\BoSSSexe
\BoSSScmd{
vec2.x    = -3.1;\newline 
vec2.y    = 22.4;\newline 
vec2.Name = "Another BoSSS";
 }
\BoSSSexe
\BoSSScmd{
/// We see, both objects have their independent variables:
 }
\BoSSSexe
\BoSSScmd{
vec.Name;
 }
\BoSSSexe
\BoSSScmd{
vec2.Name;
 }
\BoSSSexe
\BoSSScmd{
/// % ============================================
/// \subsection{Member Functions (Methods)}
/// % ============================================
 }
\BoSSSexe
\BoSSScmd{
/// The next important idea of object orientation is to bundle data structures
/// with the code which manipulates this data. In imperative
/// programming languages (e.g. C, FORTRAN, Matlab) these code is organized
/// in so-called \emph{functions}, aka. \emph{subroutines}. 
/// In C\# such functions are put into the class and they can access all
/// variables in the class, including \code{private} ones.
/// These functions are often called \emph{methods} or \emph{member functions}.
///
/// A function consists of the following parts:
/// \begin{itemize}
/// \item
/// An optional access modifier, e.g. \code{public} or \code{private}; if no 
/// access modifier is given, the method is private.
///
/// \item
/// The type of the value returned by the method; this can be either a 
/// build-in type like \code{double} or some class. If the method should 
/// not return a value, its return type is \code{void}.
///
/// \item
/// A function name
///
/// \item
/// A list of function arguments, i.e. variables which are used as inputs
/// to the function
///
/// \item
/// The function body which contains the actual code
/// \end{itemize}
 }
\BoSSSexe
\BoSSScmd{
class MyAdvancedVector \{\newline 
\btab // two private variables, which cannot be accessed from outside:\newline 
\btab double x;    \newline 
\btab double y;\newline 
 \newline 
\btab // a function which only manipulates internal data, but has neither \newline 
\btab // a return value nor an input argument:\newline 
\btab public void Rotate90Deg() \{\newline 
\btab \btab double temp = x; // back-up x in a local variable\newline 
\btab \btab x           = y; // overwrite x\newline 
\btab \btab y           = -temp; // set y\newline 
\btab \}\newline 
 \newline 
\btab // a function with an input argument:\newline 
\btab public void Scale(double factor) \{\newline 
\btab \btab x *= factor; \newline 
\btab \btab y *= factor;   \newline 
\btab \}\newline 
 \newline 
\btab // a function which only returns a value:\newline 
\btab public double LengthSquared() \{\newline 
\btab \btab double sqSum = 0;;\newline 
\btab \btab sqSum += x*x; \newline 
\btab \btab sqSum += y*y; \newline 
\btab \btab return sqSum; // returning the result of the function   \newline 
\btab \}\newline 
 \newline 
\btab // a function with multiple input arguments, and a return value;\newline 
\btab // to avoid confusion, it is better to chose names for the input arguments\newline 
\btab // that do not conflict with member variables:\newline 
\btab public double Set\_xy(double new\_x, double new\_y) \{\newline 
\btab \btab x = new\_x;\newline 
\btab \btab y = new\_y;\newline 
\btab \btab return LengthSquared();\newline 
\btab \}\newline 
\}
 }
\BoSSSexe
\BoSSScmd{
MyAdvancedVector vecX = new MyAdvancedVector();
 }
\BoSSSexe
\BoSSScmd{
/// Next, we demonstrate how to execute member functions of our new class:
 }
\BoSSSexe
\BoSSScmd{
vecX.LengthSquared();
 }
\BoSSSexe
\BoSSScmd{
vecX.Set\_xy(2,3);
 }
\BoSSSexe
\BoSSScmd{
vecX.Rotate90Deg();
 }
\BoSSSexe
\BoSSScmd{
vecX.LengthSquared();
 }
\BoSSSexe
\BoSSScmd{
/// % =======================
/// \subsection{Properties}
/// % =======================
 }
\BoSSSexe
\BoSSScmd{
/// Properties are special methods, which look like functions. 
/// Most of the time, they are used to execute additional code before setting 
/// a variable.
 }
\BoSSSexe
\BoSSScmd{
class PropertiesDemo \{\newline 
\btab // a private variable:\newline 
\btab int i1;\newline 
 \newline 
\btab // reading/setting the value through methods:\newline 
\btab // (old-fashioned)\newline 
\btab public int Get\_i1() \{\newline 
\btab \btab return i1;    \newline 
\btab \}\newline 
\btab public void Set\_i1(int new\_i1) \{\newline 
\btab \btab // here, one can put code to perform additional operations,\newline 
\btab \btab // e.g. check that the passed argument is positive, (if this is desired)    \newline 
\btab \btab i1 = new\_i1;\newline 
\btab \}\newline 
 \newline 
\btab // reading/setting the value i1 through a property\newline 
\btab public int I1 \{\newline 
\btab \btab get \{\newline 
\btab \btab \btab // the get-part: like a function, \newline 
\btab \btab \btab //    * no arguments\newline 
\btab \btab \btab //    * returns int\newline 
\btab \btab \btab return i1;\newline 
\btab \btab \}    \newline 
\btab \btab set \{\newline 
\btab \btab \btab // the set-part: like a function, \newline 
\btab \btab \btab //    * one argument of type int, vamed value\newline 
\btab \btab \btab //    * no return value/void return value\newline 
\btab \btab \btab i1 = value;\newline 
\btab \btab \}\newline 
\btab \}\newline 
\}
 }
\BoSSSexe
\BoSSScmd{
/// Again, we instantiate the class:
 }
\BoSSSexe
\BoSSScmd{
PropertiesDemo p = new PropertiesDemo();
 }
\BoSSSexe
\BoSSScmd{
/// Access to the member variable through methods would work as
 }
\BoSSSexe
\BoSSScmd{
p.Set\_i1(49);
 }
\BoSSSexe
\BoSSScmd{
p.Get\_i1();
 }
\BoSSSexe
\BoSSScmd{
/// while the property-notation is more compact, but has the same functionality
/// (syntactic sugar):
 }
\BoSSSexe
\BoSSScmd{
p.I1;
 }
\BoSSSexe
\BoSSScmd{
p.I1 = 33;
 }
\BoSSSexe
\BoSSScmd{
p.I1;
 }
\BoSSSexe
\BoSSScmd{
/// % ============================================
/// \subsection{Static functions}
/// % ============================================
 }
\BoSSSexe
\BoSSScmd{
/// So far, always had to instantiate a class to execute its methods.
/// Now think e.g. about mathematical functions, like the sinus. In 
/// languages like C or FORTRAN, one would just evaluate them as 
/// \code{sin(x)}. The designers of C\# could have allowed functions like in 
/// C, but decided against it because it would compromise the object-oriented
/// syntax. On the other hand, creating an object before one is able to 
/// evaluate a sinus function also seems over-complicated. The compromise 
/// are \emph{static} functions:
 }
\BoSSSexe
\BoSSScmd{
class StaticDemo \{\newline 
\btab // in addition to normal functions, static functions are decorated\newline 
\btab // with the keyword static:\newline 
\btab static public int Summation(int a, int b) \{\newline 
\btab \btab return a + b;\newline 
\btab \}    \newline 
\}
 }
\BoSSSexe
\BoSSScmd{
/// Then, static functions can be involved without instantiating an object:
 }
\BoSSSexe
\BoSSScmd{
StaticDemo.Summation(1, 2);
 }
\BoSSSexe
\BoSSScmd{
/// % ============================================
/// \section{Class libraries and Namespaces}
/// % ============================================
 }
\BoSSSexe
\BoSSScmd{
/// Like most programming languages, C\# also ships with an extensive 
/// programming library (class libraries), 
/// e.g. to read and save files, mathematical functions,
/// graphical output; The full list of standard libraries can be found at  
/// \url{https://msdn.microsoft.com/en-us/library/gg145045(v=vs.110).aspx}.
///
/// These libraries are collections of classes; they are organized in 
/// \emph{namespaces}. In \BoSSSpad, the list of currently loaded 
/// namespaces can be 
/// shown with the \code{ShowUsing()} command:
 }
\BoSSSexe
\BoSSScmd{
ShowUsing();
 }
\BoSSSexe
\BoSSScmd{
/// Any class within a loaded namespace can be accessed directly; e.g. the 
/// classes \code{Console} and \code{Math}, which are part of the 
/// \code{System} namespace:
 }
\BoSSSexe
\BoSSScmd{
Console.WriteLine("Hello World!");
 }
\BoSSSexe
\BoSSScmd{
/// (By the way: the \code{Console} class, resp. the 
/// \code{Console.WriteLine} method is the preffered way to write 
/// text output (for non-windowed apps without graphical user interface).
/// In \BoSSSpad{}, it is very handy if one wants to write more information
/// than just the last object in a snippet.)
 }
\BoSSSexe
\BoSSScmd{
Math.Sin(2.0);
 }
\BoSSSexe
\BoSSScmd{
/// This is the same as calling the methods by their full 
/// \emph{namespace path}:
 }
\BoSSSexe
\BoSSScmd{
System.Console.WriteLine("Hello World!");
 }
\BoSSSexe
\BoSSScmd{
System.Math.Sin(2.0);
 }
\BoSSSexe
\BoSSScmd{
/// If a namespace is not loaded, its classes must be accessed via their 
/// full path, e.g.
 }
\BoSSSexe
\BoSSScmd{
System.Collections.BitArray ba = new System.Collections.BitArray(3);
 }
\BoSSSexe
\BoSSScmd{
/// If we load the namespace by the \code{using} keyword, we can 
/// access the \code{BitArray} class directly:
 }
\BoSSSexe
\BoSSScmd{
using System.Collections;
 }
\BoSSSexe
\BoSSScmd{
BitArray ba = new BitArray(3);
 }
\BoSSSexe
\BoSSScmd{
/// Note that, in order to work with a class library, 
/// it may not be enough to just add a using directive.
/// All libraries are contained in \texttt{dll} files, e.g. 
/// \texttt{System.dll} (standard C\# library) or \texttt{BoSSS.foundation.dll}
/// (part of \BoSSS).
///
/// These files must be loaded before any class from the library can be used.
/// (Namespaces are only for the internal organization of the library.)
/// Initially, \BoSSSpad{} load a set of libraries which are useful for working
/// with \BoSSS. For most use-cases, this set may be sufficient; however,
/// not that by the 
/// \code{Mono.\allowbreak CSharp.\allowbreak InteractiveBase.\allowbreak LoadAssembly(string)} method, 
/// it is possible to load further assemblies.
 }
\BoSSSexe
\BoSSScmd{
/// % ============================================
/// \section{Control Flow}
/// % ============================================
 }
\BoSSSexe
\BoSSScmd{
/// Regarding the standard syntax of the language for implementing 
/// a function, C\# designers tried to stay close to popular
/// existing languages like C, C++ and Java.
/// In this tutorial, we briefly discuss two control structures, branches
/// and loops.
 }
\BoSSSexe
\BoSSScmd{
/// At first, the \code{if}-branch;
/// We use a random number generator create random input data for our 
/// example. Therefore, the following snippet runs differently each time
/// it is executed:
 }
\BoSSSexe
\BoSSScmd{
Random rnd       = new Random(); // Generate new random number generator object\newline 
double randomVal = rnd.NextDouble();\newline 
Console.WriteLine("Random number is " + randomVal);\newline 
 \newline 
// here, the if-clause\newline 
if(randomVal > 0.5) \{ // some condition, which is ether true or false\newline 
\btab // Code that will be executed when the condition is true\newline 
\btab Console.WriteLine("greater than 1/2");    \newline 
\} else \{\newline 
\btab // Code that will be executed when the condition is false\newline 
\btab Console.WriteLine("smaller or equal to 1/2");  \newline 
\}
 }
\BoSSSexe
\BoSSScmd{
/// Next, the \code{for}-loop. It has three `Arguments':
/// \begin{itemize} 
/// \item
/// An initialization code (\code{int j = 0}): executed \emph{before} the 
/// loop starts.
///
/// \item
/// A runtime condition (\code{j < J}): this is checked \emph{before} each 
/// iteration of the loop; if it is false, the loop terminates.
///
/// \item
/// An increment statement (\code{j++}): this is executed \emph{after} each
/// iteration of the loop and usually used to increase or decrease some 
/// variables value.
/// \end{itemize}
 }
\BoSSSexe
\BoSSScmd{
int J = 7;\newline 
for(int j = 0; j < J; j++) \{\newline 
\btab // code which is executed repeadedly \newline 
\btab double s = Math.Sin(j);\newline 
\btab Console.WriteLine("The sinus of " + j + " is " + s);\newline 
\}
 }
\BoSSSexe
\BoSSScmd{
/// Another kind of loops is the \code{foreach} statement. It executes
/// the loop body for each element of a list or array:
 }
\BoSSSexe
\BoSSScmd{
double[] SomeValues = new double[] \{ 1.2, -3.4, 7.3, 3.1, 3.14 \};\newline 
foreach(double x in SomeValues) \{\newline 
\btab double s = Math.Sin(x);\newline 
\btab Console.WriteLine("The sinus of " + x + " is " + s);\newline 
\}
 }
\BoSSSexe
\BoSSScmd{
/// % ============================================
/// \section{Functions, Methods and Delegates}
/// % ============================================
 }
\BoSSSexe
\BoSSScmd{
/// In its heart, C\# is an object-oriented programming language, and 
/// methods are implemented in a imperative style. However, the designers
/// also included concepts from the functional programming model.
/// A key idea of functional programming is that functions/methods can be 
/// treated as variables; such variables are called \emph{delegates}.
///
/// In order to demonstrate the use of delegates, we create a class with 
/// some static methods:
 }
\BoSSSexe
\BoSSScmd{
class DelDemo \{\newline 
\btab public static double Pow2(int x, float z) \{\newline 
\btab \btab return x*x*z;\newline 
\btab \}\newline 
\btab public static void Print(int x, double y) \{\newline 
\btab \btab Console.Write("Got the numbers " + x + " and " + y + ".");\newline 
\btab \}    \newline 
\}
 }
\BoSSSexe
\BoSSScmd{
/// Now, we want to create variables/delegates which \emph{point} to the 
/// functions/methods defined above. As for any variable declaration, we 
/// need to specify type and name.
 }
\BoSSSexe
\BoSSScmd{
/// For functions which have a return type, it works like this:
 }
\BoSSSexe
\BoSSScmd{
// Note: the last type in the '< >' - List:\newline 
// * first, all arguments of the function\newline 
// * finally, the return type \newline 
Func<int,float,double> delPow2 = DelDemo.Pow2;
 }
\BoSSSexe
\BoSSScmd{
/// Then, the method can be called via the delegate:
 }
\BoSSSexe
\BoSSScmd{
// Note: the 'f'-suffix is for single precision types; this also shows round-of\newline 
// errors of floating point arithmetics \newline 
delPow2(2, 1.1f);
 }
\BoSSSexe
\BoSSScmd{
/// Functions without a return type (void) are also called \emph{actions}:
 }
\BoSSSexe
\BoSSScmd{
Action<int,double> a = DelDemo.Print;
 }
\BoSSSexe
\BoSSScmd{
a(1,1.2);
 }
\BoSSSexe
\BoSSScmd{
/// By the keyword \code{delagate}, C\# designers finally (in the C\# version 
/// 3.0) reversed their original decision that functions are only allowed 
/// within classes by the introduction of anonymous functions:
 }
\BoSSSexe
\BoSSScmd{
delPow2 = delegate(int x, float z) \{\newline 
\btab double retval;\newline 
\btab retval = x*x*z*2.0;\newline 
\btab return retval;\newline 
\};
 }
\BoSSSexe
\BoSSScmd{
/// Now, the variable \code{delPow2} does not anymore points to the method 
/// \code{DelDemo.Pow2}, but to the newly defined, anonymous function:
 }
\BoSSSexe
\BoSSScmd{
delPow2(2, 1.1f);
 }
\BoSSSexe
\BoSSScmd{
/// To make notation more compact, a function-like syntax was introduced:
 }
\BoSSSexe
\BoSSScmd{
delPow2 = (x, z) => x*x*z*4.0;
 }
\BoSSSexe
\BoSSScmd{
delPow2(2, 1.1f);
 }
\BoSSSexe
\BoSSScmd{
/// Since delegates can be treated as normal variables, the can also be used
/// as arguments to other functions; here is a more advanced example,
/// an action (no return value) which takes as inputs 
/// another action, a function and an integer value:
 }
\BoSSSexe
\BoSSScmd{
Action<Action<int,double>,Func<int,float,double>,int> apply = \newline 
\btab delegate(Action<int,double> act, Func<int,float,double> fun, int i) \{\newline 
\btab \btab float f = 1.0f;\newline 
 \newline 
\btab \btab double y = fun(i, f); // apply function 'fun' to argument 'i'\newline 
\btab \btab act(i, y);            // hand both numbers to action 'act'\newline 
 \newline 
\btab \btab return; \newline 
\btab \};
 }
\BoSSSexe
\BoSSScmd{
apply(a,delPow2,2);
 }
\BoSSSexe
\BoSSScmd{
/// % ============================================
/// \section{Inheritance}
/// % ============================================
 }
\BoSSSexe
\BoSSScmd{
/// As already discussed, classes are blueprints for objects: once a class
/// is implemented, it can be used to create an infinite number of 
/// objects. Of course, only until the computers memory is full.
///
/// The model of a single blueprint is, however, sometimes a bit inflexible.
/// engineers know this for a long time: in a group of products, you 
/// may have two products that have a lot in common, but also significant
/// differences regarding some functionality.
///
/// Object oriented programming languages provide the concept of inheritance:
/// Common functionality is implemented in a common base class.
 }
\BoSSSexe
\BoSSScmd{
/// First we define some class; functions or properties which are decorated
/// as \code{virtual}, can be
 }
\BoSSSexe
\BoSSScmd{
class BaseClass \{\newline 
\btab public int num;     \newline 
 \newline 
\btab // this is a normal method, which will be accessible form \newline 
\btab // any derived class\newline 
\btab public void Method1() \{\newline 
\btab \btab Console.WriteLine("BaseClass: Hello from Method1, num is " + num);    \newline 
\btab \}\newline 
 \newline 
\btab // this is a virtual method, its functionality can be overriden in derived\newline 
\btab // classes\newline 
\btab public virtual void VirtMeth(int i) \{\newline 
\btab \btab Console.WriteLine("BaseClass: VirtMethod, i is " + i);\newline 
\btab \}\newline 
\}
 }
\BoSSSexe
\BoSSScmd{
/// Now, we derive two classes from \code{BaseClass}:
 }
\BoSSSexe
\BoSSScmd{
class DerivA : BaseClass \{ // indicates that thes class is derived \newline 
\btab \btab \btab \btab \btab \btab    // from 'BaseClass'\newline 
 \newline 
\btab // We can add new members:\newline 
\btab public double SomeVal;\newline 
 \newline 
\btab // a private member:\newline 
\btab Random rnd = new Random();\newline 
 \newline 
\btab // And override functionality from the base class\newline 
\btab public override void VirtMeth(int i) \{\newline 
\btab \btab SomeVal = rnd.NextDouble() + i;\newline 
\btab \btab Console.WriteLine("DerivA: VirtMethod, random value is " + SomeVal);     \newline 
\btab \}\newline 
\}
 }
\BoSSSexe
\BoSSScmd{
class DerivB : BaseClass \{\newline 
 \newline 
\btab // Again, some override:\newline 
\btab public override void VirtMeth(int i) \{\newline 
\btab \btab num += i; // we can modify members of the base class\newline 
\btab \btab Console.WriteLine("DerivB: VirtMethod, num is " + num);     \newline 
\btab \}\newline 
\}
 }
\BoSSSexe
\BoSSScmd{
/// As usual, we can instantiate and use \code{BaseClass}:
 }
\BoSSSexe
\BoSSScmd{
BaseClass b = new BaseClass();\newline 
b.Method1();\newline 
b.VirtMeth(1); // calls base implementation
 }
\BoSSSexe
\BoSSScmd{
/// Of course, we can do the same with \code{DerivA}; since it is derived from
/// \code{BaseClass}, an object of type \code{DerivA} can be stored in
/// (resp. referenced from) a variable of type \code{BaseClass}. 
/// In short, any \code{DerivA}-object is autamtically also 
/// a \code{BaseClass}-object, due to inhertance (but not the other way).
 }
\BoSSSexe
\BoSSScmd{
BaseClass b = new DerivA(); // we instantiate an object of type 'DerivA'\newline 
\btab \btab \btab \btab \btab \btab \btab // but the variable is of type 'BaseClass'.\newline 
b.Method1();   // non-overridden functionality: from base implementation\newline 
b.VirtMeth(1); // calls override implementation
 }
\BoSSSexe
\BoSSScmd{
/// We can also store \code{DerivB}-objects in variable \code{b}:
 }
\BoSSSexe
\BoSSScmd{
b = new DerivB();\newline 
b.Method1();\newline 
b.VirtMeth(1); // 'DerivB.VirtMeth' modifies the internal state of the object,\newline 
b.VirtMeth(1); // so the result is different for each call
 }
\BoSSSexe
\BoSSScmd{
/// Note: implicitly, all classes in C\# are derived vom the class 
/// \code{System.Object}, and has all the methods of this class.
 }
\BoSSSexe
\BoSSScmd{
object o; // The 'object' keyword is a build-in alias for 'System.Object'\newline 
o = new BaseClass();\newline 
o.ToString(); // 'ToString' is a virtual method of 'System.Object', which,\newline 
\btab \btab \btab   // if not overriden only retunrs the name of the class
 }
\BoSSSexe
\BoSSScmd{
/// % ============================================
/// \subsection{Abstract Classes}
/// % ============================================
 }
\BoSSSexe
\BoSSScmd{
/// In lots of situations, there is no reasonable base-implementation for
/// a method.
 }
\BoSSSexe
\BoSSScmd{
abstract class AbstractBase \{ // if the class conatains any abstract member,\newline 
\btab \btab \btab \btab \btab \btab \btab   // it must be declared as abstract.\newline 
 \newline 
\btab public int num;     \newline 
 \newline 
\btab // like above, a normal method:\newline 
\btab public void Method1() \{\newline 
\btab \btab Console.WriteLine("AbstractBase: Hello from Method1, num is " + num);    \newline 
\btab \}\newline 
 \newline 
\btab // this is an abstract method, for which only the \newline 
\btab // signature (return value, arguments, etc.) is specified, but no \newline 
\btab // implementation is given:\newline 
\btab public abstract void VirtMeth(int i);\newline 
\}
 }
\BoSSSexe
\BoSSScmd{
/// Note:
/// \begin{itemize}
/// \item
/// Virtual methods can be overridden, i.e. the override is optional
///
/// \item
/// Abstract methods \emph{must} be overridden
///
/// \item
/// Abstract classes cannot be instantiated, since functionality is missing
///
/// \item
/// Abstract classes can be used as a variable type
/// \end{itemize}
 }
\BoSSSexe
\BoSSScmd{
class DerivC : AbstractBase \{\newline 
\btab // we are enforced to override the base method\newline 
\btab public override void VirtMeth(int i) \{\newline 
\btab \btab Console.WriteLine("DerivC: VirtMeth, i is " + i);    \newline 
\btab \}\newline 
\}
 }
\BoSSSexe
\BoSSScmd{
AbstractBase a;   // use 'AbstractBase' as a variable type\newline 
a = new DerivC(); // instantiate a 'DerivC' object, store it in 'a'\newline 
a.VirtMeth(2);    // call the objects methods
 }
\BoSSSexe
\BoSSScmd{
/// % ============================================
/// \subsection{Interfaces}
/// % ============================================
 }
\BoSSSexe
\BoSSScmd{
/// When we finish the concept of abstract classes, we come to classes that 
/// \emph{only} have abstract members. These are called \emph{interfaces} 
/// and C\# has a special keyword for them:
 }
\BoSSSexe
\BoSSScmd{
interface IMethodsInterface \{\newline 
\btab void MethodA(); // methods in an interface are always public and need   \newline 
\btab void MethodB(); // no modifier.\newline 
\}
 }
\BoSSSexe
\BoSSScmd{
interface IPropertyInterface \{\newline 
\btab int Property \{ get; set; \}    \newline 
\}
 }
\BoSSSexe
\BoSSScmd{
/// Note: there is a widly used convention to start interface names with 
/// capital letter `I'. This is only a convention, not a rule.
 }
\BoSSSexe
\BoSSScmd{
/// An important difference between classes and interfaces is, that
/// any class can have \emph{only one base-class}, but it can implement
/// an \emph{unlimited number of interfaces}:
 }
\BoSSSexe
\BoSSScmd{
class DerivD : AbstractBase, IMethodsInterface, IPropertyInterface \{\newline 
\btab // form 'AbstractBase'\newline 
\btab public override void VirtMeth(int i) \{\newline 
\btab \btab Console.WriteLine("DerivD: VirtMeth, i is " + i);    \newline 
\btab \}\newline 
 \newline 
\btab // from 'IMethodsInterface'\newline 
\btab public void MethodA() \{\newline 
\btab \btab Console.WriteLine("DerivD: MethodA");\newline 
\btab \}\newline 
\btab public void MethodB() \{\newline 
\btab \btab Console.WriteLine("DerivD: MethodB");\newline 
\btab \}\newline 
 \newline 
\btab // form 'IPropertyInterface'\newline 
\btab public int Property \{ \newline 
\btab \btab get \{ \newline 
\btab \btab \btab Console.WriteLine("This is the get property.");\newline 
\btab \btab \btab return num;  \newline 
\btab \btab \} \newline 
\btab \btab set \{ num = value; \}\newline 
\btab \}\newline 
\}
 }
\BoSSSexe
\BoSSScmd{
/// Of course, we can instantiate \code{DerivD} and call any method or 
/// property:
 }
\BoSSSexe
\BoSSScmd{
DerivD d = new DerivD();\newline 
d.VirtMeth(1);\newline 
d.MethodB();
 }
\BoSSSexe
\BoSSScmd{
/// But we can also store the object in references with the interface type:
 }
\BoSSSexe
\BoSSScmd{
IMethodsInterface imd = new DerivD();\newline 
imd.MethodA();\newline 
imd.MethodB();
 }
\BoSSSexe
\BoSSScmd{
IPropertyInterface ipd = new DerivD();\newline 
ipd.Property           = 44;\newline 
ipd.Property;
 }
\BoSSSexe
\BoSSScmd{
/// % ============================================
/// \section{The execution model of \BoSSSpad}
/// % ============================================
 }
\BoSSSexe
\BoSSScmd{
/// The term which describes the execution model of \BoSSSpad{} best is 
/// called \emph{read-eval-print-loop (REPL)}:
/// commands are read from the input window, evaluated by the interpreter,
/// the result is printed to the screen and this is repeated in an infinite 
/// loop (until the app is terminated).
///
/// Each time a command is executed, the internal state of the interpreter
/// is changed. E.g. new variables or classes are defined. If a variable or  
/// class name is already used, it will be overwritten. The special command
/// \code{restart} resets the interpreter to an initial state.
///
/// On top of that, \BoSSSpad{} puts a document model:
/// the user can navigate backward and re-execute any command; this may or 
/// may not change the behavior of the entire script. A simple example is:
 }
\BoSSSexe
\BoSSScmd{
/// We are defining an \code{int} variable:
 }
\BoSSSexe
\BoSSScmd{
int i = 9;           // int-definition
 }
\BoSSSexe
\BoSSScmd{
/// Next, we overwrite the name \code{i} with an new variable
 }
\BoSSSexe
\BoSSScmd{
double i = 1.1;      // double-definition
 }
\BoSSSexe
\BoSSScmd{
/// And execute some code which depends on \code{i}, e.g.:
 }
\BoSSSexe
\BoSSScmd{
/// Obviously, if we would navigate back to the 
/// \texttt{int-definition}-statement, re-execute it, but skip the execution
/// of the \texttt{double-definition} statement, the following statement 
/// will behave differently:
 }
\BoSSSexe
\BoSSScmd{
Math.Cos(i);         // depends on last 'i'
 }
\BoSSSexe
\BoSSScmd{
/// As we see,...
/// \begin{itemize}
/// \item
/// the result of a snippet depends on the sequence of previously 
/// \emph{executed} snippets
///
/// \item
/// navigation within the worksheet \emph{alters the sequence} in which 
/// snippets are executed -- this may have side-effects.
/// \end{itemize}
/// This behavior might seem confusing (and very often, it is). However, 
/// it is desired because it speeds up the workflow, because it allows to 
/// alter or correct a script, without (time-consuming) re-evaluation
/// of the entire script.
///
/// As a rule of thumb, take this:
/// \begin{itemize}
/// \item
/// If we want to alter a snippet above, we only need to re-execute everything
/// which depends on this snippet. The dependent snippets must be identified 
/// by the user, which is a good coding exercise.
///
/// \item
/// A simple solution is to re-evaluate everything downward of the altered 
/// snippet.
///
/// \item
/// As an ultima ratio, one should re-execute the entire worksheet.
/// \end{itemize}
 }
\BoSSSexe
\BoSSScmd{
/// % ============================================
/// \section{Further reading}
/// % ============================================
 }
\BoSSSexe
\BoSSScmd{
/// Obviously, this minimal tutorial does not cover the entire C\# language.
/// It should, however, demonstrate everything which is needed to understand
/// about 90\% of the tutorials in the \BoSSS{} handbook.
///
/// Finally: there are lots of tutorials on C\# online. Some of them are 
/// excellent, some are not so good. 
/// And: Google is your friend (at least in this respect)...
 }
\BoSSSexe
