\BoSSSopen{tutorial10-PoissonSystem/Poisson}
\graphicspath{{tutorial10-PoissonSystem/Poisson.texbatch/}}

\BoSSScmd{
/// \section*{What's new?}
/// \label{sec:poissonSystem_new}
/// \begin{itemize}
///     \item {implementation of a saddle point problem into \BoSSS{} 
///           (e.g. the Poisson equation as a system)}
///     \item {central-difference-form and strong form}
///     \item {comparison of both formulations}
///     \item advanced: {algebraic reduction of the poisson problem}
/// \end{itemize}
/// %==========================================
/// \section*{Prerequisites}
/// %==========================================
/// \begin{itemize}
///    \item {implementation of numerical fluxes, chapter \ref{NumFlux}} 
///    \item {spatial operator, chapter \ref{SpatialOperator}} 
/// \end{itemize}
/// %==========================================
/// \section{Problem statement}
/// %==========================================
/// Within this exercise, we are going to investigate 
/// the discretization of a Poisson equation as a system.
/// Obviously, it is possible to discretize the Poisson equation as a system of
/// first-order-PDE's, introducing a vector field $\vec{\sigma}$:
/// \begin{alignat}{3}
///  \vec{\sigma}  + \nabla u & = 0, & & \text{ in } \domain
///    \label{eq:PoissonSys1} \\
///  \operatorname{div}(\vec{\sigma}) &  = g_{\domain}, & & \text{ in } \domain
///    \label{eq:PoissonSys2} \\
///   u                                               & = g_D, & & \text{ on } \Gamma_D \\
///   - \vec{\sigma} \cdot \vec{n}_{\partial \domain} & = g_N, & & \text{ on } \Gamma_N
/// \end{alignat}
/// resp. in matrix-notation:
/// \begin{align*}
///   \begin{bmatrix}
///     \one & \nabla \\
///     \operatorname{div} & 0 \\
///   \end{bmatrix}\cdot
///   \begin{bmatrix}
///     \vec{\sigma}\\
///     u
///   \end{bmatrix}=
///   \begin{bmatrix}
///     0 \\
///     g_{\domain}
///   \end{bmatrix}
/// \end{align*}
/// This exercise, together with the previous one,
/// will form the foundation for an incompressible Stokes- resp. Navier-Stokes solver.    
/// \section{Solution within the BoSSS framework}
 }
\BoSSSexeSilent
%
\BoSSScmd{
restart
 }
\BoSSSexeSilent
%
\BoSSScmd{
using ilPSP.LinSolvers; \newline 
using BoSSS.Solution.Tecplot; \newline 
using ilPSP.Connectors.Matlab;
 }
\BoSSSexe
%
\BoSSScmd{
 % 
 }
\BoSSSexe
\BoSSScmd{
/// % ===================================
/// \subsection{Tests on the divergence}
/// % ===================================
 }
\BoSSSexe
\BoSSScmd{
/// \paragraph{Common base-class for $\text{div}$-implementations}
/// We are going to implement two different formulations of the 
/// divergence-operator for which going to show equivalence. 
/// We implement a common base-class for both formulations:
abstract public class BaseDivergence :  \newline 
\btab \btab BoSSS.Foundation.IEdgeForm, // edge integrals \newline 
\btab \btab BoSSS.Foundation.IVolumeForm     // volume integrals \newline 
\{ \newline 
    /// We don't use parameters (e.g. variable viscosity, ...)
    /// at this point: so the parameter list can be null, resp. empty:
\btab public IList<string> ParameterOrdering \{  \newline 
\btab \btab get \{ return null; \}  \newline 
\btab \} \newline 
 \newline 
    /// But we have a vector argument variable, 
    /// $ [ \sigma_1, \sigma_2 ] = \vec{\sigma} $
    /// (our trial function):
\btab public IList<String> ArgumentOrdering \{  \newline 
\btab \btab get \{ return new string[] \{ "sigma1", "sigma2" \}; \}  \newline 
\btab \} \newline 
 \newline 
\btab public TermActivationFlags VolTerms \{ \newline 
\btab \btab get \{ \newline 
\btab \btab \btab return TermActivationFlags.AllOn; \newline 
\btab \btab \} \newline 
\btab \} \newline 
 \newline 
\btab public TermActivationFlags InnerEdgeTerms \{ \newline 
\btab \btab get \{ \newline 
\btab \btab \btab return (TermActivationFlags.AllOn);  \newline 
\btab \btab \} \newline 
\btab \} \newline 
 \newline 
\btab public TermActivationFlags BoundaryEdgeTerms \{ \newline 
\btab    get \{ \newline 
\btab \btab    return TermActivationFlags.AllOn; \newline 
\btab \btab \} \newline 
\btab \} \newline 
 \newline 
    /// The following functions cover the actual math.
    /// For any discretization of the divergence-operator, we have to specify:
    /// \begin{itemize}
    ///    \item a volume integrand,
    ///    \item an edge integrand for inner edges, i.e. on $ \Gamma_i$,
    ///    \item an edge integrand for boundary edges, 
    ///          i.e. on $\partial \Omega$.
    /// \end{itemize}
    /// These functions are declared as \code{abstract}, meaning that one has 
    /// to specify them in classes derived from \code{BaseLaplace}.
 \newline 
\btab abstract public double VolumeForm(ref CommonParamsVol cpv,  \newline 
\btab \btab    double[] U, double[,] GradU,  \newline 
\btab \btab    double V, double[] GradV);         \newline 
 \newline 
\btab abstract public double InnerEdgeForm(ref CommonParams inp,  \newline 
\btab \btab double[] U\_IN, double[] U\_OT, double[,] GradU\_IN, double[,] GradU\_OT,  \newline 
\btab \btab double V\_IN, double V\_OT, double[] GradV\_IN, double[] GradV\_OT); \newline 
 \newline 
\btab abstract public double BoundaryEdgeForm(ref CommonParamsBnd inp,  \newline 
\btab \btab double[] U\_IN, double[,] GradU\_IN, double V\_IN, double[] GradV\_OT); \newline 
\}
 }
\BoSSSexe
\BoSSScmd{
/// We are going to use both, Dirichlet- and Neumann-boundary conditions
/// in this exercise; the function \code{IsDirichletBndy} is used to
/// specify the type of boundary condition at point \code{X}:
Func<double[],bool> IsDirichletBndy = delegate(double[] X) \{ \newline 
\btab double x = X[0]; \newline 
\btab double y = X[1]; \newline 
\btab if(Math.Abs(x - (-1.0)) < 1.0e-8) \newline 
\btab \btab return true;     \newline 
\btab if(Math.Abs(y - (-1.0)) < 1.0e-8) \newline 
\btab \btab return true;     \newline 
\btab return false; \newline 
\}
 }
\BoSSSexe
\BoSSScmd{
/// \paragraph{Formulation (i): Central-difference-form of $\text{div}$}
/// The implementation of the central-difference form is as follows:
 }
\BoSSSexe
\BoSSScmd{
class Divergence\_cendiff : BaseDivergence \{ \newline 
 \newline 
 \newline 
    /// The volume form is equal to 
    /// $ -\vec{\sigma} \cdot \nabla v$:
\btab override public double VolumeForm(ref CommonParamsVol cpv,  \newline 
\btab \btab double[] Sigma, double[,] GradSigma,  \newline 
\btab \btab double V, double[] GradV) \{ \newline 
\btab \btab double Acc = 0; \newline 
\btab \btab for(int d = 0; d < cpv.D; d++) \{ \newline 
\btab \btab \btab Acc -= Sigma[d]*GradV[d]; \newline 
\btab \btab \} \newline 
\btab \btab return Acc; \newline 
\btab \} \newline 
 \newline 
    /// At the cell boundaries, we use a central-difference-flux,
    /// i.e. $\mean{\vec{\sigma}} \cdot \vec{n}_{\Gamma} \jump{v}$:
\btab override public double InnerEdgeForm(ref CommonParams inp,  \newline 
\btab \btab double[] Sigma\_IN, double[] Sigma\_OT, double[,] GradSigma\_IN, double[,] GradSigma\_OT,  \newline 
\btab \btab double V\_IN, double V\_OT, double[] GradV\_IN, double[] GradV\_OT) \{ \newline 
 \newline 
\btab \btab double Acc = 0; \newline 
\btab \btab for(int d = 0; d < inp.D; d++) \{ \newline 
\btab \btab \btab Acc += 0.5*(Sigma\_IN[d] + Sigma\_OT[d])*inp.Normale[d]*(V\_IN - V\_OT); \newline 
\btab \btab \} \newline 
\btab \btab return Acc; \newline 
\btab \} \newline 
 \newline 
\btab override public double BoundaryEdgeForm(ref CommonParamsBnd inp,  \newline 
\btab \btab double[] Sigma\_IN, double[,] GradSigma\_IN, double V\_IN, double[] GradV\_OT) \{ \newline 
 \newline 
\btab \btab double Acc = 0; \newline 
 \newline 
\btab \btab if(IsDirichletBndy(inp.X)) \{ \newline 
            /// Dirichlet-boundary: by taking the inner value of $\vec{\sigma}$, 
            /// this is a free boundary with respect to $\vec{\sigma}$.
\btab \btab \btab for(int d = 0; d < inp.D; d++) \{ \newline 
\btab \btab \btab \btab Acc += Sigma\_IN[d]*inp.Normale[d]*V\_IN; \newline 
\btab \btab \btab \} \newline 
\btab \btab \} else \{ \newline 
            /// Neumann-boundary
\btab \btab \btab double gNeu = 0.0; \newline 
\btab \btab \btab Acc += gNeu*V\_IN; \newline 
\btab \btab \} \newline 
\btab \btab return Acc; \newline 
\btab \} \newline 
\}
 }
\BoSSSexe
\BoSSScmd{
/// \paragraph{Formulation (ii): 'Strong' form of $\text{div}$:}
/// Here, we use the form 
/// \[
///    b(\vec{\sigma},v) = 
///    \oint_{\Gamma \backslash \Gamma_D} 
///           \mean{v} \jump{\vec{\sigma}} \cdot \vec{n}_\Gamma 
///    \dA 
///    -
///    \int_{\domain} \operatorname{div}(\vec{\sigma}) \cdot v \dV
/// \]
/// This is actually the negative divergence, which will be more useful
/// later on.
class Divergence\_strong : BaseDivergence \{ \newline 
 \newline 
    /// We have to implement \code{VolumeForm},
    /// \emph{InnerEdgeForm} and \code{BoundaryEdgeForm}:
\btab override public double VolumeForm(ref CommonParamsVol cpv,  \newline 
\btab \btab double[] Sigma, double[,] GradSigma,  \newline 
\btab \btab double V, double[] GradV) \{ \newline 
\btab \btab double Acc = 0; \newline 
\btab \btab for(int d = 0; d < cpv.D; d++) \{ \newline 
\btab \btab \btab Acc -= GradSigma[d,d]*V; \newline 
\btab \btab \} \newline 
\btab \btab return Acc; \newline 
\btab \} \newline 
 \newline 
\btab override public double InnerEdgeForm(ref CommonParams inp,  \newline 
\btab \btab double[] Sigma\_IN, double[] Sigma\_OT, double[,] GradSigma\_IN, double[,] GradSigma\_OT,  \newline 
\btab \btab double V\_IN, double V\_OT, double[] GradV\_IN, double[] GradV\_OT) \{ \newline 
 \newline 
\btab \btab double Acc = 0; \newline 
\btab \btab for(int d = 0; d < inp.D; d++) \{ \newline 
\btab \btab \btab Acc += 0.5*(V\_IN + V\_OT)*(Sigma\_IN[d] - Sigma\_OT[d])*inp.Normale[d]; \newline 
\btab \btab \} \newline 
\btab \btab return Acc; \newline 
\btab \} \newline 
 \newline 
\btab override public double BoundaryEdgeForm(ref CommonParamsBnd inp,  \newline 
\btab \btab double[] Sigma\_IN, double[,] GradSigma\_IN, double V\_IN, double[] GradV\_OT) \{ \newline 
 \newline 
\btab \btab double Acc = 0; \newline 
 \newline 
\btab \btab if(IsDirichletBndy(inp.X)) \{ \newline 
\btab \btab \btab Acc = 0;\newline 
\btab \btab \} else \{ \newline 
\btab \btab \btab double gNeu = 0.0; \newline 
\btab \btab \btab for(int d = 0; d < inp.D; d++) \{ \newline 
\btab \btab \btab \btab Acc += Sigma\_IN[d]*inp.Normale[d]*V\_IN; \newline 
\btab \btab \btab \} \newline 
\btab \btab \btab Acc -= gNeu*V\_IN; \newline 
\btab \btab \} \newline 
\btab \btab return Acc; \newline 
\btab \} \newline 
\}
 }
\BoSSSexe
\BoSSScmd{
    /// %==========================================
    /// \subsection{Equality test}
    /// %==========================================
 }
\BoSSSexe
\BoSSScmd{
    /// We are going to test the equivalence of both formulations
    /// on a 2D grid, using a DG basis of degree 1:
 var grd2D               = Grid2D.Cartesian2DGrid(GenericBlas.Linspace(-1,1,6),                                                    GenericBlas.Linspace(-1,1,7)); \newline 
 var gdata2D             = new GridData(grd2D); \newline 
 var b                   = new Basis(gdata2D, 1); \newline 
 SinglePhaseField sigma1 = new SinglePhaseField(b,"sigma1"); \newline 
 SinglePhaseField sigma2 = new SinglePhaseField(b,"sigma2"); \newline 
 CoordinateVector sigma  = new CoordinateVector(sigma1,sigma2); \newline 
 var TrialMapping        = sigma.Mapping; \newline 
 var TestMapping         = new UnsetteledCoordinateMapping(b);
 }
\BoSSSexe
\BoSSScmd{
    /// We create the matrix of the central-difference formulation:
 var OpDiv\_cendiff = (new Divergence\_cendiff()).Operator(); \newline 
 var MtxDiv\_cendiff = OpDiv\_cendiff.ComputeMatrix(TrialMapping,  \newline 
\btab \btab \btab \btab \btab \btab \btab \btab \btab \btab \btab \btab   null,  \newline 
\btab \btab \btab \btab \btab \btab \btab \btab \btab \btab \btab \btab   TestMapping);
 }
\BoSSSexe
\BoSSScmd{
    /// We create the matrix of the strong formulation 
    /// and show that the matrices of both formulations are equal.
    /// We use the \code{InfNorm(...)}-method to identify whether a 
    /// matrix is (approximately) zero or not.
 var OpDiv\_strong  = (new Divergence\_strong()).Operator(); \newline 
 var MtxDiv\_strong = OpDiv\_strong.ComputeMatrix(TrialMapping, null, TestMapping); \newline 
 var TestP         = MtxDiv\_cendiff + MtxDiv\_strong; \newline 
 TestP.InfNorm();
 }
\BoSSSexe
\BoSSScmd{
/// %==================================
/// \subsection{The gradient-operator}
/// \label{subsec:poissonSystem_gradientOp}
/// %==================================
/// For the variational formulation of the gradient operator, a vector-valued
/// test-function is required. Unfourtunately, this is not supported by 
/// \BoSSS. Therefore we have to discretize the gradent component-wise,
/// i.e. as $\partial_{x}$ and $\partial_y$. A single derivative 
/// can obviously be expressed as a divergence by the
/// identity $ \partial_{x_d} = \text{div}( \vec{e}_d u ) $.
class Gradient\_d : \newline 
\btab \btab BoSSS.Foundation.IEdgeForm, // edge integrals \newline 
\btab \btab BoSSS.Foundation.IVolumeForm     // volume integrals \newline 
\{ \newline 
\btab public Gradient\_d(int \_d) \{ \newline 
\btab \btab this.d = \_d; \newline 
\btab \} \newline 
 \newline 
    /// The component index of the gradient:
\btab int d; \newline 
 \newline 
    /// As ususal, we do not use parameters:
\btab public IList<string> ParameterOrdering \{  \newline 
\btab \btab get \{ return null; \}  \newline 
\btab \} \newline 
 \newline 
    /// We have one argument $u$:
\btab public IList<String> ArgumentOrdering \{  \newline 
\btab \btab get \{ return new string[] \{ "u" \}; \}  \newline 
\btab \} \newline 
 \newline 
\btab public TermActivationFlags VolTerms \{ \newline 
\btab \btab get \{ return TermActivationFlags.AllOn; \} \newline 
\btab \} \newline 
 \newline 
\btab public TermActivationFlags InnerEdgeTerms \{ \newline 
\btab \btab get \{ return (TermActivationFlags.AllOn); \} \newline 
\btab \} \newline 
 \newline 
\btab public TermActivationFlags BoundaryEdgeTerms \{ \newline 
\btab    get \{ return TermActivationFlags.AllOn; \} \newline 
\btab \} \newline 
 \newline 
    /// Now, we implement 
    /// \begin{itemize}
    ///    \item the volume form $u \vec{e}_d \cdot \nabla v$
    ///    \item the boundary form 
    ///       $\mean{u \ \vec{e}_d} \cdot \vec{n}_\Gamma \jump{v}$
    /// \end{itemize}
\btab public double VolumeForm(ref CommonParamsVol cpv,  \newline 
\btab \btab    double[] U, double[,] GradU,  \newline 
\btab \btab    double V, double[] GradV) \{ \newline 
 \newline 
\btab \btab double Acc = 0; \newline 
\btab \btab Acc -= U[0]*GradV[this.d]; \newline 
\btab \btab return Acc; \newline 
\btab \}         \newline 
 \newline 
\btab public double InnerEdgeForm(ref CommonParams inp,  \newline 
\btab \btab double[] U\_IN, double[] U\_OT, double[,] GradU\_IN, double[,] GradU\_OT,  \newline 
\btab \btab double V\_IN, double V\_OT, double[] GradV\_IN, double[] GradV\_OT) \{ \newline 
 \newline 
\btab \btab double Acc = 0; \newline 
\btab \btab Acc += 0.5*(U\_IN[0] + U\_OT[0])*inp.Normale[this.d]*(V\_IN - V\_OT); \newline 
\btab \btab return Acc;   \newline 
\btab  \} \newline 
 \newline 
\btab public double BoundaryEdgeForm(ref CommonParamsBnd inp,  \newline 
\btab \btab double[] U\_IN, double[,] GradU\_IN, double V\_IN, double[] GradV\_OT) \{ \newline 
 \newline 
\btab \btab double Acc = 0; \newline 
\btab \btab if(IsDirichletBndy(inp.X)) \{ \newline 
\btab \btab \btab double u\_Diri = 0.0; \newline 
\btab \btab \btab Acc += u\_Diri*inp.Normale[this.d]*V\_IN; \newline 
\btab \btab \} else \{ \newline 
\btab \btab \btab Acc += U\_IN[0]*inp.Normale[this.d]*V\_IN; \newline 
\btab \btab \} \newline 
\btab \btab return Acc;               \newline 
\btab \} \newline 
\}
 }
\BoSSSexe
\BoSSScmd{
/// Now, we are ready to assemble the full $\nabla$ operator
/// as $\left[ \begin{array}{c} \partial_x \\ \partial_y \end{array} \right]$.
var OpGrad = new SpatialOperator(1,2,QuadOrderFunc.Linear(),"u","c1","c2"); \newline 
OpGrad.EquationComponents["c1"].Add(new Gradient\_d(0)); \newline 
OpGrad.EquationComponents["c2"].Add(new Gradient\_d(1)); \newline 
OpGrad.Commit();
 }
\BoSSSexe
\BoSSScmd{
/// As an additional test, we create the gradient-matrix and verify that 
/// its transpose 
/// is equal to the negative \code{MtxDiv}-matrix:
var MtxGrad = OpGrad.ComputeMatrix(TestMapping, null, TrialMapping); \newline 
var Test2   = MtxGrad.Transpose() - MtxDiv\_strong; \newline 
Test2.InfNorm();
 }
\BoSSSexe
\BoSSScmd{
/// %====================================
/// \subsection{The complete Poisson-system}
/// %====================================
 }
\BoSSSexe
\BoSSScmd{
///\paragraph{Assembly of the system}
 }
\BoSSSexe
\BoSSScmd{
/// We also need the identity-matrix in the top-left corner 
/// of the Poisson-system:
public class Identity :  \newline 
\btab \btab BoSSS.Foundation.IVolumeForm  \newline 
\{ \newline 
\btab public IList<string> ParameterOrdering \{  \newline 
\btab \btab get \{ return new string[0]; \}  \newline 
\btab \} \newline 
 \newline 
\btab public string component;  \newline 
 \newline 
\btab public IList<String> ArgumentOrdering \{  \newline 
\btab \btab get \{ return new string[] \{ component \}; \}  \newline 
\btab \} \newline 
 \newline 
\btab public TermActivationFlags VolTerms \{ \newline 
\btab \btab get \{ \newline 
\btab \btab \btab return TermActivationFlags.AllOn; \newline 
\btab \btab \} \newline 
\btab \} \newline 
 \newline 
\btab public double VolumeForm(ref CommonParamsVol cpv,  \newline 
\btab \btab    double[] U, double[,] GradU,  \newline 
\btab \btab    double V, double[] GradV) \{ \newline 
\btab \btab return U[0]*V;            \newline 
\btab \} \newline 
\}
 }
\BoSSSexe
\BoSSScmd{
/// We are going to implement the linear Poisson-operator
/// \[
/// \left[ \begin{array}{ccc}
///      1           &  0         & \partial_x \\
///      0           &  1         & \partial_y \\
///      -\partial_x & -\partial_y & 0 
/// \end{array} \right]
/// \cdot 
/// \left[ \begin{array}{c} \sigma_0 \\ \sigma_1 \\ u \end{array} \right]
/// = 
/// \left[ \begin{array}{c} c_0 \\ c_1 \\ c_2 \end{array} \right]
/// \]
/// The variables $c_0$, $c_1$ and $c_2$, which correspond to the 
/// test functions are also called co-domain variables of the operator.
/// We are using the negative divergence, since this will lead to a 
/// symmetric matrix, instead of a anti-symmetric one.
/// By doing so, we can e.g. use a Cholesky-factorization to determine 
/// whether the system is definite or not.
var OpPoisson = new SpatialOperator(3, 3,  \newline 
\btab \btab \btab \btab \btab   QuadOrderFunc.Linear(), \newline 
\btab \btab \btab \btab \btab   "sigma1", "sigma2", "u", // the domain-variables \newline 
\btab \btab \btab \btab \btab   "c1", "c2", "c3"); //       the co-domain variables  \newline 
/// Now we add all required components to \code{OpPoisson}:
OpPoisson.EquationComponents["c1"].Add(new Gradient\_d(0)); \newline 
OpPoisson.EquationComponents["c1"].Add(new Identity() \{ component = "sigma1" \}); \newline 
OpPoisson.EquationComponents["c2"].Add(new Gradient\_d(1)); \newline 
OpPoisson.EquationComponents["c2"].Add(new Identity() \{ component = "sigma2" \}); \newline 
OpPoisson.EquationComponents["c3"].Add(new Divergence\_strong()); \newline 
OpPoisson.Commit();
 }
\BoSSSexe
\BoSSScmd{
/// We create mappings $[\sigma_1, \sigma_2, u ]$:
/// three different combinations of DG orders will be investigated:
/// \begin{itemize}
/// \item equal order: the same polynomial degree for $u$ and $\vec{\sigma}$
/// \item mixed order: the degree of $u$ is lower than the degree 
///       of $\vec{\sigma}$.
/// \item `strange' order: the degree of $u$ is higher than the degree of 
///       $\vec{\sigma}$.
/// \end{itemize}
var b3         = new Basis(gdata2D, 3); \newline 
var b2         = new Basis(gdata2D, 2); \newline 
var b4         = new Basis(gdata2D, 4); \newline 
var EqualOrder = new UnsetteledCoordinateMapping(b3,b3,b3); \newline 
var MixedOrder = new UnsetteledCoordinateMapping(b4,b4,b3); \newline 
var StrngOrder = new UnsetteledCoordinateMapping(b2,b2,b3);
 }
\BoSSSexe
\BoSSScmd{
var MtxPoisson\_Equal = OpPoisson.ComputeMatrix(EqualOrder, null, EqualOrder); \newline 
var MtxPoisson\_Mixed = OpPoisson.ComputeMatrix(MixedOrder, null, MixedOrder); \newline 
var MtxPoisson\_Strng = OpPoisson.ComputeMatrix(StrngOrder, null, StrngOrder);
 }
\BoSSSexe
\BoSSScmd{
/// We show that the matrices are symmetric 
/// (use e.g. \code{SymmetryDeviation(...)}), but indefinite
/// (use e.g. \code{IsDefinite(...)}).
 }
\BoSSSexe
\BoSSScmd{
double symDev\_Equal = MtxPoisson\_Equal.SymmetryDeviation();\newline 
symDev\_Equal;
 }
\BoSSSexe
\BoSSScmd{
double symDev\_Mixed = MtxPoisson\_Mixed.SymmetryDeviation();\newline 
symDev\_Mixed;
 }
\BoSSSexe
\BoSSScmd{
double symDev\_Strng = MtxPoisson\_Strng.SymmetryDeviation();\newline 
symDev\_Strng;
 }
\BoSSSexe
\BoSSScmd{
MtxPoisson\_Equal.IsDefinite();
 }
\BoSSSexe
\BoSSScmd{
MtxPoisson\_Mixed.IsDefinite();
 }
\BoSSSexe
\BoSSScmd{
MtxPoisson\_Strng.IsDefinite();
 }
\BoSSSexe
\BoSSScmdSilent{
/// BoSSScmdSilent BoSSSexeSilent
NUnit.Framework.Assert.LessOrEqual(symDev\_Equal, 1.0e-8);\newline 
NUnit.Framework.Assert.LessOrEqual(symDev\_Mixed, 1.0e-8);\newline 
NUnit.Framework.Assert.LessOrEqual(symDev\_Strng, 1.0e-8);
 }
\BoSSSexeSilent
\BoSSScmd{
    /// %==========================================
    /// \section{Advanced topics}
    /// %==========================================
 }
\BoSSSexe
\BoSSScmd{
/// %====================================
/// \subsection{Algebraic reduction}
/// %====================================
/// Since the top-left corner of our matrix 
/// \[ 
/// \left[ \begin{array}{cc}
/// 1   & B \\
/// B^T & 0 
/// \end{array} \right]
/// \]
/// is actually very easy to eliminate the variable $\vec{\sigma}$
/// from our system algebraically. 
/// The matrix of the reduces system is obviously $B^T \cdot B$.
 }
\BoSSSexe
\BoSSScmd{
/// \paragraph{Extraction of sub-matrices and elimination}
/// From the mapping, we can actually obtain index-lists for each variable,
/// which can then be used to extract sub-matrices from 
/// \code{MtxPoisson\_Equal}, \code{MtxPoisson\_Mixed}, resp. 
/// \code{MtxPoisson\_Strng}.
int[] SigmaIdx\_Equal = EqualOrder.GetSubvectorIndices(true, 0,1); \newline 
int[] uIdx\_Equal     = EqualOrder.GetSubvectorIndices(true, 2); \newline 
int[] SigmaIdx\_Mixed = MixedOrder.GetSubvectorIndices(true, 0,1); \newline 
int[] uIdx\_Mixed     = MixedOrder.GetSubvectorIndices(true, 2); \newline 
int[] SigmaIdx\_Strng = StrngOrder.GetSubvectorIndices(true, 0,1); \newline 
int[] uIdx\_Strng     = StrngOrder.GetSubvectorIndices(true, 2);
 }
\BoSSSexe
\BoSSScmd{
// The extraction of the sub-matrix and the elimination, for the equal order: \newline 
var MtxPoissonRed\_Equal =  \newline 
\btab   MtxPoisson\_Equal.GetSubMatrix(uIdx\_Equal, SigmaIdx\_Equal)  // -Divergence \newline 
\btab * MtxPoisson\_Equal.GetSubMatrix(SigmaIdx\_Equal, uIdx\_Equal); // Gradient
 }
\BoSSSexe
\BoSSScmd{
/// Finally, we also
/// create the reduced system for the mixed and the strange 
/// order, test for the definiteness of the reduced system.
/// Equal and mixed order are positive definite, while the strange order
/// is indefinite - a clear indication that something ist wrong:
 }
\BoSSSexe
\BoSSScmd{
var MtxPoissonRed\_Mixed =  \newline 
\btab   MtxPoisson\_Mixed.GetSubMatrix(uIdx\_Mixed, SigmaIdx\_Mixed)  // -Divergence \newline 
\btab * MtxPoisson\_Mixed.GetSubMatrix(SigmaIdx\_Mixed, uIdx\_Mixed); // Gradient   \newline 
var MtxPoissonRed\_Strng =  \newline 
\btab   MtxPoisson\_Strng.GetSubMatrix(uIdx\_Strng, SigmaIdx\_Strng)  // -Divergence \newline 
\btab * MtxPoisson\_Strng.GetSubMatrix(SigmaIdx\_Strng, uIdx\_Strng); // Gradient
 }
\BoSSSexe
\BoSSScmd{
bool isdef\_red\_Equal = MtxPoissonRed\_Equal.IsDefinite();\newline 
isdef\_red\_Equal;
 }
\BoSSSexe
\BoSSScmd{
bool isdef\_red\_Mixed = MtxPoissonRed\_Mixed.IsDefinite();\newline 
isdef\_red\_Mixed;
 }
\BoSSSexe
\BoSSScmd{
bool isdef\_red\_Strng = MtxPoissonRed\_Strng.IsDefinite();\newline 
isdef\_red\_Strng;
 }
\BoSSSexe
\BoSSScmdSilent{
/// BoSSScmdSilent BoSSSexeSilent
NUnit.Framework.Assert.IsTrue(isdef\_red\_Equal);\newline 
NUnit.Framework.Assert.IsTrue(isdef\_red\_Mixed);\newline 
NUnit.Framework.Assert.IsFalse(isdef\_red\_Strng);
 }
\BoSSSexeSilent
\BoSSScmd{
/// We compute the condition number of all three matrices; we observe that
/// the mixed as well as the equal-order discretization result give rather 
/// moderate condition numbers. For the strange orders, the condition number
/// of the system is far to high:
 }
\BoSSSexe
\BoSSScmd{
double condest\_Mixed = MtxPoissonRed\_Mixed.condest();\newline 
condest\_Mixed;
 }
\BoSSSexe
\BoSSScmd{
double condest\_Equal = MtxPoissonRed\_Equal.condest();\newline 
condest\_Equal;
 }
\BoSSSexe
\BoSSScmd{
double condest\_Strng = MtxPoissonRed\_Strng.condest();\newline 
condest\_Strng;
 }
\BoSSSexe
\BoSSScmdSilent{
/// BoSSScmdSilent BoSSSexeSilent
NUnit.Framework.Assert.LessOrEqual(condest\_Mixed, 1e5);\newline 
NUnit.Framework.Assert.LessOrEqual(condest\_Equal, 1e5);\newline 
NUnit.Framework.Assert.Greater(condest\_Strng, 1e10);
 }
\BoSSSexeSilent
