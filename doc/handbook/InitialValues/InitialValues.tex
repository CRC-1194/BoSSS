% !TeX spellcheck = en_US

\BoSSSopen{InitialValues/InitialValues}
\graphicspath{{InitialValues/InitialValues.texbatch/}}

\paragraph{What's new:} 
In this tutorial you will learn to set initial values and boundary values
\begin{itemize}
	\item from basic formulas
	\item from MATLAB scripts
\end{itemize}

\paragraph{Prerequisites:} 
\begin{itemize}
	\item basic knowledge of \BoSSSpad{}
	\item executing runs on your local machine, e.g. the quick-start guide of the \ac{cns}, see chapter \ref{sec:CNS}
    \item an understanding, what a \BoSSS{} database is
\end{itemize}
\BoSSScmd{
/// % ====================
/// \section{Introduction}
/// % ====================
 }
\BoSSSexeSilent
\BoSSScmd{
restart
 }
\BoSSSexeSilent
\BoSSScmd{
/// Note: (1) Setting Boundary values and initial values is similar; 
/// (2) For most solvers, inital and boundary values are set the same way;
/// (3) We will use the incompressible solver as an example:
 }
\BoSSSexe
\BoSSScmd{
using BoSSS.Application.XNSE\_Solver;
 }
\BoSSSexe
\BoSSScmd{
/// Create a control object:
 }
\BoSSSexe
\BoSSScmd{
var C = new XNSE\_Control();
 }
\BoSSSexe
\BoSSScmd{
/// % =======================
/// \section{From Formulas}
/// % =======================
 }
\BoSSSexe
\BoSSScmd{
/// If the Formula is simple enougth to be represented by C\# code,
/// it can be embedded in the control file. The most
 }
\BoSSSexe
\BoSSScmd{
/// First option: from code, e.g.:
 }
\BoSSSexe
\BoSSScmd{
static class MyInitialValue \{ // class should be static!\newline 
 \newline 
\btab // Warning: static constants are allowed,\newline 
\btab // but any changes outside of the current text box in BoSSSpad\newline 
\btab // will not be recorded for the code that is passed to the solver.\newline 
\btab public static double alpha = 0.7;\newline 
 \newline 
\btab // a method, which should be used for an initial value,\newline 
\btab // must be static!\newline 
\btab public static double VelocityX(double[] X, double t) \{\newline 
\btab \btab double x = X[0];\newline 
\btab \btab double y = X[1];\newline 
\btab \btab return Math.Sin(x*y*alpha);\newline 
\btab \}   \newline 
\}
 }
\BoSSSexe
\BoSSScmd{
 % 
 }
\BoSSSexe
\BoSSScmd{
/// Use the BoSSSpad-intrinsic \code{GetFormulaObject} to set tie hinital value:
C.AddInitialValue("VelocityX", GetFormulaObject(MyInitialValue.VelocityX));
 }
\BoSSSexe
\BoSSScmd{
 % 
 }
\BoSSSexe
\BoSSScmd{
/// Note: such a declaration is very restrictive;
/// \code{GetFormulaObject} works only for 
/// \begin{itemize}
/// \item use a static class
/// \item no dependence on any external parameters
/// \end{itemize}
/// E.g. the following code would only change the behavior in BoSSSpad,
/// but not the code that is passed to the solver:
 }
\BoSSSexe
\BoSSScmd{
MyInitialValue.alpha = 0.5;\newline 
MyInitialValue.VelocityX(new double[]\{ 0.5, 0.5 \}, 0.0);
 }
\BoSSSexe
\BoSSScmd{
C.InitialValues["VelocityX"].Evaluate(new double[]\{ 0.5, 0.5 \}, 0.0);
 }
\BoSSSexe
\BoSSScmd{
/// Obviously, both values are different.
 }
\BoSSSexe
\BoSSScmd{
 % 
 }
\BoSSSexe
\BoSSScmd{
 % 
 }
\BoSSSexe
\BoSSScmd{
/// \subsection{Advanced functions}
/// % =============================
 }
\BoSSSexe
\BoSSScmd{
/// Some more advanced mathematical funstions, e.g.
/// Jacobian elliptic functions $\text{sn}(u|m)$, $\text{cn}(u|m)$ and $\text{dn}(u|m)$
/// are available throug the GNU Scientific Library, for which BoSSS provides
/// bindings, see e.g.
/// \code{BoSSS.Platform.GSL.gsl\_sf\_elljac\_e}.
 }
\BoSSSexe
\BoSSScmd{
/// % ========================
/// \section{From MATLAB code}
/// % ========================
 }
\BoSSSexe
\BoSSScmd{
/// % ========================
/// \subsection{Conecting with MATLAB}
/// % ========================
 }
\BoSSSexe
\BoSSScmd{
/// Asssume e.g. the following MATLAB code; obviously, this could  
/// also be implemented in C\#, we yust use something smple for demonstration:
 }
\BoSSSexe
\BoSSScmd{
string[] MatlabCode = new string[] \{\newline 
@"[n,d2] =  size(X\_values);",\newline 
@"u=zeros(2,n);",\newline 
@"for k=1:n",\newline 
@"X=[X\_values(k,1),X\_values(k,2)];",\newline 
@"",\newline 
@"u\_x\_main     = -(-sqrt(X(1).^ 2 + X(2).^ 2) / 0.3e1 + 0.4e1 / 0.3e1 * (X(1).^ 2 + X(2).^ 2) ^ (-0.1e1 / 0.2e1)) * sin(atan2(X(2), X(1)));",\newline 
@"u\_y\_main     = (-sqrt(X(1).^ 2 + X(2).^ 2) / 0.3e1 + 0.4e1 / 0.3e1 * (X(1).^ 2 + X(2).^ 2) ^ (-0.1e1 / 0.2e1)) * cos(atan2(X(2), X(1)));",\newline 
@"",   \newline 
@"u(1,k)=u\_x\_main;",\newline 
@"u(2,k)=u\_y\_main;",\newline 
@"end" \};
 }
\BoSSSexe
\BoSSScmd{
/// We can evaluate this code in \BoSSS{} using the MATLAB connector;
/// We encapsulate it in a \code{ScalarFunction} which allows 
/// \emph{vectorized} evaluation 
/// (multiple evaluatiuons in one function call) e
/// of some function.
/// This is much more efficient, since there will be significant overhead
/// for calling MATLAB (starting MATLAB, checking the license, 
/// transfering data, etc.).
 }
\BoSSSexe
\BoSSScmd{
using ilPSP.Connectors.Matlab;
 }
\BoSSSexe
\BoSSScmd{
ScalarFunction VelocityXInitial = \newline 
delegate(MultidimensionalArray input, MultidimensionalArray output) \{\newline 
\btab int N          = input.GetLength(0); // number of points which we evaluate \newline 
\btab //                                      at once.\newline 
\btab var output\_vec = MultidimensionalArray.Create(2, N); // the MATLAB code\newline 
\btab //                        returns an entire vector.\newline 
\btab using(var bmc = new BatchmodeConnector()) \{\newline 
\btab    bmc.PutMatrix(input,"X\_values");\newline 
 \newline 
\btab    foreach(var line in MatlabCode) \{\newline 
\btab \btab    bmc.Cmd(line);   \newline 
\btab    \}\newline 
 \newline 
\btab    bmc.GetMatrix(output\_vec, "u");\newline 
 \newline 
\btab    bmc.Execute(); // Note: 'Execute' has to be *after* 'GetMatrix'\newline 
   \}\newline 
   output.Set(output\_vec.ExtractSubArrayShallow(0,-1)); // extract row 0 from \newline 
   //                       'output\_vec' and store it in 'output'\newline 
\}
 }
\BoSSSexe
\BoSSScmd{
/// We test our implementation:
 }
\BoSSSexe
\BoSSScmd{
var inputTest = MultidimensionalArray.Create(3,2); // set some test values for input\newline 
inputTest.SetColumn(0, GenericBlas.Linspace(1,2,3));\newline 
inputTest.SetColumn(1, GenericBlas.Linspace(2,3,3));\newline 
 \newline 
var outputTest = MultidimensionalArray.Create(3); // allocate memory for output
 }
\BoSSSexe
\BoSSScmd{
VelocityXInitial(inputTest, outputTest);
 }
\BoSSSexe
\BoSSScmd{
/// We recive the following velocity values for our input coordinates:
outputTest.To1DArray();
 }
\BoSSSexe
\BoSSScmd{
/// % =======================================================
/// \subsection{Projecting the MATLAB function to a DG field}
/// % =======================================================
 }
\BoSSSexe
\BoSSScmd{
/// As for a standard calculation, we create a mesh, save it to some database
/// and set the mesh in the control object.
 }
\BoSSSexe
\BoSSScmd{
var nodes        = GenericBlas.Linspace(1,2,11);\newline 
GridCommons grid = Grid2D.Cartesian2DGrid(nodes,nodes);
 }
\BoSSSexe
\BoSSScmd{
var db = CreateTempDatabase();
 }
\BoSSSexe
\BoSSScmd{
db.SaveGrid(ref grid);
 }
\BoSSSexe
\BoSSScmd{
C.SetGrid(grid);
 }
\BoSSSexe
\BoSSScmd{
/// We create a DG field for the $x$-velocity on our grid:
 }
\BoSSSexe
\BoSSScmd{
var gdata = new GridData(grid);
 }
\BoSSSexe
\BoSSScmd{
var b = new Basis(gdata, 3); // use DG degree 2
 }
\BoSSSexe
\BoSSScmd{
var VelX = new SinglePhaseField(b,"VelocityX"); // important: name the DG field\newline 
//                                 equal to initial value name
 }
\BoSSSexe
\BoSSScmd{
/// Finally, we are able to project the MATLAB function onto the DG field:
 }
\BoSSSexe
\BoSSScmd{
//VelX.ProjectField(VelocityXInitial);
 }
\BoSSSexe
\BoSSScmd{
/// One might want to check the data visually, so it can be exported
/// in the usual fashion
 }
\BoSSSexe
\BoSSScmd{
//Tecplot("initial",0.0,2,VelX);
 }
\BoSSSexe
\BoSSScmd{
 % 
 }
\BoSSSexe
\BoSSScmd{
/// % ================================================================
/// \subsection{Storing the initial value in the database and linking
///    it in the control object}
/// % ================================================================
 }
\BoSSSexe
\BoSSScmd{
/// The DG field with the initial value can be stored in the database.
/// this will create a dummy session.
 }
\BoSSSexe
\BoSSScmd{
WorkflowMgm.Init("TestProject");
 }
\BoSSSexe
\BoSSScmd{
var InitalValueTS = db.SaveTimestep(VelX); // further fields an be \newline 
//                                                  appended
 }
\BoSSSexe
\BoSSScmd{
WorkflowMgm.Sessions;
 }
\BoSSSexe
\BoSSScmd{
/// Now, we can use this timestep as a restart-value for the simulation:
 }
\BoSSSexe
\BoSSScmd{
C.SetRestart(InitalValueTS);
 }
\BoSSSexe
