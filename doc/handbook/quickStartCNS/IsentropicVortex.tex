% !TeX spellcheck = en_EN
%\documentclass[12pt,a4paper]{article}
%\usepackage{import}
%\subimport{../}{uebung.tex}
%
%%\toggletrue{showSolution}
%
%
%\title{\thetitle{}\\Exercise 4}
%\date{\WeeksAfter{3}}
%
%
%\begin{document}


\BoSSSopen{quickStartCNS/IsentropicVortex}
\graphicspath{{quickStartCNS/IsentropicVortex.texbatch/}}
%\BoSSSopen{IsentropicVortex}
%\graphicspath{{IsentropicVortex.texbatch/}}

\BoSSScmd{
///\section*{What's new}
///\begin{itemize}
/// \item running a simple simulation with the CNS solver
/// \item the important input data to solve the Euler equations
/// \item generating a simple mesh by using the \BoSSS{} internal mesher
/// \item the post-processing of the data, i.e. visualization in Paraview or VisIt
///\end{itemize}
///
///\subsection*{Prerequisites}
///\begin{itemize}
/// \item the \BoSSS{} framework
/// \item a visualization tool, e.g Paraview or VisIt
/// \item the knowledge of how to setup a database for \BoSSS{}
///\end{itemize}
///
///\section*{Problem statement}
///We only briefly describe the non dimensional Euler equations in 2D
/// \begin{equation*}
///\del{\vec{U}}{t}+\del{\vec{F}^c_j(\vec{U})}{x_j}= 0,
///\label{eq:euler_sys}
///\end{equation*}
///where $\vec{U}$ are the conserved flow variables and $\vec{F}^c_x$  are the convective fluxes, i.e.
///\begin{align*}
///\vec{U} = \begin{pmatrix}
///  \density{}\\
///  \momentum_i\\
///  \energy{}
///\end{pmatrix}, \quad \textrm{and} \quad
///\vec{F}^c_i = \begin{pmatrix}
///  \momentum_i\\
///  \momentum_i \velocity_j + \frac{1}{\heatCapacityRatio \reference{\Mach}^2}\pressure \delta_{ij}\\
///  \velocity_j \left(\energy{}+\pressure{}\right)
///\end{pmatrix}.
///\end{align*}
///Note: In our non dimensional equations, we introduce $\reference{\Mach}$. 
///
///These Quick Start tutorials are aimed to show some of the main features of the compressible flow 
///solver (Compressible Navier-Stokes (CNS)) in the \BoSSS{} framework. As already mentioned 
///in the introduction, \BoSSS{} uses a C\# code based input data and interprets these in 
///the \emph{REPL} fashion. This gives us more flexibility in the way how we can start a 
///simulation. We can have the traditional way of defining an input file, where we define all 
///important parameters in C\# code, or we can also use some predefined functions in the framework 
///to generate our input data and manipulate them interactively in the \BoSSSpad{}.\\ 
///
///As an example, we will simulate the well known isentropic vortex for the Euler equations. 
///\section{Isentropic vortex}
 }
\BoSSSexeSilent
\BoSSScmd{
restart;
 }
\BoSSSexeSilent
\BoSSScmd{
/// \subsection{Interactive mode}
/// We start with the interactive mode and load the namespace.
 }
\BoSSSexe
\BoSSScmd{
/// We create, resp. open a \BoSSS database:
var myDatabase = CreateTempDatabase();
 }
\BoSSSexe
\BoSSScmd{
using CNS;
 }
\BoSSSexe
\BoSSScmd{
typeof(CNS.Program).Assembly.Location;
 }
\BoSSSexe
\BoSSScmd{
/// We load the input data from some function within the Framework. 
/// The class \verb|ControlExamples| provides some predefined control sets for 
/// different typical test cases, i.a. the isentropic vortex.
/// For the isentropic vortex, you have to specify 
/// \begin{itemize}
/// \item path to a database (here: EMPTY path)
/// \item number of cells in each direction (here: 20)
/// \item DG order (here: 2)
/// \item advection velocity of the vortex (here: 1.0)
/// \end{itemize}
int noOfCellsPerDirection = 20;       \newline 
var c = ControlExamples_Subsonic.IsentropicVortex(myDatabase.Path,      \newline 
\btab \btab \btab \btab \btab \btab \btab \btab \btab \btab  noOfCellsPerDirection,2,1.0);\newline 
c.savetodb = true;       \newline 
/// Now, we have a set of input data stored in our variable \verb|c| and we 
/// can have a look at different parameters directly via \code{BoSSSpad}.
 }
\BoSSSexe
\BoSSScmd{
/// \subsection{How to define/change input data}
/// In this section we will walk you through the most important parameters for 
/// running the test case. We start with the grid and use the ability of \BoSSS\
/// to generate simple grids with its own mesh generator. Once we start the simulation,
/// the control object \verb|c| is parsed and the grid defined by the \verb|GridFunc|
/// is generated on the fly and stored in the database. Here, we defined a 
/// \verb|delegate| function which returns a uniform cartesian grid. First, we have to
/// define a 1-D array, which spans from -10 to 10 and is divided into the number 
/// of cells, which we previously set to \verb|int noOfCellsPerDirection = 20|. 
/// The function \verb|Grid2D.Cartesian2DGrid| generates a
/// uniform 2-D grid out of this array by using it for x and y direction
/// Additionally we specify periodic boundary conditions in x and y direction
/// by setting \verb|periodicX: true| and \verb|periodicY: true|. \\
c.GridFunc = delegate \{       \newline 
   double[] nodes = GenericBlas.Linspace(-10, 10, noOfCellsPerDirection + 1);       \newline 
   var grid = Grid2D.Cartesian2DGrid(nodes, nodes,       \newline 
\btab \btab \btab \btab \btab \btab \btab \btab \btab  periodicX: true, periodicY: true);       \newline 
   return grid;       \newline 
\};
 }
\BoSSSexe
\BoSSScmd{
/// The \code{CNS} solver is able to solve the Euler and the compressible Navier-Stokes equations. By setting
c.ActiveOperators;
 }
\BoSSSexe
\BoSSScmd{
/// we only use the convective fluxes, i.e the Euler equations, and set it to
c.ConvectiveFluxType;
 }
\BoSSSexe
\BoSSScmd{
/// Note: Optimized means in this case, that this is the classical HLLC flux, but implemented in an optimized version. \\
/// As initial conditions, we choose the analytical solution, which can be found in various publications, e.g. Hu (2006).
/// The Mach number is set in the following:
c.MachNumber
 }
\BoSSSexe
\BoSSScmd{
/// Further, we have to define a simulation time, i.e
c.Endtime;
 }
\BoSSSexe
\BoSSScmd{
/// Finally, we need a time stepping scheme
c.ExplicitScheme;
 }
\BoSSSexe
\BoSSScmd{
/// of order
c.ExplicitOrder;\newline 
/// to run the simulation.
 }
\BoSSSexe
\BoSSScmd{
/// These are all predefined input values, which were set by calling \verb|ControlExamples_Subsonic.IsentropicVortex(...)|.
/// Since we are in the interactiv mode, we can change them directly in the \code{BoSSSPad}.
/// For example, we can reduce the order of our timestepping scheme to 3, 
/// because we only use DG order 2:
c.ExplicitOrder = 3;      \newline 
c.ExplicitOrder;
 }
\BoSSSexe
\BoSSScmd{
/// Or we can change the Mach number to
c.MachNumber = 0.7;    \newline 
c.MachNumber
 }
\BoSSSexe
\BoSSScmd{
/// \subsection{Run a simulation}
/// We adjusted our input values and now we can run a simulation.
/// In the interactive mode, we can simply execute \code{Run()} on the control
/// object, which will execute the solver:
c.PrintInterval = 5;\newline 
var SI          = c.Run();
 }
\BoSSSexe
\BoSSScmd{
/// The \code{Run()} command finally returns a session info, which carries some
/// basic information on the solver run (mainly where ist was stored).
SI;
 }
\BoSSSexe
\BoSSScmd{
/// \subsection{Console mode}
/// We can also run this simulation in the ``traditional'' way, which most
/// of you are familiar with from other academical codes. We define an input file,
/// which is nothing else than the above C\# code. We can run it by calling
/// \verb|CNS.exe -c IsentropicVortex.cs|. You can find the input file in 
/// ControlExamples folder in the doc directory.
 }
\BoSSSexe
\BoSSScmd{
/// \subsection{Postprocessing}
/// We saved our data in the database and lastly we want to postprocess it, 
/// i.e visualize the individual fields like density, momentum or pressure.
/// Note: If you have run the simulation in the console mode, you now have to start 
/// the \BoSSSpad{}.
/// In our example, we find the corresponding session in our first database as first session
//myDatabase.Sessions.First();
 }
\BoSSSexe
\BoSSScmd{
/// To convert data to the Tecplot format, we just need to export it:
//databases.First().Sessions.First().Export().Do()
 }
\BoSSSexe
\BoSSScmd{
/// We can open the folder directly by using
//databases.First().Sessions.First().OpenExportDirectory()
 }
\BoSSSexe
\BoSSScmd{
/// and here we find \verb|*plt| files with our data.
/// For more information about our databases and useful commands for postprocessing,
/// we refer to our tutorials about the database and the database command overview.
 }
\BoSSSexe
