\BoSSSopen{ParameterStudy/ParameterStudy}
\graphicspath{{ParameterStudy/ParameterStudy.texbatch}}

\BoSSScmd{
restart;
 }
\BoSSSexeSilent
\BoSSScmd{
/// This guide will give you an example of how to conduct a parameter study with
/// all the necessary steps.  
/// \section{Initialization of solver, processor and workflow}
/// We start with initializing of the workflow
 }
\BoSSSexe
\BoSSScmd{
/// \section{Initialization of solver, processor and workflow}
/// We start with initializing of the workflow.
 }
\BoSSSexe
\BoSSScmd{
WorkflowMgm.Init("Name of Workflow");
 }
\BoSSSexe
\BoSSScmd{
/// This line helps us manage the sessions later on while evaluating the results. 
/// Next, we connect to the database.
 }
\BoSSSexe
\BoSSScmd{
var myDb = CreateTempDatabase();
 }
\BoSSSexe
\BoSSScmd{
/// To check all the sessions in the current workflow, use the line:
 }
\BoSSSexe
\BoSSScmd{
WorkflowMgm.Sessions;
 }
\BoSSSexe
\BoSSScmd{
/// Now, all the necessary libraries need to be loaded
 }
\BoSSSexe
\BoSSScmd{
using System.Diagnostics;\newline 
using BoSSS.Foundation.Grid.RefElements;\newline 
using BoSSS.Application.XNSE\_Solver;\newline 
using BoSSS.Platform.LinAlg;\newline 
using BoSSS.Solution.XdgTimestepping;
 }
\BoSSSexe
\BoSSScmd{
/// As an execution queue, we select the first queue defined in 
/// the {\tt $\sim$/.BoSSS/etc/BatchProcessorConfig.json}-file:
 }
\BoSSSexe
\BoSSScmd{
var myBatch = ExecutionQueues[0];
 }
\BoSSSexe
\BoSSScmd{
/// \section{Grid Generation}
///Firstly, we need to determine the boundaries of our grid/control volume. 
///Is it important to know that the number of nodes (in our case $/code{k}$)
///needed are equal
/// to the number of cells $+1$. For instance, for $10$ cells we need $11$ nodes.
/// In this example we will use the Cartesian $2D$ grid from the database
/// which requires $x$- and $y$-Nodes. The J term in the code is for doing
/// a check if the desired resolution of the volume is correctly typed.
 }
\BoSSSexe
\BoSSScmd{
int k = 10;\newline 
double[] xNodes = GenericBlas.Linspace(0, 1, k + 1);\newline 
double[] yNodes = GenericBlas.Linspace(0, 1, k + 1);\newline 
int J           = (xNodes.Length - 1)*(yNodes.Length - 1);\newline 
string GridName = string.Format(WorkflowMgm.CurrentProject + "\_J" +J);\newline 
 \newline 
Console.WriteLine("Creating grid with " + J + " cells. ");\newline 
 \newline 
GridCommons g;\newline 
g      = Grid2D.Cartesian2DGrid(xNodes, yNodes);\newline 
g.Name = GridName;
 }
\BoSSSexe
\BoSSScmd{
/// \section{Define geometrical boundaries}
/// After loading the grid and giving the dimensions, we need to adjust
/// the edges and their names. With the following code we assign every edge with
/// a number and name. Keep in mind that the name corresponds to the boundary 
///condition (in this case "Pressure Dirichlet").
/// In this particular case we will use inflow profile represented
/// via tan-function and the angle of inflow will be $30$ degrees.
 }
\BoSSSexe
\BoSSScmd{
GridCommons g;\newline 
g      = Grid2D.Cartesian2DGrid(xNodes, yNodes);\newline 
g.Name = GridName;\newline 
 \newline 
g.EdgeTagNames.Add(1, "wall");\newline 
g.EdgeTagNames.Add(2, "Velocity\_Inlet");\newline 
g.EdgeTagNames.Add(3, "Pressure\_Dirichlet\_back");\newline 
g.EdgeTagNames.Add(4, "Pressure\_Dirichlet\_top");\newline 
 \newline 
g.DefineEdgeTags(delegate (double[] X) \{\newline 
\btab byte ret = 0;\newline 
\btab if (Math.Abs(X[1]-(0.0))<= 1.0e-8)\newline 
\btab \btab ret = 1;\newline 
\btab if (Math.Abs(X[0]-(0.0))<= 1.0e-8)\newline 
\btab \btab ret = 2;\newline 
\btab if (Math.Abs(X[1]-(1.0))<= 1.0e-8)\newline 
\btab \btab ret = 3;\newline 
\btab if (Math.Abs(X[0]-(1.0))<= 1.0e-8)\newline 
\btab \btab ret = 4;\newline 
\btab return ret;\newline 
 \newline 
 \});
 }
\BoSSSexe
\BoSSScmd{
/// \section{Angle/Velocity Profile}
/// In this particular case we will use inflow profile represented via tan-function and the angle of inflow will be $30$ degrees.
 }
\BoSSSexe
\BoSSScmd{
string caseName = string.Format("k\{0\}\_\{1\}", k, g);\newline 
 \newline 
Console.WriteLine("setting up: " + caseName);\newline 
 \newline 
double beta    = 30;\newline 
string CosBeta = Math.Cos(beta*Math.PI/180.0).ToString();\newline 
string SinBeta = Math.Sin(beta*Math.PI/180.0).ToString();
 }
\BoSSSexe
\BoSSScmd{
/// These code lines set up the case name and introduce the sine and cosine 
/// functions to our simulation. Next, we define the velocities in 
/// $x$- and $y$-direction via a tan-function. These velocities and angles are only for this particular example and would not be suited for your simulation.
 }
\BoSSSexe
\BoSSScmd{
var UX = new Formula\newline 
\btab (string.Format("X=> \{0\}*Math.Atan(X[1]*5)*2.0/Math.PI",CosBeta),false);\newline 
var UY = new Formula \newline 
\btab (string.Format("X=> \{0\}*Math.Atan(X[1]*5)*2.0/Math.PI",SinBeta),false);
 }
\BoSSSexe
\BoSSScmd{
///After the velocities and boundary conditions are set. 
///We need to determine all other simulation parameters needed to proceed. 
//The variable $\textbackslash code\{ctrl\}$ is used to store the $\textbackslash code\{IBM\_Control\}$-object.\newline 
/// All other parameters are selfexplanatory.
 }
\BoSSSexe
\BoSSScmd{
var ctrl = new XNSE\_Control();\newline 
//controls.Add(ctrl);\newline 
 \newline 
ctrl.SessionName = caseName;\newline 
ctrl.SetDatabase(myDb);\newline 
ctrl.SetGrid(g);\newline 
ctrl.SetDGdegree(k);\newline 
ctrl.NoOfMultigridLevels = int.MaxValue;
 }
\BoSSSexe
\BoSSScmd{
/// \section{Boundary conditions/Initial values}
/// We move on to the part where we define
/// the boundary conditions and initial values.
 }
\BoSSSexe
\BoSSScmd{
ctrl.AddBoundaryValue("wall");\newline 
ctrl.AddBoundaryValue("Velocity\_Inlet");\newline 
ctrl.AddBoundaryValue("Pressure\_Dirichlet\_back");\newline 
ctrl.AddBoundaryValue("Pressure\_Dirichlet\_top");\newline 
ctrl.AddBoundaryValue("Velocity\_Inlet","VelocityX",UX);\newline 
ctrl.AddBoundaryValue("Velocity\_Inlet","VelocityY",UY);
 }
\BoSSSexe
\BoSSScmd{
/// and for the initial values
 }
\BoSSSexe
\BoSSScmd{
ctrl.InitialValues.Add("VelocityX", new Formula ("X=> 0.0", false));\newline 
ctrl.InitialValues.Add("VelocityY", new Formula ("X=> 0.0", false));\newline 
ctrl.InitialValues.Add("Pressure", new Formula ("X=> 0.0", false));\newline 
ctrl.InitialValues.Add("Phi", new Formula ("X=> -1.0", false));
 }
\BoSSSexe
\BoSSScmd{
/// \section{Fluid properties}
/// Here we set up the density and the Reynolds number,
/// keep in mind that the calculations are dimensionles, 
/// so leave the values as seen above ($100$ is an example value)
 }
\BoSSSexe
\BoSSScmd{
double reynolds               = 100;\newline 
ctrl.PhysicalParameters.rho\_A = 1;\newline 
ctrl.PhysicalParameters.mu\_A  = 1.0/reynolds;
 }
\BoSSSexe
\BoSSScmd{
/// \section{Simulation options}
/// We set the simulation parameters, such as time-step size,
/// end time and number of time-steps.
 }
\BoSSSexe
\BoSSScmd{
ctrl.TimeSteppingScheme = TimeSteppingScheme.ImplicitEuler;\newline 
double dt               = 7e-2;\newline 
ctrl.dtMax              = dt;\newline 
ctrl.dtMin              = dt;\newline 
ctrl.Endtime            = 1e16;\newline 
ctrl.NoOfTimesteps      = 100;
 }
\BoSSSexe
\BoSSScmd{
/// for the time-stepping scheme, you can choose either BDF2 or ImplicitEuler.
/// \section{Starting of simulation}
/// You have two possible ways to start a simulation
/// - locally on the PC via $\code{myBatch}$
/// or on the network cluster $\code{myHPC}$.
 }
\BoSSSexe
\BoSSScmd{
 \newline 
//Console.WriteLine(" Submitting to Cluster: " + ctrl.SessionName);\newline 
//ctrl.RunBatch(myHPC);\newline 
 \newline 
Console.WriteLine(" Submitting " + ctrl.SessionName);\newline 
ctrl.RunBatch(myBatch);
 }
\BoSSSexe
\BoSSScmd{
/// \section{Evaluation and Error Calculation}
/// After all of the desired simulation are finished,
/// you need to evaluate the different parameters and their effect on 
///the whole system. Typing the following command gives you a list of all 
///simulations with their status (FinishedSuccessful or with certain errors)
 }
\BoSSSexe
\BoSSScmd{
WorkflowMgm.AllJobs.Select(kv => kv.Key + ": \textbackslash t" + kv.Value.Status);
 }
\BoSSSexe
\BoSSScmd{
/// With the next command line you are able to select a certain 
///session(simulation) and see the different time-steps for control purposes.
 }
\BoSSSexe
\BoSSScmd{
WorkflowMgm.AllJobs.ElementAt(1).Value.Stdout;
 }
\BoSSSexe
\BoSSScmd{
/// \subsection{$L^2$-Error}
/// This section introduces the calculation of the $L^2$-Error.
 }
\BoSSSexe
\BoSSScmd{
 ITimestepInfo[] AllSolutionS = WorkflowMgm.AllJobs.Select(kv => kv.Value.LatestSession.Timesteps.Last()).ToArray();
 }
\BoSSSexe
\BoSSScmd{
ITimestepInfo[] k1\_SolutionS = AllSolutionS.Where(\newline 
\btab  ts = > ts.Fields.Single(\newline 
\btab \btab    f = > f.Identification == "Pressure").Basis.Degree == 0).ToArray();\newline 
ITimestepInfo[] k2\_SolutionS = AllSolutionS.Where(\newline 
\btab  ts = > ts.Fields.Single(\newline 
\btab \btab    f = > f.Identification == "Pressure").Basis.Degree == 1).ToArray();\newline 
ITimestepInfo[] k3\_SolutionS = AllSolutionS.Where(\newline 
\btab  ts = > ts.Fields.Single(\newline 
\btab \btab    f = > f.Identification == "Pressure").Basis.Degree == 2).ToArray();
 }
\BoSSSexe
\BoSSScmd{
k1\_SolutionS.Select(\newline 
\btab  ts => ts.Fields.Single(\newline 
\btab \btab    f = > f.Identification == "Pressure").Basis.Degree);
 }
\BoSSSexe
\BoSSScmd{
double[] GridRes;\newline 
Dictionary<string, double[]> L2Errors;\newline 
DGFieldComparison.ComputeErrors(\newline 
\btab  new[]\{"VelocityX","VelocityY"\}, k1\_SolutionS, out GridRes, out L2Errors);
 }
\BoSSSexe
\BoSSScmd{
/// To check the particular errors, type
 }
\BoSSSexe
\BoSSScmd{
GridRes;
 }
\BoSSSexe
\BoSSScmd{
L2Errors["VelocityX"];
 }
\BoSSSexe
\BoSSScmd{
L2Errors["VelocityY"];
 }
\BoSSSexe
\BoSSScmd{
/// \section{Plotting of errors}
/// This section gives a brief example of how to plot the erros 
/// and all the data from the previous simulations.
 }
\BoSSSexe
\BoSSScmd{
Plot(GridRes,L2Errors["VelocityX"],"VelXErr","-oy",\newline 
\btab  GridRes,L2Errors["VelocityY"],"VelXErr","-xb",logX:true,logY:true);
 }
\BoSSSexe
\BoSSScmd{
/// for a plot with more specifics and more possible adjustments
 }
\BoSSSexe
\BoSSScmd{
var FancyPlot = new Plot2Ddata();
 }
\BoSSSexe
\BoSSScmd{
FancyPlot.LogX = true;\newline 
FancyPlot.LogY = true;
 }
\BoSSSexe
\BoSSScmd{
var k1plot = new Plot2Ddata.XYvalues(\newline 
\btab "VelXErr-k1",GridRes,L2Errors["VelocityY"]);
 }
\BoSSSexe
\BoSSScmd{
ArrayTools.AddToArray(k1plot, ref FancyPlot.dataGroups);
 }
\BoSSSexe
\BoSSScmd{
var CL = FancyPlot.ToGnuplot().PlotCairolatex();
 }
\BoSSSexe
\BoSSScmd{
CL.PlotNow();
 }
\BoSSSexe
\BoSSScmd{
/// \section{Exporting the session table}
 }
\BoSSSexe
\BoSSScmd{
static class AddCols \{\newline 
\btab static public object SipMatrixAssembly\_time(ISessionInfo SI) \{\newline 
\btab \btab var mcr = SI.GetProfiling()[0];\newline 
\btab \btab var ndS = mcr.FindChildren("SipMatrixAssembly");\newline 
\btab \btab var nd  = ndS.ElementAt(0);\newline 
\btab \btab return nd.TimeSpentInMethod.TotalSeconds  / nd.CallCount;\newline 
\btab \}\newline 
\btab static public object Aggregation\_basis\_init\_time(ISessionInfo SI) \{\newline 
\btab \btab var mcr = SI.GetProfiling()[0];\newline 
\btab \btab var ndS = mcr.FindChildren("Aggregation\_basis\_init");\newline 
\btab \btab var nd  = ndS.ElementAt(0);\newline 
\btab \btab return nd.TimeSpentInMethod.TotalSeconds  / nd.CallCount;\newline 
\btab \}\newline 
\btab static public object Solver\_Init\_time(ISessionInfo SI) \{\newline 
\btab \btab var mcr = SI.GetProfiling()[0];\newline 
\btab \btab var ndS = mcr.FindChildren("Solver\_Init");\newline 
\btab \btab var nd  = ndS.ElementAt(0);\newline 
\btab \btab //Console.WriteLine("Number of nodes: " + ndS.Count() + " cc " + nd.CallCount );\newline 
\btab \btab return nd.TimeSpentInMethod.TotalSeconds / nd.CallCount;\newline 
\btab \}\newline 
\btab static public object Solver\_Run\_time(ISessionInfo SI) \{\newline 
\btab \btab var mcr = SI.GetProfiling()[0];\newline 
\btab \btab var ndS = mcr.FindChildren("Solver\_Run");\newline 
\btab \btab var nd  = ndS.ElementAt(0);\newline 
\btab \btab return nd.TimeSpentInMethod.TotalSeconds  / nd.CallCount;\newline 
\btab \}\newline 
\}
 }
\BoSSSexe
\BoSSScmd{
/// this code adds additional/user-defined colums. Now, we want to export he 
/// saved session table in a file.
 }
\BoSSSexe
\BoSSScmd{
var SessTab = WorkflowMgm.SessionTable;
 }
\BoSSSexe
\BoSSScmd{
SessTab = SessTab.ExtractColumns(AllCols.ToArray());
 }
\BoSSSexe
\BoSSScmd{
using System.IO;
 }
\BoSSSexe
\BoSSScmd{
/// Here, we define the filename
 }
\BoSSSexe
\BoSSScmd{
var now           = DateTime.Now;\newline 
SessTab.TableName = "SolverRuns--" + now.Year + "-" + now.Month + "-" + now.Day;\newline 
string docpath    = Path.Combine(CurrentDocDir, SessTab.TableName + ".json");
 }
\BoSSSexe
\BoSSScmd{
/// saving the session table as a file could also be done in our git reposatory
 }
\BoSSSexe
\BoSSScmd{
SessTab.SaveToFile(docpath);
 }
\BoSSSexe
\BoSSScmd{
///
 }
\BoSSSexe
