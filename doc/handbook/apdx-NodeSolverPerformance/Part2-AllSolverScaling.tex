


\BoSSSopen{apdx-NodeSolverPerformance/Part2-AllSolverScaling}
\graphicspath{{apdx-NodeSolverPerformance/Part2-AllSolverScaling.texbatch/}}



\BoSSScmd{
restart
 }
\BoSSSexeSilent
\BoSSScmd{
WorkflowMgm.Init("PoissonSolverPerformance");
 }
\BoSSSexe
\BoSSScmd{
using System.IO;\newline 
using BoSSS.Application.SipPoisson;
 }
\BoSSSexe
\BoSSScmd{
string File     = "SolverRuns--2018-2-6.json";\newline 
string FilePath = Path.Combine(CurrentDocDir, File);\newline 
var SessTab     = TableExtensions.LoadFromFile(FilePath);
 }
\BoSSSexe
\BoSSScmd{
//SessTab.Print();
 }
\BoSSSexe
\BoSSScmd{
// ====================\newline 
// Evaluation, plotting\newline 
// ====================
 }
\BoSSSexe
\BoSSScmd{
SessTab.GetColumnNames();
 }
\BoSSSexe
\BoSSScmd{
var SubTab\_1 = SessTab.ExtractColumns("DGdegree:T", "Grid:NoOfCells", "solver\_name", "DOFs", "minSolRunT", "maxSolRunT", "NoIter", "RegularTerminated", "Conv");
 }
\BoSSSexe
\BoSSScmd{
var SubTab\_2 = SubTab\_1.ExtractRows(delegate(int i, IDictionary<string,object> row) \{\newline 
\btab return (Convert.ToBoolean(row["RegularTerminated"]) == true)\newline 
\btab \btab   && (Convert.ToInt32(row["Conv"]) != 0);\newline 
\btab \});
 }
\BoSSSexe
\BoSSScmd{
var SubTab = SubTab\_2.ExtractColumns("DGdegree:T", "Grid:NoOfCells", "solver\_name", "DOFs", "minSolRunT", "maxSolRunT", "NoIter");
 }
\BoSSSexe
\BoSSScmd{
SubTab.Print();
 }
\BoSSSexe
\BoSSScmd{
 % 
 }
\BoSSSexe
\BoSSScmd{
var RuntimePlot = SubTab.ToPlot("DOFs", "minSolRunT", // column for x- and y\newline 
\btab \btab \btab \btab \btab \btab "solver\_name", "DGdegree:T"); // column for group
 }
\BoSSSexe
\BoSSScmd{
//SubTab.ToCSVFile("C:\textbackslash \textbackslash tmp\textbackslash \textbackslash table.csv");
 }
\BoSSSexe
\BoSSScmd{
//RuntimePlot.dataGroups.Select(grp => grp.Name);
 }
\BoSSSexe
\BoSSScmd{
/*\newline 
RuntimePlot.LogX   = true;\newline 
RuntimePlot.LogY   = true;\newline 
RuntimePlot.Xlabel = "Degrees of freedom";\newline 
RuntimePlot.Ylabel = "Runtime [sec]";\newline 
RuntimePlot.Title  = "Solver performance -- 3D Poisson problem";\newline 
 \newline 
RuntimePlot.ModPointType("name0", PointTypes.Circle); // Pardiso\newline 
RuntimePlot.ModPointType("name1", PointTypes.UpperTriangle); // Mumps\newline 
RuntimePlot.ModPointType("name2", PointTypes.Diamond); // CG\newline 
RuntimePlot.ModPointType("name5", PointTypes.Box); // Schwarz\newline 
RuntimePlot.ModPointType("name7", PointTypes.LowerTriangle); // LAPACK\newline 
RuntimePlot.ModDashType("T2", DashTypes.Dashed); // poly. order 2\newline 
RuntimePlot.ModDashType("T3", DashTypes.DotDashed); // poly. order 3\newline 
RuntimePlot.ModDashType("T4", DashTypes.DotDotDashed); // poly. order 4\newline 
RuntimePlot.ModDashType("T5", DashTypes.Solid); // poly. order 5\newline 
RuntimePlot.ModDashType("T6", DashTypes.Dotted); // poly. order 6\newline 
RuntimePlot.ModLineColor("T2", LineColors.Blue);\newline 
RuntimePlot.ModLineColor("T3", LineColors.Red);\newline 
RuntimePlot.ModLineColor("T4", LineColors.Magenta);\newline 
RuntimePlot.ModLineColor("T5", LineColors.Orange);\newline 
RuntimePlot.ModLineColor("T6", LineColors.Green);\newline 
 \newline 
foreach(var xyGroup in RuntimePlot.dataGroups) \{\newline 
\btab xyGroup.Format.LineWidth = 2;\newline 
\btab xyGroup.Format.PointSize = 1.0;\newline 
\}\newline 
 \newline 
foreach(Plot2Ddata.XYvalues graph in RuntimePlot.dataGroups) \{\newline 
 \newline 
\btab string ret = "";\newline 
\btab if(graph.Name.Contains("name0")) \{\newline 
\btab \btab ret += "Pardiso, ";\newline 
\btab \} else if(graph.Name.Contains("name1")) \{\newline 
\btab \btab ret += "Mumps, ";\newline 
\btab \} else if(graph.Name.Contains("name2")) \{\newline 
\btab \btab //Console.WriteLine("cg");\newline 
\btab \btab ret += "Cnj Gr, ";\newline 
\btab \} else if(graph.Name.Contains("name5")) \{\newline 
\btab \btab ret += "Add Swz, ";\newline 
\btab \} else if(graph.Name.Contains("name7")) \{\newline 
\btab \btab //Console.WriteLine("lapack");\newline 
\btab \btab ret += "LAPACK, ";\newline 
\btab \} \newline 
\btab if(graph.Name.Contains("T2")) \{\newline 
\btab \btab ret += "$ k = 2 $";\newline 
\btab \} else if(graph.Name.Contains("T3")) \{\newline 
\btab \btab ret += "$ k = 3 $";\newline 
\btab \} else if(graph.Name.Contains("T4")) \{\newline 
\btab \btab ret += "$ k = 4 $";\newline 
\btab \} else if(graph.Name.Contains("T5")) \{\newline 
\btab \btab ret += "$ k = 5 $";\newline 
\btab \} else if(graph.Name.Contains("T6")) \{\newline 
\btab \btab ret += "$ k = 6 $";\newline 
\btab \}\newline 
 \newline 
\btab graph.Name = ret;\newline 
\}\newline 
*/
 }
\BoSSSexe
\BoSSScmd{
 % 
 }
\BoSSSexe
\BoSSScmd{
 % 
 }
\BoSSSexe
\BoSSScmd{
 % 
 }
\BoSSSexe
\BoSSScmd{
//RuntimePlot.PlotNow();
 }
\BoSSSexe
\BoSSScmd{
//var cl = RuntimePlot.ToGnuplot().PlotCairolatex();\newline 
//cl.Preview(trimPage:true);
 }
\BoSSSexe
\BoSSScmd{
//cl.WriteMinimalCompileableExample("C:\textbackslash \textbackslash tmp\textbackslash \textbackslash mini.tex");
 }
\BoSSSexe
\BoSSScmd{
 % 
 }
\BoSSSexe
\BoSSScmd{
Plot2Ddata[,] multiplots = new Plot2Ddata[3,2];\newline 
int pDegree = 2;\newline 
for(int iRow = 0; iRow < 3; iRow++) \{\newline 
//var RuntimePlot = SubTab.ToPlot("DOFs", "minSolRunT", // column for x- and y\newline 
//                        "solver\_name", "DGdegree:T"); // column for group\newline 
for(int iCol = 0; iCol < 2; iCol++) \{\newline 
\btab if(pDegree > 6)\newline 
\btab \btab continue;\newline 
 \newline 
\btab multiplots[iRow,iCol] = SubTab.ToPlot("DOFs", "minSolRunT", // column for x- and y\newline 
\btab    delegate (int iTabRow, \newline 
\btab \btab \btab \btab  IDictionary<string, object> Row, \newline 
\btab \btab \btab \btab  out string Nmn, \newline 
\btab \btab \btab \btab  out PlotFormat Fmt) \{ \newline 
\btab \btab    // - - - - - - - - - - - - - - - - - - - - - - - - \newline 
\btab \btab    // PlotRowSelector:\newline 
\btab \btab    // selects, which table row goes to which graph,\newline 
\btab \btab    // and the respective color  \newline 
\btab \btab    // - - - - - - - - - - - - - - - - - - - - - - - - \newline 
\btab \btab    int k = Convert.ToInt32(Row["DGdegree:T"]);\newline 
\btab \btab    if(k != pDegree) \{\newline 
\btab \btab \btab \btab // degree does not match -> not in this plot\newline 
\btab \btab \btab \btab Nmn = null;\newline 
\btab \btab \btab \btab Fmt = null;\newline 
\btab \btab \btab \btab return;\newline 
\btab \btab    \}\newline 
 \newline 
\btab \btab    SolverCodes solver\_name = (SolverCodes) Convert.ToInt32(Row["solver\_name"]);\newline 
\btab \btab    Fmt                     = new PlotFormat();\newline 
\btab \btab    switch(solver\_name) \{\newline 
\btab \btab    case SolverCodes.classic\_pardiso: \newline 
\btab \btab \btab    Nmn           = "Pardiso"; \newline 
\btab \btab \btab    Fmt.PointType = PointTypes.OpenCircle;\newline 
\btab \btab \btab    Fmt.DashType  = DashTypes.Dotted;\newline 
\btab \btab \btab    break;\newline 
\btab \btab    case SolverCodes.classic\_mumps: \newline 
\btab \btab \btab    Nmn           = "Mumps"; \newline 
\btab \btab \btab    Fmt.PointType = PointTypes.OpenUpperTriangle;\newline 
\btab \btab \btab    Fmt.DashType  = DashTypes.Dotted;\newline 
\btab \btab \btab    break;\newline 
\btab \btab    case SolverCodes.classic\_cg: \newline 
\btab \btab \btab    Nmn           = "CG"; \newline 
\btab \btab \btab    Fmt.PointType = PointTypes.Asterisk;\newline 
\btab \btab \btab    break;\newline 
\btab \btab    case SolverCodes.exp\_direct: \newline 
\btab \btab \btab    Fmt.PointType = PointTypes.OpenBox;\newline 
\btab \btab \btab    Nmn           = "Pardiso w Blk PC"; \newline 
\btab \btab \btab    break;\newline 
\btab \btab    case SolverCodes.exp\_softpcg\_schwarz: \newline 
\btab \btab \btab    Nmn           = "Add Swz"; \newline 
\btab \btab \btab    Fmt.PointType = PointTypes.LowerTriangle;\newline 
\btab \btab \btab    break;\newline 
\btab \btab    case SolverCodes.exp\_softpcg\_schwarz\_directcoarse : \newline 
\btab \btab \btab    Nmn           = "Add Swz w Coarse";\newline 
\btab \btab \btab    Fmt.PointType = PointTypes.UpperTriangle;\newline 
\btab \btab \btab    break;\newline 
\btab \btab    case SolverCodes.exp\_direct\_lapack: \newline 
\btab \btab \btab    Nmn           = "LAPACK"; \newline 
\btab \btab \btab    Fmt.PointType = PointTypes.Plus;\newline 
\btab \btab \btab    break;\newline 
\btab \btab    default:\newline 
\btab \btab \btab    Console.WriteLine("unknown: " + solver\_name); \newline 
\btab \btab \btab    Nmn = "unknown"; \newline 
\btab \btab \btab    break;\newline 
\btab \btab    \} \newline 
\btab \btab    //Console.WriteLine("name is: " + solver\_name); \newline 
\btab \btab    Fmt.PointSize = 0.5;\newline 
\btab \btab    Fmt.Style     = Styles.LinesPoints;\newline 
\btab \btab    Fmt.LineColor = LineColors.Black;\newline 
\btab    \});\newline 
 \newline 
\btab string Title = string.Format("$k = \{0\}$", pDegree);\newline 
\btab multiplots[iRow,iCol].Ylabel = Title;\newline 
\btab multiplots[iRow,iCol].LogX = true;\newline 
\btab multiplots[iRow,iCol].LogY = true;\newline 
 \newline 
\btab multiplots[iRow,iCol].YrangeMin = 1e-3;\newline 
\btab multiplots[iRow,iCol].YrangeMax = 1e+4;\newline 
\btab multiplots[iRow,iCol].XrangeMin = 1e2;\newline 
\btab multiplots[iRow,iCol].XrangeMax = 1e7;\newline 
\btab multiplots[iRow,iCol].Y2rangeMin = 1e-3;\newline 
\btab multiplots[iRow,iCol].Y2rangeMax = 1e+4;\newline 
\btab multiplots[iRow,iCol].X2rangeMin = 1e2;\newline 
\btab multiplots[iRow,iCol].X2rangeMax = 1e7;\newline 
 \newline 
\btab if(iRow == 2) \{\newline 
\btab \btab multiplots[iRow,iCol].ShowLegend = true;\newline 
\btab \btab multiplots[iRow,iCol].LegendPosition = new double[] \{1e12,1e4\};\newline 
\btab \} else \{\newline 
\btab \btab multiplots[iRow,iCol].ShowLegend = false;\newline 
\btab \}\newline 
 \newline 
\btab if(iCol == 1) \{\newline 
\btab \btab multiplots[iRow,iCol].SwapYaxes();\newline 
\btab \btab multiplots[iRow,iCol].lmargin = 1;\newline 
\btab \btab multiplots[iRow,iCol].rmargin = 10;\newline 
\btab \} else \{\newline 
\btab \btab multiplots[iRow,iCol].lmargin = 10;\newline 
\btab \btab multiplots[iRow,iCol].rmargin = 1;\newline 
\btab \}\newline 
\btab multiplots[iRow,iCol].tmargin = 1;\newline 
\btab multiplots[iRow,iCol].bmargin = 1;\newline 
\btab if(iRow == 0) \{\newline 
\btab \btab multiplots[iRow,iCol].SwapXaxes();\newline 
\btab \}\newline 
\btab if(iRow == 1 && iCol == 0) \{  \newline 
\btab \btab multiplots[iRow,iCol].ShowXtics = false;\newline 
\btab \}\newline 
\btab pDegree++;\newline 
\}                        \newline 
\}\newline 
//multiplots.PlotNow();\newline 
multiplots.AddDummyPlotsForLegend(2,0);
 }
\BoSSSexe
\BoSSScmd{
 % 
 }
\BoSSSexe
\BoSSScmd{
 % 
 }
\BoSSSexe
\BoSSScmd{
 % 
 }
\BoSSSexe
\BoSSScmd{
var CL = multiplots.ToGnuplot().PlotCairolatex(xSize:17, ySize:17);
 }
\BoSSSexe
\BoSSScmd{
//CL.SaveTo(Path.Combine(CurrentDocDir, "ConstCoeffPoissonScaling.tex"));
 }
\BoSSSexe
\BoSSScmd{
CL.PlotNow();
 }
\BoSSSexe
\BoSSScmd{
//cmt
 }
\BoSSSexe
