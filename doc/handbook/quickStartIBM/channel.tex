% !TeX spellcheck = en_EN
%\documentclass[12pt,a4paper]{article}
%\usepackage{import}
%\subimport{../}{uebung.tex}
%
%%\toggletrue{showSolution}
%
%
%\title{\thetitle{}\\Exercise 4}
%\date{\WeeksAfter{3}}
%
%
%\begin{document}


\BoSSSopen{quickStartIBM/channel}
\graphicspath{{quickStartIBM/channel.texbatch/}}


\BoSSScmd{
/// \section*{What's new}
///
///\begin{itemize}
/// \item running a simulation with the incompressible Navier-Stokes solver
/// \item how to use the immersed boundary method
/// \item post-processing, i.e. plotting and checking physical values
///\end{itemize}
///
///\section*{Prerequisites}
///
///\begin{itemize}
///\item the \BoSSS{} framework
///\item a visualization tool, e.g Paraview or VisIt
///\item the knowledge of how to setup a database for \BoSSS{}
///\end{itemize}
/// This tutorial will explain the basic features of the incompressible 
/// Navier-Stokes solver in the \BoSSS{} framework. 
/// First, the simple testcase of a 2D channel flow will be explained. After 
/// that, there will be a short part about the immersed boundary feature of our incompressible flow solver. 
/// Therefore the flow around a cylinder will be investigated using the immersed boundary method.
///
/// Note that \BoSSS{}, at the present time contains no stand-alone single-phase  
/// solver that is fully recomended - although there are some legacy solvers, e.g. SIMPLE.
/// Instead, the two-phase-solver with immersed boundary is used, where the two-phase option ist deactivated.
///
///\section*{Problem statement}
///
///The flow is described by the unsteady Navier-Stokes equations in the fluid region
///\begin{subequations}
/// \label{NavierStokes}
/// \begin{equation}
/// \label{NavierStokes_mom}
/// \rho_f\Big(\frac{\partial \vec{u}}{\partial t}+ \vec{u} \cdot \nabla \vec{u}\Big) +\nabla p - \mu_f \Delta \vec{u} = \vec{f} \nonumber
/// \end{equation}  
/// and the continuity equation
/// \begin{equation}
/// \label{NavierStokes_conti}
/// \nabla \cdot \vec{u} = 0 \quad \forall\ t \in (0,T)\quad \textrm{in}\ \Omega \nonumber
/// \end{equation}  
/// \end{subequations}
/// \newline
/// In the equations above $\vec{u}$ is the velocity vector and $p$ the pressure. The fluid density is denoted by $\rho_f$, while $\mu_f=\rho_f \cdot \nu_f$ is the dynamic viscosity of the fluid.
///
///\section{Channel}
/// \label{sec:Channel}
///\subsection{Interactive mode}
 }
\BoSSSexeSilent
\BoSSScmd{
restart;
 }
\BoSSSexeSilent
\BoSSScmd{
/// For the interactive mode of the solver in \BoSSSpad{} we have to load several dependencies first:
using System.Diagnostics;\newline 
using BoSSS.Foundation.Grid.RefElements;\newline 
using BoSSS.Application.XNSE\_Solver;
 }
\BoSSSexe
\BoSSScmd{
/// Now, a new database has to be created.
/// In this worksheet, we use a temporary database which will be deleted
/// after the worksheet has been executed.
/// For your calculation, you might consider some non-temporary alternative,
/// cf. \code{OpenOrCreateDatabase} or \code{OpenOrCreateDefaultDatabase}:
var myDb = CreateTempDatabase();
 }
\BoSSSexe
\BoSSScmd{
/// Create a new control object for setting up the simulation:
var c = new XNSE\_Control();
 }
\BoSSSexe
\BoSSScmd{
/// \subsection {How to define/change input data}
 }
\BoSSSexe
\BoSSScmd{
/// First, the DG polynomial degree is set:
/// (degree 2 for velocity and 1 for pressure).
c.SetDGdegree(2);
 }
\BoSSSexe
\BoSSScmd{
/// Domain and Grid variables
double xMin        = -2; \newline 
double xMax        = 20;\newline 
double yMin        = -2; \newline 
double yMax        = 2.1;\newline 
int numberOfCellsX = 44; \newline 
int numberOfCellsY = 8;
 }
\BoSSSexe
\BoSSScmd{
/// Basic database options 
c.SetDatabase(myDb);\newline 
c.savetodb   = true;\newline 
c.saveperiod = 1;
 }
\BoSSSexe
\BoSSScmd{
/// Setting some variables for database saving. Here it is also possible to define tags which can be helpful for finding a particular simulation in the \BoSSS{} database
string sessionName   = "dt = 1E20\_" + numberOfCellsX + "x" + numberOfCellsY + "\_k2";\newline 
c.SessionName        = sessionName;\newline 
c.ProjectDescription = "Incompressible Solver Examples";\newline 
c.Tags.Add("numberOfCellsX\_" + numberOfCellsX);\newline 
c.Tags.Add("numberOfCellsY\_" + numberOfCellsY);\newline 
c.Tags.Add("k2");
 }
\BoSSSexe
\BoSSScmd{
/// The grid is generated using the previously defined parameters.
c.GridFunc       = null; \newline 
var xNodes       = GenericBlas.Linspace(xMin, xMax , numberOfCellsX);\newline 
var yNodes       = GenericBlas.Linspace(yMin, yMax, numberOfCellsY);\newline 
GridCommons grid = Grid2D.Cartesian2DGrid(xNodes, yNodes, CellType.Square\_Linear, false);
 }
\BoSSSexe
\BoSSScmd{
/// Set the geometric location of boundary conditions by edge tags;
/// Edges that get assigned "0" are "inner edges".
grid.DefineEdgeTags(delegate (double[] X) \{\newline 
\btab if (Math.Abs(X[1] - (-2)) <= 1.0e-8)\newline 
\btab \btab return "wall"; // wall at y = -2\newline 
\btab if (Math.Abs(X[1] - (+2.1 )) <= 1.0e-8)\newline 
\btab \btab return "wall"; // wall at y = +2.1\newline 
\btab if (Math.Abs(X[0] - (-2)) <= 1.0e-8)\newline 
\btab \btab return "Velocity\_Inlet"; // velocity inlet at x = -2\newline 
\btab if (Math.Abs(X[0] - (+20.0)) <= 1.0e-8)\newline 
\btab \btab return "Pressure\_Outlet"; // pressure outlet at x = +20\newline 
\btab throw new ArgumentException("unexpected domain boundary"); \newline 
\});
 }
\BoSSSexe
\BoSSScmd{
/// Save the grid in the database so that the simulation can use it
grid;\newline 
myDb.SaveGrid(ref grid);\newline 
c.SetGrid(grid);
 }
\BoSSSexe
\BoSSScmd{
/// Specification of boundary conditions with a parabolic velocity profile for the inlet
c.BoundaryValues.Clear();\newline 
c.AddBoundaryValue("Velocity\_Inlet", "VelocityX", \newline 
\btab   (X => (4.1 * 1.5 * (X[1] + 2) * (4.1 - (X[1] + 2)) / (4.1 * 4.1))));
 }
\BoSSSexe
\BoSSScmd{
/// Fluid Properties
/// Note: The characteristic length and fluid density are choosen to one. 
/// Therefore, the viscosity can be defined by \code{1.0/reynolds}.
double reynolds            = 20;\newline 
c.PhysicalParameters.rho\_A = 1;\newline 
c.PhysicalParameters.mu\_A  = 1.0/reynolds;
 }
\BoSSSexe
\BoSSScmd{
/// Bool parameter whether the Navier-Stokes or Stokes equations
/// should be solved
c.PhysicalParameters.IncludeConvection = true;
 }
\BoSSSexe
\BoSSScmd{
/// Initial Values are set to 0; Note that the following lines are only for 
/// demonstration -- if no initial value is specified, 0 is set automatically.
c.InitialValues.Clear();\newline 
c.InitialValues.Add("VelocityX", new Formula("X => 0.0", false));\newline 
c.InitialValues.Add("VelocityY", new Formula("X => 0.0", false));\newline 
c.InitialValues.Add("Pressure", new Formula("X => 0.0", false));
 }
\BoSSSexe
\BoSSScmd{
/// Timestepping properties:
/// Most solvers in \BoSSS{} simulate transient equations. Configuring
/// a steady simulation confiures one very large timestep.
c.TimesteppingMode = AppControl.\_TimesteppingMode.Steady;
 }
\BoSSSexe
\BoSSScmd{
/// \subsection{Run a simulation}
 }
\BoSSSexe
\BoSSScmd{
/// The solver can be run inline (i.e. within the \BoSSSpad process) by 
/// executing the \code{Run} method on the control objece \code{c}.
/// An inline run will block \BoSSSpad until the solver exits.
c.Run();
 }
\BoSSSexe
\BoSSScmd{
LastError
 }
\BoSSSexe
\BoSSScmd{
/// \subsection{Postprocessing}
 }
\BoSSSexe
\BoSSScmd{
/// Plot the current session
//myDb.Sessions.First().Export().Do();
 }
\BoSSSexe
\BoSSScmd{
/// Open the ExportDirectory to view the *.plt files
//myDb.Sessions.First().OpenExportDirectory();
 }
\BoSSSexe
\BoSSScmd{
/// Some information like the console output or a log containing various physical values can be found in the session directory
//myDb.Sessions.First().OpenSessionDirectory();
 }
\BoSSSexe
\BoSSScmd{
/// Delete database
//DatabaseUtils.DeleteDatabase(myDb.Path);
 }
\BoSSSexe
\BoSSScmd{
/// \subsection{Immersed boundary method}
/// It is also possible to use the immersed boundary feature of our incompressible Navier-Stokes Solver.
/// For this example we have to change two parts of the code: First, for a good result, we have to refine 
/// the grid at the position of the cylinder.
 }
\BoSSSexe
\BoSSScmd{
/// x-Direction (using also hyperbolic tangential distribution)
//var \_xNodes1 = Grid1D.TanhSpacing(-2, -1, 10, 0.5, false); \newline 
//\_xNodes1     = \_xNodes1.GetSubVector(0, (\_xNodes1.Length - 1));\newline 
//var \_xNodes2 = GenericBlas.Linspace(-1, 2, 35); \newline 
//\_xNodes2     = \_xNodes2.GetSubVector(0, (\_xNodes2.Length - 1));\newline 
//var \_xNodes3 = Grid1D.TanhSpacing(2, 20, 60 , 1.5, true);  \newline 
//var xNodes   = ArrayTools.Cat(\_xNodes1, \_xNodes2, \_xNodes3);
 }
\BoSSSexe
\BoSSScmd{
/// y-Direction
//var \_yNodes1 = Grid1D.TanhSpacing(-2, -1, 7, 0.9, false); \newline 
//\_yNodes1     = \_yNodes1.GetSubVector(0, (\_yNodes1.Length - 1));\newline 
//var \_yNodes2 = GenericBlas.Linspace(-1, 1, 25); \newline 
//\_yNodes2     = \_yNodes2.GetSubVector(0, (\_yNodes2.Length - 1));\newline 
//var \_yNodes3 = Grid1D.TanhSpacing(1, 2.1, 7, 1.1, true);  \newline 
//var yNodes   = ArrayTools.Cat(\_yNodes1, \_yNodes2, \_yNodes3);
 }
\BoSSSexe
\BoSSScmd{
/// Furthermore, the cylinder immersing the fluid should be described
/// by using the zero contour of a level set function. The radius of the cylinder is set to 0.5.
 }
\BoSSSexe
\BoSSScmd{
//c.InitialValues.Add("Phi", new Formula("X => -(X[0]).Pow2() + -(X[1]).Pow2() + 0.25", false));
 }
\BoSSSexe
\BoSSScmd{
/// Example control files for both,
/// the channel and the flow around a cylinder can be found in the ControlExample directory. As soon as we run 
/// the simulation again we can take a look at the plots and the PhysicalData file in the session directory. 
/// There we can find for example lift and drag forces acting on the cylinder.
 }
\BoSSSexe
