
\BoSSSopen{tutorial11-Stokes/StokesEq}
\graphicspath{{tutorial11-Stokes/StokesEq.texbatch/}}


\BoSSScmd{
/// \label{chap:stokesEquation}
/// \section*{What's new?}
/// \label{sec:stokesEquation_new}
/// \begin{itemize}
///     \item {implementation of the incompressible, steady Stokes equation}
///     \item advanced: {implementation of the Stokes flow behind a grid as an application}
/// \end{itemize}
///
///\section*{Prerequisites}
///\label{sec:stokesEquation_prereq}
///\begin{itemize}
/// \item implementation of numerical fluxes, chapter \ref{NumFlux} 
/// \item spatial operator, chapter \ref{SpatialOperator}
/// \item implementation of the gradient operator, chapter \ref{sec:PoissonAsASystem}
/// \item implementation of the SIP operator, chapter \ref{sec:SIP}
///\end{itemize}
    /// %==========================================
    /// \section{Problem statement}
    /// %==========================================
/// The Stokes-equation is given as
/// \begin{alignat*}{3}
/// -\frac{1}{\reynolds} \Delta \vec{u}
///     & + \nabla \psi
///        & \ & = \vec{g}_\domain
/// \\ 
/// \operatorname{div} (\vec{u})
///     &
///        & \ & = 0
/// \end{alignat*}
/// We consider two types of boundary conditions for the Stokes equation,
/// Dirichlet (on $\Gamma_D$) and Neumman (on $\Gamma_N$). Those are defined as
/// \begin{alignat*}{3}
///         \vec{u} & =\vec{u}_D 
///         & \ & \text{ on } \Gamma_D\ \text{ (Dirichlet)}, \\
///         %
///         \left( -\frac{1}{\reynolds}\ \nabla  \vec{u} +  \one_p \psi \right) 
///                \vec{n}_{\partial \domain}        & = 0 
///         & \ & \text{ on } \Gamma_N\ \text{ (Neumann) } .
/// \end{alignat*}
/// %==========================================
/// \section{Solution within the BoSSS framework}
/// %==========================================
 }
\BoSSSexeSilent
\BoSSScmd{
restart
 }
\BoSSSexeSilent
\BoSSScmd{
using ilPSP.LinSolvers;\newline 
using BoSSS.Solution.Tecplot;\newline 
using ilPSP.Connectors.Matlab;
 }
\BoSSSexe
\BoSSScmdSilent{
/// BoSSScmdSilent BoSSSexeSilent
using NUnit.Framework;
 }
\BoSSSexeSilent
\BoSSScmd{
/// To implement at which point of the boundary which condition is valid, we
/// define a global function 
Func<double[],bool> IsDirichletBndy = null;\newline 
/// which defines a mapping 
/// \[
///    \vec{x} \mapsto \{ \code{true}, \code{false} \},
/// \]
/// where \code{true} actually indicates a Dirichlet boundary.
/// Since this function is defined as a global delegate, it can be altered 
/// later on. In the same manner, the function
Func<double[],double[]> UDiri = null;\newline 
/// defines the Dirichlet-value for the velocity at the boundary.
 }
\BoSSSexe
\BoSSScmd{
/// %==========================================
/// \subsection{Velocity divergence and pressure gradient}
/// %==========================================
 }
\BoSSSexe
\BoSSScmd{
/// At first, we implement the velocity divergence, i.e. 
/// the continuity equation. We use the strong form, i.e.
/// \[
///    b(\vec{u},v) = 
///    \oint_{\Gamma \backslash \Gamma_D} 
///           \mean{v} \jump{\vec{u}} \cdot \vec{n}_\Gamma 
///    \dA 
///    -
///    \int_{\domain} \operatorname{div}(\vec{u}) v \dV.
/// \]
public class Divergence : \newline 
\btab \btab BoSSS.Foundation.IEdgeForm, // edge integrals\newline 
\btab \btab BoSSS.Foundation.IVolumeForm     // volume integrals\newline 
\{\newline 
    /// The parameter list for the divergence is empty:
\btab public IList<string> ParameterOrdering \{ \newline 
\btab \btab get \{ return null; \} \newline 
\btab \}\newline 
 \newline 
    /// We have a vector argument variable, 
    /// the velocity $ [ u, v ] = \vec{u} $
    /// (our trial function):
\btab public IList<String> ArgumentOrdering \{ \newline 
\btab \btab get \{ return new string[] \{ "u", "v" \}; \} \newline 
\btab \}\newline 
 \newline 
\btab public TermActivationFlags VolTerms \{\newline 
\btab \btab get \{\newline 
\btab \btab \btab return TermActivationFlags.AllOn;\newline 
\btab \btab \}\newline 
\btab \}\newline 
 \newline 
\btab public TermActivationFlags InnerEdgeTerms \{\newline 
\btab \btab get \{\newline 
\btab \btab \btab return TermActivationFlags.AllOn; \newline 
\btab \btab \}\newline 
\btab \}\newline 
 \newline 
\btab public TermActivationFlags BoundaryEdgeTerms \{\newline 
\btab    get \{\newline 
\btab \btab    return TermActivationFlags.AllOn;\newline 
\btab \btab \}\newline 
\btab \}\newline 
 \newline 
    /// In the volume part, the integrand is $\operatorname{div}(\vec{u}) v$:
\btab public double VolumeForm(ref CommonParamsVol cpv, \newline 
\btab \btab double[] U, double[,] GradU, \newline 
\btab \btab double V, double[] GradV) \{\newline 
\btab \btab double Acc = 0;\newline 
\btab \btab for(int d = 0; d < cpv.D; d++) \{\newline 
\btab \btab \btab Acc -= GradU[d,d]*V;\newline 
\btab \btab \}\newline 
\btab \btab return Acc;\newline 
\btab \}\newline 
 \newline 
    /// On interior cell boundaries, we use a \emph{velocity penalty},
    /// $\mean{v} \jump{\vec{u}} \cdot \vec{n}_\Gamma$:
\btab public double InnerEdgeForm(ref CommonParams inp, \newline 
\btab \btab double[] U\_IN, double[] U\_OT, double[,] GradU\_IN, double[,] GradU\_OT, \newline 
\btab \btab double V\_IN, double V\_OT, double[] GradV\_IN, double[] GradV\_OT) \{\newline 
 \newline 
\btab \btab double Acc = 0;\newline 
\btab \btab for(int d = 0; d < inp.D; d++) \{\newline 
\btab \btab \btab Acc += 0.5*(V\_IN + V\_OT)*(U\_IN[d] - U\_OT[d])*inp.Normale[d];\newline 
\btab \btab \}\newline 
\btab \btab return Acc;\newline 
\btab \}\newline 
 \newline 
    /// On the domain boundary, we have to distinguish between 
    /// Dirichlet- and Neumann-boundary conditions; the function
    /// \code{uDiri} defines which of the two actually applies:
\btab public double BoundaryEdgeForm(ref CommonParamsBnd inp, \newline 
\btab \btab double[] U\_IN, double[,] GradU\_IN, double V\_IN, double[] GradV\_OT) \{\newline 
 \newline 
\btab \btab double Acc = 0;\newline 
 \newline 
\btab \btab if(!IsDirichletBndy(inp.X)) \{\newline 
\btab \btab \btab // On the Neumann boundary, we do not know an outer value for the\newline 
\btab \btab \btab // velocity, so there is no penalization at all:\newline 
\btab \btab \btab Acc = 0;    \newline 
\btab \btab \} else \{\newline 
\btab \btab \btab // On the Dirichlet boundary, the outer value for the velocity\newline 
\btab \btab \btab // is given by the function/delegate 'UDiri':\newline 
\btab \btab \btab double[] UD = UDiri(inp.X);\newline 
\btab \btab \btab for(int d = 0; d < inp.D; d++) \{\newline 
\btab \btab \btab \btab Acc += (U\_IN[d] - UD[d])*inp.Normale[d]*V\_IN;\newline 
\btab \btab \btab \}\newline 
\btab \btab \}\newline 
\btab \btab return Acc;\newline 
\btab \}\newline 
\}
 }
\BoSSSexe
\BoSSScmd{
/// %==================================
/// \subsection{The gradient-operator}
/// %==================================
/// We use the variational formulation of the gradient operator, as it is explained in section \ref{subsec:poissonSystem_gradientOp}.
class Gradient\_d :\newline 
\btab \btab BoSSS.Foundation.IEdgeForm, // edge integrals\newline 
\btab \btab BoSSS.Foundation.IVolumeForm     // volume integrals\newline 
\{\newline 
\btab public Gradient\_d(int \_d) \{\newline 
\btab \btab this.d = \_d;\newline 
\btab \}\newline 
 \newline 
    /// The component index of the gradient:
\btab int d;\newline 
 \newline 
    /// As usual, we do not use parameters:
\btab public IList<string> ParameterOrdering \{ \newline 
\btab \btab get \{ return null; \} \newline 
\btab \}\newline 
 \newline 
    /// We have one argument, the pressure $\psi$:
\btab public IList<String> ArgumentOrdering \{ \newline 
\btab \btab get \{ return new string[] \{ "psi" \}; \} \newline 
\btab \}\newline 
 \newline 
\btab public TermActivationFlags VolTerms \{\newline 
\btab \btab get \{ return TermActivationFlags.AllOn; \}\newline 
\btab \}\newline 
 \newline 
\btab public TermActivationFlags InnerEdgeTerms \{\newline 
\btab \btab get \{ return (TermActivationFlags.AllOn); \}\newline 
\btab \}\newline 
 \newline 
\btab public TermActivationFlags BoundaryEdgeTerms \{\newline 
\btab    get \{ return TermActivationFlags.AllOn; \}\newline 
\btab \}\newline 
 \newline 
    /// The volume integrand, for a vector-valued test-function $\vec{v}$
    /// would be $-\operatorname{div}{\vec{v}} \psi$. Our test function $v$
    /// is scalar-valued, so e.g. for $\code{d} = 0$ we have
    /// $\vec{v} = (v,0)$. In this case, our volume integrand reduces as 
    /// $-\operatorname{div}{\vec{v}} \psi = -\partial_x v \psi$:
\btab public double VolumeForm(ref CommonParamsVol cpv, \newline 
\btab \btab    double[] Psi, double[,] GradPsi, \newline 
\btab \btab    double V, double[] GradV) \{\newline 
 \newline 
\btab \btab double Acc = 0;\newline 
\btab \btab Acc -= Psi[0]*GradV[d];\newline 
\btab \btab return Acc;\newline 
\btab \}        \newline 
 \newline 
    /// On interior cell edges, we simply use a central-difference flux.
    /// Again, we consider a scalar test function, so we have
    /// $ \jump{\psi} \vec{v} \cdot \vec{n} = \jump{\psi} v n_d $,
    /// where $n_d$ is the $d$--th component of $\vec{n}$:
\btab public double InnerEdgeForm(ref CommonParams inp, \newline 
\btab \btab double[] Psi\_IN, double[] Psi\_OT, \newline 
\btab \btab double[,] GradPsi\_IN, double[,] GradPsi\_OT, \newline 
\btab \btab double V\_IN, double V\_OT, double[] GradV\_IN, double[] GradV\_OT) \{\newline 
 \newline 
\btab \btab double Acc = 0;\newline 
\btab \btab Acc += 0.5*(Psi\_IN[0] + Psi\_OT[0])*inp.Normale[this.d]*(V\_IN - V\_OT);\newline 
\btab \btab return Acc;  \newline 
\btab  \}\newline 
 \newline 
\btab public double BoundaryEdgeForm(ref CommonParamsBnd inp, \newline 
\btab \btab double[] Psi\_IN, double[,] GradPsi\_IN, double V\_IN, double[] GradV\_OT) \{\newline 
 \newline 
\btab \btab double Acc = 0;\newline 
\btab \btab if(!IsDirichletBndy(inp.X)) \{\newline 
\btab \btab \btab // On the Neumann boundary, we want the total stress to be zero,\newline 
\btab \btab \btab // so there is no contribution from the pressure:\newline 
\btab \btab \btab Acc = 0;\newline 
\btab \btab \} else \{\newline 
\btab \btab \btab // On the Dirichlet boundary, we do not know an outer value for \newline 
\btab \btab \btab // the pressure, so we have to take the inner value:\newline 
\btab \btab \btab Acc += Psi\_IN[0]*inp.Normale[this.d]*V\_IN;\newline 
\btab \btab \}\newline 
\btab \btab return Acc;              \newline 
\btab \}\newline 
\}
 }
\BoSSSexe
\BoSSScmd{
/// %==============================================================
/// \subsection{Tests on pressure gradient and velocity divergence}
/// %==============================================================
 }
\BoSSSexe
\BoSSScmd{
/// If our implementation is correct, we created a discretization of 
/// \[
///  \left[ \begin{array}{cc}
///     0                 & \nabla \\
///   -\operatorname{div} & 0      \\
///  \end{array} \right]
/// \]
/// so the matrix should have the form 
/// \[
///  \left[ \begin{array}{cc}
///     0     & B      \\
///     B^T   & 0      \\
///  \end{array} \right]
///  =: M,
/// \]
/// i.e. $M$ should be symmetric.
/// We are testing this using a channel flow configuration:
/// \begin{align*}
///     \Omega    := & (0,10) \times (-1,1) , \\
///     \Gamma_N  := & \{ (x,y) | \ x = 10 \} , \\
///     \Gamma_D  := & \partial \Omega \setminus \Gamma_D , \\
///     \vec{u}_D := & (1 - y^2, 0), 
/// \end{align*}
/// using an equidistant grid.
 }
\BoSSSexe
\BoSSScmd{
/// We create a grid, a DG basis for velocity and pressure 
/// and a variable mapping:
var xNodesChannel = GenericBlas.Linspace(0,10,31);// 30 cells in x-direction\newline 
var yNodesChannel = GenericBlas.Linspace(-1,1,7); // 6 cells in y-direction\newline 
var grdChannel    = Grid2D.Cartesian2DGrid(xNodesChannel,yNodesChannel);\newline 
var grdDatChannel = new GridData(grdChannel);\newline 
var VelBChannel   = new Basis(grdDatChannel, 2);  // velocity basis\newline 
var PsiBChannel   = new Basis(grdDatChannel, 1);  // pressure basis \newline 
var varMapChannel = new UnsetteledCoordinateMapping(\newline 
\btab \btab \btab \btab \btab    VelBChannel,VelBChannel,PsiBChannel); // variable mapping
 }
\BoSSSexe
\BoSSScmd{
/// We specify the boundary conditions as delegates:
Func<double[],bool> IsDirichletBndy\_Channel \newline 
\btab \btab = (X => Math.Abs(X[0] - 10) > 1.0e-10); // its Dirichlet, if x != 10\newline 
Func<double[],double[]> UDiri\_Channel \newline 
\btab \btab = (X => new double[2] \{ 1.0 - X[1]*X[1], 0\});
 }
\BoSSSexe
\BoSSScmd{
/// Let's create the operator which contains only the pressure gradient
/// and velocity divergence:
SpatialOperator GradDiv = new SpatialOperator(3,3, // 3 vars. in dom. & codom.\newline 
\btab \btab \btab \btab \btab \btab    QuadOrderFunc.Linear(), // linear operator\newline 
\btab \btab \btab \btab \btab \btab    "u", "v", "psi",  // names of domain variables\newline 
\btab \btab \btab \btab \btab \btab    "mom\_x", "mom\_y", "conti"); // names of codom. vars\newline 
GradDiv.EquationComponents["mom\_x"].Add(new Gradient\_d(0)); \newline 
GradDiv.EquationComponents["mom\_y"].Add(new Gradient\_d(1)); \newline 
GradDiv.EquationComponents["conti"].Add(new Divergence());\newline 
GradDiv.Commit();
 }
\BoSSSexe
\BoSSScmd{
/// We create the matrix of the \code{GradDiv}-operator for 
/// the channel configuration. Before that, we have to set values for the 
/// global \code{IsDirichletBndy} and \code{UDiri}-variables.
IsDirichletBndy           = IsDirichletBndy\_Channel;\newline 
UDiri                     = UDiri\_Channel;\newline 
var GradDivMatrix\_Channel = GradDiv.ComputeMatrix(varMapChannel,\newline 
\btab \btab \btab \btab \btab \btab \btab \btab \btab \btab \btab \btab   null,\newline 
\btab \btab \btab \btab \btab \btab \btab \btab \btab \btab \btab \btab   varMapChannel);
 }
\BoSSSexe
\BoSSScmd{
///Finally, we can test the symmetry of the matrix:
var ErrMtx = GradDivMatrix\_Channel - GradDivMatrix\_Channel.Transpose();\newline 
ErrMtx.InfNorm();
 }
\BoSSSexe
\BoSSScmdSilent{
/// NUnit test (few random tests) BoSSScmdSilent
Assert.LessOrEqual(ErrMtx.InfNorm(), 1e-12);
 }
\BoSSSexe
\BoSSScmd{
/// %====================================================================
/// \subsection{Adding the viscous operator, forming the Stokes operator}
/// %====================================================================
 }
\BoSSSexe
\BoSSScmd{
/// We use the SIP-operator from chapter \ref{sec:SIP} to model the viscous terms:
public class Viscous : \newline 
\btab \btab IEdgeForm, // edge integrals\newline 
\btab \btab IVolumeForm     // volume integrals\newline 
\{\newline 
    /// The velocity component:
\btab int d;\newline 
 \newline 
\btab public Viscous(int \_d) \{\newline 
\btab \btab this.d = \_d;    \newline 
\btab \}\newline 
 \newline 
    /// We implement Reynolds number and the polynomial degree, 
    /// as well as the cell-wise length scales (required for 
    /// the computation of the penalty factor) as global, static variables.
\btab public static double Re;\newline 
\btab public static int PolynomialDegree;\newline 
\btab public static MultidimensionalArray cj;\newline 
 \newline 
    /// We do not use parameters:
\btab public IList<string> ParameterOrdering \{ \newline 
\btab \btab get \{ return new string[0]; \} \newline 
\btab \}\newline 
 \newline 
    /// Depending on \code{d}, the argument variable
    /// should be either $u$ or $v$:
\btab public IList<String> ArgumentOrdering \{ \newline 
\btab \btab get \{ \newline 
\btab \btab \btab switch(d) \{\newline 
\btab \btab \btab \btab case 0  : return new string[] \{ "u" \}; \newline 
\btab \btab \btab \btab case 1  : return new string[] \{ "v" \}; \newline 
\btab \btab \btab \btab default : throw new Exception();\newline 
\btab \btab \btab \}\newline 
\btab \btab \} \newline 
\btab \}\newline 
 \newline 
    /// The \code{TermActivationFlags}, as usual:
\btab public TermActivationFlags VolTerms \{\newline 
\btab \btab get \{\newline 
\btab \btab \btab return TermActivationFlags.GradUxGradV;\newline 
\btab \btab \}\newline 
\btab \}\newline 
 \newline 
\btab public TermActivationFlags InnerEdgeTerms \{\newline 
\btab \btab get \{\newline 
\btab \btab \btab return (TermActivationFlags.AllOn);\newline 
\btab \btab \}\newline 
\btab \}\newline 
 \newline 
\btab public TermActivationFlags BoundaryEdgeTerms \{\newline 
\btab    get \{\newline 
\btab \btab    return TermActivationFlags.AllOn;\newline 
\btab \btab \}\newline 
\btab \}\newline 
 \newline 
    /// The integrand for the volume integral:
\btab public double VolumeForm(ref CommonParamsVol cpv, \newline 
\btab \btab    double[] U, double[,] GradU,\newline 
\btab \btab    double V, double[] GradV) \{               \newline 
\btab \btab double acc = 0;\newline 
\btab \btab for(int d = 0; d < cpv.D; d++)\newline 
\btab \btab \btab acc += GradU[0, d] * GradV[d];\newline 
\btab \btab return (1/Re)*acc;\newline 
\btab \}\newline 
 \newline 
 \newline 
    /// The integrand for the integral on the inner edges:
\btab public double InnerEdgeForm(ref CommonParams inp, \newline 
\btab \btab double[] U\_IN, double[] U\_OT, double[,] GradU\_IN, double[,] GradU\_OT, \newline 
\btab \btab double V\_IN, double V\_OT, double[] GradV\_IN, double[] GradV\_OT) \{\newline 
 \newline 
\btab \btab double eta = PenaltyFactor(inp.jCellIn, inp.jCellOut);\newline 
 \newline 
\btab \btab double Acc = 0.0;\newline 
\btab \btab for(int d = 0; d < inp.D; d++) \{ // loop over vector components \newline 
\btab \btab \btab // consistency term: -(\{\{ \textbackslash /u \}\} [[ v ]])*Normale\newline 
\btab \btab \btab // index d: spatial direction\newline 
\btab \btab \btab Acc -= 0.5 * (GradU\_IN[0, d] + GradU\_OT[0, d])*(V\_IN - V\_OT)\newline 
\btab \btab \btab \btab \btab    * inp.Normale[d];\newline 
 \newline 
\btab \btab \btab // the symmetry term -(\{\{ \textbackslash /v \}\} [[ u ]])*Normale\newline 
\btab \btab \btab Acc -= 0.5 * (GradV\_IN[d] + GradV\_OT[d])*(U\_IN[0] - U\_OT[0])\newline 
\btab \btab \btab \btab \btab    * inp.Normale[d];;\newline 
\btab \btab \}\newline 
 \newline 
\btab \btab // the penalty term eta*[[u]]*[[v]]\newline 
\btab \btab Acc += eta*(U\_IN[0] - U\_OT[0])*(V\_IN - V\_OT);\newline 
\btab \btab return (1/Re)*Acc;\newline 
 \newline 
\btab \}\newline 
 \newline 
    /// The integrand on boundary edges, i.e. on $\partial \Omega$:
\btab public double BoundaryEdgeForm(ref CommonParamsBnd inp, \newline 
\btab \btab double[] U\_IN, double[,] GradU\_IN, double V\_IN, double[] GradV\_IN) \{\newline 
 \newline 
 \newline 
\btab \btab double Acc = 0.0;\newline 
 \newline 
\btab \btab if(!IsDirichletBndy(inp.X)) \{\newline 
\btab \btab \btab // Neumann boundary conditions, i.e. zero-stress:\newline 
\btab \btab \btab Acc = 0;\newline 
\btab \btab \} else \{\newline 
\btab \btab \btab // Dirichlet boundary conditions\newline 
\btab \btab \btab double uBnd = UDiri(inp.X)[d];\newline 
 \newline 
\btab \btab \btab for(int d = 0; d < inp.D; d++) \{ // loop over vector components \newline 
\btab \btab \btab \btab // consistency term:\newline 
\btab \btab \btab \btab Acc -= (GradU\_IN[0, d])*(V\_IN) * inp.Normale[d];\newline 
\btab \btab \btab \btab // symmetry term:\newline 
\btab \btab \btab \btab Acc -= (GradV\_IN[d])*(U\_IN[0]- uBnd) * inp.Normale[d];\newline 
\btab \btab \btab \}\newline 
 \newline 
\btab \btab \btab // penalty term\newline 
\btab \btab \btab double eta = PenaltyFactor(inp.jCellIn, -1);\newline 
\btab \btab \btab Acc += eta*(U\_IN[0] - uBnd)*(V\_IN);\newline 
\btab \btab \}\newline 
 \newline 
\btab \btab return (1/Re)*Acc;\newline 
\btab \}\newline 
 \newline 
\btab double PenaltyFactor(int jCellIn, int jCellOut) \{\newline 
\btab \btab double PenaltySafety = 2;\newline 
\btab \btab double cj\_in         = cj[jCellIn];\newline 
\btab \btab double penalty\_base  = PenaltySafety*PolynomialDegree*PolynomialDegree;\newline 
\btab \btab double eta           = penalty\_base * cj\_in;\newline 
\btab \btab if(jCellOut >= 0) \{\newline 
\btab \btab \btab double cj\_out = cj[jCellOut];\newline 
\btab \btab \btab eta           = Math.Max(eta, penalty\_base * cj\_out);\newline 
\btab \btab \}\newline 
\btab \btab return eta;\newline 
\btab \}\newline 
\}
 }
\BoSSSexe
\BoSSScmd{
/// Finally, we are ready to implement the Stokes operator:
SpatialOperator Stokes = new SpatialOperator(3,3, // 3 vars. in dom. & codom.\newline 
\btab \btab \btab \btab \btab \btab  QuadOrderFunc.Linear(), // linear operator\newline 
\btab \btab \btab \btab \btab \btab  "u", "v", "psi",  // names of domain variables\newline 
\btab \btab \btab \btab \btab \btab  "mom\_x", "mom\_y", "conti"); // names of codom. vars\newline 
Stokes.EquationComponents["mom\_x"].Add(new Gradient\_d(0)); \newline 
Stokes.EquationComponents["mom\_x"].Add(new Viscous(0)); \newline 
Stokes.EquationComponents["mom\_y"].Add(new Gradient\_d(1)); \newline 
Stokes.EquationComponents["mom\_y"].Add(new Viscous(1));\newline 
Stokes.EquationComponents["conti"].Add(new Divergence());\newline 
Stokes.Commit();
 }
\BoSSSexe
\BoSSScmd{
/// Again, we create the matrix (now, for the Stokes operator) and check its
/// symmetry; we also have to set the Reynolds number and the polynomial
/// degree \emph{before} calling \code{ComputeMatrix} (since we are doing a
/// rather dirty trick by using global variables).
IsDirichletBndy          = IsDirichletBndy\_Channel;\newline 
UDiri                    = UDiri\_Channel;\newline 
Viscous.Re               = 20.0;\newline 
Viscous.PolynomialDegree = VelBChannel.Degree;\newline 
Viscous.cj               = grdDatChannel.Cells.cj;\newline 
var StokesMatrix\_Channel = Stokes.ComputeMatrix(varMapChannel,\newline 
\btab \btab \btab \btab \btab \btab \btab \btab \btab \btab \btab \btab null,\newline 
\btab \btab \btab \btab \btab \btab \btab \btab \btab \btab \btab \btab varMapChannel);
 }
\BoSSSexe
\BoSSScmd{
/// Testing the symmetry:
var ErrMtx1 = StokesMatrix\_Channel - StokesMatrix\_Channel.Transpose();\newline 
ErrMtx1.InfNorm();
 }
\BoSSSexe
\BoSSScmdSilent{
/// NUnit test (few random tests) BoSSScmdSilent
Assert.LessOrEqual(ErrMtx1.InfNorm(), 1e-12);
 }
\BoSSSexe
\BoSSScmd{
/// We also verify that our Stokes-matrix has full rank, i.e. we show that 
/// matrix size and rank are equal:
 }
\BoSSSexe
\BoSSScmd{
StokesMatrix\_Channel.NoOfRows;
 }
\BoSSSexe
\BoSSScmd{
StokesMatrix\_Channel.rank();
 }
\BoSSSexe
\BoSSScmdSilent{
/// NUnit test (few random tests) BoSSScmdSilent
Assert.AreEqual(StokesMatrix\_Channel.rank(), StokesMatrix\_Channel.NoOfRows);
 }
\BoSSSexe
\BoSSScmd{
/// %======================================================
/// \subsection{Solving the Stokes equation in the channel}
/// %======================================================
 }
\BoSSSexe
\BoSSScmd{
/// We set the parameters and see whether we actually obtain the correct 
/// solution; the exact solution of our problem is obviously
/// \begin{align*}
///     \vec{u}_{\text{ex}} = & (1 - y^2, 0 ), \\
///     \psi_{\text{ex}}    = & \frac{200}{\text{Re}} - x \frac{2}{\text{Re}} \\
/// \end{align*}
/// and since it is polynomial we should be able to obtain it 
/// \emph{exactly} in our velocity-pressure-space of degrees $(2,1)$.
IsDirichletBndy          = IsDirichletBndy\_Channel;\newline 
UDiri                    = UDiri\_Channel;\newline 
Viscous.Re               = 20.0;\newline 
Viscous.PolynomialDegree = VelBChannel.Degree;\newline 
var StokesMatrix\_Channel = Stokes.ComputeMatrix(varMapChannel,\newline 
\btab \btab \btab \btab \btab \btab \btab \btab \btab \btab \btab \btab null,\newline 
\btab \btab \btab \btab \btab \btab \btab \btab \btab \btab \btab \btab varMapChannel);\newline 
var StokesAffine\_Channel = Stokes.ComputeAffine(varMapChannel,\newline 
\btab \btab \btab \btab \btab \btab \btab \btab \btab \btab \btab \btab null,\newline 
\btab \btab \btab \btab \btab \btab \btab \btab \btab \btab \btab \btab varMapChannel);
 }
\BoSSSexe
\BoSSScmd{
/// Now, we are ready to solve the stokes equation. \BoSSS\ provides us with
/// a system 
/// \[
///   \code{StokesMatrix\_Channel} \cdot (u,v,\psi) 
///   + \code{StokesAffine\_Channel} = 0,
/// \]
/// so we have to multiply \code{StokesAffine\_Channel} with $-1$ to get a 
/// right-hand-side.
double[] RHS = StokesAffine\_Channel.CloneAs();\newline 
RHS.ScaleV(-1.0);
 }
\BoSSSexe
\BoSSScmd{
/// In order to store our solution, we have to create DG fields:
SinglePhaseField u               = new SinglePhaseField(VelBChannel,"u");\newline 
SinglePhaseField v               = new SinglePhaseField(VelBChannel,"v");\newline 
SinglePhaseField psi             = new SinglePhaseField(PsiBChannel,"psi");\newline 
CoordinateVector SolutionChannel = new CoordinateVector(u,v,psi);
 }
\BoSSSexe
\BoSSScmd{
/// Solve the linear system using a direct method:
StokesMatrix\_Channel.Solve\_Direct(SolutionChannel, RHS);
 }
\BoSSSexe
\BoSSScmd{
/// We export the solution to a Tecplot file:
Tecplot("Channel", 0.0, 4, u, v, psi);
 }
\BoSSSexe
\BoSSScmd{
/// \emph{TODO}: Use Visit (or any other visualization software)
/// to inspect the solution!
 }
\BoSSSexe
\BoSSScmd{
/// %=====================================
/// \section{Advanced topics}
/// %=====================================
 }
\BoSSSexe
\BoSSScmd{
/// %=====================================
/// \subsection{Stokes flow behind a grid}
/// %=====================================
 }
\BoSSSexe
\BoSSScmd{
/// We use the following setting:
/// \begin{align*}
///     \Omega    := & (0,5) \times (-2,2) \\
///     \Gamma_N  := & \{ (x,y) | \ x = 5 \} , \\
///     \Gamma_D  := & \partial \Omega \setminus \Gamma_D , \\
///     \vec{u}_D := & (1 - (2 (y - \operatorname{floor}(y)) - 1)^2, 0), \\
/// \end{align*}
 }
\BoSSSexe
\BoSSScmd{
/// So, the boundary functions are:
Func<double[],bool> IsDirichletBndy\_GridFlow \newline 
\btab \btab = (X => Math.Abs(X[0] - 5) > 1.0e-10); \newline 
Func<double[],double[]> UDiri\_GridFlow \newline 
\btab \btab = (X => new double[2] \{ 1.0 - (2*(X[1] - Math.Floor(X[1])) - 1).Pow2(),\newline 
\btab \btab \btab \btab \btab \btab \btab \btab 0\});
 }
\BoSSSexe
\BoSSScmd{
/// \emph{TODO}: the rest is for you! One hint: in $y$-direction, use some 
/// spacing so that you have cell boundaries at (least at) $y = -1, 0, 1$.
 }
\BoSSSexe
