
%
% FDY TEMPLATE for ANUAL REPORT
% ===================================================================
\NeedsTeXFormat{LaTeX2e}
\documentclass[11pt,twoside,a4paper]{fdyartcl}
\usepackage[utf8]{inputenc}
\usepackage[T1]{fontenc}
\usepackage[ngerman,english]{babel} % selectlanguage wird nur dann gebraucht, wenn
                                    % mehrere Sprachen-packages verwendet werden,
                                    % das zuletzt angegebene package ist die aktive
                                    % Sprache, mit \selectlanguage kann umgeschaltet
                                    % werden
\usepackage[intoc]{nomencl} % zur Erstellung einer Nomenklatur
                                   % Die Option [intoc] sorgt dafuer,
                                   % dass die Nomenklatur im
                                   % Inhaltsverzeichnis eingetragen
                                   % wird. Aufruf mit:
                                   % makeindex Diss.nlo -s nomencl.ist -o Diss.nls
\usepackage{longtable}  % fuer tabellen, die evtl ueber Seiten
                        % umgebrochen werden muessen
\usepackage{graphicx}   % Stellt \includegraphics zur Verfuegung
\usepackage{parskip}    % Setzt parindent auf null und parskip auf
                        % einen angemessenen Wert
\usepackage{calc}       % Erlaubt, verschiedene Masse zu addieren
                        % z.B 1cm+2pt
\usepackage[a4paper,twoside,outer=2.2cm,inner=3cm,top=1.5cm,bottom=2.7cm,includehead]{geometry}
                                        % erheblich verbesserte
                                        % Papieranassung
\usepackage{setspace}   % Stellt \singlespacing, \onehalfspacing und
                        % \doublespacig zur Verfuegung
                        % erlaubt ausserdem die Verwendung der
                        % Umgebung \begin{spacing}{2.3}
\setstretch{1.05}       % minimal vergroesserter Zeilenabstand
\usepackage{amsmath}    % Stellt verschiedene Mathematik Operatoren
                        % und Befehle bereit und verbessert die
                        % Darstellung von Gleichungen ermoeglicht
                        % ausserdem die Verwendung von \boldsymbol
                        % fuer z.B. griechische Buchstaben
\usepackage{amsfonts}
\usepackage{amsthm}
\usepackage{mathpazo}   % Aenderung der Standardschrift auf Palatino
\usepackage{upgreek}    % nicht-kursive grichische Buchstaben
\usepackage{fancyhdr}

\graphicspath{{./Figures/}}%
% \input hyphen_dt.tex  % Spezielle Trennregeln fuer deutsche
                        % Woerter - gehoert in den Vorspann
\clubpenalty = 10000%
\widowpenalty = 10000%
\displaywidowpenalty =10000


%%%%%%%%%%%%%%%%%%%%%%%%%%%%%%%%%%%%%%%%%%%%%%%%%%%%%%%%%%%%%%%%%%%%%
%%%%%%%%%%%%%%%%%%%%%%%%%%%%%%%%%%%%%%%%%%%%%%%%%%%%%%%%%%%%%%%%%%%%%
% TITLE AND AUTHOR
%%%%%%%%%%%%%%%%%%%%%%%%%%%%%%%%%%%%%%%%%%%%%%%%%%%%%%%%%%%%%%%%%%%%%
%%%%%%%%%%%%%%%%%%%%%%%%%%%%%%%%%%%%%%%%%%%%%%%%%%%%%%%%%%%%%%%%%%%%%
\title{Compute cluster guide\\TU Darmstadt - Chair of Fluid Dynamic (FDY)}
\author{Björn Müller}
%%%%%%%%%%%%%%%%%%%%%%%%%%%%%%%%%%%%%%%%%%%%%%%%%%%%%%%%%%%%%%%%%%%%%
%%%%%%%%%%%%%%%%%%%%%%%%%%%%%%%%%%%%%%%%%%%%%%%%%%%%%%%%%%%%%%%%%%%%%
%%%%%%%%%%%%%%%%%%%%%%%%%%%%%%%%%%%%%%%%%%%%%%%%%%%%%%%%%%%%%%%%%%%%%

%%%%%%%%%%%%%%%%%%%%%%%%%%%%%%%%%%%%%%%%%%%%%%%%%%%%%%%%%%%%%%%%%%%%%
%%%%%%%%%%%%%%%%%%%%%%%%%%%%%%%%%%%%%%%%%%%%%%%%%%%%%%%%%%%%%%%%%%%%%
% Kopfzeile
\pagestyle{fancy}
\fancyhf{}
\fancyhead[EL]{\sffamily \small  \thepage\ }
\fancyhead[OR]{\sffamily \small  \thepage\ }
\fancyhead[OC]{\sffamily \small TU Darmstadt - FDY}
\fancyhead[EC]{\sffamily \small BoSSS tutorial}
%%%%%%%%%%%%%%%%%%%%%%%%%%%%%%%%%%%%%%%%%%%%%%%%%%%%%%%%%%%%%%%%%%%%%
%%%%%%%%%%%%%%%%%%%%%%%%%%%%%%%%%%%%%%%%%%%%%%%%%%%%%%%%%%%%%%%%%%%%%
%%%%%%%%%%%%%%%%%%%%%%%%%%%%%%%%%%%%%%%%%%%%%%%%%%%%%%%%%%%%%%%%%%%%%
%\renewcommand{\vec}[1]{\textrm{\textbf{#1}}}
\renewcommand{\vec}[1]{\mathbf{#1}}
\newcommand{\grvec}[1]{\boldsymbol{#1}}
\renewcommand{\div}[1]{\textrm{div}\left( #1 \right)}

\newcommand{\real}{\mathbb{R}}
\newcommand{\complex}{\mathbb{C}}
\newcommand{\realpos}{\mathbb{R}_{\geq0}}
\newcommand{\natWoZero}{\mathbb{N}_{>0}}
\newcommand{\natInclZero}{\mathbb{N}_{\geq0}}
\newcommand{\nat}{\mathbb{N}}
\newcommand{\integers}{\mathbb{Z}}
\newcommand{\code}[1]{\sffamily{#1}}
\newcommand{\coderm}[1]{\footnote{\code{#1}}}


\newcounter{cnt}

\newtheoremstyle{myPlain} % namei
 {15pt}% Space above
 {3pt}% Space below
 {\itshape}% Body font
 {}% Indent amount
 {\bfseries}% Theorem head font %\scshape
 {:}% Punctuation after theorem head
 {.5em}% hSpace after theorem headi2
 {\thmname{#1} \thmnumber{#2} \thmnote{ -- #3}}% Theorem head spec (can be left empty, meaning `normal')


\theoremstyle{myPlain}
%\theoremstyle{plain}
\newtheorem{myTheorem}[cnt]{Theorem}
%\newtheorem{satzDef}[cnt]{Satz und Definition}

\newtheoremstyle{myDefinition} % namei
 {15pt}% Space above
 {3pt}% Space below
 {}% Body font
 {}% Indent amount
 {\bfseries}% Theorem head font
 {:}% Punctuation after theorem head
 {.5em}% hSpace after theorem headi2
 {\thmname{#1} \thmnumber{#2} \thmnote{ -- #3}}% Theorem head spec (can be left empty, meaning `normal')

\theoremstyle{myDefinition}
%\theoremstyle{definition}
%\newtheorem{bsp}[cnt]{Beispiel}
\newtheorem{myDef}[cnt]{Definition}
\newtheorem{myRem}[cnt]{Remark}
\newtheorem{myNot}[cnt]{Notation}


%%%%%%%%%%%%%%%%%%%%%%%%%%%%%%%%%%%%%%%%%%%%%%%%%%%%%%%%%%%%%%%%%%%%%

%       DOKUMENT
\begin{document}

\pagenumbering{arabic} % Zurueckschalten auf arabische Ziffern, dabei wird der Zaehler auf 1 gesetzt


% Titelseite
% -------------------
\maketitle


\begin{abstract}
This document explains the usage of the compute cluster at the chair of fluid dynamics, in particular for the execution of BoSSS applications.
\end{abstract}


\section{Notation}

\begin{description}
	\item[\$executable] The path to the BoSSS binary on your local machine  
	  which you want to  execute on the cluster
	\item[\$home] \verb|\\fdyprime\usersprace\$yourName\| where
		\verb|$yourName| is your user name at the FDY
	\item[\$executionDir] An arbitrary sub-directory of \verb|$home| to 
		which the executables will be deployed
	\item[\$databaseDir] The location of a BoSSS database on the server
		(usually a sub-directory of \verb|$home|)
\end{description}


\section{Prerequisites}

Before you can run a BoSSS application on the FDY cluster, you need to have the following tools installed on your local machine:
\begin{enumerate}
	\item BoSSS via an installer from
		\\ \verb|\\fdyprime\bosss\install\|
	\item Microsoft MPI
		\\ \verb|\\fdyprime\software\microsoft\HPC Pack 2012 R2 64bit\MPI\MSMPISetup.exe|
	\item Microsoft HPC Pack 2012 R2
		\\ \verb|\\fdyprime\software\microsoft\HPC Pack 2012 R2 64bit\|;
\end{enumerate}
In general, both Microsoft tools should already be installed on your machine via OPSI. If not, please contact an administrator.

Additionally, you need a BoSSS database located at \verb|$databaseDir|. If you
have not already created one, you can do so by executing the command
\begin{verbatim}
  bcl init-db $databaseDir
\end{verbatim}
from a location where you can access bcl.


\section{Deployment}

Before an application can be run on the cluster, it has to be transfered to a location on the server including all its dependencies. You can do by using the
\begin{verbatim}
  bcl deploy-at $executable $executionDir
\end{verbatim}
command.

Now, you need to create a control file for you application in
\verb|$executionDir$|. The exact content of this file depends on the
application but in every case, the path to be database has to be set using the
\textbf{full} path (see \verb|$databaseDir|).

Finally, the deployment should be done but it is highly recommended to test 
your deployed application before starting the execution on the cluster. This 
can simply be done by starting the application in \verb|$executionDir| by hand
(possibly with a smaller problem size) and verifying that the initialization 
completes successfully.


\section{Execution}

\begin{figure}
	\centering
  \includegraphics[width=0.9\columnwidth]{step1.png}%
  \caption{Main screen of the HPC Job Manager}%
  \label{fig:step1}%
\end{figure}

\begin{figure}
	\centering
  \includegraphics[width=0.9\columnwidth]{step2.png}%
  \caption{\emph{Create New Job from Description File} dialog}%
  \label{fig:step2}%
\end{figure}

\begin{figure}
	\centering
  \includegraphics[width=0.9\columnwidth]{step3.png}%
  \caption{\emph{Edit Task} dialog}%
  \label{fig:step3}%
\end{figure}

Runs on the FDY cluster can only be initiated by means of the HPC Job Manager.
During the first start it will ask you to select a \emph{head node} which in 
our case simply is \verb|HPCCLUSTER|. For every new run on the cluster you will
need to create a new \emph{job}. The easiest way of doing this is to use the
template \verb|job-template.xml| which is being distributed with this tutorial.
That way, you can use the option \emph{Create New Job from Description File}
(see Figure \ref{fig:step1}) of the Job Manager. Simply select the mentioned 
file in the dialog and you will be led to the screen displayed in Figure
\ref{fig:step2} where you can specify a name for the job.

Finally, you can proceed to the \emph{task list}. Every run consists of a list 
of \emph{tasks} which can be specified here. When using the job template 
mentioned before, one task will already be present. A click on \emph{Edit} 
will open the screen shown in in Figure \ref{fig:step3} where you can specify
the task information according to your needs. The most important parts 
(highlighted in red) are:
\begin{description}
	\item[Command Line] Replace the name of the executable (\verb|NSE2.exe| by default) with the name of your executable (while keeping the \verb|mpiexec| command!). Note that you can explicitly specify the name of a control file as a command line argument to the application if required (e.g. \verb|NSE2.exe -c myControlFile.xml|).
	\item[Working directory] Replace with the \textbf{full} path to \verb|$executionDir|
	\item[Standard output] Specify the path (either absolute or relative to the working directory) to a file in which you want to store the standard output stream of your program
	\item[Standard error] Specify the path (either absolute or relative to the working directory) to a file in which you want to store the standard error stream of your program
	\item[Number of resources] Choose the (minimum and maximum) number of processors you want to use
\end{description}

When you have completed this step you may want to store your settings by means 
of \emph{Save Job as} in order to be able to reuse your settings for later 
runs. You are now ready to enqueue your job via \emph{Submit} using your 
username and your password. You can view the status of your job in the
\emph{My Jobs} list in the \emph{Job Management} section of the HPC Job 
Manager. It will appear in the list of \emph{Active} jobs until it is 
finished. Afterwards, it will be listed either in the list of \emph{Finished} 
or \emph{Failed} jobs depending on the outcome of the run.


\end{document} 