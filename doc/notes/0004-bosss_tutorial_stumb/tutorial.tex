%
% FDY TEMPLATE for ANUAL REPORT
% ===================================================================
\NeedsTeXFormat{LaTeX2e}
\documentclass[11pt,twoside,a4paper]{fdyartcl}
\usepackage[utf8]{inputenc}
\usepackage[T1]{fontenc}
\usepackage[ngerman,english]{babel} % selectlanguage wird nur dann gebraucht, wenn
                                    % mehrere Sprachen-packages verwendet werden,
                                    % das zuletzt angegebene package ist die aktive
                                    % Sprache, mit \selectlanguage kann umgeschaltet
                                    % werden
\usepackage[intoc]{nomencl} % zur Erstellung einer Nomenklatur
                                   % Die Option [intoc] sorgt dafuer,
                                   % dass die Nomenklatur im
                                   % Inhaltsverzeichnis eingetragen
                                   % wird. Aufruf mit:
                                   % makeindex Diss.nlo -s nomencl.ist -o Diss.nls
\usepackage{longtable}  % fuer tabellen, die evtl ueber Seiten
                        % umgebrochen werden muessen
\usepackage{graphicx}   % Stellt \includegraphics zur Verfuegung
\usepackage{parskip}    % Setzt parindent auf null und parskip auf
                        % einen angemessenen Wert
\usepackage{calc}       % Erlaubt, verschiedene Masse zu addieren
                        % z.B 1cm+2pt
\usepackage[a4paper,twoside,outer=2.2cm,inner=3cm,top=1.5cm,bottom=2.7cm,includehead]{geometry}
                                        % erheblich verbesserte
                                        % Papieranassung
\usepackage{setspace}   % Stellt \singlespacing, \onehalfspacing und
                        % \doublespacig zur Verfuegung
                        % erlaubt ausserdem die Verwendung der
                        % Umgebung \begin{spacing}{2.3}
\setstretch{1.05}       % minimal vergroesserter Zeilenabstand
\usepackage{amsmath}    % Stellt verschiedene Mathematik Operatoren
                        % und Befehle bereit und verbessert die
                        % Darstellung von Gleichungen ermoeglicht
                        % ausserdem die Verwendung von \boldsymbol
                        % fuer z.B. griechische Buchstaben
\usepackage{amsfonts}
\usepackage{amsthm}
\usepackage{harvard}    % neue citation Befehle und anderes Layout der
                        % Bibliographie
\usepackage{mathpazo}   % Aenderung der Standardschrift auf Palatino
\usepackage{showkeys}   % Spaeter auskommentieren
\usepackage{upgreek}    % nicht-kursive grichische Buchstaben
\usepackage{fancyhdr}

\graphicspath{{./Figures/}}%
% \input hyphen_dt.tex  % Spezielle Trennregeln fuer deutsche
                        % Woerter - gehoert in den Vorspann
\clubpenalty = 10000%
\widowpenalty = 10000%
\displaywidowpenalty =10000


%%%%%%%%%%%%%%%%%%%%%%%%%%%%%%%%%%%%%%%%%%%%%%%%%%%%%%%%%%%%%%%%%%%%%
%%%%%%%%%%%%%%%%%%%%%%%%%%%%%%%%%%%%%%%%%%%%%%%%%%%%%%%%%%%%%%%%%%%%%
% TITLE AND AUTHOR
%%%%%%%%%%%%%%%%%%%%%%%%%%%%%%%%%%%%%%%%%%%%%%%%%%%%%%%%%%%%%%%%%%%%%
%%%%%%%%%%%%%%%%%%%%%%%%%%%%%%%%%%%%%%%%%%%%%%%%%%%%%%%%%%%%%%%%%%%%%
\title{BoSSS tutorial}
\author{TU Darmstadt - Chair of Fluid Dynamic (FDY)}
%%%%%%%%%%%%%%%%%%%%%%%%%%%%%%%%%%%%%%%%%%%%%%%%%%%%%%%%%%%%%%%%%%%%%
%%%%%%%%%%%%%%%%%%%%%%%%%%%%%%%%%%%%%%%%%%%%%%%%%%%%%%%%%%%%%%%%%%%%%
%%%%%%%%%%%%%%%%%%%%%%%%%%%%%%%%%%%%%%%%%%%%%%%%%%%%%%%%%%%%%%%%%%%%%

%%%%%%%%%%%%%%%%%%%%%%%%%%%%%%%%%%%%%%%%%%%%%%%%%%%%%%%%%%%%%%%%%%%%%
%%%%%%%%%%%%%%%%%%%%%%%%%%%%%%%%%%%%%%%%%%%%%%%%%%%%%%%%%%%%%%%%%%%%%
% Kopfzeile
\pagestyle{fancy}
\fancyhf{}
\fancyhead[EL]{\sffamily \small  \thepage\ }
\fancyhead[OR]{\sffamily \small  \thepage\ }
\fancyhead[OC]{\sffamily \small TU Darmstadt - FDY}
\fancyhead[EC]{\sffamily \small BoSSS tutorial}
%%%%%%%%%%%%%%%%%%%%%%%%%%%%%%%%%%%%%%%%%%%%%%%%%%%%%%%%%%%%%%%%%%%%%
%%%%%%%%%%%%%%%%%%%%%%%%%%%%%%%%%%%%%%%%%%%%%%%%%%%%%%%%%%%%%%%%%%%%%
%%%%%%%%%%%%%%%%%%%%%%%%%%%%%%%%%%%%%%%%%%%%%%%%%%%%%%%%%%%%%%%%%%%%%
%\renewcommand{\vec}[1]{\textrm{\textbf{#1}}}
\renewcommand{\vec}[1]{\mathbf{#1}}
\newcommand{\grvec}[1]{\boldsymbol{#1}}
\renewcommand{\div}[1]{\textrm{div}\left( #1 \right)}

\newcommand{\real}{\mathbb{R}}
\newcommand{\complex}{\mathbb{C}}
\newcommand{\realpos}{\mathbb{R}_{\geq0}}
\newcommand{\natWoZero}{\mathbb{N}_{>0}}
\newcommand{\natInclZero}{\mathbb{N}_{\geq0}}
\newcommand{\nat}{\mathbb{N}}
\newcommand{\integers}{\mathbb{Z}}
\newcommand{\code}[1]{\sffamily{#1}}
\newcommand{\coderm}[1]{\footnote{\code{#1}}}


\newcounter{cnt}

\newtheoremstyle{myPlain} % namei
 {15pt}% Space above
 {3pt}% Space below
 {\itshape}% Body font
 {}% Indent amount
 {\bfseries}% Theorem head font %\scshape
 {:}% Punctuation after theorem head
 {.5em}% hSpace after theorem headi2
 {\thmname{#1} \thmnumber{#2} \thmnote{ -- #3}}% Theorem head spec (can be left empty, meaning `normal')


\theoremstyle{myPlain}
%\theoremstyle{plain}
\newtheorem{myTheorem}[cnt]{Theorem}
%\newtheorem{satzDef}[cnt]{Satz und Definition}

\newtheoremstyle{myDefinition} % namei
 {15pt}% Space above
 {3pt}% Space below
 {}% Body font
 {}% Indent amount
 {\bfseries}% Theorem head font
 {:}% Punctuation after theorem head
 {.5em}% hSpace after theorem headi2
 {\thmname{#1} \thmnumber{#2} \thmnote{ -- #3}}% Theorem head spec (can be left empty, meaning `normal')

\theoremstyle{myDefinition}
%\theoremstyle{definition}
%\newtheorem{bsp}[cnt]{Beispiel}
\newtheorem{myDef}[cnt]{Definition}
\newtheorem{myRem}[cnt]{Remark}
\newtheorem{myNot}[cnt]{Notation}


%%%%%%%%%%%%%%%%%%%%%%%%%%%%%%%%%%%%%%%%%%%%%%%%%%%%%%%%%%%%%%%%%%%%%

%       DOKUMENT
\begin{document}

\pagenumbering{arabic} % Zurueckschalten auf arabische Ziffern, dabei wird der Zaehler auf 1 gesetzt


% Titelseite
% -------------------
\maketitle


\begin{abstract}
This document is an introductional tutorial for BoSSS;
It will be later integrated into the full BoSSS documentation.
\end{abstract}

\section{A introductional example: 2D scalar transport}

\subsection{Theoretical introduction}

\subsubsection{Problem Definition}

Solve the equation
\[
    \frac{\partial}{\partial t } c + \vec{u} \cdot \nabla c = 0.
\]
We assume $\div{u} = 0$, so it can be put into BoSSS framework:
\[
    \frac{\partial}{\partial t } c + \div{ \vec{u} \ c } = 0.
\]
We choose $\vec{u}(x,y) = (-y,x)^T$ (a curl field);
\begin{itemize}
  \item characterisic curves of the problem?
  \item consistent boundary conditions (defined by characteristics)?
\end{itemize}

\subsubsection{Numerical treatment}

Describe the upwind fluxes that should can be used in DG or Finite volume methods to 
simulate this problem;

\subsection{First Implementation in BoSSS}



\subsubsection{Visual Studio: creating ``Project'' and ``Solution''}

We assume that the user is able to checkout the code and binaries
by using Git and pull-lib-win32.bat;
We assume to work in c:$\backslash$BoSSS

\begin{itemize}
 \item In which directory should we create the Solution? 
 \item What type of c\# - project should we create, in which directory?
 \item Which binary libs do we bind as references to our project?
\end{itemize}

\subsubsection{Implementing the flux}

Create a class which implements BoSSS.Foundation.INonlinearFlux

- or --

derive a class from BoSSS.Solution.Utils.NonlinearFlux

Describe the difference between those two methods.

\subsubsection{Implementing the program}

\begin{itemize}
  \item Derive a  class from BoSSS.Solution.Application
        (Hint: use visual Studio feature: ``implement abstract baseclass'');
  \item Implement the Main -- method (entry point)
  \item Create a cartesian grid (implement CreateOrLoadGrid(...))
  \item Create fields (implement CreateFields(...))
  \item Defining an initial value: override BoSSS.Solution.Application.SetInitial(...) 
        and define a algebraic function;
  \item Create a Runge-Kutta Timestepper (implementing CreateEquationsAndSolvers(...));
  \item usinge the Timestepper: implement RunSolverOneStep(...)
        Howto set the timestep size and the number of timesteps?
\end{itemize}

\subsection{Using IO}

\subsubsection{Working with a grid in the database}

\begin{itemize}
  \item Saving a (Gambit-) grid to the database
  \item Loading the grid by hard-coding it's GUID;
  \item Loading the grid by instructions in the control file.
\end{itemize}

\subsubsection{Working with sessions}

\begin{itemize}
  \item what is a session?
  \item saving the state of DG fields to the database;
        Hint: BoSSS.Solution.Application.m\_IOFields;
  \item plotting files from the database using the DBE
\end{itemize}

\subsection{Preview: Working with systems of equations}

\subsubsection{Problem Definition}

Solve the system
\begin{eqnarray*}
    \frac{\partial}{\partial t } c_1  +  \div{ \vec{u} \ c_1 }  +  \sin (c_2) & = 0. \\
    \frac{\partial}{\partial t } c_2  +  \div{ \vec{u} \ c_2 }  +  \cos (c_1) & = 0.
\end{eqnarray*}
%\[
%    \frac{\partial}{\partial t } c_1  +  \div{ \vec{u} \ c_1 }  +  \sin (c_2)  = 0. 
%    \frac{\partial}{\partial t } c_2  +  \div{ \vec{u} \ c_2 }  +  \cos (c_1)  = 0.
%\]
We choose $\vec{u}(x,y) = (-y,x)^T$ (a curl field);
\begin{itemize}
  \item characterisic curves of the problem?
  \item consistent boundary conditions (defined by characteristics)?
\end{itemize}

\subsubsection{Implementing sources}

Create a class which implements BoSSS.Foundation.INonlinearSource

- or --

derive a class from BoSSS.Solution.Utils.NonlinearSource

\subsubsection{Modify the code}

\begin{itemize}
  \item Create an additional field (implement CreateFields(...))
  \item Define the system of equations and add the source terms
\end{itemize}


\end{document} 