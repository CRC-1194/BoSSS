\section{A introductional example: 2D scalar transport}

\subsection{Theoretical introduction}

\subsubsection{Problem Definition}

Solve the equation
\[
    \frac{\partial}{\partial t } c + \vec{u} \cdot \nabla c = 0.
\]
We assume $\div{u} = 0$, so it can be put into BoSSS framework:
\[
    \frac{\partial}{\partial t } c + \div{ \vec{u} \ c } = 0.
\]
We choose $\vec{u}(x,y) = (-y,x)^T$ (a curl field);
\begin{itemize}
  \item characterisic curves of the problem?
  \item consistent boundary conditions (defined by characteristics)?
\end{itemize}

\subsubsection{Numerical treatment}

Describe the upwind fluxes that should can be used in DG or Finite volume methods to 
simulate this problem;

\subsection{First Implementation in BoSSS}



\subsubsection{Visual Studio: creating ``Project'' and ``Solution''}

We assume that the user is able to checkout the code and binaries
by using Git and pull-lib-win32.bat;
We assume to work in c:$\backslash$BoSSS

\begin{itemize}
 \item In which directory should we create the Solution? 
 \item What type of c\# - project should we create, in which directory?
 \item Which binary libs do we bind as references to our project?
\end{itemize}

\subsubsection{Implementing the flux}

Create a class which implements BoSSS.Foundation.INonlinearFlux

- or --

derive a class from BoSSS.Solution.Utils.NonlinearFlux

Describe the difference between those two methods.

\subsubsection{Implementing the program}

\begin{itemize}
  \item Derive a  class from BoSSS.Solution.Application
        (Hint: use visual Studio feature: ``implement abstract baseclass'');
  \item Implement the Main -- method (entry point)
  \item Create a cartesian grid (implement CreateOrLoadGrid(...))
  \item Create fields (implement CreateFields(...))
  \item Defining an initial value: override BoSSS.Solution.Application.SetInitial(...) 
        and define a algebraic function;
  \item Create a Runge-Kutta Timestepper (implementing CreateEquationsAndSolvers(...));
  \item usinge the Timestepper: implement RunSolverOneStep(...)
        Howto set the timestep size and the number of timesteps?
\end{itemize}

\subsection{Using IO}

\subsubsection{Working with a grid in the database}

\begin{itemize}
  \item Saving a (Gambit-) grid to the database
  \item Loading the grid by hard-coding it's GUID;
  \item Loading the grid by instructions in the control file.
\end{itemize}

\subsubsection{Working with sessions}

\begin{itemize}
  \item what is a session?
  \item saving the state of DG fields to the database;
        Hint: BoSSS.Solution.Application.m\_IOFields;
  \item plotting files from the database using the DBE
\end{itemize}

\subsection{Preview: Working with systems of equations}

\subsubsection{Problem Definition}

Solve the system
\begin{eqnarray*}
    \frac{\partial}{\partial t } c_1  +  \div{ \vec{u} \ c_1 }  +  \sin (c_2) & = 0. \\
    \frac{\partial}{\partial t } c_2  +  \div{ \vec{u} \ c_2 }  +  \cos (c_1) & = 0.
\end{eqnarray*}
%\[
%    \frac{\partial}{\partial t } c_1  +  \div{ \vec{u} \ c_1 }  +  \sin (c_2)  = 0. 
%    \frac{\partial}{\partial t } c_2  +  \div{ \vec{u} \ c_2 }  +  \cos (c_1)  = 0.
%\]
We choose $\vec{u}(x,y) = (-y,x)^T$ (a curl field);
\begin{itemize}
  \item characterisic curves of the problem?
  \item consistent boundary conditions (defined by characteristics)?
\end{itemize}

\subsubsection{Implementing sources}

Create a class which implements BoSSS.Foundation.INonlinearSource

- or --

derive a class from BoSSS.Solution.Utils.NonlinearSource

\subsubsection{Modify the code}

\begin{itemize}
  \item Create an additional field (implement CreateFields(...))
  \item Define the system of equations and add the source terms
\end{itemize}

