
%
% FDY TEMPLATE for ANUAL REPORT
% ===================================================================
\NeedsTeXFormat{LaTeX2e}
\documentclass[11pt,twoside,a4paper]{fdyartcl}
\usepackage[utf8]{inputenc}
\usepackage[T1]{fontenc}
\usepackage[ngerman,english]{babel} % selectlanguage wird nur dann gebraucht, wenn
                                    % mehrere Sprachen-packages verwendet werden,
                                    % das zuletzt angegebene package ist die aktive
                                    % Sprache, mit \selectlanguage kann umgeschaltet
                                    % werden
\usepackage[intoc]{nomencl} % zur Erstellung einer Nomenklatur
                                   % Die Option [intoc] sorgt dafuer,
                                   % dass die Nomenklatur im
                                   % Inhaltsverzeichnis eingetragen
                                   % wird. Aufruf mit:
                                   % makeindex Diss.nlo -s nomencl.ist -o Diss.nls
\usepackage{longtable}  % fuer tabellen, die evtl ueber Seiten
                        % umgebrochen werden muessen
\usepackage{graphicx}   % Stellt \includegraphics zur Verfuegung
\usepackage{parskip}    % Setzt parindent auf null und parskip auf
                        % einen angemessenen Wert
\usepackage{calc}       % Erlaubt, verschiedene Masse zu addieren
                        % z.B 1cm+2pt
\usepackage[a4paper,twoside,outer=2.2cm,inner=3cm,top=1.5cm,bottom=2.7cm,includehead]{geometry}
                                        % erheblich verbesserte
                                        % Papieranassung
\usepackage{setspace}   % Stellt \singlespacing, \onehalfspacing und
                        % \doublespacig zur Verfuegung
                        % erlaubt ausserdem die Verwendung der
                        % Umgebung \begin{spacing}{2.3}
\setstretch{1.05}       % minimal vergroesserter Zeilenabstand
\usepackage{amsmath}    % Stellt verschiedene Mathematik Operatoren
                        % und Befehle bereit und verbessert die
                        % Darstellung von Gleichungen ermoeglicht
                        % ausserdem die Verwendung von \boldsymbol
                        % fuer z.B. griechische Buchstaben
\usepackage{amsfonts}
\usepackage{amsthm}
\usepackage{mathpazo}   % Aenderung der Standardschrift auf Palatino
\usepackage{upgreek}    % nicht-kursive grichische Buchstaben
\usepackage{fancyhdr}

\graphicspath{{./Figures/}}%
% \input hyphen_dt.tex  % Spezielle Trennregeln fuer deutsche
                        % Woerter - gehoert in den Vorspann
\clubpenalty = 10000%
\widowpenalty = 10000%
\displaywidowpenalty =10000


%%%%%%%%%%%%%%%%%%%%%%%%%%%%%%%%%%%%%%%%%%%%%%%%%%%%%%%%%%%%%%%%%%%%%
%%%%%%%%%%%%%%%%%%%%%%%%%%%%%%%%%%%%%%%%%%%%%%%%%%%%%%%%%%%%%%%%%%%%%
% TITLE AND AUTHOR
%%%%%%%%%%%%%%%%%%%%%%%%%%%%%%%%%%%%%%%%%%%%%%%%%%%%%%%%%%%%%%%%%%%%%
%%%%%%%%%%%%%%%%%%%%%%%%%%%%%%%%%%%%%%%%%%%%%%%%%%%%%%%%%%%%%%%%%%%%%
\title{How to use Blender for visualising BoSSS results\\TU Darmstadt - Chair of Fluid Dynamic (FDY)}
\author{Simone Stange}
%%%%%%%%%%%%%%%%%%%%%%%%%%%%%%%%%%%%%%%%%%%%%%%%%%%%%%%%%%%%%%%%%%%%%
%%%%%%%%%%%%%%%%%%%%%%%%%%%%%%%%%%%%%%%%%%%%%%%%%%%%%%%%%%%%%%%%%%%%%
%%%%%%%%%%%%%%%%%%%%%%%%%%%%%%%%%%%%%%%%%%%%%%%%%%%%%%%%%%%%%%%%%%%%%

%%%%%%%%%%%%%%%%%%%%%%%%%%%%%%%%%%%%%%%%%%%%%%%%%%%%%%%%%%%%%%%%%%%%%
%%%%%%%%%%%%%%%%%%%%%%%%%%%%%%%%%%%%%%%%%%%%%%%%%%%%%%%%%%%%%%%%%%%%%
% Kopfzeile
\pagestyle{fancy}
\fancyhf{}
\fancyhead[EL]{\sffamily \small  \thepage\ }
\fancyhead[OR]{\sffamily \small  \thepage\ }
\fancyhead[OC]{\sffamily \small TU Darmstadt - FDY}
\fancyhead[EC]{\sffamily \small BoSSS tutorial}
%%%%%%%%%%%%%%%%%%%%%%%%%%%%%%%%%%%%%%%%%%%%%%%%%%%%%%%%%%%%%%%%%%%%%
%%%%%%%%%%%%%%%%%%%%%%%%%%%%%%%%%%%%%%%%%%%%%%%%%%%%%%%%%%%%%%%%%%%%%
%%%%%%%%%%%%%%%%%%%%%%%%%%%%%%%%%%%%%%%%%%%%%%%%%%%%%%%%%%%%%%%%%%%%%
%\renewcommand{\vec}[1]{\textrm{\textbf{#1}}}
\renewcommand{\vec}[1]{\mathbf{#1}}
\newcommand{\grvec}[1]{\boldsymbol{#1}}
\renewcommand{\div}[1]{\textrm{div}\left( #1 \right)}

\newcommand{\real}{\mathbb{R}}
\newcommand{\complex}{\mathbb{C}}
\newcommand{\realpos}{\mathbb{R}_{\geq0}}
\newcommand{\natWoZero}{\mathbb{N}_{>0}}
\newcommand{\natInclZero}{\mathbb{N}_{\geq0}}
\newcommand{\nat}{\mathbb{N}}
\newcommand{\integers}{\mathbb{Z}}
\newcommand{\code}[1]{\sffamily{#1}}
\newcommand{\coderm}[1]{\footnote{\code{#1}}}


\newcounter{cnt}

\newtheoremstyle{myPlain} % namei
 {15pt}% Space above
 {3pt}% Space below
 {\itshape}% Body font
 {}% Indent amount
 {\bfseries}% Theorem head font %\scshape
 {:}% Punctuation after theorem head
 {.5em}% hSpace after theorem headi2
 {\thmname{#1} \thmnumber{#2} \thmnote{ -- #3}}% Theorem head spec (can be left empty, meaning `normal')


\theoremstyle{myPlain}
%\theoremstyle{plain}
\newtheorem{myTheorem}[cnt]{Theorem}
%\newtheorem{satzDef}[cnt]{Satz und Definition}

\newtheoremstyle{myDefinition} % namei
 {15pt}% Space above
 {3pt}% Space below
 {}% Body font
 {}% Indent amount
 {\bfseries}% Theorem head font
 {:}% Punctuation after theorem head
 {.5em}% hSpace after theorem headi2
 {\thmname{#1} \thmnumber{#2} \thmnote{ -- #3}}% Theorem head spec (can be left empty, meaning `normal')

\theoremstyle{myDefinition}
%\theoremstyle{definition}
%\newtheorem{bsp}[cnt]{Beispiel}
\newtheorem{myDef}[cnt]{Definition}
\newtheorem{myRem}[cnt]{Remark}
\newtheorem{myNot}[cnt]{Notation}


%%%%%%%%%%%%%%%%%%%%%%%%%%%%%%%%%%%%%%%%%%%%%%%%%%%%%%%%%%%%%%%%%%%%%

%       DOKUMENT
\begin{document}

\pagenumbering{arabic} % Zurueckschalten auf arabische Ziffern, dabei wird der Zaehler auf 1 gesetzt


% Titelseite
% -------------------
\maketitle


\begin{abstract}
This document explains how to basically use Blender for creating nicely rendered pictures and videos from BoSSS data. Programs used during the process are VisIt (v2.9.1), ParaView (v3.14.1) and Blender (v2.79)
\end{abstract}


\textbf{For analysing and visualising data, VisIt is a very helpful tool. If you want to impress your audience during a conference, however, you might wish to have some nicely rendered pictures or videos which represent your hard work. In this tutorial you will therefore learn how to use Blender to make some nice pictures, camera tours or videos of your BoSSS data.}

\section{Main workflow}
Using Blender for rendering BoSSS data is quite time-consuming when doing it for the first few times, nevertheless it can produce very nice pictures and videos. The main workflow is a bit different depending on whether you want to render an individual scene or a series of time dependent scenes. 
The main steps are:
\begin{enumerate}
	\item Converting plt-files to vtk-files using VisIt
	\item Importing those to ParaView, visualising your data
	\item Exporting scene(s) to x3d-format using ParaView
	\item Importing x3d-file to Blender
	\item Setting up camera and lights
	\item Rendering picture/video 
\end{enumerate}
In the following we are going to elaborate on these steps using an example which can be found at \textit{Z:/Blender}. This example is an instationary simulation of a flow around a sphere.

\section{Rendering an individual scene}
In this section we will first consider a single scene. As ParaView is able to read single plt-files but does not detect plt-files connected through time, we can skip the first step.
\subsection{Visualisation using ParaView}
For the more experienced user we will always give a short summary before starting with the step-by-step tutorial.\\
\textit{Start by opening your file, then visualise everything until you are happy with your scene. Keep in mind that only surfaces will be visible in the rendered picture afterwards, e.g. put tubes around streamlines etc. to make them visible. Finally export your scene to x3d-format.}
\begin{enumerate}
	\item Use \textit{Open} to open the last time step of \textit{Z:/Blender/Sphere/}, Sphere2\_494.plt.
	\item Check \textit{Cell/Point Array Status} to import all of your variables and click \textit{Apply}
	\item First, we will start by using the variable \textit{Phi} to visualise the sphere. Therefore, click \textit{Filters $\rightarrow$ Alphabetical $\rightarrow$ Contour}, select \textit{Phi}, delete the automatically created value at \textit{Isosurfaces} and add 0 as new value. Now you can even colour the sphere by changing \textit{Solid Color} in the drop-down menu above to the variable you want to colour it with, e.g. \textit{Pressure}. 
	\item Next we want to create some stream lines. Therefore we need calculate a velocity vector as we only have the scalar components. Click \textit{Filters $\rightarrow$ Alphabetical $\rightarrow$ Calculator}. Name your result (e.g. vel) and type in \textit{VelocityX*iHat+VelocityY*jHat+VelocityZ*kHat} to construct the velocity vector. Apply.
	\item As we do not want our stream lines to be inside the sphere, click \textit{Filters $\rightarrow$ Alphabetical $\rightarrow$ Clip}, select sphere and set it to the origin with a radius of 2.5.
	\item Next we can generate the stream lines. Click \textit{Filters $\rightarrow$ Alphabetical $\rightarrow$ StreamTracer} and hit \textit{Apply}. You will already see the stream lines. Change the number of points and the radius until you are happy with the outcome. In the drop-down menu above change \textit{Solid Color} to the variable you want your stream lines colored with, e.g. the vorticity.
	\item As Blender only renders surfaces/volumes, we need to give those stream lines some volume. Click on your StreamTracer in the Pipeline Browser and add tubes via \textit{Filters $\rightarrow$ Alphabetical $\rightarrow$ Tube}. Hit \textit{Apply}, then adjust the radius until you are happy.
	\item When everything that you want to see in your rendered picture is visualised, click \textit{File $\rightarrow$ Export} and export your scene as x3d-file.
\end{enumerate}
	
\subsection{Rendering a picture}
\textit{Now you need to import your x3d-file to Blender. Then delete/restrict everything that should not be rendered like the your box or imported lights from paraview. Afterwards you need to set up lights and your camera. When done, press "Render image".}

\begin{enumerate}
	\item It's time to start using Blender! Open Blender and click \textit{File $\rightarrow$ Import $\rightarrow$ x3d-file}, choose your file and open it. Now you will see your exported scene from ParaView in the middle of the screen. To the right of it you will have the \textit{Outliner} with a list of all objects and below the properties for rendering. To the left of your 3D object is the place where you can add objects, lamps, cameras etc. Below your 3D-model there is a button where you can change between different screens like the 3D-model, the Text editor, ...
	\item First, start by pressing \textit{A} to deselect everything, then click the small circle beneath the window and select \textit{Textured}. You should be able to see the colours of your stream lines now. 
	\item Click the objects in the outliner and delete everything (like the surrounding box) that you do not want to be rendered.
	\item Click the camera in the outliner to find it. Then click right to start moving and left to stop moving your camera. By pressing \textit{0} on the number block, you can see the camera's view, press it again to get back to the 3D-view. Move and turn the camera until you are happy. \\
	Alternatively, you can press \textit{NumBlock 0} to get into camera view, then press \textit{Shift+F} to get into "flying mode". Use your mouse and \textit{WASD}-keys to move your camera, left click to fix it in place. 
	\item By default, Blender already positions a lamp somewhere and imports a lot of \textit{DirectLight} from ParaView. Therefore, everything should be clearly visible and there are almost no shadows. If you want to adjust your light, this is the time to remove lights and create new ones by clicking \textit{Create $\rightarrow$ Point/Sun/Spot}.
	\item Now let's render your image. Set your properties in the right column like output format and path and click \textit{Render}. You will see your result. Change back to 3D view by clicking the icon on the bottom left of your image. 
\end{enumerate}
\subsection{Camera tour}
We have three possibilities of generating a camera tour: 
\begin{enumerate}
	\item We record ourselves flying through the scene -- fastest and easiest way, as long as you are able to move and steer evenly, otherwise the result will be quite chaotic.
	\item Defining the camera position at a few specific frames. Blender will interpolate location and rotation of your camera in between those frames. Easy and smooth camera tour.
	\item Connecting the camera to a path through/around your scene. Works well if you want it to move around in a circle e.g., for self-made bezier curves it takes a while until it always faces the way you want it.
\end{enumerate}
\subsubsection{Fly through the scene}
\textit{Use Shift+F to switch into "flying mode", record your path, then use "Animation" to render it.}
\begin{enumerate}
	\item Have your scene open in 3D-view, then press \textit{Shift+F} to switch to "flying mode". Use WASD-keys and the mouse to move and practice the path you want to record.
	\item Set your frame rate to 5, get into camera view by pressing \textit{0}, then press the record button (red circle) on the bottom. 
	\item Click the play button and then \textit{Shift+F}. Fly through your scene, then press \textit{Escape} and click the stop button.
	\item Use the timeline to select start and end frame and adjust them in the animation properties on the right. Set your frame rate to something around 15 fps and select output format and path. 
	\item Click \textit{Animation} to render your video.
\end{enumerate}
\subsubsection{Using camera positions at specific frames}
\textit{Define camera position at specific frames, then Blender interpolates positions between those frames. Animate.}
\begin{enumerate}
	\item Click the frame where you want to set a keyframe. select your camera and press \textit{NumBlock 0} to get into camera mode
	\item At the camera properties column to the right click the cube to edit the selected object (camera).
	\item Use \textit{Shift+F} to get into flying mode. Turn and position your camera.
	\item Press Escape to get out of flying mode. Right click into \textit{Location} and \textit{Rotation } and select \textit{Insert Keyframes} both times.
	\item Do these steps several times until you have a few key frames. Press play to see the resulting camera movement.
	\item Set your animation properties and click \textit{Animation} to render your video.
\end{enumerate}
\subsubsection{Connecting the camera to a path}
\textit{Insert a curve, connect camera to curve and render your animation.}
\begin{enumerate}
	\item Insert curve which represents the path your camera should follow
	\item Select camera, hold \textit{shift} to select curve, press \textit{Ctrl+P} and select \textit{Follow path}
	\item Hit \textit{Alt+A} to view camera movement
	\item Adjust path or camera rotation
	\item Set your animation properties and click \textit{Animation} to render your video.
\end{enumerate}


\section{Rendering a time dependent video}
Now we will turn to rendering a video of a instationary flow simulation. As doing a camera tour while having a time dependent case is even more complex as for the single scene, we will focus on generating a video using a fixed camera position.
\subsection{Converting plt-files to vtk-files using VisIt}
ParaView unfortunately does not recognize plt-files that are connected through time as combined files. Therefore, we need to open those first using VisIt and export them to a vtk-format.\\
\textit{Open your plt-files in VisIt, create mesh, then export database with all time steps and all needed variables to vtk-format.}\\
\begin{enumerate}
	\item Use VisIt to open your batch of plt-files
	\item Click \textit{Add $\rightarrow$ Mesh $\rightarrow$ Zone\_Cube} and \textit{Draw}.
	\item Now click \textit{File $\rightarrow$ Export database...}, then in the appearing window check \textit{Export all time states}, select \textit{Export to VTK} and add all variables that will be important for the visualisation, namely \textit{VelocityX VelocityY VelocityZ Pressure Phi}. Adjust file name and export directory and click \textit{Export}. 
\end{enumerate}
\subsection{Visualisation using ParaView, export using a python script}
\textit{Open the vtk-files, visualise as you want then adjust the python script "export.py" and run it.}\\
The visualisation works in the same way as for a single scene. Open the vtk-files and visualise to your needs. When you are done, follow these steps:
\begin{enumerate}
	\item Use \textit{Notepad++} to open the python script \textit{export.py} which you can find in \textit{/Z:/Blender/}.
	\item Adjust start and end frame. Adjust the path and name of your x3d-files. Save when done.
	\item In ParaView, click  \textit{Tools $\rightarrow$ Python Shell} and then \textit{Run Script} and select export.py. Now every time step is being exported to x3d-format.
\end{enumerate}
\subsection{Creating rendered pictures of each scene automatically}
\textit{Use one scene to set up camera and lights, then delete everything of that scene and run the python script "render.py".}
\begin{enumerate}
	\item While the export script is still running, you can already open a scene to set up camera and lights. As we did for a single scene before, find a camera/light constellation that you want for your video later. Check by rendering an image of your test scene.
	\item When you are done, delete everything that has been imported from ParaView like all \textit{DirectLight}s, \textit{Shape\_IndexedFaceSet}s and \textit{Shape\_IndexedLineSet}s. Save your set up in case Blender crashes.
	\item Now switch to \textit{Text editor} mode and open another python script \textit{render.py}. Again, adjust the paths and start and end frame. Click \textit{Windows $\rightarrow$ Toggle System Console} to see if any errors occur. Start rendering by clicking \textit{Run script}. To make sure everything is correct you can also start by rendering the first few frames first and then the whole batch. 
\end{enumerate}
\subsection{Generating video}
Now we have generated a lot of single pictures which shall be combined into a movie.\\
\textit{In Blender, add your images to the Video Sequence Editor. Set your video properties in the right column and click Animation.}
\begin{enumerate}
	\item In Blender, switch to the \textit{Video Sequence Editor}. Click \textit{Add $\rightarrow$ Image} and select all of your generated images (hold control and shift to select).
	\item Now you will see an image strip in your editor. In the column on the right side you can adjust the properties of the animation. Set start and end frame, the go down to set the output format and the output path. 
	\item Click \textit{Animation} and wait. Enjoy your result.
\end{enumerate}
\end{document} 