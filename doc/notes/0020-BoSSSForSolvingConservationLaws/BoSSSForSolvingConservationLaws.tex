%
% FDY TEMPLATE for ANNUAL REPORT
% ===================================================================
% Status: In preparation
% Reviewer 1: Florian Kummer (done)
% Reviewer 2: Roozbeh Mousavi (done)
\NeedsTeXFormat{LaTeX2e}
%\documentclass[11pt,twoside,a4paper]{fdyartcl}
\documentclass[11pt,twoside,a4paper]{fdyarticle}
%\documentclass[11pt,twoside,a4paper]{article}

\usepackage[utf8]{inputenc}
\usepackage[T1]{fontenc}
\usepackage[ngerman,english]{babel} % selectlanguage wird nur dann gebraucht, wenn
                                    % mehrere Sprachen-packages verwendet werden,
                                    % das zuletzt angegebene package ist die aktive
                                    % Sprache, mit \selectlanguage kann umgeschaltet
                                    % werden
\usepackage[intoc]{nomencl} % zur Erstellung einer Nomenklatur
                                   % Die Option [intoc] sorgt dafuer,
                                   % dass die Nomenklatur im
                                   % Inhaltsverzeichnis eingetragen
                                   % wird. Aufruf mit:
                                   % make index Diss.nlo -s nomencl.ist -o Diss.nls
\usepackage{longtable}  % fuer tabellen, die evtl ueber Seiten
                        % umgebrochen werden muessen
\usepackage{graphicx}   % Stellt \includegraphics zur Verfuegung
\usepackage{parskip}    % Setzt parindent auf null und parskip auf
                        % einen angemessenen Wert
\usepackage{calc}       % Erlaubt, verschiedene Masse zu addieren
                        % z.B 1cm+2pt
\usepackage[a4paper,twoside,outer=2.2cm,inner=3cm,top=1.5cm,bottom=2.7cm,includehead]{geometry}
                                        % erheblich verbesserte
                                        % Papieranassung
\usepackage{setspace}   % Stellt \singlespacing, \onehalfspacing und
                        % \doublespacig zur Verfuegung
                        % erlaubt ausserdem die Verwendung der
                        % Umgebung \begin{spacing}{2.3}
\setstretch{1.05}       % minimal vergroesserter Zeilenabstand
\usepackage{amsmath}    % Stellt verschiedene Mathematik Operatoren
                        % und Befehle bereit und verbessert die
                        % Darstellung von Gleichungen ermoeglicht
                        % ausserdem die Verwendung von \boldsymbol
                        % fuer z.B. griechische Buchstaben
\usepackage{amsfonts}
\usepackage{amsthm}
\usepackage{harvard}    % neue citation Befehle und anderes Layout der
                        % Bibliographie
\usepackage{mathpazo}   % Aenderung der Standardschrift auf Palatino
\bibliographystyle{diss_harv_babel}
\input babelbst.tex     % sonst funktioniert das zweite \cite Kommando
                        % nicht, weil im harvardstyle \bbletal{}
                        % herausgeschrieben wird
%usepackage{showkeys}   % Spaeter auskommentieren
\usepackage{upgreek}    % nicht-kursive grichische Buchstaben
\usepackage{fancyhdr}

\usepackage{ngerman}    %erlaubt auf allen Btriebssystemen die Eingabe der Umlaute als "a,"o,"u und "s mit dem Ergebnis
                        % ?,?,ü und ?.(Gleiches gilt für Gro?buchstaben)
\usepackage{hyperref}
\usepackage[all]{hypcap}

\usepackage{subfiles}   %For having "child" documents

\usepackage{epstopdf}   %Converting eps pictures to pdf

\usepackage{verbatim}   %Allows for multi-line comments

\graphicspath{{./Figures/}}%
% \input hyphen_dt.tex  % Spezielle Trennregeln fuer deutsche
                        % Woerter - gehoert in den Vorspann
\clubpenalty = 10000%
\widowpenalty = 10000%
\displaywidowpenalty =10000


%%%%%%%%%%%%%%%%%%%%%%%%%%%%%%%%%%%%%%%%%%%%%%%%%%%%%%%%%%%%%%%%%%%%%
%%%%%%%%%%%%%%%%%%%%%%%%%%%%%%%%%%%%%%%%%%%%%%%%%%%%%%%%%%%%%%%%%%%%%
% TITLE AND AUTHOR
%%%%%%%%%%%%%%%%%%%%%%%%%%%%%%%%%%%%%%%%%%%%%%%%%%%%%%%%%%%%%%%%%%%%%
%%%%%%%%%%%%%%%%%%%%%%%%%%%%%%%%%%%%%%%%%%%%%%%%%%%%%%%%%%%%%%%%%%%%%
\title{BoSSS for solving conservation laws}
\author{Nehzat Emamy}
%%%%%%%%%%%%%%%%%%%%%%%%%%%%%%%%%%%%%%%%%%%%%%%%%%%%%%%%%%%%%%%%%%%%%
%%%%%%%%%%%%%%%%%%%%%%%%%%%%%%%%%%%%%%%%%%%%%%%%%%%%%%%%%%%%%%%%%%%%%
%%%%%%%%%%%%%%%%%%%%%%%%%%%%%%%%%%%%%%%%%%%%%%%%%%%%%%%%%%%%%%%%%%%%%

%%%%%%%%%%%%%%%%%%%%%%%%%%%%%%%%%%%%%%%%%%%%%%%%%%%%%%%%%%%%%%%%%%%%%
%%%%%%%%%%%%%%%%%%%%%%%%%%%%%%%%%%%%%%%%%%%%%%%%%%%%%%%%%%%%%%%%%%%%%
% Kopfzeile
\pagestyle{fancy}
\fancyhf{}
\renewcommand{\headrulewidth}{0pt}

\fancyhead[EL]{\sffamily \small  \thepage\ }
\fancyhead[OR]{\sffamily \small  \thepage\ }
\fancyhead[OC]{\sffamily \small Annual report 2010}
\fancyhead[EC]{\sffamily \small Emamy}


%%%%%%%%%%%%%%%%%%%%%%%%%%%%%%%%%%%%%%%%%%%%%%%%%%%%%%%%%%%%%%%%%%%%%
%%%%%%%%%%%%%%%%%%%%%%%%%%%%%%%%%%%%%%%%%%%%%%%%%%%%%%%%%%%%%%%%%%%%%

% Dieser Befehl stellt sicher, dass neue Kapitel auf rechten (ungeraden) Seiten beginnen
\newcommand{\clearemptydoublepage}%
{\newpage{\pagestyle{empty}\cleardoublepage}}
\newfont{\myrm}{cmr12 at 12 pt}

\newlength{\FigureHeight}
\newlength{\FigureHeightHalf}
\newcommand{\FigureXYLabel}[7]{%
\settoheight{\FigureHeight}{#1}%
\setlength{\FigureHeightHalf}{0.5\FigureHeight}%
\addtolength{\FigureHeightHalf}{#7}%
\raisebox{\FigureHeightHalf}{\makebox[0cm][r]{#5\makebox[#6]{}}}%
#1\\%
\vspace{#4}%
{\makebox{#2\makebox[#3]{}}}}

\input defs.tex
\makenomenclature


%       DOKUMENT
\begin{document}

\pagenumbering{arabic} % Zurueckschalten auf arabische Ziffern, dabei wird der Zaehler auf 1 gesetzt

%%%%%%%%%%%%%%%%%%%%%%%%%%%%%%%%%%%%%%%%%%%%%%%%%%%%%%%%%%%%%%%%%%%%%%%%%%%%%%%%55
% Seitennummer der ersten Seite
\setcounter{page}{1} % must be an odd number
                       % muss eine ungerade Zahl sein
%%%%%%%%%%%%%%%%%%%%%%%%%%%%%%%%%%%%%%%%%%%%%%%%%%%%%%%%%%%%%%%%%%%%%%%%%%%%%%%%55
%%%%%%%%%%%%%%%%%%%%%%%%%%%%%%%%%%%%%%%%%%%%%%%%%%%%%%%%%%%%%%%%%%%%%%%%%%%%%%%%55
%%%%%%%%%%%%%%%%%%%%%%%%%%%%%%%%%%%%%%%%%%%%%%%%%%%%%%%%%%%%%%%%%%%%%%%%%%%%%%%%55

% Titelseite
% -------------------
\maketitle

\begin{abstract}
This document shows how a scalar conservation law is discretized and solved with Discontinuous Galerkin Finite Element Method (DG-FEM) using BoSSS libraries.
\end{abstract}

%==================== Hauptteil =====================================
%
\section*{Preamble}
This document starts with a short introduction to the Discontinuous Galerkin (DG) method and shows how it can be used for solving a scalar conservation law. For further reading about the DG method refer to \cite{HesthavenWarburton08}. Each section of this document starts with describing some theory and it is followed by a subsection that shows how this theory is implemented in BoSSS. Although a scalar conservation law is used in the theory part, the implementation part is general for solving system of equations. In the implementation part references to source code are given as footnotes; for which referring to reference manual, created from inline documentation, or looking directly to the code is recommended. A successive reading of sections is expected but it is up to reader, in which details and to what extent is interested. Reading sections \ref{sec:Polynomials} and \ref{sec:Element-wise operations} about polynomials or sections \ref{sec:QuadRules} to \ref{sec:LECQuadrature} about quadratures are not necessary for following the rest. In sections \ref{sec:QuadRules} to \ref{sec:LECQuadrature} and \ref{sec:Linear2ndDerivativeTerms} (about discretizing terms with $2$nd order derivatives) some semi-code lines are written to facilitate understanding the procedure and refereing to the source code. They are not real code lines; for example the reader should imagine that there are loops over the indices or the arrays may be arrays of arrays in the source code. The last section, section \ref{sec:Linear2ndDerivativeTerms} can also be considered as an example how to modify the linear quadrature to be able to discretize some new terms if they can not be handled in the current frame work. To understand how the different pieces described in this documentation are put together to solve a problem with BoSSS see BoSSS tutorial and example applications based on BoSSS libraries in L$4$ layer.

\newpage
\tableofcontents

\newpage
\printnomenclature

%Files included
%===================================================================
\newpage
\subfile{AboutDG.tex}

\newpage
\section{Orthonormal polynomials on a reference element}
\label{sec:Polynomials}
\subfile{Polynomials.tex}

\newpage
\section{Element-wise operations}
\label{sec:Element-wise operations}
\subfile{ElementwiseOperations.tex}

\newpage
\section{Spatial Differential Operator}
\label{sec:SpatialDifferentialOperator}
\subfile{SpatialDifferentialOperator.tex}

\newpage
\section{DG Fields}
\label{sec:DGFields}
\subfile{DGFields.tex}

\newpage
\section{Quadrature rules}
\label{sec:QuadRules}
\subfile{QuadRules.tex}

\newpage
\section{Nonlinear equation components}
\label{sec:NECQuadrature}
\subfile{NECQuadrature.tex}

\newpage
\section{Linear equation components}
\label{sec:LECQuadrature}
\subfile{LECQuadrature.tex}

\newpage
\section{Fluxes and sources}
\label{sec:FluxAndSource}
\subfile{FluxAndSource.tex}

\newpage
\section{Time discretization}
\label{sec:Timediscretization}
\subfile{TimeDiscretization.tex}

\newpage
\section{Linear solvers}
\label{sec:LinSolvers}
\subfile{LinSolvers.tex}

\newpage
\section{Linear second order derivative terms}
\label{sec:Linear2ndDerivativeTerms}
\subfile{Linear2ndDerivativeTerms.tex}


%==================== Literatur =====================================
% Bibliography
% -------------
\newpage
\addcontentsline{toc}{section}{Bibliography}
\bibliography{BibDaten}

\end{document} 