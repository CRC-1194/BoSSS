\documentclass[BoSSSForSolvingConservationLaws.tex]{subfiles}

\begin{document}
\section{DG-FEM in comparison with other numerical methods}
\begin{enumerate}
  \item Finite difference
  \begin{itemize}
    \item PDE is satisfied in a point-wise manner
    \item Not suitable for complex geometries
    \item Grid smoothness required
    \item Not well-suited for problems with discontinuities
  \end{itemize}
  \item Finite volume
  \begin{itemize}
    \item PDE is satisfied on conservation form
    \item Not suitable for high-order accuracy and hp-adaptivity
    \item Grid smoothness requirement
  \end{itemize}
  \item Finite Element
  \begin{itemize}
    \item PDE is satisfied in a global manner
    \item Solution is represented globally
    \item Not well-suited for problems with direction
    \item Implicit in time
  \end{itemize}
\end{enumerate}
DG-FEM has local statement of FVM, local high order elements of FEM and the geometric flexibility of FVM and FEM. The only problem is that it is not well-suited for elliptic problems which is the same as with FVM.

\section{Basics of DG-FEM}
Physical domain $\Omega$\nomenclature{$\Omega$}{Physical domain} with boundary $\partial\Omega$\nomenclature{$\partial\Omega$}{Boundary of physical domain} is divided into $J$\nomenclature{$J$}{Number of elements} non overlapping elements, $K_j$\nomenclature{$K_j$}{Element (cell) $j$}\nomenclature{$j$}{Index for elements}. The global solution $u(\mathbf{x},t)$\nomenclature{$u(\mathbf{x},t)$}{Global solution} is approximated by piecewise polynomials approximation $u^h(\vec{x},t)$\nomenclature{$u^h(\vec{x},t)$}{Polynomial approximation of the global solution} of degree $N$\nomenclature{$N$}{Polynomial degree}. The approximate solution $u^h(\vec{x},t)$ is direct sum of the $J$ local polynomial solutions $u_j^h(\vec{x},t)$\nomenclature{$u_j^h(\vec{x},t)$}{Local polynomial solution}.\\
\begin{equation}
\label{ApproximateSolution}
  u(\vec{x},t)\simeq u^h(\vec{x},t)=\bigoplus_{j=0}^{J-1} u_j^h(\vec{x},t)
\end{equation}
Local polynomial solution in element $K_j$ is represented by\\
\begin{equation}
\label{PolynomialRepresentation}
u_j^h(\vec{x},t)=\sum_{n=0}^{N_p-1} \hat u_{jn}(t) \phi_{jn}(\vec{x})=\sum_{n=0}^{N_p-1} u_j^h(\vec{x}_{jn},t) l_{jn}(\vec{x})
\end{equation}
\nomenclature{$n$}{Index for basis polynomials}
$N_p$\nomenclature{$N_p$}{Number of polynomials} is number of basis polynomials of order $N$ and $D$\nomenclature{$D$}{Spatial dimension} is spatial dimension.\\
\[
N_p=\frac{1}{D!}\prod_{1\le l \le D}(N+l)
\]
The first representation in equation \eqref{PolynomialRepresentation} is called \emph{modal} and the second one \emph{nodal}. Coefficients which belong to each cell $j$ and change with time are called DG coordinates\nomenclature{$\hat u_{jn}$}{$n$th DG coordinate for cell j}. In nodal representation lagrange interpolating polynomials are used which have the cardinal property $l_m(\vec{x}_n)=\delta_{mn}$. In this case coefficients are in fact the solution at interpolating points $\vec{x}_{jn}$. In modal representation, polynomials need to be orthogonal; here we use orthonormal ones.\\
\[
\int_{K_j} \phi_{jm}(\vec{x}) \phi_{jn}(\vec{x}) d\vec{x}=\delta_{mn}
\]
$\delta_{mn}$ is Kronecker's delta\\
\[
\delta_{mn}=
\begin{cases}
  0 \qquad m\neq n \\
  1 \qquad m=n
\end{cases}
\]

\subsection*{Implementation in BoSSS}
The approximate polynomial representation $u^h(\vec{x},t)$ in equation \eqref{ApproximateSolution} is called a \emph{field}\coderm{BoSSS.Foundation.Field} in BoSSS. A field is characterized by a \emph{basis}\coderm{BoSSS.Foundation.Field.Basis} (see section \ref{sec:Polynomials}, \nameref{sec:Polynomials}) and a 2D matrix, \emph{coordinates}\coderm{BoSSS.Foundation.Field.Coordinates}, which stores the coefficients $\hat u_{jn}$ for all cells (elements).\\
A field can be evaluated\coderm{BoSSS.Foundation.Field.Evaluate(...)} in cells of domain at some specified nodes, which are called a \emph{node set} (see section \ref{sec:Element-wise operations}, \nameref{sec:Element-wise operations}). For evaluating a field, the first step is to evaluate\coderm{BoSSS.Foundation.Basis.Evaluate(...)} basis polynomials at the node set. Then at each node these values of all polynomials are multiplied by the corresponding DG coordinates and summed up. Polynomials are defined on reference elements so the result needs to be scaled (see section \ref{sec:Element-wise operations}) to find the field values at the node set. Resultant of field evaluation is in form of a 2D array $result[j,n\_node]$\nomenclature{$n\_node$}{node index in node set} for which the first index is cell index and the second index corresponds to nodes of the node set.

\section{Solving a scalar conservation law}
We want to solve following conservation law with DG-FEM.
\begin{subequations}
\label{ScalarConservationLaw}
\begin{align}
  &\partial_t u(\vec{x},t) +\nabla\cdot\vec{f}(u(\vec{x},t),\vec{x},t)
  +q(u(\vec{x},t),\vec{x},t)=0,\qquad\vec{x}\in\Omega \\
  &u(\vec{x},t)=g(\vec{x},t),\qquad \vec{x} \in \partial \Omega \\
  &u(\vec{x},0)=f(\vec{x})
\end{align}
\end{subequations}
\nomenclature{$\vec{f}$}{Flux function}
\nomenclature{$q$}{Source term}
To do that, first we form the local residual on each element $j$ and require this to vanish locally in a Galerkin sense by choosing the space of test functions to be the same as the solution space.
\[
R_j^h(\vec{x},t)=\partial_t u_j^h +\nabla\cdot\vec{f}_j^h+q_j^h, \qquad \vec{x} \in K_j
\]
\begin{equation}
\label{residual}
\int_{K_j} R_j^h(\vec{x},t)\psi_{jm}(\vec{x})d\vec{x}=0, \qquad 0\leq m \leq N_p-1
\end{equation}
\begin{itemize}
\item Residual is orthogonal to test functions $\psi$
\item This yields $N_p$ equations for $N_p$ local unknowns on each element
\end{itemize}

\subsection{DG-FEM weak formulation}
Here we start from equation \eqref{residual} and use integration by part to arrive at the following equation\\
\[
\int_{K_j} [\partial_t u_j^h \psi_{jm}-\vec{f}_j^h\cdot \nabla \psi_{jm} +q_j^h \psi_{jm}] d\vec{x}
+\oint_{\partial K_j} \hat{n}\cdot \vec{f}_j^h \psi_{jm} d\vec{x}=0, \qquad 0\leq m \leq N_p-1
\]
We know that the polynomial representation of the solution in domain $\Omega$ is piecewise continuous and therefore at boundaries of elements the solution is not unique. To calculate the integral on $\partial K_j$ we have to approximate the physical flux function $\vec{f}_j^h$ by introducing a numerical flux function $\vec{f}_j^*$\nomenclature{$\vec{f}^*$}{Numerical flux function}. This gives us the weak form of DG-FEM\\
\begin{equation}
\label{WeakFormulation}
\int_{K_j} [\partial_t u_j^h \psi_{jm}-\vec{f}_j^h\cdot \nabla \psi_{jm} +q_j^h \psi_{jm}] d\vec{x}
+\oint_{\partial K_j} \hat{n}\cdot \vec{f}_j^* \psi_{jm} d\vec{x}=0, \qquad 0\leq m \leq N_p-1
\end{equation}
\nomenclature{$\hat{n}$}{Normal vector}

\begin{comment}

\subsubsection*{Quadrature free formulation}
In the following we choose modal representation and substitute the polynomial representations of the solution $u_j^h(\vec{x},t)$, physical flux function $\vec{f}_j^h$ and source term $q_j^h(\vec{x},t)$ of element $j$.\\
\begin{align*}
  \partial_t \hat u_{jm}
  &-\sum_{n=0}^{N_p-1} \hat f_{jn}^1 \int_{K_j} \phi_{jn}(\vec{x})\frac{ \partial \phi_{jm}(\vec{x})}{\partial x_1} d(\vec{x}) \\
  &-\sum_{n=0}^{N_p-1} \hat f_{jn}^2 \int_{K_j} \phi_{jn}(\vec{x})\frac{ \partial \phi_{jm}(\vec{x})}{\partial x_2} d(\vec{x}) \\
  &-\sum_{n=0}^{N_p-1} \hat f_{jn}^3 \int_{K_j} \phi_{jn}(\vec{x})\frac{ \partial \phi_{jm}(\vec{x})}{\partial x_3} d(\vec{x}) \\
  &+\hat q_{jm}+\oint_{\partial K_j} \hat{n}\cdot \vec{f}_j^* \phi_{jm}(\vec{x}) d(\vec{x})=0, \qquad 0\leq m \leq N_p-1
\end{align*}

A matrix representation
\begin{subequations}
\label{MatrixRepresentation}
\begin{align}
&\partial_t U_j-\sum_{l=1}^{d} (S_j^l)^T F_j^l +Q_j+F_j^*=0 \\
&S_{jmn}^l=\int_{K_j} \phi_{jm}(\vec{x}) \frac{\partial \phi_{jn}(\vec{x})}{\partial x_l} d(\vec{x}), \qquad 1\leq m,n \leq N_p \\
&F_{jm}^*=\oint_{\partial K_j} \hat{n}\cdot \vec{f}^* \phi_{jm}(\vec{x}) d(\vec{x}), \qquad 0\leq m \leq N_p-1
\end{align}
\end{subequations}
$S$ is called stiffness matrix.

\end{comment}

\subsection{DG-FEM strong formulation}
Strong formulation can be found from the weak formulation. They are mathematically the same but computationally different. Starting from equation \eqref{WeakFormulation} and applying integration by part once more we find the following strong formulation.\\
\[
\int_{K_j} [\partial_t u_j^h +\nabla\cdot\vec{f}_j^h +q_j^h ] \psi_{jm} d\vec{x}
+\oint_{\partial K_j} \hat{n}\cdot [\vec{f}_j^*-\vec{f}_j^h] \psi_{jm} d\vec{x}=0, \qquad 0\leq m \leq N_p-1
\]

\subsection{Solving a scalar conservation law in BoSSS}
BoSSS uses the weak formulation and modal representation therefore equation \eqref{WeakFormulation} can be written here again as
\begin{equation}
\label{WeakFormulation-BoSSSStyle}
\int_{K_j} \partial_t u_j^h \phi_{jm} d\vec{x}
-\int_{K_j}[\vec{f}_j^h\cdot \nabla \phi_{jm} -q_j^h \phi_{jm}] d\vec{x}
+\oint_{\partial K_j} \hat{n}\cdot \vec{f}_j^* \phi_{jm} d\vec{x}=0, \qquad 0\leq m \leq N_p-1
\end{equation}
\nomenclature{$m$}{index for test functions}
The terms involving the physical and numerical flux functions and the source term are handled as \emph{equation components} of a \emph{spatial differential operator} (see section \ref{sec:SpatialDifferentialOperator}). In our case of a scalar conservation equation \eqref{ScalarConservationLaw}, the spatial differential operator has one domain variable $u$ and one codomain variable $w=\nabla\cdot \vec{f}+q$.

\end{document} 