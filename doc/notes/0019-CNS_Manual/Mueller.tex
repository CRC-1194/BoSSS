%
% FDY TEMPLATE for ANUAL REPORT
% ===================================================================
% Status: Ready for print
% Reviewer 1: C. Kallendeorf
% Reviewer 2: R. Mousavi Belfeh Teymouri
\NeedsTeXFormat{LaTeX2e}
\documentclass[11pt,twoside,a4paper]{fdyartcl}
\usepackage[utf8]{inputenc}
\usepackage[T1]{fontenc}
\usepackage[ngerman,english]{babel} % selectlanguage wird nur dann gebraucht, wenn
                                    % mehrere Sprachen-packages verwendet werden,
                                    % das zuletzt angegebene package ist die aktive
                                    % Sprache, mit \selectlanguage kann umgeschaltet
                                    % werden
\usepackage[intoc]{nomencl} % zur Erstellung einer Nomenklatur
                                   % Die Option [intoc] sorgt dafuer,
                                   % dass die Nomenklatur im
                                   % Inhaltsverzeichnis eingetragen
                                   % wird. Aufruf mit:
                                   % makeindex Diss.nlo -s nomencl.ist -o Diss.nls
\usepackage{longtable}  % fuer tabellen, die evtl ueber Seiten
                        % umgebrochen werden muessen
\usepackage{graphicx}   % Stellt \includegraphics zur Verfuegung
\usepackage{parskip}    % Setzt parindent auf null und parskip auf
                        % einen angemessenen Wert
\usepackage{calc}       % Erlaubt, verschiedene Masse zu addieren
                        % z.B 1cm+2pt
\usepackage[a4paper,twoside,outer=2.2cm,inner=3cm,top=1.5cm,bottom=2.7cm,
includehead]{geometry}
                                        % erheblich verbesserte
                                        % Papieranassung
\usepackage{setspace}   % Stellt \singlespacing, \onehalfspacing und
                        % \doublespacig zur Verfuegung
                        % erlaubt ausserdem die Verwendung der
                        % Umgebung \begin{spacing}{2.3}
\setstretch{1.05}       % minimal vergroesserter Zeilenabstand
\usepackage{amsmath}    % Stellt verschiedene Mathematik Operatoren
                        % und Befehle bereit und verbessert die
                        % Darstellung von Gleichungen ermoeglicht
                        % ausserdem die Verwendung von \boldsymbol
                        % fuer z.B. griechische Buchstaben
\usepackage{amsfonts}
\usepackage{amsthm}
\usepackage{harvard}    % neue citation Befehle und anderes Layout der
                        % Bibliographie
\usepackage{mathpazo}   % Aenderung der Standardschrift auf Palatino
\bibliographystyle{diss_harv_babel}
%% This file was renamed to englbst.tex from babelbst.tex
%%
%%
%%
%% This is file `babelbst.tex',
%% generated with the docstrip utility.
%%
%% The original source files were:
%%
%% merlin.mbs  (with options: `bblbst')
%%
%% IMPORTANT NOTICE:
%%
%% For the copyright see the source file.
%%
%% Any modified versions of this file must be renamed
%% with new filenames distinct from babelbst.tex.
%%
%% For distribution of the original source see the terms
%% for copying and modification in the file merlin.mbs.
%%
%% This generated file may be distributed as long as the
%% original source files, as listed above, are part of the
%% same distribution. (The sources need not necessarily be
%% in the same archive or directory.)
%% Copyright 1994-2005 Patrick W Daly
 % ===============================================================
 % IMPORTANT NOTICE:
 % This bibliographic style (bst) file has been generated from one or
 % more master bibliographic style (mbs) files, listed above.
 %
 % This generated file can be redistributed and/or modified under the terms
 % of the LaTeX Project Public License Distributed from CTAN
 % archives in directory macros/latex/base/lppl.txt; either
 % version 1 of the License, or any later version.
 % ===============================================================
 % Name and version information of the main mbs file:
 % \ProvidesFile{merlin.mbs}[2005/10/17 4.14 (PWD, AO, DPC)]
 % This is babelbst.tex for English.
 % It should serve as a model for other languages.
 % Alternatively, store it under a different name (e.g. englbst.tex)
 % and then \input it with a command in babelbst.tex.
\def\bbland{and}                \def\bbletal{et~al.}
\def\bbleditors{editors}        \def\bbleds{eds.}
\def\bbleditor{editor}          \def\bbled{ed.}
\def\bbledby{edited by}
\def\bbledition{edition}        \def\bbledn{edn.}
\def\bblvolume{volume}          \def\bblvol{vol.}
\def\bblof{of}
\def\bblnumber{number}          \def\bblno{no.}
\def\bblin{in}
\def\bblpages{pages}            \def\bblpp{pp.}
\def\bblpage{page}              \def\bblp{p.}
\def\bbleidpp{pages}
\def\bblchapter{chapter}        \def\bblchap{chap.}
\def\bbltechreport{Technical Report}
\def\bbltechrep{Tech. Rep.}
\def\bblmthesis{Master's thesis}
\def\bblphdthesis{Ph.D. thesis}
\def\bblfirst{First}            \def\bblfirsto{1st}
\def\bblsecond{Second}          \def\bblsecondo{2nd}
\def\bblthird{Third}            \def\bblthirdo{3rd}
\def\bblfourth{Fourth}          \def\bblfourtho{4th}
\def\bblfifth{Fifth}            \def\bblfiftho{5th}
\def\bblst{st}  \def\bblnd{nd}  \def\bblrd{rd}
\def\bblth{th}
\def\bbljan{January}  \def\bblfeb{February}  \def\bblmar{March}
\def\bblapr{April}    \def\bblmay{May}       \def\bbljun{June}
\def\bbljul{July}     \def\bblaug{August}    \def\bblsep{September}
\def\bbloct{October}  \def\bblnov{November}  \def\bbldec{December}
\endinput
%%
%% End of file `babelbst.tex'.
    % sonst funktioniert das zweite \cite Kommando
                        % nicht, weil im harvardstyle \bbletal{}
                        % herausgeschrieben wird
%\usepackage{showkeys}   % Spaeter auskommentieren
\usepackage{upgreek}    % nicht-kursive grichische Buchstaben
\usepackage{fancyhdr}
\usepackage{ngerman}    % ä,ö,ü und ß.(Gleiches gilt für Großbuchstaben)
\usepackage{setspace}
\usepackage{stmaryrd}
\usepackage{listings}
\usepackage[hang,tight,raggedright]{subfigure}

\lstset{language=XML,tabsize=4,numbers=left,numberstyle=\tiny,stepnumber=5,
numbersep=10pt,frame=single,captionpos=b,basicstyle={\small \ttfamily},
aboveskip=\bigskipamount}
\graphicspath{{./Figures/}}%
% \input hyphen_dt.tex  % Spezielle Trennregeln fuer deutsche
                        % Woerter - gehoert in den Vorspann
\clubpenalty = 10000%
\widowpenalty = 10000%
\displaywidowpenalty =10000


%%%%%%%%%%%%%%%%%%%%%%%%%%%%%%%%%%%%%%%%%%%%%%%%%%%%%%%%%%%%%%%%%%%%%
%%%%%%%%%%%%%%%%%%%%%%%%%%%%%%%%%%%%%%%%%%%%%%%%%%%%%%%%%%%%%%%%%%%%%
% TITLE AND AUTHOR
%%%%%%%%%%%%%%%%%%%%%%%%%%%%%%%%%%%%%%%%%%%%%%%%%%%%%%%%%%%%%%%%%%%%%
%%%%%%%%%%%%%%%%%%%%%%%%%%%%%%%%%%%%%%%%%%%%%%%%%%%%%%%%%%%%%%%%%%%%%
\title{Development of a solver for compressible single-phase flows in BoSSS}
\author{Bj"orn M"uller}
%%%%%%%%%%%%%%%%%%%%%%%%%%%%%%%%%%%%%%%%%%%%%%%%%%%%%%%%%%%%%%%%%%%%%
%%%%%%%%%%%%%%%%%%%%%%%%%%%%%%%%%%%%%%%%%%%%%%%%%%%%%%%%%%%%%%%%%%%%%
%%%%%%%%%%%%%%%%%%%%%%%%%%%%%%%%%%%%%%%%%%%%%%%%%%%%%%%%%%%%%%%%%%%%%

%%%%%%%%%%%%%%%%%%%%%%%%%%%%%%%%%%%%%%%%%%%%%%%%%%%%%%%%%%%%%%%%%%%%%
%%%%%%%%%%%%%%%%%%%%%%%%%%%%%%%%%%%%%%%%%%%%%%%%%%%%%%%%%%%%%%%%%%%%%
% Kopfzeile
\pagestyle{fancy}
\fancyhf{}
\renewcommand{\headrulewidth}{0pt}

\fancyhead[CE]{\sffamily \small \thepage \quad \hrulefill \quad \sffamily
\small Annual report 2010 } \fancyhead[CO]{\sffamily \small B. Müller \quad
\hrulefill \quad \sffamily \small \thepage }

%%%%%%%%%%%%%%%%%%%%%%%%%%%%%%%%%%%%%%%%%%%%%%%%%%%%%%%%%%%%%%%%%%%%%
%%%%%%%%%%%%%%%%%%%%%%%%%%%%%%%%%%%%%%%%%%%%%%%%%%%%%%%%%%%%%%%%%%%%%

% Angaben fuer die Nomenklatur
%\makenomenclature
%\renewcommand{\nomname}{Nomenklatur}
%\setlength{\nomitemsep}{-0.2\parsep}% Der Abstand zweier Eintraege in
                                    % der Nomeklatur betraegt

% Dieser Befehl stellt sicher, dass neue Kapitel auf rechten (ungeraden) Seiten beginnen
\newcommand{\clearemptydoublepage}%
{\newpage{\pagestyle{empty}\cleardoublepage}}
\newfont{\myrm}{cmr12 at 12 pt}
% FigureXYLabel - urspruenglich in defin.tex, von Prof.~Dr.-Ing.~M. Oberlack
% Urspruengliche Version umfasste 5 Parameter -  geaenderte 7 Parameter - Bauerbach
% 1 Figurename: \includegraphics[......
% 2 Beschriftung der x Achse
% 3 x-Beschriftung - Verruecken horizontal positiv nach links
% 4 x Beschriftung - Verruecken vertikal positiv nach unten
% 5 Beschriftung der y Achse
% 6 y-Beschriftung - Verruecken horizontal positiv nach links
% 7 y Beschriftung - Verruecken vertikal positiv nach oben
\newlength{\FigureHeight}
\newlength{\FigureHeightHalf}
\newcommand{\FigureXYLabel}[7]{%
\settoheight{\FigureHeight}{#1}%
\setlength{\FigureHeightHalf}{0.5\FigureHeight}%
\addtolength{\FigureHeightHalf}{#7}%
\raisebox{\FigureHeightHalf}{\makebox[0cm][r]{#5\makebox[#6]{}}}%
#1\\%
\vspace{#4}%
{\makebox{#2\makebox[#3]{}}}}
%
%%%%%%%%%%%%%%%%%%%%%%%%%%%%%%%%%%%%%%%%%%%%%%%%%%%%%%%%%%%%%%%%%%%%%

%       DOKUMENT
\begin{document}

% Zurueckschalten auf arabische Ziffern, dabei wird der Zaehler auf 1 gesetzt
\pagenumbering{arabic}

%%%%%%%%%%%%%%%%%%%%%%%%%%%%%%%%%%%%%%%%%%%%%%%%%%%%%%%%%%%%%%%%%%%%%%%%%%%%%%%%
% Seitennummer der ersten Seite
\setcounter{page}{1} % must be an odd number
%%%%%%%%%%%%%%%%%%%%%%%%%%%%%%%%%%%%%%%%%%%%%%%%%%%%%%%%%%%%%%%%%%%%%%%%%%%%%%%%
%%%%%%%%%%%%%%%%%%%%%%%%%%%%%%%%%%%%%%%%%%%%%%%%%%%%%%%%%%%%%%%%%%%%%%%%%%%%%%%%
%%%%%%%%%%%%%%%%%%%%%%%%%%%%%%%%%%%%%%%%%%%%%%%%%%%%%%%%%%%%%%%%%%%%%%%%%%%%%%%%

% Titelseite
% -------------------
\maketitle
\doublespace

\begin{abstract}
This report reviews the current state of the implementation of a compressible
Navier-Stokes solver (\emph{CNS}) in the Discontinuous Galerkin framework 
\emph{BoSSS}. At this point, the implementation of the inviscid part (i.e. the Euler 
equations) has been completed and this report summarizes the numerical formulations 
used in the solver. Additionally, the usage of \emph{CNS} is described and some
numerical examples are given.
\end{abstract}

\section{Introduction}
\label{sec:introduction}

\subsection{Governing equations}
The three-dimensional compressible Navier-Stokes equations in dimensionless,
conservative form read
\begin{equation}
	\frac{\partial U}{\partial t}
	+ \frac{\partial F_i(U)}{\partial x_i}
	- \frac{1}{Re} \frac{\partial G_i(U, \nabla U)}{\partial x_i}
	= 0, \quad i=1,\ldots,3
\end{equation}
where
\begin{equation}
	U(x_1, x_2, x_3, t)
	= \begin{pmatrix}
		\rho \\
		\rho u_1 \\
		\rho u_2 \\
		\rho u_3 \\
		\rho E
	\end{pmatrix}
\end{equation}
denotes the vector of unknowns, $F_i(U)$ denotes the convective fluxes and
$G_i(U, \nabla U)$ denotes diffusive fluxes (e.g. see \cite{Prandtl2008}).
Throughout this work, the diffusive fluxes will be neglected which leads to the
well-known Euler equations
\begin{equation}
\label{equ:Euler}
	\frac{\partial U}{\partial t}
	+ \frac{\partial F_i(U)}{\partial x_i}
	= 0
\end{equation}
for inviscid, compressible flow where
\begin{equation}
	F_i(U)
	= \begin{pmatrix}
		\rho u_i\\
		\rho u_i u_1 + \delta_{1i} p\\
		\rho u_i u_2 + \delta_{2i} p\\
		\rho u_i u_3 + \delta_{3i} p\\
		u_i (\rho E + p)
	\end{pmatrix}.
\end{equation}

The resulting system of equations is yet unclosed since we have not specified an
equation for the pressure $p$. Obviously, such an equation depends on the
applied material law. \emph{CNS} currently assumes an ideal gas which is why the
respective (dimensionless) equations of state
\begin{align}
	p &= \kappa \rho T \mathrm{Ma}_\infty^2\\
	e &= \frac{T}{(\kappa - 1) \kappa \mathrm{Ma}_\infty^2} 
\end{align}
are used in the following. They depend on the heat capacity ratio $\kappa$, the
specific inner energy per volume $e$ and the reference Mach number
$\mathrm{Ma}_\infty$.


\subsection{Discontinuous Galerkin approximation}

Suppose we have a decomposition of the considered domain $\Omega$ into $M$
non-overlapping cells $K_1, \ldots, K_M$. Let
\begin{equation}
	\frac{\partial u}{\partial t} + \nabla \cdot \vec{f}(u) = 0
\end{equation}
be any of the equations from the Euler system (\ref{equ:Euler}). In any cell
$K_i$, we multiply this equation by an arbitrary test function $\varphi_j$
and integrate by parts. This leads to the weak formulation
\begin{equation}
	\label{eqn:Euler_weak}
	\int \limits_{K_i} \frac{\partial u}{\partial t} \varphi_j dV
	+ \int \limits_{\partial K_i} \left(\vec{f}(u) \cdot \vec{n} \right) \varphi_j dS
	- \int \limits_{K_i} \vec{f}(u) \cdot \nabla \varphi_j dV
	= 0
\end{equation}
which serves as a starting point for the Discontinuous Galerkin approximation.

In the next step, we approximate the unknown quantity $u$ by a weighted sum of
basis polynomials $\Phi_k$ ($k = 1, \ldots, N$) which satisfy the orthonormality
condition
\begin{equation}
    \label{eqn:orthonormality}
    \int\limits_{K_i} \Phi_j \Phi_k dV = \delta_{jk} \quad \forall j,k \in \{1 \ldots N\}.
\end{equation}
and which may be discontinuous across cell boundaries.
The approximated solution $\tilde{u} = \tilde{u}^i(\vec{x}, t)$ is then defined by
\begin{equation}
	\label{eqn:approximation}
	u \approx \tilde{u}
	= \sum \limits_{k=1}^N \tilde{u}_k^i \Phi_k
\end{equation}
with yet unknown coefficients $\tilde{u}_k^i(t)$. If we insert this equation
into the weak formulation (\ref{eqn:Euler_weak}) and choose the test functions
$\varphi_j$ from the same polynomial space as the Ansatz functions $\Phi_k$
(i.e. if we follow the Galerkin approach), we arrive at
\begin{equation}
	\int \limits_{K_i} \frac{\partial \tilde{u}}{\partial t} \Phi_j dV
	+ \int \limits_{\partial K_i}
		\left(\vec{f}(\tilde{u}) \cdot \vec{n} \right) \Phi_j dS
	- \int \limits_{K_i} \vec{f}(\tilde{u}) \cdot \nabla \Phi_j dV
	= 0.
\end{equation}
Note that the second integral on the left hand side is not uniquely defined
because we did not enforce continuity of the approximation
(\ref{eqn:approximation}) between adjacent cells. We thus introduce the
so called \emph{numerical flux function}
\begin{equation}
	\tilde{f}(\tilde{u}^-, \tilde{u}^+) \approx \vec{f}(\tilde{u}) \cdot \vec{n}
\end{equation}
which reconstructs a unique value at the interface using the inner value
$\tilde{u}^-$ and the outer value $\tilde{u}^+$ from the cell sharing the
considered part of the cell boundary $\partial \Omega$. Many choices for
$\tilde{f}$ are possible and the particular choice for the Euler equations
applied in \emph{CNS} will be presented in section
\ref{sec:numerics_flux_function}.

Having specified the numerical flux
function, the Discontinuous Galerkin formulation of (\ref{eqn:Euler_weak}) is
completely defined and reads
\begin{equation}
	\label{eqn:Euler_DG}
	\int \limits_{K_i} \frac{\partial \tilde{u}}{\partial t} \Phi_j dV
	+ \int \limits_{\partial K_i} \tilde{f}(\tilde{u}^-, \tilde{u}^+) \Phi_j dS
	- \int \limits_{K_i} \vec{f}(\tilde{u}) \cdot \nabla \Phi_j dV
	= 0.
\end{equation}
This equation can be used to compute the unknown coefficients $\tilde{u}_k^i$
for given initial and boundary conditions.

\section{Numerical formulation}
\label{sec:numerics}

This section deals with the numerical formulation of equation
(\ref{eqn:Euler_DG}) which is implemented in \emph{CNS}. Generally, \emph{BoSSS}
only requires the implementation of the flux function, boundary conditions and a
time integration scheme by an application programmer. In \emph{CNS}, the
employed time-integration scheme is not a predefined setting but rather a
configuration option (see section \ref{sec:usage}). As a result, only the
numerical flux function and the available boundary conditions will be discussed
in the following.


\subsection{Flux function}
\label{sec:numerics_flux_function}

The quality of any solution of the Euler system calculated via
(\ref{eqn:Euler_DG}) highly depends in the particular choice of $\tilde{f}$.
A reasonable flux function should approximate the solution of the Riemann
problem
\begin{equation}
	\label{eqn:Riemann_problem}
	\frac{\partial U}{\partial t} + \frac{\partial F_1}{\partial x_1} = 0
\end{equation}
with initial conditions
\begin{equation}
	\label{eqn:Riemann_initial}
	U(x_1, x_2, x_3, 0) =
	\begin{cases}
		U_L & \mathrm{if~} x_1 \leq 0\\
		U_R & \mathrm{if~} x_1 > 0
	\end{cases}
\end{equation}
where we have assumed (without loss of generality) that the $x_1$ axis is
perpendicular to the cell boundary which is located at $x_1=0$. For given
constants $U_L$ and $U_R$ it is generally possible to solve this initial
value problem exactly (e.g. see \cite{Toro2009}).

The basic structure of this exact solution as described by \cite{Batten1997} is
outlined in figure \ref{fig:riemann_fan}. It consists of a contact wave with
speed $S_M$ and two acoustic waves with speeds $S_L$ and $S_R$ satisfying
$S_L < S_M < S_R$. An acoustic wave may either be an expansion wave or a shock
wave. In the case of an expansion wave with speed $S_L$, region 2 represents a
continuous transition from region 1 to region 3. On the other hand, in case of
a shock wave with speed $S_L$, region 2 vanishes and an abrupt transition from
region 1 to region 3 occurs. Analogously, the same holds true for
expansion/shock waves with speed $S_R$ for the regions 4, 5 and 6 in figure
\ref{fig:riemann_fan}. Finally, the transition between regions 3 and 4 may in
general be discontinuous but is always continuous in the pressure.

\begin{figure}
	\centering
	\includegraphics[height=5cm]{wavespeeds}
	\caption[dummy]{Riemann fan for the Euler equations. Source: \protect\cite{Batten1997}}
	\label{fig:riemann_fan}
\end{figure}

Exact solutions for this Riemann problem exist. It is, however, very costly to
compute this exact solution. As a result, a lot of effort has been put on the
development of approximate Riemann solvers which provide a solution with
reasonable computational effort. The \emph{HLLC} solver presented in
\cite{Toro2009} achieves very accurate results with tolerable computational
effort and has thus been implemented in \emph{CNS}.

The corresponding simplified Riemann fan is depicted in figure
\ref{fig:riemann_fan_hllc}. Here, a three wave-speed model is used to
differentiate between four states. The states $U_L$ and $U_R$ correspond to
the undisturbed initial values while $U_L^*$ and $U_R^*$ correspond to
suitably averaged states.

\begin{figure}
	\centering
	\includegraphics[height=5cm]{wavespeeds_HLLC}
	\caption{Simplified Riemann fan used in the HLLC flux. Source: \protect\cite{Batten1997}}
	\label{fig:riemann_fan_hllc}
\end{figure}

Given estimates for the required wave-speeds, the approximate value of $U$ at 
$x_1=0$ may be calculated depending on the wave-speed configuration:
If $S_L > 0$ (i.e. if no information propagates from the right to the left
cell), the flow is supersonic from the left and upwinding of $U_L$ is
appropriate. Analogously, if $S_R < 0$, the flow is supersonic from the right
and $U_R$ will be used to calculate the flux at the edge.

In the remaining situations $S_L \leq 0 < S_M$ (which is displayed in figure
\ref{fig:riemann_fan_hllc}) and $S_M \leq 0 \leq S_R$, we need a suitable method
to calculate the average state vectors $U_L^*$ and $U_R^*$. \cite{Toro2009}
gives several formulas based on the Rankine-Hugoniot jump conditions for
each of the waves. The variant implemented in \emph{CNS}
\begin{equation}
	\label{eqn:hllc_intermediate_states}
	U_K^* = \rho_K \left(\frac{S_K - u_{1,K}}{S_K - S_M} \right)
		\begin{pmatrix}
			1\\
			S_M\\
			u_2\\
			u_3\\
			\frac{E_K}{\rho_K} + (S_M - u_{1,K})
				\left(S_M + \frac{p_K}{\rho_K(S_K - u_{1,K})} \right)
		\end{pmatrix}
\end{equation}
(for $K \in \{R, L\}$) is characterized by the exact enforcement of the
aforementioned condition $p_L^*=p_R^*$.

Having determined $U_K^*$, we can easily compute the fluxes at cell boundaries. 
However, we still have to specify estimations for the wave-speeds required in the 
flux formulation. Pressure-based estimates have been proven to very reliable in this
context and thus we follow \cite{ToroSpruceSpeares1994} in setting \cite{ToroSpruceSpeares1994}
\begin{align}
	S_L &= u_{1,L} - a_L q_L\\
	S_R &= u_{1,R} + a_R q_R\\
	S_M &= \frac{p_R - p_L + u_{1,L} c_L - u_{1,R} c_R}{c_L - c_R}
\end{align}
where
\begin{align}
	c_K &= \rho_K (S_K - u_{1,K})\\
	q_K &= \begin{cases}
		1
			& \mathrm{if~} p^* \leq p_K\\
		\sqrt{1 + \frac{1 + \kappa}{2 \kappa} (\frac{p^*}{p_K} - 1)}
			& \mathrm{if~} p^* > p_K
	\end{cases}\\
	p^* &= \frac{1}{2} \max \{0, (p_L + p_R) - \frac{1}{4} (u_{1,R} - u_{1,L}) (\rho_L + \rho_R) (a_L + a_R)\}
\end{align}
and $a_K$ denotes the respective local speed of sound.


\subsection{Boundary conditions}
\label{sec:numerics_bcs}

In general, \emph{CNS} uses standard boundary conditions recommended for Finite 
Volume methods and similar approaches. This is why we do not explicitly give their 
formulation here but rather refer to some well-known literature on the subject.

An overview of the boundary conditions currently supported in \emph{CNS} is given
in table \ref{fig:boundary_conditions}. In particular, the right-most column depicts
quantities that have to be prescribed on the given boundary by the user. Please note
that the boundary conditions imposing a no-slip condition for the velocity lead to
an ill-posed problem in case of the Euler equation.

\begin{table}
	\centering
	\begin{tabular}{l | p{8cm} | c}
		\hline
		Type & Description & Parameters\\
		
		\hline\hline
		Adiabatic slip wall
		& Isolating wall with a slip condition for the velocity
		& -\\
		
		\hline
		Adiabatic wall
		& Isolating wall with a no-slip condition for the velocity
		& -\\
		
		\hline
		Isothermal wall
		& Wall with a fixed temperature and a no-slip condition for the velocity
		& $T_\mathrm{wall}$\\
		
		\hline
		Symmetry plane
		& Mimics a symmetric flow on the other side of the plane. Equivalent
		to an adiabatic slip wall in case of inviscid flows
		& -\\
		
		\hline
		Subsonic inlet
		& Standard inlet for Mach numbers smaller than one
		& $\rho_\mathrm{inlet}$, $\vec{m}_\mathrm{inlet}$\\
		
		\hline
		Subsonic pressure inlet
		& Inlet based on reservoir (i.e. \emph{stagnation} or \emph{total})
		conditions for pressure and temperature
		& $p_{t,\mathrm{inlet}}$, $T_{t,\mathrm{inlet}}$\\
		
		\hline
		Supersonic inlet
		& Standard inlet for Mach numbers greater than one
		& $\rho_{inlet}$, $\vec{m}_\mathrm{inlet}$, $p_\mathrm{inlet}$\\
		
		\hline
		Subsonic outlet
		& Standard outlet for Mach numbers smaller than one
		& $p_\mathrm{outlet}$\\
		
		\hline
		Supersonic outlet
		& Standard outlet for Mach numbers greater than one
		& -\\
		
		\hline
	\end{tabular}
	
	\caption{Currently supported boundary conditions in \emph{CNS}}
	\label{fig:boundary_conditions}
\end{table}

The basic requirements for specifying boundary conditions for both, viscid and 
inviscid flows, can be found in \cite{PoinsotLelef1992}. All boundary conditions 
except the adiabatic slip wall and the subsonic pressure inlet have been
implemented according to the principles stated there.

For adiabatic slip wall boundaries, a classical mirror method has been applied.
For information on that topic e.g. see \cite{VegtVen2002} and the references
therein.

Concerning the subsonic pressure inlet, the formulation proposed 
in \cite{FerzigerPeric2001} has been applied without mass flux correction.

\section{Controlling \emph{CNS}}
\label{sec:usage}

All \emph{BoSSS} applications are controlled by means of an XML control file.
In the following, it will be assumed that the reader is familiar with the basic
structure of this file. As a result, only the most important points specific to
\emph{CNS} will be discussed in detail.

First of all, the polynomial degrees for the density (denoted by \texttt{rho}),
the momentum field (denoted by \texttt{m}) and the energy (denoted by
\texttt{rhoE}) have to be specified (see listing \ref{lst:degrees}, lines 3 to 5).
Since the momentum is a vector-valued quantity, it will consist of
one, two or three components depending on the spatial dimension of the considered
problem domain (i.e. of the grid). The components of this vector have the same
polynomial degree and can be addressed via \texttt{m0} to \texttt{m2} in other
parts of the control file.

\begin{lstlisting}[caption={Specification of the polynomial degrees}, 
label={lst:degrees}]
<fields_degree>
	<field identification="rho" degree="1"/>
	<field identification="m" degree="2"/>
	<field identification="rhoE" degree="3"/>
	
	<!-- Optional auxiliary variables -->
	<field identification="p" degree="1"/>
	<field identification="u" degree="1"/>
	<field identification="e" degree="1"/>
	<field identification="T" degree="1"/>
</fields_degree>
\end{lstlisting}

In addition to these required specifications, \emph{CNS} supports the
automatic calculation of auxiliary variables (see listing \ref{lst:degrees},
lines 8 to 11). The values of auxiliary variables specified in the
\texttt{fields\_degree} section of the control file are calculated in every
time-step (using a basis of the specified degree) and are included in all IO
operations (i.e. they will be included in plots and/or stored in the database).
Currently supported variables are the pressure (denoted by \texttt{p}), the
velocity field (denoted by \texttt{u}), the specific inner energy (denoted by
\texttt{e}) and the absolute temperature (denoted by \texttt{T}). Please note
that, just as in the case of the momentum field, the velocity field is a
vector-valued quantity. That is, it consists of one, two or theree components
which can be addressed via \texttt{u0} to \texttt{u2} in other parts of the
control file.

Auxiliary variables are also supported when specifying initial conditions. In
listing \ref{lst:initial_conditions}, for example, initial values have been
specified for the density, the components of the momentum field and the pressure
\texttt{p}. \emph{CNS} will use this information to calculate the initial energy
automatically. Using an auxiliary variable in the initial conditions is,
however, independent of using an auxiliary variable in the list of polynomial
degrees. In the given example, this means that the pressure will not be an IO
variable by default. It is important to note that this calculation may have
influences on the quality of the projection onto the Discontinuous Galerkin
space if the projected function cannot be represented exactly by the chosen
polynomial space.


\begin{lstlisting}[caption={Specification of initial conditions}, 
label={lst:initial_conditions},float]
<initial mode="values">
	<values>
		<formula>rho(x,y,z) = 1</formula>
		<formula>m0(x,y,z) = y</formula>
		<formula>m1(x,y,z) = z</formula>
		<formula>m2(x,y,z) = x</formula>
		<formula>p(x,y,z) = x^2 + y^2 + z^2</formula>
	</values>
</initial>
\end{lstlisting}

In general, all auxiliary variables mentioned above are also supported in the initial
conditions sections. It should be obvious, however, that not all combinations of
initial values lead to a well-defined problem. This fact can be illustrated by
considering a case where initial values for the components of the momentum field,
the inner energy and the temperature are specified in the control file. Since the inner
energy and the temperature of an ideal gas are equivalent, \emph{CNS} will report
an error in such a situation.

Finally, \emph{CNS} defines some additional parameters in the \texttt{properties}
section of the control file (see listing \ref{lst:properties}). Table
\ref{fig:properties} summarizes the available options and the respective admissible 
settings.

\begin{lstlisting}[caption={Specification of \emph{CNS}-specific properties}, 
label={lst:properties},float]
<properties>
	<string key="equationSystem">Euler</string>
	<string key="convectiveFluxType">HLLC</string>
	<string key="diffusiveFluxType">dGRP</string>
	<float key="kappa">1.4</float>
	<float key="Reynolds">10.0</float>
	<float key="Mach">0.1</float>
	<float key="Prandtl">0.71</float>
	<string key="timeStepping">explicit</string>
	<string key="explicitScheme">explicitEuler</string>
	<string key="implicitScheme">implicitEuler</string>
	<string key="implicitSolver">monkey</string>
</properties>
\end{lstlisting}

\begin{table}
	\centering

	\begin{tabular}{l | p{8cm} | p{3cm}}
		\hline
		Option & Description & Admissible \hspace{3cm} values\\
		
		\hline\hline
		Equation system
		& Defines the system of partial differential equations to be solved
		& \texttt{Euler} \hspace{3cm} \texttt{Stokes}* \hspace{3cm} 
		\texttt{Navier-Stokes}* \\
		
		\hline
		Convective flux type
		& Chooses the flux function for the convective part of the equation system.
		Irrelevant if the equation system is set to \texttt{Stokes}
		& \texttt{HLL} \hspace{3cm} \texttt{HLLC}\\
		
		\hline
		Diffusive flux type
		& Chooses the flux function for the diffusive part of the equation system.
		Irrelevant if the equation system is set to \texttt{Euler}
		& \texttt{dGRP}*\\
		
		\hline
		Kappa
		& The heat capacity ratio $\kappa$
		& Positive floats\\
		
		\hline
		Reynolds
		& The reference Reynolds number. Irrelevant if the equation system is set to 
		\texttt{Euler}
		& Positive floats\\
		
		\hline
		Mach
		& The reference Mach number
		& Positive floats\\
		
		\hline
		Prandtl
		& The reference Prandtl number. Irrelevant if the equation system is set to 
		\texttt{Euler}
		& Positive floats\\
		
		\hline
		Time stepping
		& Defines the general structure of the time-stepping scheme for the
		convective and the diffusive part of the configured equation system. The
		option \texttt{mixed} treats the convective part explicitly and the
		diffusive part implicitly and is thus illegal if the equation system is
		set to \texttt{Euler}
		& \texttt{explicit} \hspace{3cm} \texttt{mixed}\\
		
		\hline
		Explicit scheme
		& The explicit time-stepping scheme to be used
		& \texttt{explicitEuler} \hspace{3cm} \texttt{rungeKutta} \hspace{3cm}
		\texttt{heun}\\
		
		\hline
		Implicit scheme
		& The implicit scheme to be used. Only relevant if the time stepping is set
		to \texttt{mixed}
		& \texttt{implicitEuler} \hspace{3cm} \texttt{crankNicolson}\\
		
		\hline
		Implicit solver
		& The solver used for the solution of the system of equations emerging
		from the implicit treatment of the diffusive part. Irrelevant if the
		time stepping is set to \texttt{explicit}
		& Any defined linear solver\\
		
		\hline
	\end{tabular}
	
	\caption{Summary of \emph{CNS}-specific control options. Admissible values marked
	with an asterisk are still under development}
	\label{fig:properties}
\end{table}

\section{Numerical examples}
\label{sec:examples}

This section briefly outlines two numerical experiments and the respective
results that have been used to validate \emph{CNS}. It should however be noted
that the given examples only cover a subset of all possible applications and
that a more thorough validation has to be a subject of further studies.
Nevertheless, the obtained results are encouraging and suggest that the overall
implementation produces reasonable results. Further studies will focus on
the role of the boundary conditions in different flow configurations and the
convergence properties of the implemented scheme.


\subsection{Shock tube problems}

Shock tube problems have been studied extensively in the context of approximate
Riemann solvers (e.g., see \cite{Toro2009}) because they are realizations of the
Riemann problem \ref{eqn:Riemann_problem}. Hence, they are an adequate measure
for the validation of the implementation of the numerical flux function
introduced in section \ref{sec:numerics}.

Figure \ref{fig:shock_tube_initial}
shows the initial configuration of such a shock tube problem. The diaphragm in
the middle separates two regions (denoted by 1 and 4) with different densities
and pressures. At $t=0$, the diaphragm breaks and a set of waves starts to
spread from the plane of discontinuity. A qualitative outline of this pattern is
given in figure \ref{fig:shock_tube_final}. In regions 1 and 4, the fluid is
still unaffected by the disturbances due to the finite speed of wave
propagation. On the other hand, two new regions (denoted by 2 and 3) have
formed. While the transition between region 1 and region 2 is given by a
right-travelling shock wave, regions 2 and 3 are separated by a 
contact wave (i.e. the pressure is continuous as discussed in section
\ref{sec:introduction}) following the shock. On the other side of the tube,
regions 3 and 4 are separated by smooth, left-travelling expansion wave.

\begin{figure}
	\centering
	\includegraphics[width=0.8\textwidth]{shockTubeDescriptionInitial}
	\caption{Initial configuration of a shock tube problem. Regions 1 and
	4 are separated by a diaphragm that breaks at $t=0$}
	\label{fig:shock_tube_initial}
\end{figure}

\begin{figure}
	\centering
	\includegraphics[width=0.8\textwidth]{shockTubeDescriptionFinal}
	\caption{Configuration in the shock tube after the diaphragm has broken
	($t>0$)}
	\label{fig:shock_tube_final}
\end{figure}

The classical configuration often referred to as \emph{Sod shock tube} with
initial conditions $\vec{u} = \vec{0}$, $\rho_1 = 0.125$, $p_1 = 0.1$,
$\rho_4 = 1.0$ and $p_4 = 1.0$ has been simulated in one, two and three space
dimensions. The rectilinear grid invariably consisted of 100 cells in x
direction, while the y and the z direction were, where applicable, discretized
each with 10 cells. Please note that this test case could only be conducted
using zeroth order polynomials because the discontinuous nature of the exact
solution requires the application of limiting techniques in order to avoid
non-physical oscillations (also known as \emph{Gibbs phenomenon}) if higher
order polynomials are employed.

\begin{figure}
	\centering
	\subfigure[Pressure]{
		\includegraphics[width=.45\textwidth]{shockTube2d2dsPressure}
		\label{fig:shockTubePressure}}
	\quad
	\subfigure[Mach number]{
		\includegraphics[width=.45\textwidth]{shockTube2d2dsMach}
		\label{fig:shockTube1dMach}}
	\caption{State inside the shock tube at $t=0.2$}
	\label{fig:shockTubeResults}
\end{figure}

As expected, the numerical results were virtually identical in all three cases
which is why only the one-dimensional results will be discussed in the
following. In figure \ref{fig:shockTubeResults}, the numerical solution obtained
is compared to the exact solution which can e.g. be found in
\cite{Anderson2002}. The numerical solution obviously is in good agreement
with the analytical results which implies a correct implementation of the HLLC
flux presented in section \ref{sec:numerics}. However, the given test case is
hardly influenced by boundary conditions which is why this aspect will be the
focus of the following test cases.


\subsection{Subsonic flow through a converging-diverging nozzle}

As a second example, we investigate the subsonic flow through a symmetric
converging-diverging nozzle. The upper half of the problem geometry (which goes
back to \cite{Liou1987}) is displayed in figure \ref{fig:nozzle_grid}. Its cross
sectional area $A$ is parametrized according to
\begin{equation}
	\label{eqn:parametrizationNozzle}
	A = \begin{cases}
		1.75 - 0.75 \cos((0.2 x - 1.0) \pi)
			& \mathrm{if~~} x < 5.0\\
		1.25 - 0.25 \cos((0.2 x - 1.0) \pi)
			& \mathrm{if~~} 5.0 \leq x \leq 10.0
	\end{cases}
\end{equation}
which leads to a throat area of $1.0$. Due to the symmetry of the problem, only
the upper part of the geometry is meshed while the centreline is treated as a
symmetry plane. For the following considerations, a coarse, a medium and a fine
grid consisting of 1520, 6376 and 10015 cells respectively have been used.

\begin{figure}
	\centering
	\includegraphics[width=\textwidth]{nozzle81CoarseGrid}
	\caption{Computational domain and coarse grid for the flow through a
	converging-diverging nozzle}
	\label{fig:nozzle_grid}
\end{figure}

\cite{Anderson2002} gives an analytical solution for the flow conditions at the
centreline for a quasi one-dimensional flow starting from reservoir conditions
(i.e. the total pressure $p_t$ and the total temperature $T_t$ of the fluid at
rest are given at the inlet) with a prescribed pressure $p_\mathrm{exit}$ at the
outlet. In this particular case, the settings $p_t = 1.0$, $T_t = 1.0$ and
$p_\mathrm{exit} = 0.81$ have been used. This leads to subsonic flow conditions
throughout the whole nozzle and a Mach number of
$\mathrm{Ma}_\mathrm{throat} \approx 0.87$ in centre of the throat. Moreover,
the throat area exhibits strong gradients of the flow parameters in this
configuration which renders it an ideal candidate for the evaluation of the
quality of numerical solutions.

\begin{figure}
	\centering
	\subfigure[Pressure]{
		\includegraphics[width=.45\textwidth]{nozzle81comparisonPressure}
		\label{fig:nozzle81comparisonPressure}}
	\quad
	\subfigure[Mach number]{
		\includegraphics[width=.45\textwidth]{nozzle81comparisonMach}
		\label{fig:nozzle81comparisonMach}}
	\caption{Comparison of the results at the centreline of
		the converging-diverging}
	\label{fig:nozzle81comparison}
\end{figure}

An illustration of the results obtained using \emph{CNS} can be found in figure
\ref{fig:nozzle81comparison} where the computed pressures and Mach numbers
are compared to the respective theoretical values. The zeroth order solution on
the coarse grid significantly underestimates the Mach number and overestimates
the pressure in the throat. As expected, the agreement with the exact solution
can be improved significantly by either increasing the number cells or the
polynomial degree.

It should however be pointed out that the use of first order Ansatz functions
leads to a considerably better agreement with the analytical solution compared
to the zeroth order solution on grids with a higher overall number of degrees
of freedom. This result is encouraging and demonstrates the potential advantage
over classical Finite Volume schemes.

\section{Outlook}
\label{sec:outlook}

This report covers the first milestone of the development of the compressible
Navier-Stokes solver \emph{CNS}. Two additional milestones have been specified
which will be addressed in the future.

The first task is the generalization for multiphase flows by means
of a cut-cell approach. Therefore, a level set method will be used to divide the
problem domain into two or more distinct parts which contain different
(immiscible) fluids separated by a sharp interface. The cut-cell method can then
be used to track the interaction of fluids \emph{within} a cell by the
introduction of local degrees of freedom for jumps and kinks. These additional
degrees of freedom can eventually be resolved via the continuum mechanical jump
conditions. In the following months, the details of this approach will be a
major subject of research.

The second task will be the introduction of viscosity effects. At this
point, it is planned to use the so called \emph{dGRP} flux outlined in
\cite{GassnerMunz2007} or the more well-known symmetric interior penalty method
which can e.g. be found in \cite{Arnold1982}. In combination with the
cut-cell approach described above, this will allow for the examination of
highly dynamic practical problems like the collapse of a cavitation bubble with
high order methods and without the need of remeshing during the computation.


\bibliography{BibDatabase}

%\appendix


\clearemptydoublepage

\end{document}
