\section{Introduction}
\label{sec:introduction}

\subsection{Governing equations}
The three-dimensional compressible Navier-Stokes equations in dimensionless,
conservative form read
\begin{equation}
	\frac{\partial U}{\partial t}
	+ \frac{\partial F_i(U)}{\partial x_i}
	- \frac{1}{Re} \frac{\partial G_i(U, \nabla U)}{\partial x_i}
	= 0, \quad i=1,\ldots,3
\end{equation}
where
\begin{equation}
	U(x_1, x_2, x_3, t)
	= \begin{pmatrix}
		\rho \\
		\rho u_1 \\
		\rho u_2 \\
		\rho u_3 \\
		\rho E
	\end{pmatrix}
\end{equation}
denotes the vector of unknowns, $F_i(U)$ denotes the convective fluxes and
$G_i(U, \nabla U)$ denotes diffusive fluxes (e.g. see \cite{Prandtl2008}).
Throughout this work, the diffusive fluxes will be neglected which leads to the
well-known Euler equations
\begin{equation}
\label{equ:Euler}
	\frac{\partial U}{\partial t}
	+ \frac{\partial F_i(U)}{\partial x_i}
	= 0
\end{equation}
for inviscid, compressible flow where
\begin{equation}
	F_i(U)
	= \begin{pmatrix}
		\rho u_i\\
		\rho u_i u_1 + \delta_{1i} p\\
		\rho u_i u_2 + \delta_{2i} p\\
		\rho u_i u_3 + \delta_{3i} p\\
		u_i (\rho E + p)
	\end{pmatrix}.
\end{equation}

The resulting system of equations is yet unclosed since we have not specified an
equation for the pressure $p$. Obviously, such an equation depends on the
applied material law. \emph{CNS} currently assumes an ideal gas which is why the
respective (dimensionless) equations of state
\begin{align}
	p &= \kappa \rho T \mathrm{Ma}_\infty^2\\
	e &= \frac{T}{(\kappa - 1) \kappa \mathrm{Ma}_\infty^2} 
\end{align}
are used in the following. They depend on the heat capacity ratio $\kappa$, the
specific inner energy per volume $e$ and the reference Mach number
$\mathrm{Ma}_\infty$.


\subsection{Discontinuous Galerkin approximation}

Suppose we have a decomposition of the considered domain $\Omega$ into $M$
non-overlapping cells $K_1, \ldots, K_M$. Let
\begin{equation}
	\frac{\partial u}{\partial t} + \nabla \cdot \vec{f}(u) = 0
\end{equation}
be any of the equations from the Euler system (\ref{equ:Euler}). In any cell
$K_i$, we multiply this equation by an arbitrary test function $\varphi_j$
and integrate by parts. This leads to the weak formulation
\begin{equation}
	\label{eqn:Euler_weak}
	\int \limits_{K_i} \frac{\partial u}{\partial t} \varphi_j dV
	+ \int \limits_{\partial K_i} \left(\vec{f}(u) \cdot \vec{n} \right) \varphi_j dS
	- \int \limits_{K_i} \vec{f}(u) \cdot \nabla \varphi_j dV
	= 0
\end{equation}
which serves as a starting point for the Discontinuous Galerkin approximation.

In the next step, we approximate the unknown quantity $u$ by a weighted sum of
basis polynomials $\Phi_k$ ($k = 1, \ldots, N$) which satisfy the orthonormality
condition
\begin{equation}
    \label{eqn:orthonormality}
    \int\limits_{K_i} \Phi_j \Phi_k dV = \delta_{jk} \quad \forall j,k \in \{1 \ldots N\}.
\end{equation}
and which may be discontinuous across cell boundaries.
The approximated solution $\tilde{u} = \tilde{u}^i(\vec{x}, t)$ is then defined by
\begin{equation}
	\label{eqn:approximation}
	u \approx \tilde{u}
	= \sum \limits_{k=1}^N \tilde{u}_k^i \Phi_k
\end{equation}
with yet unknown coefficients $\tilde{u}_k^i(t)$. If we insert this equation
into the weak formulation (\ref{eqn:Euler_weak}) and choose the test functions
$\varphi_j$ from the same polynomial space as the Ansatz functions $\Phi_k$
(i.e. if we follow the Galerkin approach), we arrive at
\begin{equation}
	\int \limits_{K_i} \frac{\partial \tilde{u}}{\partial t} \Phi_j dV
	+ \int \limits_{\partial K_i}
		\left(\vec{f}(\tilde{u}) \cdot \vec{n} \right) \Phi_j dS
	- \int \limits_{K_i} \vec{f}(\tilde{u}) \cdot \nabla \Phi_j dV
	= 0.
\end{equation}
Note that the second integral on the left hand side is not uniquely defined
because we did not enforce continuity of the approximation
(\ref{eqn:approximation}) between adjacent cells. We thus introduce the
so called \emph{numerical flux function}
\begin{equation}
	\tilde{f}(\tilde{u}^-, \tilde{u}^+) \approx \vec{f}(\tilde{u}) \cdot \vec{n}
\end{equation}
which reconstructs a unique value at the interface using the inner value
$\tilde{u}^-$ and the outer value $\tilde{u}^+$ from the cell sharing the
considered part of the cell boundary $\partial \Omega$. Many choices for
$\tilde{f}$ are possible and the particular choice for the Euler equations
applied in \emph{CNS} will be presented in section
\ref{sec:numerics_flux_function}.

Having specified the numerical flux
function, the Discontinuous Galerkin formulation of (\ref{eqn:Euler_weak}) is
completely defined and reads
\begin{equation}
	\label{eqn:Euler_DG}
	\int \limits_{K_i} \frac{\partial \tilde{u}}{\partial t} \Phi_j dV
	+ \int \limits_{\partial K_i} \tilde{f}(\tilde{u}^-, \tilde{u}^+) \Phi_j dS
	- \int \limits_{K_i} \vec{f}(\tilde{u}) \cdot \nabla \Phi_j dV
	= 0.
\end{equation}
This equation can be used to compute the unknown coefficients $\tilde{u}_k^i$
for given initial and boundary conditions.
