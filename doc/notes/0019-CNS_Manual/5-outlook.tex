\section{Outlook}
\label{sec:outlook}

This report covers the first milestone of the development of the compressible
Navier-Stokes solver \emph{CNS}. Two additional milestones have been specified
which will be addressed in the future.

The first task is the generalization for multiphase flows by means
of a cut-cell approach. Therefore, a level set method will be used to divide the
problem domain into two or more distinct parts which contain different
(immiscible) fluids separated by a sharp interface. The cut-cell method can then
be used to track the interaction of fluids \emph{within} a cell by the
introduction of local degrees of freedom for jumps and kinks. These additional
degrees of freedom can eventually be resolved via the continuum mechanical jump
conditions. In the following months, the details of this approach will be a
major subject of research.

The second task will be the introduction of viscosity effects. At this
point, it is planned to use the so called \emph{dGRP} flux outlined in
\cite{GassnerMunz2007} or the more well-known symmetric interior penalty method
which can e.g. be found in \cite{Arnold1982}. In combination with the
cut-cell approach described above, this will allow for the examination of
highly dynamic practical problems like the collapse of a cavitation bubble with
high order methods and without the need of remeshing during the computation.
