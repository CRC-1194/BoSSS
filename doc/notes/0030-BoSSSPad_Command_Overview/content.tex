\section{Command line basics}

The command line uses C\# syntax.
On startup, the variable \lstinline{IList<IDatabaseInfo> databases} stores a list of all loaded databases. They can be accessed and modified by \BoSSSpad{}.

For most variables, the method \lstinline{Actions()} will display a (in rare situations, potentially incomplete) list of methods that are available.
\begin{lstlisting}
> databases.Actions()

You can invoke the following methods (more actions may exist):
- Actions()
- As()
- Cast()
- Actions(Boolean showBuiltinMethods)
- Describe()
- Describe(String methodName)
- Summary()
\end{lstlisting}

The most important methods are described in this document, but if you want to learn more about a particular method you can always use \lstinline{Describe("nameOfSomeMethod")} to get the full documentation from the source code.
\begin{lstlisting}
> databases.First().Sessions.Describe("Find")

ISessionInfo Find(PartialGuid guid):
   * Summary: Returns the single session that matches the given guid .
   * guid: The guid of session the session in question.
   * Returns: A single session that matches guid , if it exists. Otherwise,
     an error will be thrown.
\end{lstlisting}

Other useful commands are:
\begin{description}
	\item[Clear()]
	Clears the console window
	
	\item[ShowVars()]
	Displays all the variables that have been created during the \BoSSSpad{} session.

	\item[OpenConfigFile()]
	Opens the \lstinline{DBE.xml} with the default viewer for text files.
	
	\item[OpenConfigDirectory()]
	Opens an Explorer window of the folder containing the \lstinline{DBE.xml}, the \lstinline{DBErc.cs} and another configuration files.
	
	\item[LastError]
	Displays details concerning the last \lstinline{Exception} that occurred.
	
	\item[LoadAssembly(string pathToAssembly)]
	Loads the specified assembly into memory. Useful for loading assemblies that are not loaded by BoSSSPad by default.

	\item[SaveSessionAsWorksheet(string path)]
	Saves all commands with their respective results in the format ".bws" format than can be read by \BoSSSpad{}.
\end{description}

\subsection{Navigating through history}
\BoSSSpad{} maintains a history of the last 500 issued commands. The individual entries can be accessed in the style of the standard Unix/Linux shell \lstinline{Bash}. That is, commands can be accessed in the order of their submission using the up and down keys. Moreover, the page up and page down keys can be used to \emph{serach} the history for commands that are equal to the current command up the current cursor position. As an example, consider the following command:
\begin{lstlisting}[title=Example history search]
databases.First().Sessions.Newest()
\end{lstlisting}
If the cursor position is at the end of the line, hitting the page up key will search the history for the last command that started with \lstinline{databases.First().Sessions.Newest()}. Here, it is important to note that only the sub-expression appearing left of the cursor position will be considered in the search. 


\subsection{DBErc}
\label{sec:dberec}
Every user can define custom startup commands, much like in a \lstinline{bashrc} file. The C\# code in \lstinline{DBErc.cs} will be executed once during \BoSSSpad{} startup. This way, variables can be pre-defined and initialized for user convenience.

For example, one could wish to have a variable \lstinline{newest} which always contains the newest session from the first database. This could be achieved by means of a \lstinline{DBErc.cs} with the following content:

\begin{lstlisting}[title=Example DBErc.cs]
var newest = databases.First().Sessions.Newest();
Console.WriteLine();
Console.WriteLine("Newest session ($newest):");
Console.WriteLine(newest);
\end{lstlisting}


\section{Accessing databases and their entities}
This sections contains information about the most common commands and how to use them.

\subsection{IDatabaseInfo}
Variables of type \lstinline{IDatabaseInfo} store basic information about a database. They also provide access to a database's sessions and grids.
The following fields and methods can be used:
\begin{description}
	\item[Clean()]
	Remove unused files (e.~g. leftover timesteps from missing sessions).
	
	\item[Clear()]
	Remove \emph{all} files in the database.
	
	\item[Grids]
	Access all grids in the database.
	
	\item[ImportGrid(string filePath)]
	Imports the grid located at \lstinline{filePath} into the current database.
	
	\item[OpenBaseDirectory()]
	Open an Explorer window with the database's root folder.
	
	\item[Sessions]
	Access all sessions in the database.
	
	\item[Summary()]
	Display a concise summary of the database.
\end{description}

\subsection{ISessionInfo}
Variables of type \lstinline{ISessionInfo} store basic information about a session. They also provide access to a session's timesteps and grid(s).
The following fields and methods can be used:
\begin{description}
	\item[AddTags(String{[}{]} newTags)]
	Add tags to a sessions collection of tags.
	
	\item[Ancestors()]
	Retrieves all ancestors of the current session. That is, the session from which the current has been restarted (if any, c.f. \lstinline{RestartedFrom}), and so on.
	
	\item[AncestorsAndSelf()]
	Retrieves a list containing this session and all its \lstinline{Ancestors()}
	
	\item[Copy(IDatabaseInfo targetDB)]
	Copy the session to another database. Grids are automatically copied as well, if needed.
	
	\item[Delete(\emph{{[}bool force{]}})]
	Delete the session. Use of \lstinline{Delete(true)} will not ask for any confirmation which is useful for deleting a lot of sessions.
	
	\item[Description]
	Description of the session. Can be more detailed than the name.
	
	\item[Diff(ISessionInfo otherSession)]
	Compares the control file of this session and \lstinline{otherSession}. If differences were found, a visual comparison is displayed in a browser window.
	
	\item[Export()]
	Start export of session data (e.g., for plotting purposes), see section \ref{sec:exporting}.
	
	\item[GetApproximateRunTime({[}int firstIndex{]}, {[}lastIndex{]})]
	Estimates the run-time of the time spans between the timesteps within the given range of indices. If no indices are given, the timespan between the first and the last timestep will be considered.
	
	\item[GetAverageComputingTimePerTimestep({[}int firstIndex{]}, {[}lastIndex{]})]
	Estimates the average computing time per timestep by taking the average of the time spans between the timesteps within the given range of indices. If no indices are given, the timespan between the first and the last timestep will be considered.
	
	\item[GetAverageCPUTimePerTimestep({[}int firstIndex{]}, {[}lastIndex{]})]
	Similar to \lstinline{GetAverageComputingTimePerTimestep}, but multiplied by the number of CPU cores.
	
	\item[GetConfig()]
	Reads the control file (old XML version) stored for the session (if any) and returns the corresponding instance of \lstinline{AppControl}
	
	\item[GetConfig<T>()]
	Reads the control file stored (new REPL version) for the session (if any) and returns the corresponding instance of \lstinline{T}, where \lstinline{T} is derived from \lstinline{AppControlV2}. That is, \lstinline{GetConfig<CNSControl>()} will return an instance of \lstinline{CNSControl} which is the specialized control file format for the \lstinline{CNS} solver. Note that the corresponding solver-assembly has to be loaded via \lstinline{LoadAssembly("assemblyName.exe")} first. The best practice is to store the corresponding code in the DBErec (see Section \ref{sec:dberec})
	
	\item[GetDOF(string fieldName)]
	Computes the total number of DOF for the DG field with name \lstinline{fieldName} in this session.
	
	\item[GetOrder(string variableName)]
	Retrieves the polynomial degree associated with a DG field named \lstinline{variableName} in the current session. Note that the result is only meaningful if the polynomial degree does not vary over time.
	
	\item[GetRun()]
	If the given session was part of a parameter study, extracts information about the case index within that study.
	
	\item[GetGrids()]
	Returns the session's grid(s).
	
	\item[Move(IDatabaseInfo targetDB)]
	Move the session to another database.
	
	\item[OpenExportDirectory()]
	Open an explorer window with the session's export directory, i.~e. where the files from \lstinline{Export()} are located.
	
	\item[OpenSessionDirectory()]
	Open an explorer window with the session's base directory inside the database's folder structure.
	
	\item[PrintExportDirectory()]
	Show the path to the session's export directory.
	
	\item[PrintSessionDirectory()]
	Show the path to the session's base directory.
	
	\item[QueryResults(\emph{{[}string query{]}})]
	Show the results of the Queries. With the optional argument \lstinline{query} this returns the value of the selected query.
	
	\item[Residuals(String norm,\emph{{[}int stride{]},{[}String{[}{]} variables{]}})]
	Access a session's residuals. The first argument \lstinline{norm} defines which residual file\footnote{The residual files must have the file name structure \lstinline{residuals-xyz.txt} where \lstinline{xyz} is to be replaced with an identifier such as \lstinline{L2}, \lstinline{Linf}, etc. To read \lstinline{residual-L2.txt}, the method \lstinline{mySess.Residuals("L2")} is called.} will be read. The retrieved residual data can be analyzed in the following ways
	
	\item[Residuals(...).Tail({[}int n{]})]
	Prints the last \lstinline{n} lines of the selected residual data to the console
	
	\item[Residuals(...).TailLive()]
	Prints the last \lstinline{n} lines of the selected residual file to the console
	
	\item[Residuals(...).Plot()]
	Displays a residual plot window (Gnuplot required) for the selected residuals
	
	\item[Residuals(...).PlotLive()]
	Displays a residual plot window that continuously updated when new residual data arrives
	
	\item[RestartedFrom]
	If this session has been restarted from another session, this field contains the original session's ID.
	
	\item[Summary()]
	Display a summary of the session properties.	
	
	\item[Tags]
	The session's tags.
	
	\item[Timesteps]
	Access all timesteps of the session.
\end{description}

\subsection{IEnumerable<ISessionInfo>}
In addition to the use of standard LINQ operators, lists of sessions offer convenient search methods:
\begin{description}
	\item[CollectTags()]
	Collects all tags in the given set of sessions

	\item[Find(string name)]
	Determines exactly one session with the given session name. If more than one session are found, an error will occur.

	\item[FindByGuid(PartialGuid guid)]
	Find exactly one session from a \emph{Partial Guid}\footnote{Partial Guids can be implicitly converted from strings, e.~g. \lstinline{myDB.FindSession("c2")} finds the session that has an ID beginning with \emph{c2}.} and returns a single session. If more than one session are found, an error will occur.
	
	\item[LastDays({[}noOfPreviousDays = 0{]})]
	Returns all sessions which have been created within the last \lstinline{noOfPreviousDays} days.
	
	\item[RestartedFrom(PartialGuid id)]
	Filters all sessions which have been restarted from a session with the given id.
	
	\item[ToDataSet(\ldots)]
	Each of the numerous variants of this method extracts user-defined data from the last timestep of the respective sessions and turns the result into a \lstinline{DataSet} (cf. section \ref{sec:DataSet})
	
	\item[ToEfficiencyData({[}var groupKeySelector{]}, {[}int firstIndex{]}, {[}lastIndex{]})]
	Summarizes the information about the parallel efficiency of the considered sessions. For the meaning of \lstinline{firstIndex} and \lstinline{lastIndex}, see \lstinline{GetApproximateRunTime}.
	
	\item[ToGridConvergenceData(\ldots)]
	Each of the numerous variants of this method extracts data concerning the mesh size and the error from the last timestep of the respective sessions and turns the result into a \lstinline{DataSet} (cf. section \ref{sec:DataSet})
	
	\item[ToEstimatedGridConvergenceData(string fieldName)]
	Extracts data concerning the mesh size and the error of the field with name \lstinline{fieldName} with respect to the finest grid (for each polynomial degree) from the last timestep of the respective sessions and turns the result into a \lstinline{DataSet} (cf. section \ref{sec:DataSet}). Note that the different grid levels must be strictly nested (in the sense that associated edges have to lie exactly on top of each other) in order to get a meaningful result.
	
	\item[ToSpeedUpData({[}var groupKeySelector{]}, {[}int firstIndex{]}, {[}lastIndex{]})]
	Summarizes the information about the parallel speed up of the considered sessions. For the meaning of \lstinline{firstIndex} and \lstinline{lastIndex}, see \lstinline{GetApproximateRunTime}.
	
	\item[ToTimeConvergenceData(\ldots)]
	Each of the numerous variants of this method extracts data concerning the timestep size and the error from the last timestep of the respective sessions and turns the result into a \lstinline{DataSet} (cf. section \ref{sec:DataSet})
	
	\item[ToPerformanceData({[}var groupKeySelector{]}, {[}int firstIndex{]}, {[}lastIndex{]})]
	Extracts data concerning the mesh size and the average CPU time per timestep of the respective sessions and turns the result into a \lstinline{DataSet} (cf. section \ref{sec:DataSet}). For the meaning of \lstinline{firstIndex} and \lstinline{lastIndex}, see \lstinline{GetApproximateRunTime}.
	
	\item[WithGuid(PartialGuid guid)]
	Find multiple, i.~e. one or more, sessions from a partial guid. Returns a list of sessions.

	\item[WithName(string name)]
	Find multiple sessions with a given session name.
	
	\item[WithProject(string projectName)]
	Find all sessions with the given \lstinline{projectName}.
	
	\item[WithTag(String{[}{]} tags)]
	Find multiple sessions from session tags.
\end{description}

\subsection{IGridInfo}
Variables of type \lstinline{IGridInfo} store basic information about a grid.
The following fields and methods can be used:
\begin{description}
	\item[Copy(IDatabaseInfo targetDB)]
	Copy the grid to another database.
	
	\item[Delete(\emph{{[}bool force{]}})]
	Deletes the grid. If no argument or the argument \lstinline{false} is given, the grid will only be removed if it is not used by any session. Use \lstinline{Delete(true)} to force delete the grid.
	
	\item[Export()]
	Start export of grid data (e.g., for plotting purposes), see section \ref{sec:exporting}
	
	\item[GetSessions()]
	Retrieves the sessions that use this grid.
	
	\item[OpenExportDirectory()]
	Open an explorer window with the grid's export directory, i.~e. where the files from \lstinline{Export()} are located.
	
	\item[PrintExportDirectory()]
	Show the path to the grid's export directory.

	\item[Summary()]
	Display a summary of the grid properties.
\end{description}

\subsection{ITimestepInfo}
\label{sec:ITimestepInfo}
Variables of type \lstinline{ITimestepInfo} store basic information about a timestep. Each \lstinline{ITimestepInfo} comes with a \lstinline{TimeStepNumber} that can represent a timestep (e.g.. "1"), a sub-timestep (e.g. "1.2" for the second iteration in the first timestep) or even a sub-sub-timestep (e.g., "1.2.1" for the first sub-iteration of the second timestep in the first timestep)
The following fields and methods can be used:
\begin{description}
	\item[Fields]
	The saved state of the DG fields for this timestep. Individual fields can be accessed by their name/identification.
	
	\item[Export()]
	Start export of timestep data (e.g., for plotting purposes), see section \ref{sec:exporting}
	
	\item[Summary()]
	Display a summary of the timestep properties.
\end{description}

\subsection{IEnumerable<ITimestepInfo>}
In addition to the use of standard LINQ operators, lists of timesteps offer convenient search methods:
\begin{description}
	\item[Find(TimeStepNumber number)]
	Find exactly one timestep with the given \lstinline{TimeStepNumber}\footnote{Numbers of full timesteps can be implicitly converted from \lstinline{int}. Number of arbitrary timesteps can implicitly be converted from \lstinline{string}}
	
	\item[Export()]
	Start export of timestep data (e.g., for plotting purposes), see section \ref{sec:exporting} for the given timesteps. Note that, at the moment, the timesteps must belong to the same session.
	
	\item[From(TimestepNumber number)]
	Filters all timesteps with a timestep number greater than or equal to \lstinline{number}.
	
	\item[To(TimestepNumber number)]
	Filters all timesteps with a timestep number less than or equal to \lstinline{number}.
	
	\item[ToDataSet(\ldots)]
	Each of the numerous variants of this method extracts user-defined data from the timesteps and turns the result into a \lstinline{DataSet} (cf. section \ref{sec:DataSet})
	
	\item[ToGridConvergenceData(\ldots)]
	Each of the numerous variants of this method extracts data concerning the mesh size and the error from the timesteps and turns the result into a \lstinline{DataSet} (cf. section \ref{sec:DataSet})
	
	\item[ToEstimatedGridConvergenceData(string fieldName)]
	Extracts data concerning the mesh size and the error of the field with name \lstinline{fieldName} with respect to the finest grid (for each polynomial degree) from the timesteps and turns the result into a \lstinline{DataSet} (cf. section \ref{sec:DataSet}). Note that the different grid levels must be strictly nested (in the sense that associated edges have to lie exactly on top of each other) in order to get a meaningful result.
	
	\item[ToTimeConvergenceData(\ldots)]
	Each of the numerous variants of this method extracts data concerning the timestep size and the error from the timesteps and turns the result into a \lstinline{DataSet} (cf. section \ref{sec:DataSet})
	
	\item[WithoutSubSteps()]
	Filters all sub-timesteps from the given list of timesteps.
\end{description}

\subsection{Collections of database entities}
The following methods can be used with collections \lstinline{IEnumerable<T>} of entities. As of now, \lstinline{T} can only be of type \lstinline{ISessionInfo} or (for most of those methods) \lstinline{IGridInfo}.
\begin{description}
	\item[CopyAll(IDatabaseInfo targetDB)]
	Copies all sessions or grids to \lstinline{targetDB}.
	
	\item[DeleteAll()]
	Deletes all sessions or grids in the collection from their database.
	
	\item[MoveAll(IDatabaseInfo targetDB)]
	Moves all sessions to \lstinline{targetDB}.
\end{description}

\subsection{DataSet}
\label{sec:DataSet}

A \lstinline{DataSet} represents a potentially grouped set of data in the form of key-value pairs. As a result, its main purpose is the analysis of sets of calculations and the preparation of the respective plots.
\begin{description}
	\item[Extract(params string{[}{]} groupNames))]
	Extracts the data rows associated with the given group name(s)
	
	\item[Merge(DataSet otherDataSet)]
	Merges the given data set into this data set
	
	\item[Plot()]
	Visualizes the data
	
	\item[Regression()]
	Calculates the coefficients in the form of the linear fit for each group of values
	
	\item[SaveToTextFile(String path)]
	Save the data to a text file in tabular form
	
	\item[SaveGnuplotFile(String path)]
	Save the data in a gnuplot file
	
	\item[SavePgfplotsFile(String path)]
	Save the data in a latex file for creating a plot in a publication
	
	\item[SaveTextFileToPublish(String path)]
	Save the data to a text file in tabular form, addionally a second text file saves the linear regression values of this data.\\ Naming convention: path+Data.txt and path+Rgrs.txt
	
	\item[Without(params string{[}{]} groupNames))]
	Subtracts the data rows associated with the given group name(s)
\end{description}


\section{Exporting}
\label{sec:exporting}

The command \lstinline{Export()} creates a so-called \lstinline{ExportInstruction} for a session, a timestep or a grid with default options. The default options can be modified by calling one or more of the below commands via method chaining (also, see section \ref{sec:examples}). However, the actual export process does not start before the command \lstinline{Do()} is called.

The following methods are available for all types of \lstinline{ExportInstruction} (Sessions, timesteps and grids):
\begin{description}
	\item[AndOpenYouMust()]
	Equivalent (but funnier) alternative to \lstinline{DoAndOpen()}.
	
	\item[Do()]
	Starts the export process.
	
	\item[DoAndOpen()]
	Starts the export process and opens the first exported file using the default application for the corresponding file type.
	
	\item[To(string path)]
	Overrides the default path for the output files and writes them to \lstinline{path} instead.
	
	\item[WithSupersampling(int n)]
	Number of recursive sub-divisions of each individual grid cell. Default is zero.
	
	\item[WithGhostCells()]
	Enables the export of ghost cells for parallel runs (which is turned off by default).
	
	\item[WithMPI(uint n)]
	Uses \lstinline{n} processes for export process, thus resulting in \lstinline{n} separate plot files. Default is one.
	
	\item[WithFormat(string format)]
	Sets the export format \lstinline{format} (case insensitive). Supported formats are 'TecPlot', 'CGNS', 'CSV', 'Curve' (one-dimensional data only).
	
	\item[YouMust()]
	Equivalent (but funnier) alternative to \lstinline{Do()}.
\end{description}

The following methods are available for sessions and timesteps:
\begin{description}
	\item[WithFields(params string{[}{]} fieldNames)]
	Restricts the set of exported fields to the fields with the given names. By default, all stored fields are included in the export.
	
	\item[WithReconstruction()]
	Enables reconstruction of a smooth variable field (which is turned off by default) by calculating node-wise averages
\end{description}

The following methods are available for sessions only:
\begin{description}
	\item[WithTimesteps(params TimestepNumber{[}{]} numbers)]
	Restricts the set of exported timesteps to the timesteps with the given \lstinline{numbers}. Full timesteps can be given as integer numbers (cf. section \ref{sec:ITimestepInfo}).
\end{description}


\section{LINQ basics}
LINQ is a very powerful tool for filtering, selecting and sorting data which makes it especially useful in the context of BoSSS databases.
The following section will provide a very brief overview of the most useful commands. A more extensive documentation of its features can be found at
\url{http://en.wikipedia.org/wiki/LINQ#Standard\_Query\_Operators}

LINQ queries usually work on collections, such as \lstinline{IList} or \lstinline{IEnumerable}.
Multiple queries can be combined to create highly sophisticated filters.

\subsection{Picking entities}
Commands to pick a certain element from a collection include the following.
\begin{description}
	\item[First()]
	Picks the first entity in a collection. \lstinline{Second()} and \lstinline{Third()} also exist for convenience.
	
	\item[Pick(int n)]
	Picks the $n$-th entity in a collection, starting from zero (i.e., it is a synonym for the built-in function \lstinline{ElementAt}).
	
	\item[Pick(params int[] indices)]
	Picks the elements at the given \lstinline{indices} from given collection, where each index is zero-based.
	
	\item[Newest()]
	Retrieves the newest element from a collection of \lstinline{IDatabaseEntityInfo} objects, sorted by creation time.
	
	\item[Oldest()]
	Retrieves the oldest \lstinline{IDatabaseEntityInfo} object.
\end{description}

\begin{lstlisting}[title=Picking entities with LINQ]
> databases.First() // First database in the list
> databases.First().Sessions.Newest() // Newest session of first database
\end{lstlisting}

\subsection{Querying collections}
The most important query commands are shown in this section.
The method's arguments are usually functions. Therefore, the use of queries requires basic knowledge of \emph{lambda expressions}, a way to anonymously define inline functions.
A few basic examples will illustrate how to use lambda expressions for queries.
\begin{description}
	\item[ForEach(...)]
	Iterates over the whole collection. Similar to the well-known foreach loop in standard C\# syntax.
	
	\item[Select(...)]
	Maps aspects of the collections entities to a new collection.
	
	\item[SelectMany(...)]
	Same as \lstinline{Select}, but flattens collections of collections.
	
	\item[Where(...)]
	Provides basic filtering. The argument has to be a function that returns a \lstinline{bool}.
\end{description}

\begin{lstlisting}[title=Querying collections with LINQ]
// Let 'sessions' be a collection of ISessionInfo objects

// Loop over all sessions and delete them one by one
> sessions.ForEach(sess => sess.Delete())

// Select each session's ID and save them in a separate list.
// Note that the argument name in the lambda expression (here: s)
// chosen freely
> var sessionIDs = sessions.Select(s => s.ID)

// Gather all session's grids and flatten the list of lists to a single list
> var sessionGrids = sessions.SelectMany(s => s.GetGrids())

// Collect all sessions that have "test" in their name
> var testSessions = sessions.Where(s => s.Name.ToLower().Contains("test"))
\end{lstlisting}

%\subsection{Miscellaneous}
%Note that LINQ queries usually create a query or implicit collection that is not evaluated right away.
%\todo[inline]{ToList(), ToArray()}


\section{Examples}
\label{sec:examples}
In this section, examples are shown to illustrate basic and more advanced workflows that are possible with the command line.
\begin{lstlisting}[title=Basic examples]
// Display a summary of all available databases
> databases

// Create a new variable
> var db1 = databases.First()

// Confirm the variable has been created
> ShowVars()

// Show a (potentially incomplete) list of methods that can be called on $db1
> db1.Actions()

// Display the first database's sessions
> db1.Sessions
\end{lstlisting}

Assuming that the field \lstinline{db1} still exists, the following commands are possible.
\begin{lstlisting}[title=Advanced Examples]
// Finds all sessions whose ID starts with "a4c" and stores the result in
// the variable 'session'
> var session = db1.Sessions.WithGuid("a4c")

// Same as above, but with LINQ
> var session = db1.Sessions.Where(s => s.ID.ToString().StartsWith("a4c"))

// Plot the L2-Residuals
> db1.Sessions.Newest().Residuals("L2").Plot()

// Same as above, but now only plot for every 5th iteration/timestep
// and only for the "u" field
> db1.Sessions.Newest().Residuals("L2", 5 ,"u").Plot()

// Calculate the L2 norm of a field
> session.Timesteps.Last().Fields["rho"].L2Norm()

// Exports the data for all timesteps of the session with standard settings
> session.Export().Do()

// Open the folder containing the exported files
> session.OpenExportDirectory()

// Same as above, but using Yoda style
> session.Export().YouMust()

// List all timesteps in the session
> session.Timesteps

// Exports every 5th timestep in the session
> session.Timesteps.Every(5).Export().Do()

// Exports the data for the last timestep with two levels of supersampling
> session.Timesteps.Last().Export().WithSupersampling(2).YouMust()

// Create a list of sessions by taking the first 5 sessions from $db1 and 
// appending sessions 11 to 15
> var sessions = db1.Sessions.Take(5).Concat(db1.Sessions.Skip(10).Take(5))

// Copy $sessions to another database
foreach(ISessionInfo s in sessions) { s.Copy(databases.Third()); }

// Same as above, using LINQ
> sessions.ForEach(s => s.Copy(databases.Third()))

// Still the same, this time with a built-in command
> sessions.CopyAll(databases.Third())

\end{lstlisting}

\begin{lstlisting}[title=Advanced example: Exporting a Dataset]
//The database in this example contains only files of comparable calculations.

// Select the Timesteps with the ID "500"
> var mytimesteps = databases.Second().Sessions.Select(s => s.Timesteps.Find("500"))

// Create a Dataset with the Meshsize as x-axis,
// the L2Norm of the Field "DivAfter" as y- axis and
// the polynomial degree of the Field as grouing parameter
> var mydataset = mytimesteps.ToDataSet(
	t => t.Grid.GetMeshSize(),
	t => t.Fields.Find("DivAfter").L2Norm(),
	t => t.Fields.Find("VelocityX").Basis.Degree.ToString()
	)
	
// Plot Dataset in a double-logarithmic diagram
> mydataset.WithLogX().WithLogY().Plot()

// Export Dataset in Different Formats
> mydataset.WithLogX().WithLogY().SaveGnuplotFile(@"C:\tmp\gnuplot.gp")
> mydataset.WithLogX().WithLogY().SavePgfplotsFile(@"C:\tmp\pgfplots.tex")
> mydataset.WithLogX().WithLogY().SaveToTextFile(@"C:\tmp\text.txt")
\end{lstlisting}

\begin{lstlisting}[title=Advanced example: Time history in a specific point]
// Select some session
> var session = databases.First().Sessions.First()

// Create a data set with the physical time on the x-axis
// and the pressure at (1.0, 0.0) on the y-axis
> var dataset = Timesteps.ToDataSet(
	t => t.PhysicalTime,
	t => t.Fields.Find("pressure").ProbeAt(1.0, 0.0))

// Plot data set
> dataset.Plot()

// Export data set in different formats
> dataset.SaveGnuplotFile(@"C:\tmp\gnuplot.gp")
> dataset.SavePgfplotsFile(@"C:\tmp\pgfplots.tex")
> dataset.SaveToTextFile(@"C:\tmp\text.txt")
\end{lstlisting}